\title{Лекция 9\\Базы знаний интеллектуальных систем \vspace{-2em}}   
%\author[]{Шункевич Д.В.}
%\institute[]{Белорусский государственный университет информатики и радиоэлектроники}

\begin{frame}
	\titlepage
\end{frame}

\begin{frame}{\\Содержание лекции}
	\topline
	\justifying

	\begin{SCn}
		\scnheader{Лекция 9. Базы знаний интеллектуальных систем}
		\begin{scnrelfromset}{структура}
			\scnitem{Виды знаний и модели их представления, база знаний, требования, предъявляемые к базам знаний, критерии качеств}
			\scnitem{Четкие и нечеткие множества и знания, смысл, смысловое представление знаний}
			\scnitem{Предметная область, онтология}
			\scnitem{Высказывания и формальные теории}
		\end{scnrelfromset}
	\end{SCn}
\end{frame}

\begin{frame}
\centering
\Huge
\textbf{Виды знаний и модели их представления, база знаний, требования, предъявляемые к базам знаний, критерии качеств}
\end{frame}

\begin{frame}{\\Знание}
	\topline
	\justifying
	
	\begin{SCn}
		\scnheader{знание}
		\scnidtf{синтаксически корректная (для соответствующего языка) и семантически целостная информационная конструкция}
		\scnrelfrom{покрытие}{вид знаний}
		\begin{scnindent}
			\scnidtf{Множество всевозможных всевозможных видов знаний}
		\end{scnindent}
	\end{SCn}

	Тот факт, что семейство видов знаний является покрытием Множества всевозможных знаний, означает то, что каждое знание принадлежит по крайней мере одному выделенному нами виду знаний.
\end{frame}

\begin{frame}{\\Виды знаний}
	\topline
	\justifying

	\begin{SCn}
		\scnheader{вид знаний}
		\scnhaselement{спецификация}
		\begin{scnindent}
			\scnidtf{описание заданной сущности}
		\end{scnindent}
		\scnhaselement{метазнание}
		\begin{scnindent}
			\scnidtf{спецификация самих знаний}
		\end{scnindent}
		\scnhaselement{задача}
		\begin{scnindent}
			\scnidtf{спецификация действия}
		\end{scnindent}
		\scnhaselement{сравнение}
		\scnhaselement{высказывание}
		\scnhaselement{формальная теория}
		\scnhaselement{предметная область}
		\scnhaselement{предметная область и онтология}
	\end{SCn}
\end{frame}

\begin{frame}{\\Виды знаний}
	\topline
	\justifying

	\begin{SCn}
		\scnheader{вид знаний}
		\scnhaselement{сравнение}
		\scnhaselement{план}
		\scnhaselement{протокол}
		\scnhaselement{результативная часть протокола}
		\scnhaselement{метод}
		\scnhaselement{технология}
		\scnhaselement{база знаний}
	\end{SCn}

	Даже небольшой перечень видов знаний свидетельствует об огромном многообразии видов знаний.
\end{frame}

\begin{frame}{Знание}
	\topline
	\justifying

	\begin{SCn}
		\scnheader{знание}
		\begin{scnrelfromset}{разбиение}
			\scnitem{декларативное знание}
			\begin{scnindent}
				\scnidtf{знание, имеющее только денотационную семантику в виде
				семантической спецификации системы используемых понятий}
			\end{scnindent}
			\scnitem{процедурное знание}
			\begin{scnindent}
				\scnidtf{знание, имеющее не только денотационную семантику, но
				и операционную семантику в виде семейства спецификаций агентов,
				интерпретирующих знание, направленные на решение некоторой
				задачи}
			\end{scnindent}
		\end{scnrelfromset}
	\end{SCn}
\end{frame}

\begin{frame}{\\Отношения на множестве знаний}
	\topline
	\justifying

	\begin{SCn}
		\scnheader{отношение, заданное на множестве знаний}
		\scnhaselement{дочернее знание*}
		\begin{scnindent}
			\scnidtf{знание, которое от "материнского"{} знания наследует все описанные там свойства объектов исследования}
		\end{scnindent}
		\scnhaselement{спецификация*}
		\begin{scnindent}
			\scnidtf{быть знанием, которое является спецификацией (описанием) заданной сущности}
		\end{scnindent}
		\scnhaselement{онтология*}
		\begin{scnindent}
			\scnidtf{быть семантической спецификацией заданного знания*}
		\end{scnindent}
		\scnhaselement{следовательно*}
		\begin{scnindent}
			\scnidtf{логическое следствие}
		\end{scnindent}
		\scnhaselement{семантическая эквивалентность*}
		\scnhaselement{логическая эквивалентность*}
	\end{SCn}
\end{frame}

\begin{frame}{\\База знаний}
	\topline
	\justifying

	\begin{SCn}
		\scnheader{база знаний}
		\scnidtf{совокупность знаний, хранимых в памяти интеллектуальной компьютерной системы и достаточных для того, чтобы указанная система удовлетворяла соответствующим предъявляемым к ней требованиям (в частности, чтобы она имела соответствующий уровень интеллекта)}
		\scnidtf{систематизированная совокупность знаний, хранимая в памяти
		интеллектуальной компьютерной системы и достаточная для обеспечения
		целенаправленного (целесообразного, адекватного) функционирования (поведения) этой системы как в своей внешней среде, так и в своей внутренней среде (в собственной базе знаний)}
		\scniselement{вид знаний}
	\end{SCn}
\end{frame}

\begin{frame}{\\Критерии качества баз знаний}
	\topline
	\justifying

	\begin{SCn}
		\scnheader{критерий качества баз знаний}
		\scnhaselement{структуризация}
		\begin{scnindent}
			\scntext{пояснение}{при накоплении больших объемов информации в базе знаний возникает необходимость выделять целые фрагменты базы знаний и иметь возможность их специфицировать, рассматривая как отдельные сущности}
		\end{scnindent}
		\scnhaselement{стратификация}
		\begin{scnindent}
			\scntext{пояснение}{каждый фрагмент базы знаний должен иметь свою
			семантическую <<полочку>> (никакого дублирования)}
		\end{scnindent}
	\end{SCn}
\end{frame}

\begin{frame}
\centering
\Huge
\textbf{Четкие и нечеткие множества и знания, смысл, смысловое представление знаний}
\end{frame}

\begin{frame}{\\Множества}
	\topline
	\justifying

	\begin{SCn}
		\scnheader{множество}
		\scnidtf{соединение в некое целое M определенных хорошо различимых предметов m нашего созерцания или нашего мышления (которые будут называться <<элементами>> множества M)}
		\scnidtf{мысленная сущность, которая связывает одну или несколько сущностей в целое}
		\scnidtf{абстрактный математический объект, состоящий из элементов. Связь множеств с их элементами задается бинарным ориентированным отношением \textit{принадлежность*}}
	\end{SCn}
\end{frame}

\begin{frame}{\\Четкие и нечеткие множества}
	\topline
	\justifying

	\begin{SCn}
		\scnheader{множество}
		\begin{scnrelfromset}{разбиение}
			\scnitem{нечеткое множество}
			\begin{scnindent}
				\scntext{пояснение}{нечеткое множество -- это множество, которое представляет собой совокупность элементов произвольной природы, относительно которых нельзя точно утверждать -- обладают ли эти элементы некоторым характеристическим свойством, которое используется для задания этого нечеткого множества}
			\end{scnindent}
			\scnitem{четкое множество}
			\begin{scnindent}
				\scntext{пояснение}{четкое множество -- это множество, принадлежность элементов которому достоверна}
			\end{scnindent}
		\end{scnrelfromset}		
	\end{SCn}
\end{frame}

\begin{frame}{\\Смысл}
	\topline
	\justifying

	\begin{SCn}
		\scnheader{смысл}
		\scnidtf{абстрактная знаковая конструкция, принадлежащая внутреннему языку компьютерной системы, являющаяся инвариантом максимального класса семантически эквивалентных знаковых конструкций (текстов), принадлежащих самым разным языкам}
	\end{SCn}	
\end{frame}

\begin{frame}{\\Требования ко смыслу}
	\topline
	\justifying

	\begin{SCn}
		\scnheader{смысл}
		\begin{scnrelfromset}{требования}
			\scnitem{универсальность}
			\scnitem{отсутствие синонимии знаков}
			\scnitem{отстутсвие дублирования знаний}
			\scnitem{отсутствие омонимичных знаков}
			\scnitem{атомарность знаков}
			\scnitem{отсутствие фрагментов знаковой конструкции, не являющихся знаками}
			\scnitem{выделение знаков связей}
		\end{scnrelfromset}
	\end{SCn}	
\end{frame}

\begin{frame}{\\Требования ко смыслу}
	\topline
	\justifying

	\begin{SCn}
		\scnheader{универсальность}
		\scnidtf{возможность представления любой информации}
		\scnheader{отсутствие синонимии знаков}
		\scnidtf{отсутствие многократного вхождения знаков с одинаковыми денотатами}
		\scnheader{отсутствие дублирования информации}
		\scnidtf{отсутствие семантически эквивалентных текстов (не путать с логической эквивалентностью)}
	\end{SCn}
\end{frame}

\begin{frame}{\\Требования ко смыслу}
	\topline
	\justifying
	
	\begin{SCn}
		\scnheader{атомарность знаков}
		\scnidtf{отсутствие у знаков внутренней структуры}
		\scnheader{выделение знаков связей}
		\scntext{пояснение}{компонентами знаков связей могут быть любые знаки, с которыми знаки
		связей связываются синтаксически задаваемыми отношениями инцидентности}
	\end{SCn}
\end{frame}

\begin{frame}{\\Требования ко смыслу}
	\topline
	\justifying
	
	Следствием указанных принципов смыслового представления информации в памяти
	компьютерной системы является то, что знаки сущностей, входящие в смысловое
	представление информации, не являются именами и, следовательно, не привязаны ни к
	какому естественному языку.\\~\\

	Также, эти принципы приводят к нелинейным знаковым конструкциям (к графовым структурам),
	что усложняет реализацию памяти компьютерных систем, но существенно упрощает ее
	логическую организацию.
\end{frame}

\begin{frame}{\\Смысловое представление знаний}
	\topline
	\justifying

	\begin{SCn}
		\scnheader{смысловое представление знаний}
		\scnidtf{запись (представление) информационной конструкции на смысловом уровне}
		\scnidtf{информационная конструкция, синтаксическая структура которой близка ее
		смыслу, то есть близка описываемой конфигурации связей между описываемыми сущностями}
		\scnsuperset{семантическая сеть}
		\begin{scnindent}
			\scnsuperset{рафинированная семантическая сеть}
		\end{scnindent}
	\end{SCn}
\end{frame}

\begin{frame}
\centering
\Huge
\textbf{Предметная область, онтология}
\end{frame}

\begin{frame}{\\Предметная область}
	\topline
	\justifying

	\begin{SCn}
		\scnheader{предметная область}
		\scnsubset{знание}
		\scnsubset{бесконечное множество}

		\scnidtf{результат объединения частичных семантических окрестностей, описывающих
		все исследуемые сущности заданного класса и имеющих общий предмет исследования}
	\end{SCn}
\end{frame}

\begin{frame}{\\Предметная область}
	\topline
	\justifying
	
	\begin{SCn}
		\scnheader{предметная область}
		\scnidtf{
			система связей некоторого множества объектов исследования, ключевыми
			элементами которой являются:
			\begin{textitemize}
				\item классы объектов исследования;
				\item конкретные объекты исследования, обладающие особыми свойствами;
				\item отношения, заданные на множестве элементов рассматриваемой системы;
				\item параметры, заданные на множестве элементов рассматриваемой системы;
				\item классы структур, являющихся фрагментами рассматриваемой системы.
			\end{textitemize}
		}
	\end{SCn}
\end{frame}

\begin{frame}{\\Предметная область}
	\topline
	\justifying

	Выделяемые в рамках базы знаний интеллектуальной системы предметные области и соответствующие им онтологии -- это, своего рода, семантические страты, кластеры, позволяющие <<разложить>> все хранимые в памяти знания по <<семантическим полочкам>> при наличии четких критериев, позволяющих однозначно определить то, на какой <<полочке>> должны находиться те или иные знания.
\end{frame}

\begin{frame}{\\Виды предметных областей}
	\topline
	\justifying

	\begin{SCn}
		\scnheader{предметная область}
		\begin{scnrelfromset}{разбиение}
			\scnitem{статическая предметная область}
			\begin{scnindent}
				\scnidtf{предметная область, в которой связи между сущностями,
				входящими в ее состав, не зависят от времени}
			\end{scnindent}
			\scnitem{квазистатическая предметная область}
			\begin{scnindent}
				\scnidtf{предметная область, решение задач в которой не требует
				учета темпоральных свойств объектов исследования}
			\end{scnindent}
			\scnitem{динамическая предметная область}
			\begin{scnindent}
				\scnidtf{предметная область, в которой некоторые связи между
				сущностями, входящими в ее состав, меняются со временем}
			\end{scnindent}
		\end{scnrelfromset}
	\end{SCn}
\end{frame}

\begin{frame}{\\Предметная область}
	\topline
	\justifying

	\begin{SCn}
		\scnheader{предметная область}
		\scnhaselement{Предметная область предметных областей}
		\begin{scnindent}
			\scntext{пояснение}{объектами исследования являются
			всевозможные предметные области, а предметом исследования
			являются -- всевозможные ролевые отношения, связывающие
			предметные области с их элементами, отношения, связывающие
			предметные области между собой, отношение, связывающее
			предметные области с их онтологиями}
		\end{scnindent}
		\scnhaselement{Предметная область сущностей}
		\begin{scnindent}
			\scntext{пояснение}{предметная область самого высокого уровня}
		\end{scnindent}
	\end{SCn}
\end{frame}

\begin{frame}{\\Онтология}
	\topline
	\justifying

	\begin{SCn}
		\scnheader{онтология}
		\scnidtf{семантическая спецификация любого знания, имеющего достаточно
		сложную структуру, любого целостного фрагмента базы знаний -- предметной области, метода решения сложных задач некоторого класса, описания истории некоторого вида деятельности, описания области выполнения некоторого множества действий (области решения задач), языка представления методов решения задач и т.д}
		\scnsubset{спецификация}
		\scnsubset{метазнание}
	\end{SCn}
\end{frame}

\begin{frame}{\\Структура онтологии}
	\topline
	\justifying

	\begin{SCn}
		\scnheader{онтология}
		\scnsuperset{типология специфируемого знания}
		\scnsuperset{связи специфируемого знания с другими знаниями}
		\scnsuperset{спецификация ключевых понятий и их экземпляров}
		\scntext{пояснение}{Если спецификация может специфицировать (описывать) любую сущность, то онтология специфицирует только различные знания. При этом наиболее важными объектами такой спецификации являются предметные области}
	\end{SCn}
\end{frame}

\begin{frame}{\\Виды онтологий}
	\topline
	\justifying

	\begin{SCn}
		\scnheader{онтология}
		\begin{scnrelfromset}{разбиение}
			\scnitem{неформальная онтология}
			\scnitem{формальная онтология}
			\begin{scnindent}
				\scnidtf{онтология, представленная на формальном языке}
				\scnidtf{формальное описание денотационной семантики
				(семантической интерпретации) специфицируемого знания}
				\scntext{пояснение}{При отсутствии достаточно полных формальных онтологий невозможно обеспечить семантическую совместимость (интегрируемость) различных знаний, хранимых в базе знаний, а также приобретаемых извне.}
			\end{scnindent}
		\end{scnrelfromset}
	\end{SCn}
\end{frame}

\begin{frame}{\\Онтология предметной области}
	\topline
	\justifying

	\begin{SCn}
		\scnheader{онтология предметной области}
		\scnidtf{описание денотационной семантики языка, определяемого (задаваемого)
		соответствующей (специфицируемой) предметной областью}
		\scnidtf{информационная надстройка (метаинформация) над соответствующей
		(специфицируемой) предметной областью, описывающая различные аспекты этой предметной области как достаточно крупного, самодостаточного и семантически целостного фрагмента базы знаний}
		\scnidtf{метаинформация (метазнание) о некоторой предметной области}
	\end{SCn}
\end{frame}

\begin{frame}{\\Виды онтологий предметной области}
	\topline
	\justifying

	\begin{SCn}
		\scnheader{онтология предметной области}
		\begin{scnrelfromset}{разибение}
			\scnitem{частная онтология предметной области}
			\begin{scnindent}
				\scnidtf{онтология, представляющая спецификацию соответствующей
				предметной области в том или ином аспекте}
			\end{scnindent}
			\scnitem{объединенная онтология предметной области}
			\begin{scnindent}
				\scnidtf{онтология предметной области, являющаяся результатом
				объединения всех известных частных онтологий этой предметной области}
			\end{scnindent}
		\end{scnrelfromset}
	\end{SCn}	
\end{frame}

\begin{frame}{\\Частная онтология предметной области}
	\topline
	\justifying

	\begin{SCn}
		\scnheader{частная онтология предметной области}
		\begin{scnrelfromset}{разбиение}
			\scnitem{структурная спецификация предметной области}
			\begin{scnindent}
				\scnidtf{вид метазнаний, описывающих соответствующие этому виду
				метазнаний свойства предметных областей}
				\scnidtf{схема предметной области}
			\end{scnindent}
			\scnitem{теоретико-множественная онтология предметной области}
			\begin{scnindent}
				\scnidtf{спецификация заданной предметной области в рамках
				Предметной области множеств}
			\end{scnindent}
			\scnitem{логическая онтология предметной области}
			\begin{scnindent}
				\scnidtf{текст формальной теории заданной предметной области}
			\end{scnindent}
			\scnitem{терминологическая онтология предметной области}
		\end{scnrelfromset}
	\end{SCn}
\end{frame}

\begin{frame}{\\Частная онтология предметной области}
	\topline
	\justifying

	\begin{SCn}
		\scnheader{структурная спецификация предметной области}
		\scnidtf{схема ролей понятий предметной области и ее связи со смежными предметными областями}
		\scnheader{теоретико-множественная онтология предметной области}
		\scnidtf{семантическая окрестность специфицируемой предметной области в рамках
		Предметной области множеств, описывающая теоретико-множественные связи между
		понятиями специфицируемой предметной области, включая связи отношений с их
		областями определения и доменами, связи используемых параметров и классов
		структур их областями определения}
	\end{SCn}
\end{frame}

\begin{frame}{\\Частная онтология предметной области}
	\topline
	\justifying

	\begin{SCn}
		\scnheader{логическая онтология предметной области}
		\scnidtf{формальная теория заданной (специфицируемой) предметной области, описывающая с помощью переменных, кванторов, логических связок, формул различные свойства экземпляров понятий, используемых в специфицируемой предметной области}
		\scnheader{терминологическая онтология предметной области}
		\scnidtf{онтология, описывающая правила построения терминов, соответствующих элементам, принадлежащим специфицируемой предметной области, а также описывающая различного рода терминологические связи между используемыми терминами, характеризующие происхождение этих терминов}
	\end{SCn}
\end{frame}

\begin{frame}{\\Объединенная онтология предметной области}
	\topline
	\justifying

	\begin{SCn}
		\scnheader{объединенная онтология предметной области}
		\scnidtf{объединение всех частных онтологий, соответствующих одной предметной области}
		\begin{scnreltoset}{обобщенное объединение}
			\scnitem{структурная спецификация предметной области}
			\scnitem{теоретико-множественная онтология предметной области}
			\scnitem{логическая онтология предметной области}
			\scnitem{терминологическая онтология предметной области}
		\end{scnreltoset}
	\end{SCn}
\end{frame}

\begin{frame}{\\Предметная область и онтология}
	\topline
	\justifying

	\begin{SCn}
		\scnheader{предметная область и онтология}
		\scnidtf{интеграция некоторой предметной области c соответствующей ей объединенной онтологией}
		\begin{scnreltoset}{обобщенное объединение}
			\scnitem{предметная область}
			\scnitem{онтология}
		\end{scnreltoset}
		\scntext{пояснение}{предметные области и онтологии являются основным видом разделов баз знаний, обладающих высокой степенью их независимости друг от друга и четкими правилами их согласования, что обеспечивает их семантическую (понятную) совместимость в рамках всей базы знаний}
	\end{SCn}
\end{frame}

\begin{frame}{\\Онтологии верхнего уровня}
	\topline
	\justifying
	
	\begin{SCn}
		\scnheader{онтология верхнего уровня}
		\scnsubset{онтология}
		\scnidtf{онтология, описывающая фундаментальные понятия, которые являются
		общими для всех предметных областей}
		\scnidtf{онтология, систематизирующая знания о реальном мире безотносительно к какой-либо конкретной предметной области}
		\scntext{цель}{поддержка семантической совместимости онтологий предметных областей и прикладных онтологий}
	\end{SCn}
\end{frame}

\begin{frame}
\centering
\Huge
\textbf{Высказывания и формальные теории}
\end{frame}

\begin{frame}{\\Логическая формула}
	\topline
	\justifying

	\begin{SCn}
		\scnheader{логическая формула}
		\begin{scnrelfromset}{разбиение}
			\scnitem{атомарная логическая формула}
			\scnitem{неатомарная логическая формула}
		\end{scnrelfromset}
		\begin{scnrelfromset}{разбиение}
			\scnitem{замкнутая логическая формула}
			\scnitem{открытая логическая формула}
		\end{scnrelfromset}
	\end{SCn}
\end{frame}

\begin{frame}{\\Высказывание}
	\topline
	\justifying

	\begin{SCn}
		\scnheader{высказывание}
		\scnidtf{структура(в которую входят константы из некоторой предметной области и/или переменные) или логическая связка, которая может трактоваться как истинная или ложная в рамках какой-либо предметной области.}
	\end{SCn}
\end{frame}

\begin{frame}{\\Высказывание}
	\topline
	\justifying

	\begin{SCn}
		\scnheader{высказывание}
		\scnsubset{логическая формула}
		\begin{scnrelfromset}{разбиение}
			\scnitem{атомарное высказывание}
			\scnitem{неатомарное высказывание}
		\end{scnrelfromset}
		\begin{scnrelfromset}{разбиение}
			\scnitem{фактографическое высказывание}
			\begin{scnindent}
				\scnsubset{замкнутая логическая формула}
			\end{scnindent}
			\scnitem{нефактографическое высказывание}
		\end{scnrelfromset}
	\end{SCn}
\end{frame}

\begin{frame}{\\Атомарное и неатомарное высказывание}
\topline
\justifying

\begin{SCn}
	\scnheader{атомарное высказывание}
	\scnsubset{структура}
	\scnidtf{высказывание, которое не является неатомарным высказыванием.}
	\begin{scnrelfromset}{разбиение}
		\scnitem{атомарное фактографическое высказывание}
		\scnitem{атомарное нефактографическое высказывание}
	\end{scnrelfromset}
	
	\scnheader{неатомарное высказывание}
	\scnidtf{высказывание, в состав которого входят только знаки логических
		формул или множества связываемых переменных}
\end{SCn}
\end{frame}

\begin{frame}{\\Фактографическое высказывание}
	\topline
	\justifying

	\begin{SCn}
		\scnheader{фактографическое высказывание}
		\begin{scnrelfromset}{разбиение}
			\scnitem{атомарное фактографическое высказывание}
			\begin{scnindent}
				\scnidtf{атомарное высказывание, в состав которого не входит ни одна переменная}
			\end{scnindent}
			\scnitem{неатомарное фактографическое высказывание}
			\begin{scnindent}
				\scnidtf{неатомарное высказывание, все элементы которого также являются фактографическими высказываниями}
			\end{scnindent}
		\end{scnrelfromset}
	\end{SCn}
\end{frame}

\begin{frame}{\\Высказывание*}
	\topline
	\justifying

	\begin{SCn}
		\scnheader{высказывание*}
		\scnidtf{бинарное ориентированное отношение, где каждая пара связывает
		(1) знак некоторой предметной области и (2) знак высказывания.}
		\begin{scnrelfromset}{разбиение}
			\scnitem{ложное высказывание*}
			\scnitem{неразрешимое высказывание*}
			\scnitem{истинное высказывание*}
		\end{scnrelfromset}
		\scntext{предъявляемое требование}{Все константы, входящие в состав всех
		атомарных логических формул, входящих в состав всех высказываний,
		описывающих некоторую предметную область должны входить в состав 
		описываемой предметной области.}
	\end{SCn}
\end{frame}

\begin{frame}{\\Высказывание*}
	\topline
	\justifying

	\begin{SCn}
		\scnheader{следует отличать*}
		\begin{scnhaselementset}
			\scnitem{высказывание*}
			\begin{scnindent}
				\scniselement{бинарное ориентированное отношение}
				\scnidtf{быть высказыванием, описывающим заданную предметную область*}
			\end{scnindent}
			\scnitem{высказывание}
			\begin{scnindent}
				\scnsubset{логическая формула}
				\scnidtf{Второй домен отношения “быть высказыванием”}
			\end{scnindent}
		\end{scnhaselementset}
	\end{SCn}
\end{frame}

\begin{frame}{\\Нефактографическое высказывание}
	\topline
	\justifying

	\begin{SCn}
		\scnheader{нефактографическое высказывание}
		\begin{scnrelfromset}{разбиение}
			\scnitem{атомарное нефактографическое высказывание}
			\begin{scnindent}
				\scnidtf{атомарное высказывание, в состав которого входит хотя бы одна переменная}
			\end{scnindent}
			\scnitem{неатомарное нефактографическое высказывание}
			\begin{scnindent}
				\scnidtf{неатомарное высказывание, хотя бы один элемент которого является нефактографическим высказыванием}
			\end{scnindent}
		\end{scnrelfromset}
	\end{SCn}
\end{frame}

\begin{frame}{\\Формальная теория}
	\topline
	\justifying

	\begin{SCn}
		\scnheader{формальная теория}
		\scnidtf{множество высказываний, которые считаются истинными в рамках данной формальной теории}
	\end{SCn}
\end{frame}