\title{Лекция 10\\Задачи в интеллектуальных системах \vspace{-2em}}   
\author[]{Шункевич Д.В.}
\institute[]{Белорусский государственный университет информатики и радиоэлектроники}

\begin{frame}
	\titlepage
\end{frame}

\begin{frame}{\\Содержание лекции}
	\topline
	\justifying
	Вопросы и информационные задачи. Поведенческие цели и поведенческие задачи. Информационно-поисковые задачи. Интеллектуализация информационного поиска. Трудноформализуемые задачи. Задачи, для которых отсутствует четкая постановка. Классы задач и хранимые (интерпретируемые) способы их решения – программы. Типология хранимых программ и их интерпретаторов.
\end{frame}

\begin{frame}{\\Содержание лекции}
	\topline
	\justifying
    \begin{scnrelfromset}{Разбиение}
        \scnitem{Вопросы и информационные задачи}
        \scnitem{Поведенческие цели и поведенческие задачи}
        \scnitem{Информационно-поисковые задачи}
        \scnitem{Интеллектуализация информационного поиска}
        \scnitem{Трудноформализуемые задачи}
        \scnitem{Задачи, для которых отсутствует четкая постановка}
        \scnitem{Классы задач и хранимые (интерпретируемые) способы их решения – программы}
        \scnitem{Типология хранимых программ и их интерпретаторов}
    \end{scnrelfromset}
\end{frame}

\begin{frame}{\\Вопросы и информационные задачи}
	\topline
 	\begin{SCn}
    \scnheader{Информационная задача}
    \begin{scnrelfromset}{Определение}
	   \scnitem{это конкретная проблема, цель или задание, которые система должна решить или выполнить. Задачи в интеллектуальных системах могут быть разнообразными и зависят от области применения системы.}
    \end{scnrelfromset}
    \end{SCn}
\end{frame}

\begin{frame}{\\Вопросы и информационные задачи}
	\topline
 	\begin{SCn} 
    \scnheader{Вопросы и информационные задачи}
    \begin{scnrelfromset}{Виды задач}
            \scnitem{хранение и передача информации}
            \scnitem{анализ больших объемов данных}
            \scnitem{классификация и распознавание образов}
            \scnitem{построение рекомендательных систем}
    \end{scnrelfromset}
    \end{SCn}
\end{frame}

\begin{frame}{\\Поведенческие цели и поведенческие задачи}
	\topline
 	\begin{SCn} 
    \scnheader{Поведенческие цели и поведенческие задачи}
    \begin{scnrelfromset}{Поведенческие цели}
            \scnitem{Поведенческие цели определяют желаемое поведение или реакцию интеллектуальной системы}
            \scnitem{Цели могут быть разнообразными: от выполнения конкретной задачи до взаимодействия с пользователем}
    \end{scnrelfromset}
    \end{SCn}
\end{frame}

\begin{frame}{\\Поведенческие цели и поведенческие задачи}
	\topline
 	\begin{SCn} 
    \scnheader{Поведенческие цели и поведенческие задачи}
    \begin{scnrelfromset}{Примеры поведенческих целей}
            \scnitem{Распознавание и классификация объектов на изображениях}
            \scnitem{Автоматизированное принятие решений на основе анализа данных}
            \scnitem{Предоставление рекомендаций пользователю}
            \scnitem{Взаимодействие с пользователем через голосового помощника}
    \end{scnrelfromset}
    \end{SCn}
\end{frame}

\begin{frame}{\\Поведенческие цели и поведенческие задачи}
	\topline
 	\begin{SCn} 
    \scnheader{Поведенческие цели и поведенческие задачи}
    \begin{scnrelfromset}{Поведенческие задачи}
            \scnitem{Поведенческие задачи представляют собой конкретные задания или подзадачи, которые необходимо решить для достижения поведенческой цели}
            \scnitem{Решение задач требует использования различных методов и техник в интеллектуальных системах}
    \end{scnrelfromset}
    \end{SCn}
\end{frame}

\begin{frame}{\\Поведенческие цели и поведенческие задачи}
	\topline
 	\begin{SCn} 
    \scnheader{Поведенческие цели и поведенческие задачи}
    \begin{scnrelfromset}{Примеры поведенческих задач}
            \scnitem{Извлечение и анализ информации из больших объемов данных}
            \scnitem{Разработка и обучение моделей машинного обучения для классификации или прогнозирования}
            \scnitem{Обработка и анализ текстовых данных для понимания пользовательских запросов}
    \end{scnrelfromset}
    \end{SCn}
\end{frame}

\begin{frame}{\\Поведенческие цели и поведенческие задачи}
	\topline
 	\begin{SCn} 
    \scnheader{Поведенческие цели и поведенческие задачи}
    \begin{scnrelfromset}{Применение поведенческих целей и задач}
            \scnitem{Автоматизированные системы принятия решений в медицине, финансах, транспорте и других областях}
            \scnitem{Персонализированные рекомендации в интернет-магазинах и стриминговых сервисах}
            \scnitem{Голосовые помощники и чат-боты для улучшения взаимодействия с пользователями}
    \end{scnrelfromset}
    \end{SCn}
\end{frame}

\begin{frame}{\\Информационно-поисковые задачи}
	\topline
 	\begin{SCn} 
    \scnheader{Информационно-поисковые задачи}
    \begin{scnrelfromset}{Определение информационно-поисковых задач}
            \scnitem{Информационно-поисковые задачи заключаются в поиске, извлечении и анализе информации для ответа на вопросы пользователей или решения конкретных задач.}
            \scnitem{Цель заключается в предоставлении точных и релевантных результатов на основе пользовательских запросов.}
    \end{scnrelfromset}
    \end{SCn}
\end{frame}

\begin{frame}{\\Информационно-поисковые задачи}
	\topline
 	\begin{SCn} 
    \scnheader{Информационно-поисковые задачи}
    \begin{scnrelfromset}{Примеры информационно-поисковых задач}
            \scnitem{Поиск информации: поиск ответов на конкретные вопросы или запросы в базе данных или в сети Интернет.}
            \scnitem{Извлечение информации: извлечение ключевых фактов или информации из текстового или мультимедийного контента.}
            \scnitem{Классификация и фильтрация: организация и классификация информации для лучшего доступа и понимания.}
    \end{scnrelfromset}
    \end{SCn}
\end{frame}

\begin{frame}{\\Информационно-поисковые задачи}
	\topline
 	\begin{SCn} 
    \scnheader{Информационно-поисковые задачи}
    \begin{scnrelfromset}{Применение информационно-поисковых задач}
            \scnitem{Интернет-поиск: поиск и предоставление релевантных результатов для пользователей.}
            \scnitem{Автоматизированное анализ и обработка данных в различных областях, таких как медицина, финансы, наука и т.д.}
            \scnitem{Рекомендательные системы: предоставление персонализированных рекомендаций на основе предыдущих действий и предпочтений пользователя.}
    \end{scnrelfromset}
    \end{SCn}
\end{frame}

\begin{frame}{\\Информационно-поисковые задачи}
	\topline
 	\begin{SCn} 
    \scnheader{Информационно-поисковые задачи}
    \begin{scnrelfromset}{Преимущества информационно-поисковых задач в интеллектуальных системах}
            \scnitem{Быстрый доступ к информации: системы могут быстро найти и предоставить необходимую информацию.}
            \scnitem{Улучшение качества решений: системы помогают в принятии информированных решений на основе обработки больших объемов информации.}
            \scnitem{Автоматизация задач: системы могут автоматически анализировать информацию и предоставлять релевантные результаты.}
    \end{scnrelfromset}
    \end{SCn}
\end{frame}

\begin{frame}{\\Интеллектуализация информационного поиска}
	\topline
 	\begin{SCn}
    \scnheader{Интеллектуализация информационного поиска}
    \begin{scnrelfromset}{Определение}
	   \scnitem{это процесс применения различных технологий и методов искусственного интеллекта для улучшения качества и эффективности информационного поиска.}
    \end{scnrelfromset}
    \begin{scnrelfromset}{Цель}
	   \scnitem{Цель состоит в том, чтобы понять запрос пользователя, интерпретировать его и предоставить наиболее релевантные результаты.}
    \end{scnrelfromset}
    \end{SCn}
\end{frame}

\begin{frame}{\\Интеллектуализация информационного поиска}
	\topline
 	\begin{SCn} 
    \scnheader{Интеллектуализация информационного поиска}
    \begin{scnrelfromset}{Преимущества интеллектуализации информационного поиска}
            \scnitem{Улучшенная релевантность результатов: системы могут использовать алгоритмы машинного обучения и анализ данных для предоставления наиболее релевантных результатов пользователю.}
            \scnitem{Расширенные возможности поиска: интеллектуализация позволяет расширить функциональность поиска, включая фильтрацию, классификацию, анализ и другие функции.}
            \scnitem{Персонализированный подход: системы могут учитывать предыдущие действия и предпочтения пользователя для предоставления персонализированных результатов.}
    \end{scnrelfromset}
    \end{SCn}
\end{frame}

\begin{frame}{\\Интеллектуализация информационного поиска}
	\topline
 	\begin{SCn} 
    \scnheader{Интеллектуализация информационного поиска}
    \begin{scnrelfromset}{Применение}
            \scnitem{Интернет-поиск: поиск информации в Интернете с использованием интеллектуальных алгоритмов для улучшения результатов.}
            \scnitem{Корпоративные системы: применение интеллектуализации информационного поиска в организациях для более эффективного поиска и доступа к корпоративным данным.}
            \scnitem{Аналитика данных: использование интеллектуализации для анализа и обработки больших объемов данных для выявления тенденций, паттернов и важных информационных фактов.}
    \end{scnrelfromset}
    \end{SCn}
\end{frame}

\begin{frame}{\\Трудноформализуемые задачи}
	\topline
 	\begin{SCn}
    \scnheader{Трудноформализуемые задачи}
    \begin{scnrelfromset}{Определение}
	   \scnitem{это задачи, для которых сложно разработать точный и эффективный алгоритм решения, или для которых нет известного полиномиального алгоритма решения.}
    \end{scnrelfromset}
    \end{SCn}
\end{frame}

\begin{frame}{\\Трудноформализуемые задачи}
	\topline
 	\begin{SCn} 
    \scnheader{Трудноформализуемые задачи}
    \begin{scnrelfromset}{Примеры}
            \scnitem{Задача коммивояжера (Traveling Salesman Problem): поиск самого короткого пути, проходящего через все города.}
            \scnitem{Рюкзаковая задача (Knapsack Problem): выбор наиболее ценных предметов, чтобы заполнить рюкзак с ограниченной вместимостью.}
            \scnitem{Рюкзаковая задача (Knapsack Problem): выбор наиболее ценных предметов, чтобы заполнить рюкзак с ограниченной вместимостью.}
    \end{scnrelfromset}
    \end{SCn}
\end{frame}

\begin{frame}{\\Трудноформализуемые задачи}
	\topline
 	\begin{SCn} 
    \scnheader{Трудноформализуемые задачи}
    \begin{scnrelfromset}{Сложность}
            \scnitem{Трудноформализуемые задачи имеют высокую вычислительную сложность, что означает, что время и ресурсы, необходимые для их решения, растут экспоненциально с увеличением размера задачи.}
            \scnitem{Некоторые из этих задач являются NP-полными, что означает, что нет известного полиномиального алгоритма решения для них.}
    \end{scnrelfromset}
    \end{SCn}
\end{frame}

\begin{frame}{\\Трудноформализуемые задачи}
	\topline
 	\begin{SCn} 
    \scnheader{Трудноформализуемые задачи}
    \begin{scnrelfromset}{Подходы к решению}
            \scnitem{При решении трудноформализуемых задач часто используются эвристические и приближенные методы.}
            \scnitem{Методы оптимизации, метаэвристики и алгоритмы машинного обучения могут помочь в приближенном решении этих задач.}
    \end{scnrelfromset}
    \end{SCn}
\end{frame}

\begin{frame}{\\Трудноформализуемые задачи}
	\topline
 	\begin{SCn} 
    \scnheader{Трудноформализуемые задачи}
    \begin{scnrelfromset}{Применение}
            \scnitem{Трудноформализуемые задачи имеют широкое применение в различных областях, таких как логистика, планирование, проектирование сетей, криптография и многое другое.}
            \scnitem{Решение этих задач может привести к оптимизации и повышению эффективности в различных бизнес-и технических задачах.}
    \end{scnrelfromset}
    \end{SCn}
\end{frame}

\begin{frame}{\\Задачи, для которых отсутствует четкая постановка}
	\topline
 	\begin{SCn}
    \scnheader{Задачи, для которых отсутствует четкая постановка}
    \begin{scnrelfromset}{Определение}
	   \scnitem{это задачи, для которых отсутствует точное определение цели, ограничений и решения.}
    \end{scnrelfromset}
    \end{SCn}
\end{frame}

\begin{frame}{\\Задачи, для которых отсутствует четкая постановка}
	\topline
 	\begin{SCn} 
    \scnheader{Задачи, для которых отсутствует четкая постановка}
    \begin{scnrelfromset}{Примеры}
            \scnitem{Принятие решений в неопределенных ситуациях: например, решение, какой продукт разработать на основе нечетких требований рынка.}
            \scnitem{Моделирование сложных систем: задачи, связанные с моделированием и анализом сложных систем, таких как экономические системы, социальные сети или климатические изменения.}
    \end{scnrelfromset}
    \end{SCn}
\end{frame}

\begin{frame}{\\Задачи, для которых отсутствует четкая постановка}
	\topline
 	\begin{SCn} 
    \scnheader{Задачи, для которых отсутствует четкая постановка}
    \begin{scnrelfromset}{Особенности}
            \scnitem{Множественные цели и критерии: в нечетких задачах часто существует множество возможных целей и критериев, которые могут противоречить друг другу.}
            \scnitem{Неопределенность и нечеткость: такие задачи часто содержат неопределенность и нечеткость в данных, ограничениях или целях.}
    \end{scnrelfromset}
    \end{SCn}
\end{frame}

\begin{frame}{\\Задачи, для которых отсутствует четкая постановка}
	\topline
 	\begin{SCn} 
    \scnheader{Задачи, для которых отсутствует четкая постановка}
    \begin{scnrelfromset}{Подходы к решению}
            \scnitem{Интуитивный подход: понимание проблемы и использование интуиции для принятия решений.}
            \scnitem{Использование нечеткой логики: применение математического формализма нечеткой логики для моделирования и анализа нечетких задач.}
    \end{scnrelfromset}
    \end{SCn}
\end{frame}

\begin{frame}{\\Задачи, для которых отсутствует четкая постановка}
	\topline
 	\begin{SCn} 
    \scnheader{Задачи, для которых отсутствует четкая постановка}
    \begin{scnrelfromset}{Применение}
            \scnitem{Финансовое планирование: принятие решений в инвестициях и портфельном управлении, учитывая неопределенность рынка.}
            \scnitem{Управление проектами: управление проектами, в которых существует неопределенность в ресурсах, сроках и требованиях.}
    \end{scnrelfromset}
    \end{SCn}
\end{frame}

\begin{frame}{\\Классы задач и хранимые (интерпретируемые) способы их решения – программы}
	\topline
 	\begin{SCn} 
    \scnheader{Классы задач и хранимые (интерпретируемые) способы их решения – программы}
    \begin{scnrelfromset}{Классификация}
            \scnitem{Задачи могут быть классифицированы по различным критериям, таким как вычислительная сложность, тип задачи (оптимизация, поиск, классификация и др.) и многое другое.}
            \scnitem{Классификация задач помогает определить подходящий метод и программу для их решения.}
    \end{scnrelfromset}
    \end{SCn}
\end{frame}

\begin{frame}{\\Классы задач и хранимые (интерпретируемые) способы их решения – программы}
	\topline
 	\begin{SCn} 
    \scnheader{Классы задач и хранимые (интерпретируемые) способы их решения – программы}
    \begin{scnrelfromset}{Примеры}
            \scnitem{Задачи оптимизации: поиск оптимального решения с учетом заданных ограничений и целевых функций.}
            \scnitem{Задачи машинного обучения: создание моделей и алгоритмов, которые способны изучать и адаптироваться на основе данных.}
            \scnitem{Задачи графического моделирования: моделирование и анализ систем с использованием графических моделей и алгоритмов.}
    \end{scnrelfromset}
    \end{SCn}
\end{frame}

\begin{frame}{\\Хранимые (интерпретируемые) программы}
	\topline
 	\begin{SCn} 
    \scnheader{Хранимые (интерпретируемые) программы}
    \begin{scnrelfromset}{определение}
            \scnitem{это программы, которые представляют собой данные, которые можно сохранить и повторно использовать для решения задач.}
    \end{scnrelfromset}
    \end{SCn}
\end{frame}

\begin{frame}{\\Хранимые (интерпретируемые) программы}
	\topline
 	\begin{SCn} 
    \scnheader{Хранимые (интерпретируемые) программы}
    \begin{scnrelfromset}{применение}
            \scnitem{Хранимые программы широко применяются в различных областях, включая науку, инженерию, финансы и многое другое.}
            \scnitem{Они позволяют автоматизировать процессы и повысить эффективность решения задач.}
    \end{scnrelfromset}
    \end{SCn}
\end{frame}

\begin{frame}{\\Типология хранимых программ и их интерпретаторов}
	\topline
 	\begin{SCn} 
    \scnheader{Типология хранимых программ и их интерпретаторов}
    \begin{scnrelfromset}{определение}
            \scnitem{это программы, которые сохраняются и могут быть выполнены или интерпретированы позднее.}
    \end{scnrelfromset}
    \begin{scnrelfromset}{возможности}
            \scnitem{Типология хранимых программ и их интерпретаторов помогает классифицировать и организовывать различные типы программ.}
    \end{scnrelfromset}
    \end{SCn}
\end{frame}

\begin{frame}{\\Типология хранимых программ и их интерпретаторов}
	\topline
 	\begin{SCn} 
    \scnheader{Типология хранимых программ и их интерпретаторов}
    \begin{scnrelfromset}{классификация}
            \scnitem{Программы на языках общего назначения: программы, написанные на языках программирования, таких как Python, Java, C++ и других.}
            \scnitem{Скрипты и команды оболочки: наборы инструкций, которые выполняются в интерпретаторе командной строки.}
            \scnitem{SQL-запросы: язык структурированных запросов для работы с базами данных.}
    \end{scnrelfromset}
    \end{SCn}
\end{frame}

\begin{frame}{\\Типология хранимых программ и их интерпретаторов}
	\topline
 	\begin{SCn} 
    \scnheader{Типология хранимых программ и их интерпретаторов}
    \begin{scnrelfromset}{типология}
            \scnitem{Исполняемые программы: программы, которые могут быть непосредственно выполнены без дополнительной обработки или интерпретации.}
            \scnitem{Интерпретируемые программы: программы, которые требуют интерпретатора для выполнения их кода.}
    \end{scnrelfromset}
    \end{SCn}
\end{frame}

\begin{frame}{\\Типология хранимых программ и их интерпретаторов}
	\topline
 	\begin{SCn} 
    \scnheader{Интерпретаторы}
    \begin{scnrelfromset}{виды}
            \scnitem{Общие интерпретаторы: интерпретаторы, способные выполнять программы на различных языках программирования.}
            \scnitem{Специализированные интерпретаторы: интерпретаторы, разработанные для конкретного языка программирования или окружения.}
    \end{scnrelfromset}
    \end{SCn}
\end{frame}

\begin{frame}{\\Типология хранимых программ и их интерпретаторов}
	\topline
 	\begin{SCn} 
    \scnheader{Интерпретаторы}
    \begin{scnrelfromset}{примеры}
            \scnitem{Python: интерпретатор для выполнения программ, написанных на языке Python.}
            \scnitem{R: интерпретатор для выполнения программ, написанных на языке R, который широко используется в области статистики и анализа данных.}
            \scnitem{MATLAB: интерпретатор для выполнения программ, написанных на языке MATLAB, часто используется в научных и инженерных расчетах.}
    \end{scnrelfromset}
    \end{SCn}
\end{frame}

\begin{frame}{\\Типология хранимых программ и их интерпретаторов}
	\topline
 	\begin{SCn} 
    \scnheader{Интерпретаторы}
    \begin{scnrelfromset}{преимущества}
            \scnitem{Гибкость: интерпретаторы позволяют изменять и дорабатывать программы без необходимости компиляции.}
            \scnitem{Универсальность: общие интерпретаторы поддерживают выполнение программ на различных языках программирования.}
    \end{scnrelfromset}
    \end{SCn}
\end{frame}