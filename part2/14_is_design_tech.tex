\title{Лекция 14\\Технологии проектирования интеллектуальных систем \vspace{-2em}}   
\author[]{Шункевич Д.В.}
\institute[]{Белорусский государственный университет информатики и радиоэлектроники}

\begin{frame}
	\titlepage
\end{frame}

\begin{frame}{\\Содержание лекции}
	\topline
	\justifying
\begin{SCn}
	\scnheader{Структура лекции}

	\begin{scnrelfromset}{разбиение}
				\scnitem{Многократно используемые компоненты интеллектуальной системы}
				\scnitem{Технология проектирования интеллектуальных компьютерных систем}
				\scnitem{Требования, предъявляемые к технологии проектирования интеллектуальных систем}
	\end{scnrelfromset}
				
\end{SCn}
\end{frame}

\begin{frame}{Многократно используемые компоненты интеллектуальной системы}
\topline
\justifying
\begin{SCn}
	\scnheader{Многократно используемые компоненты в интеллектуальных системах }

	\scnidtf{компоненты, которые могут использоваться повторно в различных системах или в разных проектах той же системы.}

\end{SCn}
\end{frame}

\begin{frame}{Многократно используемые компоненты интеллектуальной системы}
\topline
\justifying
\begin{SCn}
	\scnheader{Многократно используемые компоненты в интеллектуальных системах }

	\begin{scnrelfromset}{особенности применения}
				\scnitem{Использование значительно сокращает время и затраты на разработку новых систем или компонентов}
				\scnitem{Многократно используемые компоненты используются для обработки данных, настройки моделей, извлечения признаков и других задач. Компоненты разрабатываются и тестируются в отдельности, а затем могут быть включены в различные системы машинного обучения без дополнительной работы. Это помогает сократить время разработки и повторного использования кода.}
			\end{scnrelfromset}

\end{SCn}
\end{frame}

\begin{frame}{Многократно используемые компоненты интеллектуальной системы}
\topline
\justifying
\begin{SCn}
	\scnheader{Многократно используемые компоненты в интеллектуальных системах }

	\begin{scnrelfromset}{особенности применения}
				\scnitem{Используются для создания общей базы знаний, которая может использоваться многими системами}
				\scnitem{Многократно используемые компоненты используются для унификации интерфейсов и стандартов между различными системами, что повышает совместимость и обмен данными. Например, в системах распознавания речи представлены многократно используемые компоненты для обработки звука и выделения речевых признаков. Эти компоненты могут использоваться многими системами, чтобы сделать обработку речи более унифицированной и удобной для пользователя.}
			\end{scnrelfromset}

\end{SCn}
\end{frame}

\begin{frame}{Многократно используемые компоненты интеллектуальной системы}
\topline
\justifying
\begin{SCn}
	\scnheader{Компонентное проектирование интеллектуальных систем}

	\begin{scnrelfromset}{предъявляемые требования}
				\scnitem{Использование универсального языка представления знаний, используемого в интеллектуальной системе.}
				\scnitem{Наличие универсальной процедуры интеграции знаний в рамках указанного языка.}
                \scnitem{Наличие стандарта, обеспечивающего семантическую совместимость интегрируемых знаний (таким стандартом является согласованная система используемых понятий и унифицированная спецификация компонентов).}
			\end{scnrelfromset}

\end{SCn}
\end{frame}

\begin{frame}{Многократно используемые компоненты интеллектуальной системы}
\topline
\justifying
\begin{SCn}
	\scnheader{Компонентное проектирование интеллектуальных систем}

	\begin{scnrelfromset}{предъявляемые требования}
                \scnitem{Многократно используемый компонент должен выполнять свои функции наиболее общим образом, чтобы круг возможных систем, в которые он может быть встроен, был наиболее широким.}
                \scnitem{Самодостаточность компонентов (или групп компонентов) технологии, т.е. способности их функционировать отдельно от других компонентов без утраты целесообразности их использования.}
			\end{scnrelfromset}

\end{SCn}
\end{frame}

\begin{frame}{Технология проектирования интеллектуальных компьютерных систем}
    \topline
    \justifying
    \begin{SCn}
    
    \scnheader{Технология проектирования интеллектуальных компьютерных систем}
    \scnidtf{набор методов и инструментов, используемых для создания систем, способных к самообучению, анализу данных и принятию решений на основе имеющейся информации.}
    
    \scntext{примеры}{системы автоматической обработки естественного языка, системы машинного обучения, системы компьютерного зрения, системы экспертных систем, системы робототехники.}
    
    \end{SCn}
\end{frame}

\begin{frame}{Технология проектирования интеллектуальных компьютерных систем}
    \topline
    \justifying
    \begin{SCn}
    
    \scnheader{Технология проектирования интеллектуальных компьютерных систем}
    
    \begin{scnrelfromset}{ключевые аспекты}
                \scnitem{способность анализировать большие объемы данных и использовать их для принятия решений. В этой технологии могут быть использованы различные методы машинного обучения, такие как классификация, кластеризация, регрессия и обработка естественного языка, которые облегчают обучение компьютерных систем на основе больших объемов данных}
			\end{scnrelfromset}
   
    \end{SCn}
\end{frame}

\begin{frame}{Технология проектирования интеллектуальных компьютерных систем}
    \topline
    \justifying
    \begin{SCn}
    
    \scnheader{Технология проектирования интеллектуальных компьютерных систем}
    
    \begin{scnrelfromset}{ключевые аспекты}
                \scnitem{возможность создания систем, которые могут обучаться на основе опыта в режиме реального времени. Такие системы могут быть настроены на определенное поведение и могут изменять свое поведение на основе нового опыта, полученного в процессе эксплуатации}
                \scnitem{включает в себя различные методы и техники по созданию интуитивных интерфейсов для взаимодействия с компьютерными системами. Эти интуитивные интерфейсы делают использование систем более удобным и понятным для людей}
			\end{scnrelfromset}
   
    \end{SCn}
\end{frame}

\begin{frame}{Требования, предъявляемые к технологии проектирования интеллектуальных систем}
    \topline
    \justifying
    \begin{SCn}
    
    \scnheader{Требования, предъявляемые к технологии проектирования интеллектуальных систем}
    
    \begin{scnrelfromset}{предъявляемые требования}
                \scnitem{реализация предлагаемой технологии разработки и сопровождения интеллектуальных компьютерных систем нового поколения в виде интеллектуальной компьютерной метасистемы, которая полностью соответствует стандартам предлагаемых интеллектуальных компьютерных систем нового поколения, разрабатываемым по предлагаемой технологии}
                \scnitem{унификация и стандартизация интеллектуальных компьютерных систем нового поколения, а также методов их проектирования, реализации, сопровождения, реинжиниринга и эксплуатации}
			\end{scnrelfromset}
   
    \end{SCn}
\end{frame}

\begin{frame}{Требования, предъявляемые к технологии проектирования интеллектуальных систем}
    \topline
    \justifying
    \begin{SCn}
    
    \scnheader{Требования, предъявляемые к технологии проектирования интеллектуальных систем}
    
    \begin{scnrelfromset}{предъявляемые требования}
                \scnitem{компонентное проектирование интеллектуальных компьютерных систем нового поколения, то есть проектирование, ориентированное на сборку интеллектуальных компьютерных систем из готовых компонентов на основе постоянно расширяемых библиотек многократно используемых компонентов}
                \scnitem{перманентная эволюция стандарта интеллектуальных компьютерных систем нового поколения, а также методов их проектирования, реализации, сопровождения, реинжиниринга и эксплуатации}
			\end{scnrelfromset}
   
    \end{SCn}
\end{frame}

\begin{frame}{Требования, предъявляемые к технологии проектирования интеллектуальных систем}
    \topline
    \justifying
    \begin{SCn}
    
    \scnheader{Требования, предъявляемые к технологии проектирования интеллектуальных систем}
    
    \begin{scnrelfromset}{предъявляемые требования}
                \scnitem{четкое согласование и оперативную формализованную фиксацию (в виде формальных онтологий) утвержденного текущего состояния иерархической системы всех понятий, лежащих в основе перманентно эволюционируемого стандарта интеллектуальных компьютерных систем нового поколения, а также в основе каждой разрабатываемой интеллектуальной компьютерной системы}
                \scnitem{достаточно полное и оперативное документирование текущего состояния каждого проекта}
                \scnitem{использование методики проектирования "сверху-вниз"}
			\end{scnrelfromset}
   
    \end{SCn}
\end{frame}

\begin{frame}{Требования, предъявляемые к технологии проектирования интеллектуальных систем}
    \topline
    \justifying
    \begin{SCn}
    
    \scnheader{Требования, предъявляемые к технологии проектирования интеллектуальных систем}
    
    \begin{scnrelfromset}{предъявляемые требования}
                \scnitem{комплексный характер предлагаемой технологии, осуществляющей поддержку проектирования не только компонентов интеллектуальных компьютерных систем нового поколения (различных фрагментов баз знаний, баз знаний в целом, различных методов решения задач, различных внутренних информационных агентов, решателей задач в целом, интерфейсов в целом), но также и интеллектуальных компьютерных систем в целом как самостоятельных объектов проектирования с учетом специфики тех классов, которым принадлежат проектируемые интеллектуальных компьютерных системы}
			\end{scnrelfromset}
   
    \end{SCn}
\end{frame}

\begin{frame}{Требования, предъявляемые к технологии проектирования интеллектуальных систем}
    \topline
    \justifying
    \begin{SCn}
    
    \scnheader{Требования, предъявляемые к технологии проектирования интеллектуальных систем}
    
    \begin{scnrelfromset}{предъявляемые требования}
                \scnitem{комплексный характер предлагаемой технологии, осуществляющей поддержку не только комплексного проектирования интеллектуальных компьютерных систем нового поколения, но также и поддержку их реализации (сборки, воспроизводства), сопровождения, реинжиниринга в ходе эксплуатации и непосредственно самой эксплуатации}
			\end{scnrelfromset}
   
    \end{SCn}
\end{frame}

\begin{frame}
	
	\begin{center}
		\begin{LARGE}
		\textbf{Спасибо за внимание!}
		\end{LARGE}

		Остались ли вопросы?
	\end{center}
\end{frame}