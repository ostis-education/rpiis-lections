\title{Лекция 9\\Базы знаний интеллектуальных систем}

\begin{frame}
	\titlepage
\end{frame}


\begin{frame}{\\Содержание лекции}
	\topline
	\justifying
	
	База знаний. Требования, предъявляемые к базам знаний. Критерии качества. Виды знаний и модели их представления. Четкие и нечеткие множества и знания. Смысл. Смысловое представление знаний. Предметная область. Онтология. Высказывания и формальные теории.
	
\end{frame}

\begin{frame}{Внутренняя информационная\\ модель окружающей среды}
	\topline
	\justifying
	
	\vspace{5mm}
	\small
	
	\begin{SCn}
		\scnheader{внутренняя информационная модель окружающей среды}
		\scnidtf{часть состояния \textit{памяти кибернетической системы}, которая используется \textit{процессором} и \textit{сенсорно-эффекторным комплексом} для организации \textit{деятельности} (поведения, функционирования) \textit{кибернетической системы} в процессе её взаимодействия со своей \textit{внешней средой}, со своей \textit{физической оболочкой} и со своей внутренней информационной средой (то есть \textit{внутренней информационной моделью окружающей среды})}
		\scnidtf{субъективная картина мира кибернетической системы}
		\scnsuperset{база знаний}
		\scntext{примечание}{Наличие у кибернетической системы внутренней информационной модели окружающей среды означает то, то кибернетическая система "живёт"{} одновременно в двух мирах --- во внешнем реальном мире и во внутреннем мире своей информационной модели (отражения) этого внешнего реального мира.}
	\end{SCn}
	
\end{frame}


\begin{frame}{\\База знаний}
	\topline
	\justifying
	
	\vspace{5mm}
	\small
	
	\begin{SCn}
		\scnheader{база знаний}
		\scnidtf{семантически структурированная внутренняя информационная модель окружающей среды интеллектуальной кибернетической системы}
		\scnidtf{совокупность знаний, хранимых в памяти интеллектуальной компьютерной системы и достаточных для того, чтобы указанная система удовлетворяла соответствующим предъявляемым к ней требованиям (в частности, чтобы она имела соответствующий уровень интеллекта)}
		\scnidtf{систематизированная совокупность знаний, хранимая в памяти		интеллектуальной компьютерной системы и достаточная для обеспечения		целенаправленного (целесообразного, адекватного) функционирования(поведения) этой системы как в своей внешней среде, так и в своей внутренней среде (в собственной базе знаний)}
		\scnidtf{совокупность знаний, которыми владеет кибернетическая система на данный момент}
	\end{SCn}
	
\end{frame}

\begin{frame}{\\Знание}
	\topline
	\justifying
	
	\begin{SCn}
		\scnheader{знание}
		\scnidtf{синтаксически корректная (для соответствующего языка) и семантически целостная информационная конструкция}
		\scnrelfrom{покрытие}{вид знаний}
		\begin{scnindent}
			\scnidtf{Множество всевозможных всевозможных видов знаний}
		\end{scnindent}
	\end{SCn}
	
\end{frame}

\begin{frame}{\\Виды знаний}
	\topline
	\justifying
	
	\begin{SCn}
		\scnheader{вид знаний}
		\scnhaselement{спецификация}
		\begin{scnindent}
			\scnidtf{описание заданной сущности}
		\end{scnindent}
		\scnhaselement{метазнание}
		\begin{scnindent}
			\scnidtf{спецификация самих знаний}
		\end{scnindent}
		\scnhaselement{задача}
		\begin{scnindent}
			\scnidtf{спецификация действия}
		\end{scnindent}
		\scnhaselement{ситуация}
		\scnhaselement{событие}
		\scnhaselement{процесс}
		\scnhaselement{план}
		\scnhaselement{протокол}
		\scnhaselement{метод}
		\begin{scnindent}
			\scnsuperset{алгоритм}
		\end{scnindent}
	\end{SCn}
\end{frame}

\begin{frame}{\\Виды знаний}
	\topline
	\justifying
	
	\begin{SCn}
		\scnheader{вид знаний}
		\scnhaselement{сравнение}
		\scnhaselement{высказывание}
		\scnhaselement{формальная теория}
		\scnhaselement{предметная область}
		\scnhaselement{предметная область и онтология}
		\scnhaselement{технология}
		\scnhaselement{база знаний}
	\end{SCn}
	
	Даже небольшой перечень видов знаний свидетельствует об огромном многообразии видов знаний.
\end{frame}

\begin{frame}{\\Знание}
	\topline
	\justifying
	
	\begin{SCn}
		\scnheader{знание}
		\begin{scnrelfromset}{разбиение}
			\scnitem{декларативное знание}
			\begin{scnindent}
				\scnidtf{знание, имеющее только денотационную семантику в виде	семантической спецификации системы используемых понятий}
			\end{scnindent}
			\scnitem{процедурное знание}
			\begin{scnindent}
				\scnidtf{знание, имеющее не только денотационную семантику, но
					и операционную семантику в виде семейства спецификаций программных компонентов (агентов), интерпретирующих указанное знание с целью решения некоторой задачи}
			\end{scnindent}
		\end{scnrelfromset}
	\end{SCn}
\end{frame}

\begin{frame}{\\Информационная конструкция}
	\topline
	\justifying
	
	\begin{SCn}
		\scnheader{информационная конструкция}
		\begin{scnrelfromset}{разбиение}
			\scnitem{дискретная информационная конструкция}
			\begin{scnindent}
				\scnsuperset{знаковая информационная конструкция}
				\scntext{примечание}{В рамках современных компьютерных систем мы всегда имеем дело с \textit{дискретными информационными конструкциями}.}
			\end{scnindent}
			\scnitem{непрерывная информационная конструкция}
			\begin{scnindent}
				\scnidtf{сигнал}
			\end{scnindent}
		\end{scnrelfromset}
	\end{SCn}
\end{frame}

\begin{frame}{Атомарный фрагмент\\ информационной конструкции}
	\topline
	\justifying
	\small
	
	\begin{SCn}
		\scnheader{информационная конструкция}
		\scnidtf{информация}
		\scntext{примечание}{В общем случае информационная конструкция представляет
			собой сложную иерархическую структуру, каждому уровню иерархии которой
			соответствует определенный класс информационных конструкций.}
		\scnsuperset{синтаксически элементарный фрагмент информационной конструкции}
		\begin{scnindent}
			\scnidtf{атомарный фрагмент информационной конструкции}
			\scnidtf{элемент информационной конструкции}
			\scntext{примечание}{Примерами таких элементарных фрагментов информационных
				конструкций являются буквы}
			\scnsuperset{буква}
		\end{scnindent}
	\end{SCn}
	
\end{frame}

\begin{frame}{Атомарный фрагмент\\ информационной конструкции}
	\topline
	\justifying
	\small
	
	\begin{SCn}
		\scnheader{информационная конструкция}
		\scnidtf{информация}
		\scntext{примечание}{В общем случае информационная конструкция представляет
			собой сложную иерархическую структуру, каждому уровню иерархии которой
			соответствует определенный класс информационных конструкций.}
		\scnsuperset{синтаксически элементарный фрагмент информационной конструкции}
		\begin{scnindent}
			\scnidtf{атомарный фрагмент информационной конструкции}
			\scnidtf{элемент информационной конструкции}
			\scntext{примечание}{Примерами таких элементарных фрагментов информационных
				конструкций являются буквы}
			\scnsuperset{буква}
		\end{scnindent}
	\end{SCn}
	
\end{frame}

\begin{frame}{Простой знак\\Выражение}
	\topline
	\justifying
	\small
	
	\begin{SCn}
		\scnheader{простой знак}
		\scnidtf{семантически элементарный фрагмент информационной конструкции}
		\scnsubset{знак}
		
		\scnheader{выражение}
		\scnidtf{сложный (непростой) знак}
		\scnidtf{знак, являющийся одновременно некоторым знанием обозначаемой
			сущности (спецификацией этой сущности)}
		\scnidtf{знак, построенный как выражение вида тот, который... }
		\scnidtf{знак, в состав которого входят другие знаки}
		\scnsubset{знак}
	\end{SCn}
	
\end{frame}

\begin{frame}{\\Тексты}
	\topline
	\justifying
	\scriptsize
	
	\vspace{10mm}
	
	\begin{SCn}
		\scnheader{простой текст}
		\scnidtf{минимальная синтаксически целостная и корректная (правильная)
			информационная конструкция, включающая в себя:
			\begin{textitemize}
				\item знак некоторой описываемой связи;
				\item минимальную спецификацию указанного знака связи (указание
				отношения, которому это связь принадлежит);
				\item указание \uline{всех} компонентов описываемой связи (знаков всех сущностей, связываемых этой связью, и/или всех знаков, связываемых этой связью);
				\item если описываемая связь не является бинарной, то связи с её
				компонентами могут потребовать явного представления знаков этих связей с дополнительным указанием роли этих компонентов.
		\end{textitemize}}
		\scnsubset{текст}
	
		\scnheader{сложный текст}
		\scnidtf{информационная конструкция, являющаяся результатом соединения
			нескольких простых текстов}
		\scnsubset{текст}
	\end{SCn}
	
\end{frame}

\begin{frame}{Простое знание\\Сложное знание}
	\topline
	\justifying
	\small
	
	\begin{SCn}
	\scnheader{простое знание}
	\scnidtf{минимальная семантические целостная информационная конструкция}
	\scnidtf{знание, в состав которого не входят другие знания}
	\scnsubset{знание}

	\scnheader{сложное знание}
	\scnidtf{информационная конструкция, являющаяся результатом соединения
		нескольких простых знаний}
	\scnidtf{знание, в состав которого не входят другие знания}
	\scnsubset{знание}
	\end{SCn}
	
\end{frame}

\begin{frame}{Внутреннее и внешнее\\ представление информации}
	\topline
	\justifying
	\small
	
	\begin{SCn}
		\scnheader{следует отличать*}
		\begin{scnhaselementset}
			\scnitem{внутреннее представление информации}
			\begin{scnindent}
				\scnidtf{кодирование информации в памяти компьютерной системы}
			\end{scnindent}
			\scnitem{внешнее представление информации}
			\begin{scnindent}
				\scnidtf{обеспечение однозначности интерпретации (понимания, трактовки) этой информации
					разными пользователями и разными компьютерными системами}
			\end{scnindent}
		\end{scnhaselementset}
	\end{SCn}
	
\end{frame}

\begin{frame}{\\Знак}
	\topline
	\justifying
	\small
	
	\begin{SCn}
		\scnheader{знак}
		\scnidtf{фрагмент информационной конструкции, обладающий свойством, \uline{обозначать} некоторую сущность (объект), которая наряду с другими сущностями описывается указанной информационной конструкцией}
		\scntext{примечание}{\uline{Форма} представления знаков в известной степени
			условна и является результатом соглашения между носителями соответствующего языка. Знак может быть, например, представлен:
			\begin{textitemize}
				\item  в виде фрагмента речевого сообщения (последовательностью
				фонем);
				\item в виде строки символов (последовательности букв) в
				заданном алфавите;
				\item в виде иероглифа, пиктограммы;
				\item в виде жеста.
		\end{textitemize}}
	\end{SCn}
\end{frame}

\begin{frame}{\\Типология знаков}
	\topline
	\justifying
	\small
	
	\vspace{10mm}
	Знаки, используемые в различных языках, характеризуются:
	\begin{textitemize}
		\item синтаксической структурой, по совпадению (изоморфизму)
		которых для разных знаков предполагается их синонимия;
		\item денотационной семантикой, т.е. той сущностью, которая
		обозначается соответствующим знаком;
		\item типом (классом) обозначаемой сущности, которая может
		быть:
		\begin{textitemize}
			\item материальным (физическим) элементом (точкой)
			абстрактного пространства, множеством, которое может быть связью, классом, или структурой (информационной конструкцией)
			\item реальной и вымышленной сущностью;
			\item константной (конкретной) и переменной
			(произвольной) сущностью;
			\item постоянно существующей и временно существующей
			сущностью (прошлой, настоящей, будущей);
		\end{textitemize}
		\item ...
	\end{textitemize}
	
\end{frame}

\begin{frame}{\\Типология знаков}
	\topline
	\justifying
	\small
	
	\vspace{10mm}
	Знаки, используемые в различных языках, характеризуются:
	\begin{textitemize}
		\item ...
		\item множеством тех связей, которые связывают сущность,
		обозначаемую данным знаком с другими сущностями, а также, если данный знак
		обозначает некоторую связь, множеством сущностей, которые связаны этой связью,
		т.е. сущностей, являющихся компонентом этой связи;
		\item текущим статусом самого знака в памяти кибернетической
		системы, который может быть:
		\begin{textitemize}
			\item логически удаленным знаком;
			\item настоящим знаком;
			\item предлагаемым (возможно, будущим) знаком.
		\end{textitemize}
	\end{textitemize}
	
\end{frame}

\begin{frame}{\\Денотат}
	\topline
	\justifying
	\small
	
	\begin{SCn}
		\scnheader{денотат*}
		\scnidtf{денотат заданного знака*}
		\scnidtf{объект, обозначаемый заданным знаком*}
		\scnidtf{денотационная семантика заданного знака*}
		\scnidtf{смысл заданного знака*}
		\scnidtf{Бинарное ориентированное отношение, каждая пара которого
			связывает:
			\begin{textitemize}
				\item некоторый знак, представленный в той или иной форме в тексте
				исследуемого языка;
				\item \uline{со знаком} той сущности, которая обозначается указанным
				выше знаком в рамках используемого метаязыка.
		\end{textitemize}}
	\end{SCn}
	
\end{frame}

\begin{frame}{\\Денотат}
	\topline
	\justifying
	\small
	
	Данное отношение используется, когда с помощью одного языка необходимо описать денотационную семантику другого языка. 
	
	\bigskip
	
	Фактически речь идет о переводе заданного знака, входящего в состав некоторого рассматриваемого текста, принадлежащего некоторому исследуемому языку (языку-объекту), на некоторый метаязык, денотационная семантика которого нам считается априори известной. 
	
	\bigskip
	
	Указанный перевод есть связь заданного знака с синонимичным ему знаком, входящим в состав текста, принадлежащего указанному метаязыку.
	
\end{frame}

\begin{frame}{\\Понятие языка}
	\topline
	\justifying
	
	Язык представляет собой множество текстов и задаётся четвёркой
	\[
	L = \{ A, S_n, S_m, P \},
	\]
	где
	
	A --- алфавит (множество символов),
	
	$S_n$ --- синтаксис языка,
	
	$S_m$ --- семантика языка,
	
	P --- прагматика языка.
	
\end{frame}

\begin{frame}{\\Языки}
	\topline
	\justifying
	
	\begin{SCn}
	\scnheader{язык}
	\scnsuperset{универсальный язык}
	\scnsuperset{естественный язык}
	\scnsuperset{искусственный язык}
	\begin{scnindent}
		\scnsuperset{язык смыслового представления}
	\end{scnindent}
	\scnsuperset{специализированный язык}
	\scnsuperset{формальный язык}
	\end{SCn}
	
\end{frame}

\begin{frame}{\\Инженерия знаний}
	\topline
	\justifying
	
	\begin{textitemize}
		\item \textbf{Извлечение знаний}
		\begin{textitemize}
			\item Коммуникативные методы
			\item Текстологические методы
		\end{textitemize}
		\item \textbf{Представление знаний} (запись знаний на языке, понятном интеллектуальной системе, для последующего использования при решении задач)
		
		\item \textbf{Верификация знаний}
	\end{textitemize}
	
\end{frame}

\begin{frame}{\\Модели представления знаний}
	\topline
	\justifying
	
	Модели представления знаний:
	
	\begin{textitemize}
		\item Фреймовые модели
		\item Логические модели
		\item Продукционные модели
		\item Семантические сети (графы знаний)
	\end{textitemize}
	
\end{frame}

\iffalse
% Слайд 2: Фреймовые модели
\begin{frame}{Фреймовые модели}
	\topline
	\justifying
	
	\begin{center}
		\includegraphics[width=0.8\textwidth]{image1.jpg}
	\end{center}
	
\end{frame}

% Слайд 3: Логические модели
\begin{frame}{Логические модели}
	\topline
	\justifying
	
	\begin{center}
		\includegraphics[width=0.8\textwidth]{image2.jpg}
	\end{center}
	
\end{frame}

% Слайд 4: Продукционные модели
\begin{frame}{Продукционные модели}
	\topline
	\justifying
	
	\begin{center}
		\includegraphics[width=0.8\textwidth]{image3.jpg}
	\end{center}
	
\end{frame}

% Слайд 5: Граф знаний
\begin{frame}{Граф знаний}
	\topline
	\justifying
	
	\textbf{Граф знаний}
	
	\textbf{(Семантическая сеть, Knowledge Graph, KG)}
	
	\vspace{1em}
	
	\begin{center}
		\includegraphics[width=0.45\textwidth]{image4.jpg}
		\hspace{0.5cm}
		\includegraphics[width=0.45\textwidth]{image5.jpg}
	\end{center}
	
\end{frame}

% Слайд 6: Примеры графов знаний
\begin{frame}{Примеры графов знаний}
	\topline
	\justifying
	
	\begin{columns}
		\begin{column}{0.5\textwidth}
			\centering
			\includegraphics[width=0.9\textwidth]{image6.jpg}
			
			\textbf{Google Knowledge Graph}
		\end{column}
		\begin{column}{0.5\textwidth}
			\centering
			\includegraphics[width=0.9\textwidth]{image7.jpg}
			
			\textbf{WikiData}
		\end{column}
	\end{columns}
	
\end{frame}

% Слайд 7: Онтологии (заголовок)
\begin{frame}{Онтологии}
	\topline
	\justifying
	
	\begin{center}
		\includegraphics[width=0.8\textwidth]{image8.jpg}
	\end{center}
	
\end{frame}

% Слайд 8: Определение онтологии
\begin{frame}{Онтологии}
	\topline
	\justifying
	
	\begin{textitemize}
		\item \textbf{Онтология} -- это явная формальная спецификация концептуализации некоторой предметной области, разделяемая некоторым сообществом агентов
		\begin{textitemize}
			\item Другими словами -- формальное (строгое) описание системы понятий некоторой предметной области, согласованное в рамках некоторого сообщества
			\item Решение таких проблем как синонимия, омонимия, полисемия
		\end{textitemize}
	\end{textitemize}
	
\end{frame}

% Слайд 9: Интеграция источников
\begin{frame}{}
	\topline
	\justifying
	
	\begin{center}
		\includegraphics[width=0.6\textwidth]{image9.jpg}
	\end{center}
	
	\vspace{1em}
	
	Онтология позволяет интегрировать разнородные источники информации
	
\end{frame}

% Слайд 10: Цифровые двойники данных
\begin{frame}{Цифровые двойники данных}
	\topline
	\justifying
	
	\begin{center}
		\includegraphics[width=0.8\textwidth]{image10.jpg}
	\end{center}
	
\end{frame}

% Слайд 11: Semantic Web
\begin{frame}{Semantic Web}
	\topline
	\justifying
	
	\begin{textitemize}
		\item Март 1989 -- концепция Всемирной паутины (HTML, HTTP, URL, DNS, XML,\ldots)
		\item 1997-2003 -- развитие идеи Семантического Веба (RDF, OWL, OWL2,\ldots)
	\end{textitemize}
	
	\vspace{1em}
	
	\begin{textitemize}
		\item Цель Semantic Web - создать надстройку над существующей Всемирной паутиной, которая позволит сделать веб \textit{машинно-понимаемым}
		\begin{textitemize}
			\item описывать ресурсы формальными онтологиями и связями
			\item программы смогут автоматически находить, интерпретировать и комбинировать данные, а не просто показывать страницы людям
		\end{textitemize}
	\end{textitemize}
	
\end{frame}

\fi


