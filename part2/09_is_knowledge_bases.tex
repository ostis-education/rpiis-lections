\title{Лекция 9\\Базы знаний интеллектуальных систем}

\begin{frame}
	\titlepage
\end{frame}


\begin{frame}{\\Содержание лекции}
	\topline
	\justifying
	
	База знаний. Требования, предъявляемые к базам знаний. Критерии качества. Виды знаний и модели их представления. Четкие и нечеткие множества и знания. Смысл. Смысловое представление знаний. Предметная область. Онтология. Высказывания и формальные теории.
	
\end{frame}

\begin{frame}{Внутренняя информационная\\ модель окружающей среды}
	\topline
	\justifying
	
	\vspace{5mm}
	\small
	
	\begin{SCn}
		\scnheader{внутренняя информационная модель окружающей среды}
		\scnidtf{часть состояния \textit{памяти кибернетической системы}, которая используется \textit{процессором} и \textit{сенсорно-эффекторным комплексом} для организации \textit{деятельности} (поведения, функционирования) \textit{кибернетической системы} в процессе её взаимодействия со своей \textit{внешней средой}, со своей \textit{физической оболочкой} и со своей внутренней информационной средой (то есть \textit{внутренней информационной моделью окружающей среды})}
		\scnidtf{субъективная картина мира кибернетической системы}
		\scnsuperset{база знаний}
		\scntext{примечание}{Наличие у кибернетической системы внутренней информационной модели окружающей среды означает то, то кибернетическая система "живёт"{} одновременно в двух мирах --- во внешнем реальном мире и во внутреннем мире своей информационной модели (отражения) этого внешнего реального мира.}
	\end{SCn}
	
\end{frame}


\begin{frame}{\\База знаний}
	\topline
	\justifying
	
	\vspace{5mm}
	\small
	
	\begin{SCn}
		\scnheader{база знаний}
		\scnidtf{семантически структурированная внутренняя информационная модель окружающей среды интеллектуальной кибернетической системы}
		\scnidtf{совокупность знаний, хранимых в памяти интеллектуальной компьютерной системы и достаточных для того, чтобы указанная система удовлетворяла соответствующим предъявляемым к ней требованиям (в частности, чтобы она имела соответствующий уровень интеллекта)}
		\scnidtf{систематизированная совокупность знаний, хранимая в памяти		интеллектуальной компьютерной системы и достаточная для обеспечения		целенаправленного (целесообразного, адекватного) функционирования(поведения) этой системы как в своей внешней среде, так и в своей внутренней среде (в собственной базе знаний)}
		\scnidtf{совокупность знаний, которыми владеет кибернетическая система на данный момент}
	\end{SCn}
	
\end{frame}

\begin{frame}{\\Знание}
	\topline
	\justifying
	
	\begin{SCn}
		\scnheader{знание}
		\scnidtf{синтаксически корректная (для соответствующего языка) и семантически целостная информационная конструкция}
		\scnrelfrom{покрытие}{вид знаний}
		\begin{scnindent}
			\scnidtf{Множество всевозможных всевозможных видов знаний}
		\end{scnindent}
	\end{SCn}
	
\end{frame}

\begin{frame}{\\Виды знаний}
	\topline
	\justifying
	
	\begin{SCn}
		\scnheader{вид знаний}
		\scnhaselement{спецификация}
		\begin{scnindent}
			\scnidtf{описание заданной сущности}
		\end{scnindent}
		\scnhaselement{метазнание}
		\begin{scnindent}
			\scnidtf{спецификация самих знаний}
		\end{scnindent}
		\scnhaselement{задача}
		\begin{scnindent}
			\scnidtf{спецификация действия}
		\end{scnindent}
		\scnhaselement{ситуация}
		\scnhaselement{событие}
		\scnhaselement{процесс}
		\scnhaselement{план}
		\scnhaselement{протокол}
		\scnhaselement{метод}
		\begin{scnindent}
			\scnsuperset{алгоритм}
		\end{scnindent}
	\end{SCn}
\end{frame}

\begin{frame}{\\Виды знаний}
	\topline
	\justifying
	
	\begin{SCn}
		\scnheader{вид знаний}
		\scnhaselement{сравнение}
		\scnhaselement{высказывание}
		\scnhaselement{формальная теория}
		\scnhaselement{предметная область}
		\scnhaselement{предметная область и онтология}
		\scnhaselement{технология}
		\scnhaselement{база знаний}
	\end{SCn}
	
	Даже небольшой перечень видов знаний свидетельствует об огромном многообразии видов знаний.
\end{frame}

\begin{frame}{\\Знание}
	\topline
	\justifying
	
	\begin{SCn}
		\scnheader{знание}
		\begin{scnrelfromset}{разбиение}
			\scnitem{декларативное знание}
			\begin{scnindent}
				\scnidtf{знание, имеющее только денотационную семантику в виде	семантической спецификации системы используемых понятий}
			\end{scnindent}
			\scnitem{процедурное знание}
			\begin{scnindent}
				\scnidtf{знание, имеющее не только денотационную семантику, но
					и операционную семантику в виде семейства спецификаций программных компонентов (агентов), интерпретирующих указанное знание с целью решения некоторой задачи}
			\end{scnindent}
		\end{scnrelfromset}
	\end{SCn}
\end{frame}

\begin{frame}{\\Информационная конструкция}
	\topline
	\justifying
	
	\begin{SCn}
		\scnheader{информационная конструкция}
		\begin{scnrelfromset}{разбиение}
			\scnitem{дискретная информационная конструкция}
			\begin{scnindent}
				\scnsuperset{знаковая информационная конструкция}
				\scntext{примечание}{В рамках современных компьютерных систем мы всегда имеем дело с \textit{дискретными информационными конструкциями}.}
			\end{scnindent}
			\scnitem{непрерывная информационная конструкция}
			\begin{scnindent}
				\scnidtf{сигнал}
			\end{scnindent}
		\end{scnrelfromset}
	\end{SCn}
\end{frame}

\begin{frame}{Атомарный фрагмент\\ информационной конструкции}
	\topline
	\justifying
	\small
	
	\begin{SCn}
		\scnheader{информационная конструкция}
		\scnidtf{информация}
		\scntext{примечание}{В общем случае информационная конструкция представляет
			собой сложную иерархическую структуру, каждому уровню иерархии которой
			соответствует определенный класс информационных конструкций.}
		\scnsuperset{синтаксически элементарный фрагмент информационной конструкции}
		\begin{scnindent}
			\scnidtf{атомарный фрагмент информационной конструкции}
			\scnidtf{элемент информационной конструкции}
			\scntext{примечание}{Примерами таких элементарных фрагментов информационных
				конструкций являются буквы}
			\scnsuperset{буква}
		\end{scnindent}
	\end{SCn}
	
\end{frame}

\begin{frame}{Атомарный фрагмент\\ информационной конструкции}
	\topline
	\justifying
	\small
	
	\begin{SCn}
		\scnheader{информационная конструкция}
		\scnidtf{информация}
		\scntext{примечание}{В общем случае информационная конструкция представляет
			собой сложную иерархическую структуру, каждому уровню иерархии которой
			соответствует определенный класс информационных конструкций.}
		\scnsuperset{синтаксически элементарный фрагмент информационной конструкции}
		\begin{scnindent}
			\scnidtf{атомарный фрагмент информационной конструкции}
			\scnidtf{элемент информационной конструкции}
			\scntext{примечание}{Примерами таких элементарных фрагментов информационных
				конструкций являются буквы}
			\scnsuperset{буква}
		\end{scnindent}
	\end{SCn}
	
\end{frame}

\begin{frame}{Простой знак\\Выражение}
	\topline
	\justifying
	\small
	
	\begin{SCn}
		\scnheader{простой знак}
		\scnidtf{семантически элементарный фрагмент информационной конструкции}
		\scnsubset{знак}
		
		\scnheader{выражение}
		\scnidtf{сложный (непростой) знак}
		\scnidtf{знак, являющийся одновременно некоторым знанием обозначаемой
			сущности (спецификацией этой сущности)}
		\scnidtf{знак, построенный как выражение вида тот, который... }
		\scnidtf{знак, в состав которого входят другие знаки}
		\scnsubset{знак}
	\end{SCn}
	
\end{frame}

\begin{frame}{\\Тексты}
	\topline
	\justifying
	\scriptsize
	
	\vspace{10mm}
	
	\begin{SCn}
		\scnheader{простой текст}
		\scnidtf{минимальная синтаксически целостная и корректная (правильная)
			информационная конструкция, включающая в себя:
			\begin{textitemize}
				\item знак некоторой описываемой связи;
				\item минимальную спецификацию указанного знака связи (указание
				отношения, которому это связь принадлежит);
				\item указание \uline{всех} компонентов описываемой связи (знаков всех сущностей, связываемых этой связью, и/или всех знаков, связываемых этой связью);
				\item если описываемая связь не является бинарной, то связи с её
				компонентами могут потребовать явного представления знаков этих связей с дополнительным указанием роли этих компонентов.
		\end{textitemize}}
		\scnsubset{текст}
	
		\scnheader{сложный текст}
		\scnidtf{информационная конструкция, являющаяся результатом соединения
			нескольких простых текстов}
		\scnsubset{текст}
	\end{SCn}
	
\end{frame}

\begin{frame}{Простое знание\\Сложное знание}
	\topline
	\justifying
	\small
	
	\begin{SCn}
	\scnheader{простое знание}
	\scnidtf{минимальная семантические целостная информационная конструкция}
	\scnidtf{знание, в состав которого не входят другие знания}
	\scnsubset{знание}

	\scnheader{сложное знание}
	\scnidtf{информационная конструкция, являющаяся результатом соединения
		нескольких простых знаний}
	\scnidtf{знание, в состав которого входят другие знания}
	\scnsubset{знание}
	\end{SCn}
	
\end{frame}

\begin{frame}{Внутреннее и внешнее\\ представление информации}
	\topline
	\justifying
	\small
	
	\begin{SCn}
		\scnheader{следует отличать*}
		\begin{scnhaselementset}
			\scnitem{внутреннее представление информации}
			\begin{scnindent}
				\scnidtf{кодирование информации в памяти компьютерной системы}
			\end{scnindent}
			\scnitem{внешнее представление информации}
			\begin{scnindent}
				\scnidtf{обеспечение однозначности интерпретации (понимания, трактовки) этой информации
					разными пользователями и разными компьютерными системами}
			\end{scnindent}
		\end{scnhaselementset}
	\end{SCn}
	
\end{frame}

\begin{frame}{\\Знак}
	\topline
	\justifying
	\small
	
	\begin{SCn}
		\scnheader{знак}
		\scnidtf{фрагмент информационной конструкции, обладающий свойством, \uline{обозначать} некоторую сущность (объект), которая наряду с другими сущностями описывается указанной информационной конструкцией}
		\scntext{примечание}{\uline{Форма} представления знаков в известной степени
			условна и является результатом соглашения между носителями соответствующего языка. Знак может быть, например, представлен:
			\begin{textitemize}
				\item  в виде фрагмента речевого сообщения (последовательностью
				фонем);
				\item в виде строки символов (последовательности букв) в
				заданном алфавите;
				\item в виде иероглифа, пиктограммы;
				\item в виде жеста.
		\end{textitemize}}
	\end{SCn}
\end{frame}

\begin{frame}{\\Типология знаков}
	\topline
	\justifying
	\small
	
	\vspace{10mm}
	Знаки, используемые в различных языках, характеризуются:
	\begin{textitemize}
		\item синтаксической структурой, по совпадению (изоморфизму)
		которых для разных знаков предполагается их синонимия;
		\item денотационной семантикой, т.е. той сущностью, которая
		обозначается соответствующим знаком;
		\item типом (классом) обозначаемой сущности, которая может
		быть:
		\begin{textitemize}
			\item материальным (физическим) элементом (точкой)
			абстрактного пространства, множеством, которое может быть связью, классом, или структурой (информационной конструкцией)
			\item реальной и вымышленной сущностью;
			\item константной (конкретной) и переменной
			(произвольной) сущностью;
			\item постоянно существующей и временно существующей
			сущностью (прошлой, настоящей, будущей);
		\end{textitemize}
		\item ...
	\end{textitemize}
	
\end{frame}

\begin{frame}{\\Типология знаков}
	\topline
	\justifying
	\small
	
	\vspace{10mm}
	Знаки, используемые в различных языках, характеризуются:
	\begin{textitemize}
		\item ...
		\item множеством тех связей, которые связывают сущность,
		обозначаемую данным знаком с другими сущностями, а также, если данный знак
		обозначает некоторую связь, множеством сущностей, которые связаны этой связью,
		т.е. сущностей, являющихся компонентом этой связи;
		\item текущим статусом самого знака в памяти кибернетической
		системы, который может быть:
		\begin{textitemize}
			\item логически удаленным знаком;
			\item настоящим знаком;
			\item предлагаемым (возможно, будущим) знаком.
		\end{textitemize}
	\end{textitemize}
	
\end{frame}

\begin{frame}{\\Строение знака --- треугольник Фреге}
	\topline
	\justifying
	
	\textbf{I. Вещь}, предмет, явление действительности и т. д. Иное название --- \textbf{денотат}
	
	\vspace{0.5em}
	
	\textbf{II. Знак}: в лингвистике, например, фонетическое слово или написанное слово в математике --- математический символ; иное название, принятое особенно в философии и математической логике, --- \textbf{имя}.
	
	\vspace{0.5em}
	
	\textbf{III. Понятие} о предмете, вещи. Иные названия: в лингвистике --- \textbf{сигнификат}, десигнат, в математике --- смысл имени, или концепт денотата.
	
\end{frame}

\begin{frame}{\\Треугольник Фреге}
	\topline
	\justifying
	\vspace{15mm}
	\begin{center}
		\includegraphics[width=0.9\textwidth]{part2/images/is_knowledge_bases/frege.png}
	\end{center}
	
\end{frame}

\begin{frame}{\\Денотат}
	\topline
	\justifying
	\small
	
	\begin{SCn}
		\scnheader{денотат*}
		\scnidtf{денотат заданного знака*}
		\scnidtf{объект, обозначаемый заданным знаком*}
		\scnidtf{денотационная семантика заданного знака*}
		\scnidtf{смысл заданного знака*}
		\scnidtf{Бинарное ориентированное отношение, каждая пара которого
			связывает:
			\begin{textitemize}
				\item некоторый знак, представленный в той или иной форме в тексте
				исследуемого языка;
				\item \uline{со знаком} той сущности, которая обозначается указанным
				выше знаком в рамках используемого метаязыка.
		\end{textitemize}}
	\end{SCn}
	
\end{frame}

\begin{frame}{\\Денотат}
	\topline
	\justifying
	\small
	
	Данное отношение используется, когда с помощью одного языка необходимо описать денотационную семантику другого языка. 
	
	\bigskip
	
	Фактически речь идет о переводе заданного знака, входящего в состав некоторого рассматриваемого текста, принадлежащего некоторому исследуемому языку (языку-объекту), на некоторый метаязык, денотационная семантика которого нам считается априори известной. 
	
	\bigskip
	
	Указанный перевод есть связь заданного знака с синонимичным ему знаком, входящим в состав текста, принадлежащего указанному метаязыку.
	
\end{frame}

\begin{frame}{\\Понятие языка}
	\topline
	\justifying
	
	Язык представляет собой множество текстов и задаётся четвёркой
	\[
	L = \{ A, S_n, S_m, P \},
	\]
	где
	
	A --- алфавит (множество символов),
	
	$S_n$ --- синтаксис языка,
	
	$S_m$ --- семантика языка,
	
	P --- прагматика языка.
	
\end{frame}

\begin{frame}{\\Языки}
	\topline
	\justifying
	
	\begin{SCn}
	\scnheader{язык}
	\scnsuperset{универсальный язык}
	\scnsuperset{естественный язык}
	\scnsuperset{искусственный язык}
	\begin{scnindent}
		\scnsuperset{язык смыслового представления}
	\end{scnindent}
	\scnsuperset{специализированный язык}
	\scnsuperset{формальный язык}
	\end{SCn}
	
\end{frame}

\begin{frame}{\\Инженерия знаний}
	\topline
	\justifying
	
	\begin{textitemize}
		\item \textbf{Извлечение знаний}
		\begin{textitemize}
			\item Коммуникативные методы
			\item Текстологические методы
		\end{textitemize}
		\item \textbf{Представление знаний} (запись знаний на языке, понятном интеллектуальной системе для последующего использования при решении задач)
		
		\item \textbf{Верификация знаний}
	\end{textitemize}
	
\end{frame}

\begin{frame}{\\Представление и извлечение знаний}
	\topline
	\justifying
	
	\textbf{Извлечение знаний}:
	\begin{textitemize}
		\item Где можно найти знания по проблеме?
		\item На основании чего человек сделал именно так?
		\item Основные способы: наблюдение, беседа, анализ книг.
	\end{textitemize}	
	
	\textbf{Представление знаний}:
	\begin{textitemize}
		\item Как люди представляют знания?
		\item Есть ли универсальный способ представления знаний?
		\item Как представить знания для решения этой задачи?
	\end{textitemize}
	
\end{frame}

\begin{frame}{\\Методы извлечения знаний}
	\topline
	\justifying
	
	\begin{textitemize}
		\item Коммуникативные методы
		\begin{textitemize}
			\item Активные (мозговой штурм, интервью, анкетирование и т.д.);
			\item Пассивные (наблюдения, лекции).
		\end{textitemize}
		\item Текстологические методы (анализ литературы, анализ документов);
		\item Автоматизированные методы извлечения знаний.
	\end{textitemize}
	
\end{frame}

\begin{frame}{\\Требования к представлению знаний}
	\topline
	\justifying
	
	\begin{textitemize}
		\item \textbf{Наглядность} и простота представления знаний;
		\item \textbf{Удобство представления знаний} для работы интеллектуальной системы (простота использования или обработки);
		\item \textbf{Универсальность}: способность представлять самые разные виды знаний;
		\item \textbf{Расширяемость} базы знаний, возможность интеграции различных видов знаний и целых баз знаний.
	\end{textitemize}
	
\end{frame}

\begin{frame}{\\Модель представления знаний}
	\topline
	\justifying
	
	\textbf{Модель представления знаний} --- формализм, предназначенный для описания статических и динамических свойств предметных областей (соглашение о том, как описывать знания).
	
	\vspace{1em}
	
	Классификация моделей:
	\begin{textitemize}
		\item Универсальные модели представления знаний
		\item Специализированные модели представления знаний.
	\end{textitemize}
	
\end{frame}

\begin{frame}{\\Основные модели представления знаний}
	\topline
	\justifying
	
	\begin{textitemize}
		\item Фреймы;
		\item Формальные логические модели;
		\item Продукционные модели;
		\item Семантические сети.
	\end{textitemize}
	
\end{frame}

\begin{frame}{\\Фреймы}
	\topline
	\justifying
	
	\textbf{Фрейм} (Каркас) --- абстрактный образ для представления некоего стереотипа восприятия.
	
	Фрейм имеет набор свойств (слотов).
	
	Термин предложен Марвином Минским в 1979 г.
	
	Первоначальное применение --- распознавание пространственных сцен по ключевым признакам.
	
\end{frame}

\begin{frame}{\\Фреймы: классификация}
	\topline
	\justifying
	
	Классификация фреймов:
	\begin{textitemize}
		\item Фреймы-образцы (прототипы);
		\item Фреймы-экземпляры;
	\end{textitemize}
	
	\textbf{Фрейм} --- некая структура для описания знаний, которая по мере заполнения свойствами превращается в описание факта, события или ситуации.
	
	В ходе вывода во фреймовой модели сначала подбирается прототип, а потом идет его уточнение применительно к образу.
	
\end{frame}

\begin{frame}{\\Пример фрейма}
	\topline
	\justifying
	
	\textbf{Ситуация}:
	Студент Иванов получил книгу А. Я. Архангельского <<100 компонентов Delphi>> в библиотеке ТГПУ им. Л. Н. Толстого в г. Туле
	
	\textbf{Фрейм} ПОЛУЧЕНИЕ
	\begin{textitemize}
		\item ОБЪЕКТ : (КНИГА (Автор, Название));
		\item АГЕНТ: (СТУДЕНТ(Фамилия));
		\item МЕСТО: (БИБЛИОТЕКА(Название, Расположение)).
	\end{textitemize}
	
\end{frame}

\begin{frame}{\\Средства для работы с фреймами}
	\topline
	\justifying
	\vspace{12mm}
	\begin{center}
		\includegraphics[width=0.8\textwidth]{part2/images/is_knowledge_bases/frame.png}
	\end{center}
	
\end{frame}

\begin{frame}{\\Фреймы: оценка}
	\topline
	\justifying
	\vspace{10mm}
	\textbf{Достоинства}:
	\begin{textitemize}
		\item Интеграция знаний (декларативных и процедурных)
		\item Соответствие принципам хранения знаний человеком
		\item Наглядность, гибкость, однородность
	\end{textitemize}
	
	\textbf{Недостатки}:
	\begin{textitemize}
		\item Ограниченная выразительность
		\item Сложность масштабирования
		\item Слабые механизмы вывода
		\item Плохая работа с неопределённостью и динамикой
		\item Отсутствие стандартов и неоднозначность моделирования
		\item Трудности при «пограничных» ситуациях (между фреймами)
	\end{textitemize}
	
\end{frame}

\begin{frame}{\\Формальные логические модели}
	\topline
	\justifying
	
	\vspace{10mm}
	Виды логических моделей:
	\begin{textitemize}
		\item Исчисление высказываний
		\item Исчисление предикатов
		\item Нечеткие логики
	\end{textitemize}
	
	В основе лежит формальная теория (система):
	\begin{textitemize}
		\item T --- множество термов, алфавит системы
		\item P --- синтаксические правила построения выражений из базовых термов
		\item A --- аксиомы, априорно истинные выражения
		\item R --- множество правил вывода, которые позволяют получить из одних истинных высказываний другие
	\end{textitemize}
	
	Практически не используются в промышленных разработках.
	
\end{frame}

\begin{frame}{\\Исчисление высказываний}
	\topline
	\justifying
	
	\textbf{Высказывание} --- неделимое грамматически правильное предложение, которое можно охарактеризовать как истинное или ложное.
	
	\vspace{1em}
	
	\textbf{Сложное высказывание} --- комбинация простых высказываний при помощи логических связок.
	
	\vspace{1em}
	
	Пример:
	\begin{textitemize}
		\item A --- Луна спутник Земли
		\item B --- Солнце спутник Земли
		\item A \& B --- ложь
	\end{textitemize}
	
\end{frame}

\begin{frame}{\\Исчисление предикатов}
	\topline
	\justifying
	
	\textbf{Исчисление предикатов} позволяет моделировать предметную область и проверять различные гипотезы относительно этой предметной области при помощи разработанной системы предикатов.
	
	\vspace{1em}
	
	\textbf{Предикат} --- функция на множестве M=M1*M2*…*Mn, принимающая значение истина или ложь.
	
	\vspace{1em}
	
	Пример:
	\begin{textitemize}
		\item Если философ выиграет у кого-нибудь в забеге, то этот человек будет им восхищен
		\item (любой X) (любой Y) (PHILOSOPHER(X) \textasciicircum BEATS(X, Y) > ADMIRE(Y, X)).
	\end{textitemize}
	
\end{frame}

\begin{frame}{\\Логические модели: пример}
	\topline
	\justifying
	
	\[
	\forall x \forall a \forall b \forall c \Bigl(
	RT(x) \land K(x,a) \land K(x,b) \land H(x,c)
	\rightarrow
	Pyth\bigl(a,b,c\bigr)
	\Bigr).
	\]
	
	\[
	\forall a \forall b \forall c \Bigl(
	Pyth(a,b,c) \leftrightarrow
	\exists u \exists v \exists w \bigl(
	S(a,u) \land S(b,v) \land S(c,w) \land Plus(u,v,w)
	\bigr)
	\Bigr).
	\]
	
\end{frame}

\begin{frame}{\\Логические модели: пример}
	\topline
	\justifying
	
\begin{flalign*}
	RT(x) &:\; x \text{ --- прямоугольный треугольник.} \\
	K(x, a) &:\; a \text{ --- катет треугольника } x. \\
	H(x, c) &:\; c \text{ --- гипотенуза треугольника } x. \\
	S(a, u) &:\; u \text{ --- квадрат длины отрезка } a. \\
	Plus(u, v, w) &:\; w \text{ есть сумма } u \text{ и } v. \\
	Pyth(a, b, c) &:\; \text{выполнено «квадрат гипотенузы равен} \\
	&\text{ сумме квадратов катетов» для } a, b, c.
\end{flalign*}
	
\end{frame}

\begin{frame}{\\Формальные логические модели: оценка}
	\topline
	\justifying
	
	\textbf{Достоинства}:
	\begin{textitemize}
		\item Высокий уровень формализации, что обеспечивает точность получения результата
		\item Согласованность
		\item Единый способ описания знаний о предметной области и способов решения задач в предметной области.
	\end{textitemize}
	
	\textbf{Недостатки}:
	\begin{textitemize}
		\item Ненаглядность представления знаний
		\item Очень строгие ограничения, накладываемые структурой представления знаний
	\end{textitemize}
	
\end{frame}

\begin{frame}{\\Продукционные модели}
	\topline
	\justifying
	
	\textbf{Продукционная модель} представляет собой набор продукций --- правил вида «Если» --- «То»;
	
	Формально продукция описывается как:
	\begin{textitemize}
		\item W --- сфера применения продукции.
		\item U --- предусловие --- (истинность продукции).
		\item P --- условие применения продукции.
		\item A->B --- ядро продукции, правило типа Если … , то … .
		\item C --- постусловие продукции, действия после обработки продукции.
	\end{textitemize}
	
	\textbf{Продукционная система} = база фактов + продукции + интерпретатор.
	
\end{frame}

\begin{frame}{\\Продукционные модели: пример}
	\topline
	\justifying
	\vspace{10mm}
	\begin{center}
		\includegraphics[width=\textwidth]{part2/images/is_knowledge_bases/product.png}
	\end{center}
	
\end{frame}

\begin{frame}{\\Продукционные модели: оценка}
	\topline
	\justifying
	
	\textbf{Достоинства}:
	\begin{textitemize}
		\item Простота и наглядность правил
		\item Простота пополнения базы знаний
		\item Простота вывода в базе знаний
	\end{textitemize}
	
	\textbf{Недостатки}:
	\begin{textitemize}
		\item Несоответствие представлению знаний человеком
		\item Сложно управлять выводом при больших БЗ
		\item Сложность оценки непротиворечивости БЗ
	\end{textitemize}
	
\end{frame}

\begin{frame}{\\Семантические сети}
	\topline
	\justifying
	
	\textbf{Семантическая сеть (граф знаний)} --- ориентированный граф, вершины которого понятия, а дуги --- отношения между ними.
	
	\vspace{1em}
	
	$\bm{S = (O, R)}$
	
	\vspace{1em}
	
	Наиболее общий способ представления знаний
	
	\vspace{1em}
	
\end{frame}

\begin{frame}{\\Классификация семантических сетей}
	\topline
	\justifying
	
	По количеству типов отношений:
	\begin{textitemize}
		\item однородные
		\item неоднородные
	\end{textitemize}
	
	По типу отношений:
	\begin{textitemize}
		\item бинарные
		\item N-арные
	\end{textitemize}
	
\end{frame}

\begin{frame}{\\Графы знаний: примеры}
	\topline
	\justifying
	\vspace{10mm}
	\begin{center}
		\includegraphics[width=\textwidth]{part2/images/is_knowledge_bases/kg.png}
	\end{center}
	
\end{frame}

\begin{frame}{\\Графы знаний: примеры}
	\topline
	\justifying
	
	\begin{columns}
		\begin{column}{0.49\textwidth}
			\centering
			\includegraphics[width=\textwidth]{part2/images/is_knowledge_bases/kolas.png}
			
			\textbf{Google Knowledge Graph}
		\end{column}
		\begin{column}{0.49\textwidth}
			\centering
			\includegraphics[width=\textwidth]{part2/images/is_knowledge_bases/kupala.png}
			
			\textbf{WikiData}
		\end{column}
	\end{columns}
	
\end{frame}

\begin{frame}{\\Семантические сети: оценка}
	\topline
	\justifying
	
	Достоинства:
	\begin{textitemize}
		\item Наглядность, универсальность, простота понимания
		\item Соответствуют представлению знаний у человека
	\end{textitemize}
	
	Недостатки:
	\begin{textitemize}
		\item Сложность организации процессов вывода на семантической сети
		\item Смешение различных групп знаний
	\end{textitemize}
	
\end{frame}

\begin{frame}{}
	
	\centering
	
	\Huge
	
	\textbf{Онтологии}
	
\end{frame}

\begin{frame}{\\Для чего нужны онтологии}
	\topline
	\justifying
	\vspace{15mm}
	\begin{center}
		\includegraphics[width=0.9\textwidth]{part2/images/is_knowledge_bases/ontocat.png}
	\end{center}
	
\end{frame}

\begin{frame}{Неоднозначность соответствия\\ знак --- понятие --- денотат}
	\topline
	\justifying
	
	
	\Large
	
	\begin{textitemize}
		\item Омонимия
		\item Синонимия
		\item Полисемия
	\end{textitemize}
	
\end{frame}

\begin{frame}{\\Омонимия}
	\topline
	\justifying
	
	\vspace{15mm}
	Существует несколько одинаковых по форме знаков, каждый из которых имеет свое значение, причем эти значения между собой абсолютно не связаны, так же как и соответствующие денотаты.
	
	\begin{center}
		\includegraphics[width=0.9\textwidth]{part2/images/is_knowledge_bases/jaguar.png}
	\end{center}
	
\end{frame}

\begin{frame}{\\Полисемия}
	\topline
	\justifying
	
	\vspace{10mm}
	Наличие различных значений у одного и того же знака. Обычно данный термин применяется в ситуациях, когда эти различные значения в какой-либо мере связаны между собой (в отличие от омонимии).
	
	Пример: определения понятия для словосочетания <<Капитальный ремонт>>,
	встречающегося на одном из отечественных предприятий:
	\begin{textitemize}
		\item работы по техническому обслуживанию и ремонту оборудования,
		выполняемые в капитальный останов, т.~е. останов производственной
		линии длительностью свыше 24-х часов
		\item ремонт, выполняемый для восстановления исправности и полного
		или близкого к полному восстановлению ресурса изделия с заменой
		или восстановлением любых его частей, включая базовые
	\end{textitemize}
	
\end{frame}

\begin{frame}{\\Синонимия}
	\topline
	\justifying
	
	Указывает на равнозначность, но не тождественность знаков.
	
	\vspace{1em}
	
	Под равнозначностью понимается:
	\begin{textitemize}
		\item либо соотнесенность с одним и тем же денотатом (предметом);
		\item либо соотнесенность с одним и тем же понятием, точнее с той его частью, которая содержит характеризующую информацию.
	\end{textitemize}
	
\end{frame}

\begin{frame}{\\Синонимия: пример}
	\topline
	\justifying
	
	\vspace{10mm}
	
	\begin{center}
		\includegraphics[width=0.9\textwidth]{part2/images/is_knowledge_bases/synonymy.png}
	\end{center}
	
\end{frame}


\begin{frame}{\\Решение проблемы неоднозначности}
	\topline
	\justifying
	
	Для снижения представленной выше неоднозначности соответствия знак --- понятие --- денотат необходим <<общий язык>>, включающий в себя:
	\begin{textitemize}
		\item Строго определенный словарь лексических единиц (знаков),
		\item Непротиворечивое понимание того, какие понятия обозначаются заданными лексическими единицами (знаками).
	\end{textitemize}
	
\end{frame}

\begin{frame}{\\Единое информационное пространство}
	\topline
	\vspace{15mm}
	
	\begin{columns}[T]
		\begin{column}{0.33\textwidth}
			\justifying
			Кроме потребности в общем языке организации испытывают потребность в интеграции информации и создании единого информационного пространства. Необходимые знания могут находиться в различных местах.
		\end{column}
		
		\begin{column}{0.65\textwidth}
			\centering
			\includegraphics[width=\textwidth]{part2/images/is_knowledge_bases/datasources.png}
		\end{column}
	\end{columns}
	
\end{frame}

\begin{frame}{\\Цифровые двойники данных}
	\topline
	\justifying
	
	\vspace{10mm}
	
	\begin{center}
		\includegraphics[width=\textwidth]{part2/images/is_knowledge_bases/dt.png}
	\end{center}
	
\end{frame}

\begin{frame}{\\Онтология}
	\topline
	\justifying
	
	Онтология (от древнегреч. онтос --- сущее, логос --- учение, понятие) --- термин, определяющий учение о сущем, бытии, в отличие от гносеологии --- учение о познании.
	
	\vspace{1em}
	
	В философском смысле (а этот термин заимствован из философии), онтология есть определенная система категорий, являющихся следствием определенных взглядов на мир.
	
\end{frame}

\begin{frame}{\\История понятия}
	\topline
	\justifying
	
	\vspace{10mm}
	
	В настоящее время термин «онтология» переместился в область информационных технологий, где был использован рядом исследовательских сообществ по искусственному интеллекту вначале в области инженерии знаний, в обработке естественных языков, а затем в представлении знаний [Gruber, 1993; Uschold, Jasper, 1999].
	
	\vspace{0.5em}
	
	В конце 1990-х --- начале 2000-х годов понятие онтологии также стало широко использоваться в таких областях, как интеллектуальная интеграция информации, поиск информации в Интернете и управление знаниями [Fensel, 2001; Gomez-Perez et al, 2006]
	
\end{frame}

\begin{frame}{\\История понятия (продолжение)}
	\topline
	\justifying
	
	Позже онтологии стали рассматриваться в качестве ключевого элемента в проекте Semantic Web --- нового этапа развития сети WWW (Word Wide Web).
	
	\vspace{1em}
	
	Если существующая Web-сеть --- это огромное множество документов, которые связаны перекрестными ссылками, то создаваемый Semantic Web должен добавить к существующей сети множество онтологий и метаописаний знаний, содержащихся в документах Web-сети (включая стандарты и программные инструменты) [BernersLee et al, 2001; Domingue et al, 2011].
	
\end{frame}

\begin{frame}{\\Понятие онтологии}
	\topline
	\justifying
	
	\vspace{2em}
	В целом онтология состоит из иерархии понятий предметной области, связей между ними и законов, которые действуют в рамках этой модели.
	
	\vspace{0.5em}
	Онтология строится как сеть, состоящая из понятий, связей между ними и описания свойств указанных понятий. Связи могут быть различного типа (например, «является», «состоит из», «является исполнителем» и т. п.).
	
	\vspace{0.5em}
	Для выполнения роли общего языка онтология включает в себя словарь (множество слов/лексических единиц) и набор определений понятий (формальных, описанных, например, на каком-либо логическом языке, или неформальных), задающих смысл элементов словаря.
	
	\vspace{0.5em}
	Связи между словарем (знаками) и семантикой ограничивают множество возможных интерпретаций знаков.
	
\end{frame}

\begin{frame}{\\Более строго}
	\topline
	\justifying
	
	\begin{textitemize}
		\item \textbf{Онтология} -- это явная формальная спецификация концептуализации некоторой предметной области, разделяемая некоторым сообществом агентов
		\begin{textitemize}
			\item Другими словами -- формальное (строгое) описание системы понятий некоторой предметной области, согласованное в рамках некоторого сообщества
			\item Цель разработки онтологий --- решение таких проблем как синонимия, омонимия, полисемия
		\end{textitemize}
	\end{textitemize}
	
\end{frame}

\begin{frame}{Еще более строго:\\формальная модель онтологии}
	\topline
	\justifying
	
	O = < C, P, R, А >,
	
	\vspace{1em}
	
	где: 
	\begin{textitemize}
		\item С --- конечное множество понятий (классов сущностей) предметной области;
		\item P --- конечное множество свойств этих понятий (классов);
		\item R --- конечное множество связей между понятиями (классами);
		\item А --- множество аксиом, утверждений, построенных из этих понятий, их свойств и связей между ними.
	\end{textitemize}
	
\end{frame}

\begin{frame}{\\Пример устранения омонимии}
	\topline
	\justifying
	
	После установки связи между словом «Ягуар» и интенсиональным определением «хищное животное семейства кошек» мы исключаем из множества возможных интерпретаций этого слова машины и напитки.
	
	\vspace{1em}
	В результате онтологии позволяют правильно соотносить знаки, используемые людьми и компьютерными системами, с понятиями (семантическими моделями) и обозначаемыми объектами реального мира.
	
\end{frame}

\begin{frame}{Пример интеграции\\ разнородных источников информации}
	\topline
	\justifying
	
	\vspace{10mm}
	
	\begin{center}
		\includegraphics[width=0.9\textwidth]{part2/images/is_knowledge_bases/ontodocs.png}
	\end{center}
	
\end{frame}

\iffalse

% Слайд 11: Semantic Web
\begin{frame}{Semantic Web}
	\topline
	\justifying
	
	\begin{textitemize}
		\item Март 1989 -- концепция Всемирной паутины (HTML, HTTP, URL, DNS, XML,\ldots)
		\item 1997-2003 -- развитие идеи Семантического Веба (RDF, OWL, OWL2,\ldots)
	\end{textitemize}
	
	\vspace{1em}
	
	\begin{textitemize}
		\item Цель Semantic Web --- создать надстройку над существующей Всемирной паутиной, которая позволит сделать веб \textit{машинно-понимаемым}
		\begin{textitemize}
			\item описывать ресурсы формальными онтологиями и связями
			\item программы смогут автоматически находить, интерпретировать и комбинировать данные, а не просто показывать страницы людям
		\end{textitemize}
	\end{textitemize}
	
\end{frame}

\fi

%онтологии
%Semantic Web
%Смысловое представление