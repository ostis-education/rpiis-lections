\title{Лекция 8\\Архитектура интеллектуальных систем}   
\author[]{Шункевич Д.В.}
\institute[]{Белорусский государственный университет информатики и радиоэлектроники}

\begin{frame}
	\titlepage
\end{frame}


\begin{frame}{\\Содержание лекции}
	\topline
	\justifying
\begin{SCn}
	\scnheader{Структура лекции}

	\begin{scnrelfromset}{разбиение}
				\scnitem{Интеллектуальная компьютерная система}
				\scnitem{Интеллектуальные системы, основанные на базах знаний}
				\scnitem{Интеллектуальные системы нейронных сетей}
				\scnitem{Гибридные интеллектуальные системы}
	\end{scnrelfromset}
				
\end{SCn}
\end{frame}


\begin{frame}{Интеллектуальная компьютерная система}
	\topline
	\justifying

\par Искусственный интеллект - наука и технология создания машин и программ, способных выполнять задачи, которые обычно присущи человеку. Искусственный интеллект применяется в разных сферах общественной жизни, таких как финансы, медицина, образование, военное дело и развлечения. Искусственный интеллект развивается с середины 20-го века и достиг значительных успехов в последние годы. Однако пока не существует системы, которая бы полностью заменила человека в решении задач, но методы и элементы ИИ активно внедряются в современное программное обеспечение.

\end{frame}


\begin{frame}{Интеллектуальная компьютерная система}
	\topline
	\justifying
\begin{SCn}

	\scnheader{Интеллектуальная компьютерная система}
	\scntext{определение}{компьютеризированная система  сбора, хранения, обработки, представления информации, работа которой основывается на имитации (воспроизведении) интеллектуальных возможностей человека.}
	
\end{SCn}
\end{frame}


\begin{frame}{Интеллектуальная компьютерная система}
	\topline
	\justifying
\begin{SCn}

	\scnheader{Интеллектуальная система}

	\begin{scnrelfromset}{свойства}
				\scnitem{Решение задач в условиях неопределенности}
				\scnitem{Решение неформализованных  (трудноформализуемых) задач}
				\scnitem{Эвристическое решение задач}
				\scnitem{Обучаемость и приобретение опыта}
	\end{scnrelfromset}
	
\end{SCn}
\end{frame}


\begin{frame}{Интеллектуальная компьютерная система}
	\topline
	\justifying
\begin{SCn}

	\scnheader{Традиционная информационная система}

	\begin{scnrelfromset}{преобладающие функции}
				\scnitem{Работа с количественной информацией}
				\scnitem{Математические вычисления}
				\scnitem{Хранений информации}
				\scnitem{Поиск информации}
				\scnitem{Обмен информацией}
				\scnitem{Визуализация данных}
	\end{scnrelfromset}
	
\end{SCn}
\end{frame}


\begin{frame}{Интеллектуальная компьютерная система}
	\topline
	\justifying
\begin{SCn}

	\scnheader{Интеллектуальная система}

	\begin{scnrelfromset}{преобладающие функции}
				\scnitem{Работа с качественной информацией}
				\scnitem{Логический вывод}
				\scnitem{Семантический поиск информации}
				\scnitem{Контекстнозависимый поиск информации}
				\scnitem{Интерпритация данных}
	\end{scnrelfromset}
	
\end{SCn}
\end{frame}


\begin{frame}{Интеллектуальная система}
	\topline
	\justifying
	
	Таким образом, принципиальные отличия традиционной информационной системы от интеллектуальной заключаются в том, что 
	интеллектуальная информационная система занимается получением новых знаний из уже усвоенных или получает знания в ходе
	обработки данных, в то время как традиционная может проводить только колличественную оценку информации.

\end{frame}


\begin{frame}{Интеллектуальные системы,\\ основанные на базах знаний}
	\topline
	\justifying
\begin{SCn}

	\scnheader{Интеллектуальная система, основанная на базах знаний}
		\begin{scnrelfromset}{характерные особенности}
			\scnitem{Выразительность}
			\scnitem{Интуитивность}
			\scnitem{Модульность}
			\scnitem{Модифицируемость}
			\scnitem{Наукоёмкость}
		\end{scnrelfromset}
	
\end{SCn}
\end{frame}


\begin{frame}{Интеллектуальные системы,\\ основанные на базах знаний}
	\topline
	\justifying
\begin{SCn}
	
	\scnheader{Интеллектуальная система, основанная на базах знаний}
	\scntext{преимущество}{База знаний очень эффективно отделена от механизма вывода,
						таким образом позволяя механизму вывода быть универсальным.}
	
\end{SCn}
\end{frame}


\begin{frame}{Интеллектуальные системы,\\ основанные на базах знаний}
	\topline
	\justifying
\begin{SCn}
	
	\scnheader{Интеллектуальная система, основанная на базах знаний}
	\scntext{недостаток}{Из-за того, что база знаний не имеет иерархии, а также из-за того, что она постоянно пополняется новыми знаниями и 								правилами, система становится напрозрачной и становится медленной в работе.}
	
\end{SCn}
\end{frame}



\begin{frame}{Интеллектуальные системы,\\ основанные на базах знаний}
	\topline
	\justifying

\par  В определённый момент развития таких систем появилась необходимость в экспертном знании, т.к. от системы требуется не только принимать решения, но и обосновывать их. 
\par Задача получения экспертных знаний и передача их интеллектулаьной системе стала известна как \textbf{инженерия знаний}.

\end{frame}


\begin{frame}{Интеллектуальные системы,\\ основанные на базах знаний}
	\topline
	\justifying
\begin{SCn}
	
	\scnheader{Экспертная система}
	\begin{scnrelfromset}{характеристики}
				\scnitem{Система работает на уровне, который обычно считается равным уровню эксперта в данной области.}
				\scnitem{Система сильно завсит от предметной области. Из-за того, что система знает об узкой предметной области.}
				\scnitem{Система должна объяснять свои рассуждения, т.к. ценность работы системы заключается не только в выводе, но и в 									анализе}
				\scnitem{Если информация является вероятностной, то система должна правильно предоставлять ряд альтернативных решений с 								соответствующими вероятностями.}
	\end{scnrelfromset}
	
\end{SCn}
\end{frame}


\begin{frame}{Интеллектуальные системы,\\ основанные на базах знаний}
	\topline
	\justifying
\begin{SCn}
	
	Важным элементом интеллектуальной системы является её стратегия обучения. Каждая стратегия должна использоваться для решения подходящих 	для неё задач.

\end{SCn}
\end{frame}


\begin{frame}{Интеллектуальные системы,\\ основанные на базах знаний}
	\topline
	\justifying
\begin{SCn}	
	\scnheader{интеллектуальная система}
	\begin{scnrelfromset}{стратегия обучения}
				\scnitem{механическое обучение (без понимания смысла)}
				\scnitem{обучение по инструкции}
				\scnitem{обучние путём дедукции}
				\scnitem{обучение по аналогии}
				\scnitem{обучение путём индукции}
	\end{scnrelfromset}

\end{SCn}
\end{frame}


\begin{frame}{Интеллектуальные системы,\\ основанные на базах знаний}
	\topline
	\justifying
\begin{SCn}
	
	\scnheader{Механическое обучение}
	\scntext{определение}{\textbf{\textit{Механическое обучение}} - это базовый способ обучения как у человека, так и у машины. В машине это 								соответствует знанию напрямую записанному в память системы или базы знаний.}

\end{SCn}
\end{frame}



\begin{frame}{Интеллектуальные системы,\\ основанные на базах знаний}
	\topline
	\justifying
\begin{SCn}
	
	\scnheader{Обучение по инструкции}
	\scntext{определение}{\textbf{\textit{Обучение по инструкции}} - это обучение путём приобретения знаний от учителя или учебника и преобразуются 						системой во внутреннее представление, которое будет применимо для решения задач. Ответственность за 										структурирование и представление знаний, остаётся за учителем, но учащийся должен сделать вывод, чтобы 										преобразовать знание к виду, пригодному для использования. Роль учащегося можно рассматривать как выполнение 								синтаксической переформулировки знаний, предоставленных первоисточником.}

\end{SCn}
\end{frame}


\begin{frame}{Интеллектуальные системы,\\ основанные на базах знаний}
	\topline
	\justifying
\begin{SCn}

	\scnheader{Обучние путём дедукции}
	\scntext{определение}{\textbf{\textit{Обучение путём дедукции}} перекладывает ответственность за преобразование знаний в удобную для 									восприятия форму с учителя на ученика. Ограничения на представления знаний источника ослаблены. Учащийся 									самостоятельно делает дедуктивные выводы из знаний и переформулирует их в полезные выводы, сохраняя 										информативность. Дедуктивное обучение подразумевает переформулирование входящих знаний и их разбиение на 									фрагменты, сохраняя истинность исходного знания.}

\end{SCn}
\end{frame}


\begin{frame}{Интеллектуальные системы,\\ основанные на базах знаний}
	\topline
	\justifying
\begin{SCn}

	\scnheader{Обучение по аналогии}
	\scntext{определение}{\textbf{\textit{Обучение по аналогии}} сочетает в себе дедуктивное и индуктивное обучение. Первым шагом является 									индуктивный вывод, необходимый для нахождения общей подструктуры между задачей и схожими предметными 									областями, хранящимися в базе знаний обучающегося. Затем, идёт сопоставление решения из выбранной аналогичной 								области с проблемной областью с использованием дедуктивной логики. }

\end{SCn}
\end{frame}


\begin{frame}{Интеллектуальные системы,\\ основанные на базах знаний}
	\topline
	\justifying
\begin{SCn}

	\scnheader{Обучение путём индукции}
	\scntext{определение}{\textbf{\textit{Обучение путём индукции}} - это классификация опыта в понятия и категории. Понятия даются учителем или 									возникают по аналогии, в ходе решения задачи. Если было несколько случаев, когда конкретное действие было 									подходящим в ответ на ситуацию, то делается вывод, который обощает это действие. Обучение путём индукции 									ставит своей целью обобщение уже усвоенных понятий, в котором вывод происходит из набора примеров 										предоставленных окружающей средой. }


\end{SCn}
\end{frame}


\begin{frame}{Интеллектуальные системы,\\ основанные на базах знаний}
	\topline
	\justifying

	\par Существует набор проблем, связанных с тем, что обучающий или окружающая среда могут передавать знания с некоторой погрешностью. 	Также с течением времени системе необходимо обновлять знания, при этом возникают сложности с совместимостью.


\end{frame}



\begin{frame}{Интеллектуальные системы \\нейронных сетей}
	\topline
	\justifying
\begin{SCn}


	\par В 1985 году, в статье "The Challenge of Open Systems" Карл Хьюитт постулировал, что системы, которые мы хотим смоделировать, являются 			открытыми. Открытые системы постоянно коммуницируют и  ограничены извне.

\end{SCn}
\end{frame}


\begin{frame}{Интеллектуальные системы \\нейронных сетей}
	\topline
	\justifying
\begin{SCn}
	\scnheader{ Открытые системы}
		\begin{scnrelfromset}{характерные особенности}
			\scnitem{постоянное измененеи и эволюция}
			\scnitem{децентрализованное принятие решений}
			\scnitem{необходимость согласованности компонентов системы}
		\end{scnrelfromset}

\end{SCn}
\end{frame}



\begin{frame}{Интеллектуальные системы \\нейронных сетей}
	\topline
	\justifying
\begin{SCn}

	\scnheader{Нейронная сеть} 
	\scntext{определение}{\textbf{\textit{Нейронная сеть}} - это средство для решения задач, организованное, по всей видимости, на подобии работы 								мозга человека. В модели нейронной сети существует три основных аспекта: нейрон, топология сети и стратегия 									обучения. }

\end{SCn}
\end{frame}


\begin{frame}{Интеллектуальные системы \\нейронных сетей}
	\topline
	\justifying
	
	\par Нейроны соединены между собой, и поведение всей системы определяется структурой и силой этих связей. Обрабатывающие элементы (нейроны) 			объединены в группы или слои.

\end{frame}


\begin{frame}{Интеллектуальные системы \\нейронных сетей}
	\topline
	\justifying
\begin{SCn}

	\scnheader{Обучение нейронной сети} 
	\begin{scnrelfromset}{разбиение}
			\scnitem{Самоорганизованное обучение}
			\begin{scnindent}
					\scnrelfrom{пояснение}{Этот процесс не подразумевает внешнего учителя, и система полагается только на локальную 												информацию и внутренние стратегии контроля.}
					\begin{scnrelfromset}{пример}
						\scnitem{"Адаптивная резонансная теория" С. Гроссберг, Г. Карпетер}
						\scnitem{"Сети Хопфилда" Дж. Хопфилд}
						\scnitem{"Двунаправленная ассоциативная память" Б. Коско}
					\end{scnrelfromset}
			\end{scnindent}
	\end{scnrelfromset}

\end{SCn}
\end{frame}


\begin{frame}{Интеллектуальные системы \\нейронных сетей}
	\topline
	\justifying
\small{
\begin{SCn}

	\scnheader{Обучение нейронной сети} 
	\begin{scnrelfromset}{разбиение}
			\scnitem{Контролируемое обучение}
			\begin{scnindent}
					\scnrelfrom{пояснение}{Этот метод подразумевает внешнего учителя и/или глобальную информацию, а также такие техники 											,как метод коррекции ошибки, обучение с подкреплением и стахостические методы обучения.}
					\begin{scnrelfromset}{пример}
						\scnitem{Перцептрон - Минский, Пейперт}
						\scnitem{ADALINE (Adaptive Linear Neuron)}
						\scnitem{MADALINE (Many ADALINE)}
						\scnitem{Обратное распространение ошибки}
						\scnitem{Машина Больцмана}
					\end{scnrelfromset}
			\end{scnindent}
	\end{scnrelfromset}

\end{SCn}
}
\end{frame}


\begin{frame}{Интеллектуальные системы \\нейронных сетей}
	\topline
	\justifying
\begin{SCn}

	\scnheader{Метод коррекции ошибки} 
	\scnrelfrom{пояснение}{Корректиркет матрицу весов прямопропорционально разнице между желаемым и вычисленным результатом.}

	\scnheader{Обучение с подкреплением} 
	\scnrelfrom{пояснение}{При правильном выполнении веса увеличиваются, при неправильном уменьшаются.}

	\scnheader{Стахостические методы} 
	\scnrelfrom{пояснение}{Случайно изменяется вес в матрице весов, затем определяются свойства сети.}

\end{SCn}
\end{frame}


\begin{frame}{Гибридные интеллектуальные системы}
	\topline
	\justifying
\begin{SCn}

\scnheader{Гибридные интеллектуальные системы}
\scntext{примечание} {Гибридные интеллектуальные системы ставят своей цель интеграцию различных парадигм таких как экспертные системы, нейронные сети, нечёткая логика и другие.}

\end{SCn}
\end{frame}




\begin{frame}{Гибридные интеллектуальные системы}
	\topline
	\justifying
\begin{SCn}

	\scnheader{Подходы построения гибридных интеллектуальных систем} 
	\begin{scnrelfromset}{разбиение}
			\scnitem{Иерархический подход}
			\scnitem{Система классной доски}
	\end{scnrelfromset}

\end{SCn}
\end{frame}




\begin{frame}{Гибридные интеллектуальные системы}
	\topline
	\justifying
\begin{SCn}

\scnheader{Иерархический подход}
\scntext{пояснение} {Иерархический подход к построению интеллектуальных систем включает в себя использование несольких уровней.  Высокоуровневые задачи разбиваются на подзадачи до тех пор, пока задача не станет выполнимой. Более низкие уровни передают информацию более высоким и наоборот, что позволяет системе наладить обратную связь между слоями, тем самым более низкие уровни могут влиять на то, какая информация им приходит. Такой подход позволяет построить прозрачную и хорошо описываемую структуру управления. Однако, если системам будет иметь слишком много слоёв, то станет сложно налаживать обратную связь, которая является главным преимуществом данного подхода.}

\end{SCn}
\end{frame}


\begin{frame}{Гибридные интеллектуальные системы}
	\topline
	\justifying
\begin{SCn}

\scnheader{Система классной доски}
\scntext{пояснение} {Система классной доски хранит информацию в о состоянии в глобальной базе данных. Таким образом, у системы существует база данных  с задачами, к которой обращаются агенты и отслеживают, каким образом они могут поучаствовать в решении задачи.}

\end{SCn}
\end{frame}
































































































