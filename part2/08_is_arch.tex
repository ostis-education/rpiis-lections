\title{Лекция 8\\Архитектура интеллектуальных систем}
\author[]{Шункевич Д.В.}
\institute[]{Белорусский государственный университет информатики и радиоэлектроники}

\begin{frame}
	\titlepage
\end{frame}


\begin{frame}{\\Содержание лекции}
	\topline
	\justifying
\begin{SCn}
	\scnheader{Лекция 8. Архитектура интеллектуальных систем}
	\begin{scnrelfromset}{структура}
				\scnitem{Интеллектуальная компьютерная система}
				\scnitem{Интеллектуальные системы, основанные на базах знаний}
				\scnitem{Интеллектуальные системы нейронных сетей}
				\scnitem{Гибридные интеллектуальные системы}
	\end{scnrelfromset}
				
\end{SCn}
\end{frame}

\begin{frame}
\centering
\Huge
\textbf{Интеллектуальная компьютерная система}
\end{frame}


\begin{frame}{Интеллектуальная компьютерная система}
	\topline
	\justifying

\par Искусственный интеллект -- наука и технология создания машин и программ, способных выполнять задачи, которые обычно присущи человеку. Искусственный интеллект применяется в разных сферах общественной жизни, таких как финансы, медицина, образование, военное дело и развлечения. Искусственный интеллект развивается с середины 20-го века и достиг значительных успехов в последние годы. Однако пока не существует системы, которая бы полностью заменила человека в решении задач, но методы и элементы ИИ активно внедряются в современное программное обеспечение.

\end{frame}


\begin{frame}{Интеллектуальная компьютерная система}
	\topline
	\justifying
\begin{SCn}

	\scnheader{интеллектуальная компьютерная система}
	\scntext{определение}{компьютеризированная система сбора, хранения, обработки, представления информации, работа которой основывается на имитации (воспроизведении) интеллектуальных возможностей человека.}
	
\end{SCn}
\end{frame}


\begin{frame}{Интеллектуальная компьютерная система}
	\topline
	\justifying
\begin{SCn}

	\scnheader{интеллектуальная система}

	\begin{scnrelfromset}{свойства}
		\scnfileitem{решение задач в условиях неопределенности}
		\scnfileitem{решение неформализованных  (трудноформализуемых) задач}
		\scnfileitem{эвристическое решение задач}
		\scnfileitem{обучаемость и приобретение опыта}
	\end{scnrelfromset}
	
\end{SCn}
\end{frame}


\begin{frame}{Интеллектуальная компьютерная система}
	\topline
	\justifying
\begin{SCn}

	\scnheader{традиционная информационная система}
	\begin{scnrelfromset}{преобладающие функции}
		\scnfileitem{работа с количественной информацией}
		\scnfileitem{математические вычисления}
		\scnfileitem{хранение информации}
		\scnfileitem{поиск информации}
		\scnfileitem{обмен информацией}
		\scnfileitem{визуализация данных}
	\end{scnrelfromset}
	
\end{SCn}
\end{frame}


\begin{frame}{Интеллектуальная компьютерная система}
	\topline
	\justifying
\begin{SCn}

	\scnheader{интеллектуальная система}
	\begin{scnrelfromset}{преобладающие функции}
		\scnfileitem{работа с качественной информацией}
		\scnfileitem{логический вывод}
		\scnfileitem{семантический поиск информации}
		\scnfileitem{контекстнозависимый поиск информации}
		\scnfileitem{интерпретация данных}
	\end{scnrelfromset}
	
\end{SCn}
\end{frame}


\begin{frame}{Интеллектуальная система}
	\topline
	\justifying
	
	Таким образом, принципиальные отличия традиционной информационной системы от интеллектуальной заключаются в том, что интеллектуальная информационная система занимается получением новых знаний из уже усвоенных или получает знания в ходе обработки данных, в то время как традиционная может проводить только количественную оценку информации.

\end{frame}

\begin{frame}
\centering
\Huge
\textbf{Интеллектуальные системы, основанные на базах знаний (в традиционном понимании)}
\end{frame}


\begin{frame}{Интеллектуальные системы,\\ основанные на базах знаний}
	\topline
	\justifying
\begin{SCn}

	\scnheader{интеллектуальная система, основанная на базах знаний}
	\begin{scnrelfromset}{характерные особенности}
		\scnfileitem{выразительность}
		\scnfileitem{интуитивность}
		\scnfileitem{модульность}
		\scnfileitem{модифицируемость}
		\scnfileitem{наукоемкость}
	\end{scnrelfromset}
	
\end{SCn}
\end{frame}


\begin{frame}{Интеллектуальные системы,\\ основанные на базах знаний}
	\topline
	\justifying
\begin{SCn}
	
	\scnheader{интеллектуальная система, основанная на базах знаний}
	\scntext{преимущество}{база знаний очень эффективно отделена от механизма вывода, таким образом позволяя механизму вывода быть универсальным}
	
\end{SCn}
\end{frame}


\begin{frame}{Интеллектуальные системы,\\ основанные на базах знаний}
	\topline
	\justifying
\begin{SCn}
	
	\scnheader{интеллектуальная система, основанная на базах знаний}
	\scntext{недостаток}{Для продукционной модели: из-за того, что база знаний не имеет иерархии, а также из-за того, что она постоянно пополняется новыми знаниями и правилами, система становится непрозрачной и становится медленной в работе}
	
\end{SCn}
\end{frame}



\begin{frame}{Интеллектуальные системы,\\ основанные на базах знаний}
	\topline
	\justifying

	\scnheader{инженерия знаний}
	\scnidtf{задача получения экспертных знаний и передача их интеллектулаьной системе}
	
	\scntext{пояснение}{В определенный момент развития таких систем появилась необходимость в экспертном знании, т.к. от системы требуется не только принимать решения, но и обосновывать их}

\end{frame}


\begin{frame}{Интеллектуальные системы,\\ основанные на базах знаний}
	\topline
	\justifying
\begin{SCn}
	
	\scnheader{экспертная система}
	\begin{scnrelfromset}{характеристики}
		\scnfileitem{система работает на уровне, который обычно считается равным уровню эксперта в данной области}
		\scnfileitem{система сильно зависит от предметной области из-за того, что система знает об узкой предметной области}
		\scnfileitem{система должна объяснять свои рассуждения, т.к. ценность работы системы заключается не только в выводе, но и в анализе}
		\scnfileitem{если информация является вероятностной, то система должна правильно предоставлять ряд альтернативных решений с соответствующими вероятностями}
	\end{scnrelfromset}
	
\end{SCn}
\end{frame}


\begin{frame}{Интеллектуальные системы,\\ основанные на базах знаний}
	\topline
	\justifying
\begin{SCn}	

	\scnheader{интеллектуальная система}
	\begin{scnrelfromset}{стратегия обучения}
		\scnfileitem{механическое обучение (без понимания смысла)}
		\scnfileitem{обучение по инструкции}
		\scnfileitem{обучение путем дедукции}
		\scnfileitem{обучение по аналогии}
		\scnfileitem{обучение путем индукции}
	\end{scnrelfromset}
	\begin{scnindent}
		\scntext{примечание}{Каждая стратегия должна использоваться для решения подходящих для нее задач}
	\end{scnindent}

\end{SCn}
\end{frame}


\begin{frame}{Интеллектуальные системы,\\ основанные на базах знаний}
	\topline
	\justifying
\begin{SCn}
	
	\scnheader{механическое обучение}
	\scntext{определение}{\textbf{\textit{Механическое обучение}} -- это базовый способ обучения как у человека, так и у машины. В машине это соответствует знанию напрямую записанному в память системы или базы знаний.}

\end{SCn}
\end{frame}



\begin{frame}{Интеллектуальные системы,\\ основанные на базах знаний}
	\topline
	\justifying
\begin{SCn}
	
	\scnheader{обучение по инструкции}
	\scntext{определение}{\textbf{\textit{Обучение по инструкции}} -- это обучение путем приобретения знаний от учителя или учебника и преобразуются системой во внутреннее представление, которое будет применимо для решения задач. Ответственность за структурирование и представление знаний, остается за учителем, но учащийся должен сделать вывод, чтобы преобразовать знание к виду, пригодному для использования. Роль учащегося можно рассматривать как выполнение синтаксической переформулировки знаний, предоставленных первоисточником.}

\end{SCn}
\end{frame}


\begin{frame}{Интеллектуальные системы,\\ основанные на базах знаний}
	\topline
	\justifying
\begin{SCn}

	\scnheader{обучение путем дедукции}
	\scntext{определение}{\textbf{\textit{Обучение путем дедукции}} перекладывает ответственность за преобразование знаний в удобную для восприятия форму с учителя на ученика. Ограничения на представления знаний источника ослаблены. Учащийся самостоятельно делает дедуктивные выводы из знаний и переформулирует их в полезные выводы, сохраняя информативность. Дедуктивное обучение подразумевает переформулирование входящих знаний и их разбиение на фрагменты, сохраняя истинность исходного знания.}

\end{SCn}
\end{frame}


\begin{frame}{Интеллектуальные системы,\\ основанные на базах знаний}
	\topline
	\justifying
\begin{SCn}

	\scnheader{обучение по аналогии}
	\scntext{определение}{\textbf{\textit{Обучение по аналогии}} сочетает в себе дедуктивное и индуктивное обучение. Первым шагом является индуктивный вывод, необходимый для нахождения общей подструктуры между задачей и схожими предметными областями, хранящимися в базе знаний обучающегося. Затем, идет сопоставление решения из выбранной аналогичной области с проблемной областью с использованием дедуктивной логики.}

\end{SCn}
\end{frame}


\begin{frame}{Интеллектуальные системы,\\ основанные на базах знаний}
	\topline
	\justifying
\begin{SCn}

	\scnheader{обучение путем индукции}
	\scntext{определение}{\textbf{\textit{Обучение путем индукции}} -- это классификация опыта в понятия и категории. Понятия даются учителем или возникают по аналогии, в ходе решения задачи. Если было несколько случаев, когда конкретное действие было подходящим в ответ на ситуацию, то делается вывод, который обощает это действие. Обучение путем индукции ставит своей целью обобщение уже усвоенных понятий, в котором вывод происходит из набора примеров предоставленных окружающей средой.}

\end{SCn}
\end{frame}


\begin{frame}{Интеллектуальные системы,\\ основанные на базах знаний}
	\topline
	\justifying

	\par Существует набор проблем, связанных с тем, что обучающий или окружающая среда могут передавать знания с некоторой погрешностью. 	Также с течением времени системе необходимо обновлять знания, при этом возникают сложности с совместимостью.


\end{frame}

\begin{frame}
\centering
\Huge
\textbf{Интеллектуальные системы на базе нейронных сетей}
\end{frame}



\begin{frame}{Интеллектуальные системы \\на базе нейронных сетей}
	\topline
	\justifying
\begin{SCn}


	\par В 1985 году, в статье "The Challenge of Open Systems"{} Карл Хьюитт постулировал, что системы, которые мы хотим смоделировать, являются открытыми. Открытые системы постоянно коммуницируют и  ограничены извне.

\end{SCn}
\end{frame}


\begin{frame}{Интеллектуальные системы \\на базе нейронных сетей}
	\topline
	\justifying
\begin{SCn}

	\scnheader{открытые системы}
	\begin{scnrelfromset}{характерные особенности}
		\scnfileitem{постоянное изменение и эволюция}
		\scnfileitem{децентрализованное принятие решений}
		\scnfileitem{необходимость согласованности компонентов системы}
	\end{scnrelfromset}

\end{SCn}
\end{frame}



\begin{frame}{Интеллектуальные системы \\на базе нейронных сетей}
	\topline
	\justifying
\begin{SCn}

	\scnheader{нейронная сеть} 
	\scntext{определение}{\textbf{\textit{Нейронная сеть}} -- это средство для решения задач, организованное, по всей видимости, на подобии работы мозга человека. В модели нейронной сети существует три основных аспекта: нейрон, топология сети и стратегия обучения.}

\end{SCn}
\end{frame}


\begin{frame}{Интеллектуальные системы \\на базе нейронных сетей}
	\topline
	\justifying
	
	\par Нейроны соединены между собой, и поведение всей системы определяется структурой и силой этих связей. Обрабатывающие элементы (нейроны) объединены в группы или слои.

\end{frame}


\begin{frame}{Интеллектуальные системы \\на базе нейронных сетей}
	\topline
	\justifying
\begin{SCn}

	\scnheader{обучение нейронной сети} 
	\begin{scnrelfromset}{разбиение}
			\scnitem{самоорганизованное обучение}
			\begin{scnindent}
				\scntext{пояснение}{Этот процесс не подразумевает внешнего учителя, и система полагается только на локальную информацию и внутренние стратегии контроля}
				\begin{scnrelfromset}{примеры}
					\scnitem{"Адаптивная резонансная теория"{} С. Гроссберг, Г. Карпетер}
					\scnitem{"Сети Хопфилда"{} Дж. Хопфилд}
					\scnitem{"Двунаправленная ассоциативная память"{} Б. Коско}
				\end{scnrelfromset}
			\end{scnindent}
	\end{scnrelfromset}

\end{SCn}
\end{frame}


\begin{frame}{Интеллектуальные системы \\на базе нейронных сетей}
	\topline
	\justifying
\small{
\begin{SCn}

	\scnheader{обучение нейронной сети} 
	\begin{scnrelfromset}{разбиение}
			\scnitem{контролируемое обучение}
			\begin{scnindent}
					\scntext{пояснение}{Этот метод подразумевает внешнего учителя и/или глобальную информацию, а также такие техники, как метод коррекции ошибки, обучение с подкреплением и стахостические методы обучения}
					\begin{scnrelfromset}{пример}
						\scnitem{Перцептрон -- Минский, Пейперт}
						\scnitem{ADALINE (Adaptive Linear Neuron)}
						\scnitem{MADALINE (Many ADALINE)}
						\scnitem{Обратное распространение ошибки}
						\scnitem{Машина Больцмана}
					\end{scnrelfromset}
			\end{scnindent}
	\end{scnrelfromset}

\end{SCn}
}
\end{frame}


\begin{frame}{Интеллектуальные системы \\на базе нейронных сетей}
	\topline
	\justifying
\begin{SCn}

	\scnheader{метод коррекции ошибки} 
	\scntext{пояснение}{Корректиркет матрицу весов прямопропорционально разнице между желаемым и вычисленным результатом}

	\scnheader{обучение с подкреплением} 
	\scntext{пояснение}{При правильном выполнении веса увеличиваются, при неправильном уменьшаются}

	\scnheader{стохастические методы} 
	\scntext{пояснение}{Случайно изменяется вес в матрице весов, затем определяются свойства сети}

\end{SCn}
\end{frame}

\begin{frame}
\centering
\Huge
\textbf{Гибридные интеллектуальные системы}
\end{frame}


\begin{frame}{Гибридные интеллектуальные системы}
	\topline
	\justifying
\begin{SCn}

\scnheader{гибридные интеллектуальные системы}
\scntext{примечание}{Гибридные интеллектуальные системы ставят своей целью интеграцию различных парадигм таких как экспертные системы, нейронные сети, нечеткая логика и другие}

\end{SCn}
\end{frame}

\begin{frame}{Гибридные интеллектуальные системы}
	\topline
	\justifying
\begin{SCn}

	\scnheader{гибридные интеллектуальные системы} 
	\begin{scnrelfromset}{подходы к построению}
		\scnitem{иерархический подход}
		\scnitem{система классной доски}
	\end{scnrelfromset}

\end{SCn}
\end{frame}

\begin{frame}{Гибридные интеллектуальные системы}
	\topline
	\justifying
\begin{SCn}

\scnheader{иерархический подход}
\scntext{пояснение} {Иерархический подход к построению интеллектуальных систем включает в себя использование несольких уровней.  Высокоуровневые задачи разбиваются на подзадачи до тех пор, пока задача не станет выполнимой. Более низкие уровни передают информацию более высоким и наоборот, что позволяет системе наладить обратную связь между слоями, тем самым более низкие уровни могут влиять на то, какая информация им приходит. Такой подход позволяет построить прозрачную и хорошо описываемую структуру управления. Однако, если системам будет иметь слишком много слоев, то станет сложно налаживать обратную связь, которая является главным преимуществом данного подхода.}

\end{SCn}
\end{frame}


\begin{frame}{Гибридные интеллектуальные системы}
	\topline
	\justifying
\begin{SCn}

\scnheader{система классной доски}
\scntext{пояснение} {Система классной доски хранит информацию в о состоянии в глобальной базе данных. Таким образом, у системы существует база данных  с задачами, к которой обращаются агенты и отслеживают, каким образом они могут поучаствовать в решении задачи.}

\end{SCn}
\end{frame}
