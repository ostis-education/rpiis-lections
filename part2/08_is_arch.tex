\title{Лекция 8\\Архитектура интеллектуальных систем}

\begin{frame}
	\titlepage
\end{frame}


\begin{frame}{\\Содержание лекции}
	\topline
	\justifying
	
	Интеллектуальная компьютерная система. Принципиальные отличия интеллектуальной компьютерной системы от традиционной. Архитектура интеллектуальной системы. База знаний. Машина обработки знаний.
	
\end{frame}


\begin{frame}{\\Подходы к определению понятия ИИ}
	\topline
	\justifying
	
	\LARGE
	
	\textbf{Artificial intelligence} -- the science and engineering of making intelligent machines [McCarthy, 1956]
	
	\vspace{0.5em}
	Intelligence -- не <<интеллект>> (intellect), а <<способность рассуждать разумно>>.
	
\end{frame}

\begin{frame}{\\Подходы к определению понятия ИИ}
	\topline
	\justifying

	\vspace{1em}
	
	\Large
	
	Свойство \textbf{интеллектуальных систем} выполнять функции (творческие), которые традиционно считаются прерогативой человека [Толковый словарь по ИИ, 1992]. 
	
	\vspace{0.5em}
	Научное направление, в рамках которого ставятся и решаются задачи аппаратного или программного моделирования тех видов человеческой деятельности, которые традиционно считаются \textbf{интеллектуальными} [Толковый словарь по ИИ, 1992]. 
	
\end{frame}

\begin{frame}{\\Подходы к определению понятия ИИ}
	\topline
	\justifying
	
	\textit{Искусственный интеллект} -- комплекс технологических решений, включающих информационно-коммуникационную инфраструктуру, программное обеспечение (в том числе такое, в котором используются методы машинного обучения), процессы и сервисы обработки данных и поиска решений, позволяющих имитировать когнитивные функции человека (включая поиск решения без заранее заданного алгоритма) и получать при выполнении конкретных задач результаты, сопоставимые с результатами интеллектуальной деятельности человека или превосходящие их.
	
	\vspace{0.5em}
	[Модельный закон СНГ "О технологиях искусственного интеллекта"{},\\ 18.05.2025]
	
\end{frame}

\begin{frame}{\\Интеллектуальные системы}
	\topline
	\justifying
	
	\begin{textitemize}
	\item Система, способная решать \textbf{интеллектуальные задачи}
		
		\item Система, способная \textbf{обучаться}
		\begin{textitemize}
			\item Система, способная \textbf{обучаться хорошо решать задачи одного класса}
			\item Система, способная \textbf{обучаться навыкам решения задач новых классов}
		\end{textitemize}
		
		\item Система, способная \textbf{неограниченно обучаться}
		\begin{textitemize}
			\item Нет ограничений на вид приобретаемых ею знаний и навыков
		\end{textitemize}
	\end{textitemize}
	
	\vspace{10mm}
	\centering
	\textbf{Лучший пример интеллектуальной системы -- мы с вами}
	
\end{frame}

\begin{frame}{\\Что такое ИИ?}
	\topline
	\justifying
	
	\vspace{10mm}
	
	Искусственный интеллект -- естественный закономерный этап развития информационных технологий, «передний край» научных и технических достижений в области IT.
	
	\begin{textitemize}
		\item Нельзя противопоставлять «традиционные» информационные технологии и Искусственный интеллект
		\item Не стоит называть Искусственным интеллектом всё подряд
	\end{textitemize}	

	\vspace{1em}
	\hrule
	\vspace{1em}
	
	\begin{textitemize}
		\item В широком смысле -- научно-техническое направление
		\item В узком смысле -- конкретная система или модуль, обладающая свойствами интеллектуальных систем
	\end{textitemize}
\end{frame}

\begin{frame}{\\Типы искусственного интеллекта}
	\topline
	\justifying
	
	\vspace{10mm}
	
	\begin{figure}[h]
		\centering
		%\setlength{\tabcolsep}{3em} % можно тоже чуть меньше
		\begin{tabular}{m{0.3\textwidth} m{0.3\textwidth} m{0.3\textwidth}}
			
			% --- Колонка 1 ---
			\centering
			\includegraphics[width=0.8\linewidth]{part2/images/is_arch/ai1.png}\\[0.4em]
			{\Large\bfseries Слабый ИИ}\\[0.4em]
			\scriptsize
			\centering
			Ограниченные возможности к обучению и не способен к адаптации
			\normalsize
			&
			% --- Колонка 2 ---
			\centering
			\includegraphics[width=0.8\linewidth]{part2/images/is_arch/ai2.png}\\[0.4em]
			{\Large\bfseries Общий ИИ (AGI)}\\[0.4em]
			\scriptsize
			\centering
			Способен обучаться новым навыкам и адаптироваться к изменяющимся условиям
			\normalsize
			&
			% --- Колонка 3 ---
			\centering
			\includegraphics[width=0.8\linewidth]{part2/images/is_arch/ai3.png}\\[0.4em]
			{\Large\bfseries Сильный ИИ (ASI)}\\[0.4em]
			\scriptsize
			\centering
			Способен не просто выполнять любые задачи, но и осознавать свои действия и решения.
			
		\end{tabular}
	\end{figure}
\end{frame}

\begin{frame}
	\centering
	\Huge
	\textbf{Кибернетические системы}
\end{frame}

\begin{frame}{\\Понятие кибернетической системы}
	\topline
	\justifying
	
	\vspace{5mm}
	
	\small
	
	\begin{SCn}
	\scnheader{кибернетическая система} 
	\scnidtf{динамическая система, способная совершать некоторые действия и осуществлять некоторую деятельность}
	\scnidtftext{пояснение}{\uline{открытая} динамическая система, осуществляющая 
		
	\begin{textitemize}
		\item \textbf{мониторинг и анализ} состояния окружающей её среды и процессов, происходящих в этой среде
		\item осуществляющая \textbf{воздействия} на эту окружающую среду (в том числе и воздействия на свою физическую оболочку), обусловленные
		\begin{textitemize}
			\item \textbf{назначением} (специализацией) рассматриваемой динамической системы (чаще всего это касается искусственных кибернетических систем)
			\item стремлением обеспечить \textbf{самосохранение}, целостность, то есть обеспечить выполнение гомеостатической деятельности, что является важным направлением жизнедеятельности \uline{всех} кибернетических систем, а для естественных кибернетических систем является ключевым стимулом их эволюции
		\end{textitemize}
	\end{textitemize}}
	\end{SCn}
	
\end{frame}

\begin{frame}{\\Классификация кибернетических систем}
	\topline
	\justifying
	
	\vspace{5mm}
	
	\begin{SCn}
		\scnheader{кибернетическая система}
		\begin{scnrelfromset}{разбиение}
			\scnitem{\textbf{индивидуальная кибернетическая система}}
			\begin{scnindent}
				\scnsuperset{биологический организм}
			\end{scnindent}
			
			\scnitem{\textbf{многоагентная кибернетическая система}}
			\begin{scnindent}
				\scnidtf{распределённая кибернетическая система}
				\scnidtf{\textit{кибернетическая система}, у которой её \textit{память}, \textit{процессор}, \textit{интерфейс} и хранимая в памяти \textit{внутренняя информационная модель окружающей среды} имеют распределенный (в известной степени виртуальный) характер}
				\scnidtf{\textit{кибернетическая система}, представляющая собой совокупность \textit{кибернетических систем}, являющихся агентами \textit{многоагентной системы}, взаимодействующих между собой с помощью своих интерфейсных средств и, возможно, через специальную коммуникационную среду.}
			\end{scnindent}
		\end{scnrelfromset}
	\end{SCn}
	
\end{frame}

\begin{frame}{\\Классификация кибернетических систем}
	\topline
	\justifying
	
	\vspace{5mm}
	
	\begin{SCn}
	\scnheader{кибернетическая система}
	\begin{scnrelfromset}{разбиение}
		\scnitem{естественная кибернетическая система}
		\begin{scnindent}
			\scnidtf{кибернетическая система естественного (биологического) происхождения}
		\end{scnindent}
		\scnitem{компьютерная система}
		\begin{scnindent}
			\scnidtf{\textit{искусственная кибернетическая система}}
			\begin{scnrelfromset}{разбиение}
				\scnitem{индивидуальная компьютерная система}
				\scnitem{многоагентная компьютерная система}
			\end{scnrelfromset}
		\end{scnindent}
		\scnitem{естественно-искусственная кибернетическая система}
		\begin{scnindent}
			\scnidtf{кибернетическая система, содержащая компоненты как естественного, так и искусственного происхождения}
			\scnsuperset{\textbf{человеко-машинная кибернетическая система}}
		\end{scnindent}
	\end{scnrelfromset}
	\end{SCn}
	
\end{frame}

\begin{frame}{\\Человеко-машинные кибернетические системы}
	\topline
	\justifying
	
	\begin{SCn}
	\scnheader{человеко-машинная кибернетическая система}
	\begin{scnrelfromset}{разбиение}
		\scnitem{человеко-машинная индивидуальная кибернетическая система}
		\begin{scnindent}
			\scnidtf{система, состоящая из механически (вручную) управляемой \textit{машины} (инструмента) и пользователя, управляющего этой машиной}
			\scntext{примечание}{Указанная здесь управляемая \textit{машина} не является \textit{кибернетической системой}}
		\end{scnindent}
		
		\scnitem{человеко-машинная многоагентная кибернетическая система}
	\end{scnrelfromset}
	\end{SCn}
	
\end{frame}

\begin{frame}{\\Человеко-машинные кибернетические системы}
	\topline
	\justifying
	
	\vspace{10mm}
	
	\scriptsize
	
	\begin{SCn}
	\scnheader{человеко-машинная кибернетическая система}
	\begin{scnrelfromset}{разбиение}
		\scnitem{человеко-машинная индивидуальная кибернетическая система}
		\scnitem{человеко-машинная многоагентная кибернетическая система}
		\begin{scnindent}
			\begin{scnrelfromset}{разбиение}
				\scnitem{человеко-машинная двух-агентная кибернетическая система}
				\begin{scnindent}
					\scnidtf{двухагентная кибернетическая система, состоящая из \textit{индивидуальной компьютерной системы} и человека (пользователя), взаимодействующего с этой системой}
				\end{scnindent}
				\scnitem{человеко-машинная многоагентная кибернетическая система, имеющая более двух агентов}
				\begin{scnindent}
					\scntext{примечание}{Любое число взаимодействующих друг с другом людей и компьютерных систем. \vspace{-\baselineskip}
						\begin{textitemize}
							\item индивидуальная компьютерная система и много пользователей;
							\item многоагентная компьютерная система и один пользователь;
							\item многоагентная компьютерная система и много пользователей;
					\end{textitemize}}
				\end{scnindent}		
			\end{scnrelfromset}
		\end{scnindent}	
	\end{scnrelfromset}
	\end{SCn}
	
\end{frame}

\begin{frame}{\\Многоагентная кибернетическая система}
	\topline
	\justifying
	
	\small
	
	\vspace{10mm}
	
	\begin{SCn}
	\scnheader{многоагентная кибернетическая система}
	\begin{scnrelfromset}{разбиение}
		\scnitem{популяция}
		\begin{scnindent}
			\scnidtf{многоагентная система, в рамках которой осуществляестся самовоспроизводство новых агентов с передачей им накопленных популяцией знаний и опыта [Дж. фон Нейман, 1966]}
		\end{scnindent}
		\scnitem{многоагентная система, не являющаяся популяцией}
	\end{scnrelfromset}
	\begin{scnrelfromset}{разбиение}
		\scnitem{коллектив индивидуальных кибернетических систем}
		\scnitem{иерархическая многоагентная кибернетическая система}
	\end{scnrelfromset}
	\begin{scnrelfromset}{разбиение}
		\scnitem{многоагентная кибернетическая система с фиксированным числом агентов}
		\scnitem{многоагентная кибернетическая система с нефиксированным числом агентов}
	\end{scnrelfromset}
	\end{SCn}
	
\end{frame}

\begin{frame}{\\Структура кибернетических систем}
	\topline
	\justifying
	
	\begin{SCn}
		\scnheader{кибернетическая система}
		\begin{scnrelfromlist}{обобщенная часть}
			\scnitem{физическая оболочка кибернетической системы}
			\scnitem{внутренняя информационная модель окружающей среды}
		\end{scnrelfromlist}
	\end{SCn}
	
\end{frame}

\begin{frame}{\\Структура кибернетических систем}
	\topline
	\justifying
	
	\begin{SCn}
	\scnheader{окружающая среда}
	\begin{scnrelfromset}{обобщенная декомпозиция}
		\scnitem{внешняя среда}
		\scnitem{собственное Я}
	\end{scnrelfromset}
	\scnidtf{и внешняя среда кибернетической системы, и сама эта кибернетическая система, включая все её компоненты, в том числе и внутреннюю информационную модель окружающей среды}
	\end{SCn}
	
\end{frame}

\begin{frame}{\\Структура кибернетических систем}
	\topline
	\justifying
	
	\small
	
	\begin{SCn}
	\scnheader{собственное Я}
		\begin{scnrelfromset}{обобщенная декомпозиция}
			\scnitem{физическая оболочка кибернетической системы}
			\begin{scnindent}
				\begin{scnrelfromset}{обобщенная декомпозиция}
					\scnitem{комплекс сенсоров и эффекторов кибернетической системы}
					\scnitem{память кибернетической системы}
					\scnitem{процессор кибернетической системы}
					\scnitem{корпус кибернетической системы}
				\end{scnrelfromset}
			\end{scnindent}
			\scnitem{внутренняя информационная модель окружающей среды}  
		\end{scnrelfromset}
		\begin{scnindent}
			\scntext{примечание}{Все эти компоненты кибернетической системы могут быть как локализованными (локальными), так и \uline{распределёнными} (виртуальными) в зависимости от структурного типа кибернетической системы.}
		\end{scnindent}
	\end{SCn}
	
\end{frame}

\begin{frame}{\\Подсистема обработки информации}
	\topline
	\justifying
	
	\begin{SCn}
		\scnheader{встроенная подсистема обработки информации}
		\scntext{пояснение}{\textit{Встроенная подсистема обработки информации}, строго говоря, не является \textit{кибернетической системой}. Тем не менее, её можно рассматривать как аналог \textit{кибернетической системы}, а именно, как \textit{кибернетическую систему}, внешней средой которой является \textit{память} соответствующей \textit{индивидуальной кибернетической системы} и информация, хранимая в этой памяти.}
		\scnsuperset{процессоро-память индивидуальной кибернетической системы}
	\end{SCn}
	
\end{frame}

\begin{frame}{\\Процессор кибернетической системы}
	\topline
	\justifying
	
	\small
	\vspace{5mm}
	
	\begin{SCn}
		\scnheader{процессор кибернетической системы}
		\scnidtf{машина обработки базы знаний}
		\scnidtf{совокупность функциональных средств соответствующей кибернетической системы, обладающая достаточной полнотой (целостностью) для функционального обеспечения деятельности указанной кибернетической системы (для интерпретации информации, хранимой в памяти кибернетической системы)}
		\scnidtf{внутренняя псевдокибернетическая система обработки информации, средой деятельности которой является память соответствующей кибернетической системы}
		\scnidtf{псевдокиберсистема, встроенная в соответствующую кибернетическую систему и осуществляющая обработку информации, хранимой в памяти указанной кибернетической системы с помощью процессора этой системы (с помощью средств анализа хранимых информационных конструкций и средств их преобразования)}
	\end{SCn}
	
\end{frame}

\begin{frame}{Внутренняя информационная\\ модель окружающей среды}
	\topline
	\justifying
	
	\vspace{5mm}
	\small
	
	\begin{SCn}
		\scnheader{внутренняя информационная модель окружающей среды}
		\scnidtf{часть состояния \textit{памяти кибернетической системы}, которая используется \textit{процессором} и \textit{сенсорно-эффекторным комплексом} для организации \textit{деятельности} (поведения, функционирования) \textit{кибернетической системы} в процессе её взаимодействия со своей \textit{внешней средой}, со своей \textit{физической оболочкой} и со своей внутренней информационной средой (то есть \textit{внутренней информационной моделью окружающей среды})}
		\scnidtf{субъективная картина мира кибернетической системы}
		\scnsuperset{база знаний}
		\begin{scnindent}
			\scnidtf{семантически структурированная внутренняя информационная модель окружающей среды интеллектуальной кибернетической системы}
		\end{scnindent}
		\scntext{примечание}{Наличие у кибернетической системы внутренней информационной модели окружающей среды означает то, то кибернетическая система "живёт"{} одновременно в двух мирах --- во внешнем реальном мире и во внутреннем мире своей информационной модели (отражения) этого внешнего реального мира.}
	\end{SCn}
	
\end{frame}

\begin{frame}{\\Память <-> Информация}
	\topline
	\justifying
	
	\begin{SCn}
	\scnheader{следует отличать*}
	\begin{scnhaselementset}
		\scnitem{память}
		\begin{scnindent}
			\scnidtf{память кибернетической системы}
		\end{scnindent}
		\scnitem{объединённая совокупная информация, хранимая в памяти кибернетической системы}
		\begin{scnindent}
			\scnidtf{динамическая информационная модель окружающей среды соответствующей кибернетической системы, описывающая (отражающая) эту среду с необходимой степенью детализации}
			\scnidtf{вся информация, хранимая в памяти кибернетической системы}
			\scnsuperset{база знаний}
		\end{scnindent}
	\end{scnhaselementset}
	\end{SCn}

\end{frame}

\begin{frame}{Физическая оболочка\\ кибернетической системы}
	\topline
	\justifying
	
	\begin{SCn}
		\scnheader{физическая оболочка кибернетической системы}
		\scnidtf{материальная оболочка кибернетической системы}
		\scnidtf{тело (корпус) кибернетической системы}
		\begin{scnrelfromset}{обобщенная декомпозиция}
			\scnitem{память кибернетической системы}
			\scnitem{процессор кибернетической системы}
			\scnitem{комплекс сенсоров и эффекторов кибернетической системы}
			\scnitem{прочие материальные подсистемы, обеспечивающие обмен веществ и энергии с внешней средой кибернетической системы}
		\end{scnrelfromset}
		\scntext{примечание}{Физическая оболочка кибернетической системы подвергается постоянному разрушительному воздействию внешней среды --- этому необходимо противодействовать.}
	\end{SCn}
	
\end{frame}

\begin{frame}{\\Интерфейс кибернетической системы}
	\topline
	\justifying
	
	\vspace{5mm}
	
	\begin{SCn}
		\scnheader{интерфейс кибернетической системы}
		\scnrelfrom{обобщенная часть}{сенсорно-эффекторный комплекс кибернетической системы}
		\begin{scnindent}    
			\begin{scnrelfromset}{обобщенная декомпозиция}
				\scnitem{сенсорная подсистема кибернетической системы}
				\begin{scnindent}
					\scnrelfrom{обобщенная часть}{сенсор}
					\begin{scnindent}
						\scnidtf{рецептор}
						\scnidtf{средство анализа физического состояния своей внешней среды и физической оболочки}	
					\end{scnindent}
				\end{scnindent}
				\scnitem{эффекторная подсистема кибернетической системы}
				\begin{scnindent}
					\scnrelfrom{обобщенная часть}{эффектор}
					\begin{scnindent}
						\scnidtf{средство \uline{воздействия} на свою внешнюю среду и физическую оболочку}	
					\end{scnindent}
				\end{scnindent}
			\end{scnrelfromset}
		\end{scnindent}
	\end{SCn}
	
\end{frame}

\begin{frame}{\\Собственное Я}
	\topline
	\justifying
	
	\begin{SCn}
	\scnheader{собственное Я}
	\scnidtf{указатель на знак самой себя}
	\scnidtf{указатель на знак той кибернетической системы, каковой является она сама}
	\scntext{примечание}{В составе внутренней информационной модели окружающей среды кибернетическая система может хранить описание достаточно большого числа кибернетических систем, с которыми она взаимодействует (в частности, описание своих пользователей). Но из всего множества описываемых кибернетических систем каждая кибернетическая система должна выделить описание самой себя, что необходимо, как минимум, для осознания (осмысления) самой себя и своей деятельности в окружающей среде.}
	\end{SCn}
	
\end{frame}

\begin{frame}{Операции, заданные\\ на кибернетических системах}
	\topline
	\justifying
	
	\begin{SCn}
		\scnheader{кибернетическая система}
		\begin{scnrelfromlist}{заданная операция}
			\scnitem{конвергенция кибернетических систем*}
			\begin{scnindent}
				\scnsuperset{конвергенция внутренних информационных моделей окружающей среды кибернетических систем*}
				\begin{scnindent}
					\scnidtf{сближение у кибернетических систем субъективных картин окружающего мира*}
					\scnsuperset{обеспечение семантической совместимости*}
				\end{scnindent}        
			\end{scnindent}
			\scnitem{слияние индивидуальных кибернетических систем*}
			\scnitem{разделение индивидуальной кибернетической системы*}
			\scnitem{объединение кибернетических систем в коллектив*}
		\end{scnrelfromlist}
	\end{SCn}
	
\end{frame}

\begin{frame}{\\Свойства кибернетических систем}
	\topline
	\justifying
	
	\begin{textitemize}
		\item Наличие внутренней информационной модели окружающей среды (субъективной картины мира). Основой функционирования (поведения) кибернетических систем является использование внутренней картины мира (то есть обработка информации)
		\item эволюционируемость (неосознаваемая, осуществляемая извне)
		\begin{textitemize}
			\item гибкость
			\item стратифицированность
		\end{textitemize}
		\item самоэволюционируемость
		\begin{textitemize}
			\item рефлексия
			\item расширение многообразия (специализации) различных компонентов при повышении уровня их синергии
		\end{textitemize}
	\end{textitemize}
	
\end{frame}

\begin{frame}{\\Динамика мира}
	\topline
	\justifying
	
	Интеллектуальная система живёт одновременно в нескольких мирах:
	\begin{textitemize}
		\item В реальном внешнем мире (в простейшем случае внешним миром являются её пользователи -- конечные пользователи и разработчики разного статуса)
		\item Во внутреннем мире (мире ситуаций и событий, происходящих в её памяти, хранящей внутреннюю информационную модель некоторого фрагмента внешнего мира и обрабатываемой агентами) 
	\end{textitemize}
	
	При этом как внешний, так и внутренний мир можно декомпозировать на динамические предметные области (на несколько частных миров). При этом в каждом из этих миров система одновременно живёт в настоящем (текущем), в прошлом и будущем времени.
	
\end{frame}

\begin{frame}{\\Эволюция кибернетических систем}
	\topline
	\justifying
	\vspace{5mm}
	
	Ключевым свойством кибернетических систем является их способность \textbf{эволюционировать} (совершенствоваться), и, в том числе, эволюционировать самостоятельно (то есть самоэволюционировать). 
	
	Эта способность обусловлена наличием у кибернетической системы внутренней информационной модели окружающей её среды (внутренней субъективной картины окружающего её мира).
	
	Принципиальное достоинство этой информационной модели заключается в том, что её преобразование (осуществляемое с помощью процессора кибернетической системы) имеет значительно более низкую трудоёмкость по сравнению с трудоёмкостью преобразования среды, описываемой этой информационной моделью.
	
	Высокая скорость эволюции кибернетических систем обеспечивается гибкостью внутренней информационной модели окружающей среды и, как следствие, простотой модификации (преобразования) этой модели. 	
\end{frame}

\begin{frame}{\\Эволюция кибернетических систем}
	\topline
	\justifying
	
	Саму окружающую среду преобразовывать тоже можно, но \uline{преобразовывать} её информационную модель значительно \uline{проще} и быстрее, что позволяет:
	\begin{textitemize}
		\item прогнозировать динамику (изменения) этой среды, причиной которой не является собственная деятельность;
		\item планировать её преобразование своими эффекторами;
		\item моделировать (предусматривать) последствия своих действий во внешней среде.
	\end{textitemize}
	
\end{frame}

\begin{frame}
	\centering
	\Huge
	\textbf{Система параметров, определяющих общий уровень интеллекта (уровень самоорганизации) кибернетической системы}
\end{frame}

\begin{frame}{\\Интеллект}
	\topline
	\justifying
	
	\small
	\vspace{10mm}
	
	\begin{SCn}
		
	\scnheader{интеллект\scnsupergroupsign}
	\scnidtf{общий эволюционный уровень кибернетической системы\scnsupergroupsign}
	\scntext{примечание}{Принципиально важно подчеркнуть то, что общий уровень \textit{интеллекта} (уровень самоорганизации) \textit{кибернетической системы} определяется не только и не столько тем, какие возможности она имеет в текущий момент, а тем, насколько быстро и благодаря чему она \uline{эволюционирует}. Другими словами, основным свойством \textit{кибернетической системы} является уровень развития её \textbf{\textit{способности эволюционировать\scnsupergroupsign}}, модернизируя, трансформируя себя (иногда с помощью других субъектов --- учителей, разработчиков) в самых разных направлениях и желательно как можно быстрее.}
	\scntext{примечание}{\textit{кибернетическая система} характеризуется не только общей комплексной оценкой её текущего состояния, но также оценкой \uline{скорости} (темпа) повышения (улучшения) качественного уровня этого состояния, а также оценкой имеющегося потенциала (возможностей, способностей) системы \uline{ускорять} повышение качественного уровня своего состояния.}
	
	\end{SCn}
		
\end{frame}

\begin{frame}{\\Факторы интеллекта}
	\topline
	\justifying
	
	\begin{SCn}
		
	\scnheader{интеллект\scnsupergroupsign}
	\begin{scnrelfromlist}{параметр-фактор}
		\scnitem{\textbf{текущий уровень возможностей кибернетической системы\scnsupergroupsign}}
		
		\scnitem{\textbf{скорость эволюции кибернетической системы\scnsupergroupsign}}
		
		\scnitem{\textbf{ускорение эволюции кибернетической системы\scnsupergroupsign}}
		
	\end{scnrelfromlist}
		
	\end{SCn}
	
\end{frame}

\begin{frame}{\\Текущие возможности системы}
	\topline
	\justifying
	
	\begin{SCn}
	
	\scnheader{\textbf{текущий уровень возможностей кибернетической системы\scnsupergroupsign}}
	\scnidtf{мощность, многообразие, качество, полезность (для кибернетической системы) и целостность текущей деятельности, которую кибернетическая система умеет осуществлять в текущий момент\scnsupergroupsign}
	\scnidtf{объём и многообразие \textit{задач}, для решения которых \textit{кибернетическая система} имеет необходимые информационные ресурсы и освоенные ею \textit{методы} и методики управления собственными \textit{эффекторами} и \textit{внешними инструментами}\scnsupergroupsign}
	\scnidtf{множество \textit{технологий}, освоенных \textit{кибернетической системой}\scnsupergroupsign}
	\scntext{примечание}{\textit{деятельность кибернетической системы} не должна останавливаться прежде всего потому, что разрушительное воздействие внешней среды на кибернетическую систему не прекращается никогда и ему надо противодействовать.}
	
	\end{SCn}
		
\end{frame}

\begin{frame}{\\Текущие возможности системы}
	\topline
	\justifying
	
	\begin{SCn}
		
	\scnheader{текущий уровень возможностей кибернетической системы\scnsupergroupsign}
	\begin{scnrelfromlist}{параметр-фактор}
		\scnitem{объем памяти кибернетической системы\scnsupergroupsign}
		\scnitem{функциональная мощность процессоро-памяти кибернетической системы\scnsupergroupsign}
		\scnitem{производительность процессора кибернетической системы\scnsupergroupsign}
		\scnitem{качество внутренней информационной модели окружающей среды\scnsupergroupsign}
		\scnitem{многообразие возможных воздействий эффекторами кибернетической системы на внешнюю среду и на собственную физическую оболочку кибернетической системы\scnsupergroupsign}
		\scnitem{общее количество и многообразие видов сенсоров кибернетической системы\scnsupergroupsign}
	\end{scnrelfromlist}
		
	\end{SCn}
	
\end{frame}

\begin{frame}{\\Текущие возможности системы}
	\topline
	\justifying
	
	\begin{SCn}
		
	\scnheader{текущий уровень возможностей кибернетической системы\scnsupergroupsign}
	\begin{scnrelfromlist}{параметр-фактор}
		\scnitem{многообразие и эффективность использования технологий, которыми владеет кибернетическая система\scnsupergroupsign}
		\scnitem{самостоятельность в использовании технологий, которыми владеет кибернетическая система\scnsupergroupsign}
		\scnitem{уровень самостоятельности кибернетической системы в процессе реализации "жизненно"{} важных для неё видов деятельности\scnsupergroupsign}
	\end{scnrelfromlist}
		
	\end{SCn}
	
\end{frame}

\begin{frame}{Уровень самостоятельности\\ кибернетической системы}
	\topline
	\justifying
	
	\small
	
	\begin{SCn}
		
	\scnheader{уровень самостоятельности кибернетической системы в процессе реализации "жизненно"{} важных для неё видов деятельности\scnsupergroupsign}
	\begin{scnrelfromlist}{параметр-фактор}
		\scnitem{уровень самостоятельности кибернетической системы в процессе обеспечения её безопасности\scnsupergroupsign}
		\begin{scnindent}
			\scnidtf{уровень способности кибернетической системы к самосохранению\scnsupergroupsign}
		\end{scnindent}
		
		\scnitem{уровень самостоятельности кибернетической системы в процессе её материального обеспечения (заботы о себе)\scnsupergroupsign}
		
		\scnitem{уровень самостоятельности кибернетической системы в процессе реализации часто выполняемых видов деятельности, соответствующих ее специализации\scnsupergroupsign}
		
		\scnitem{уровень самостоятельности кибернетической системы при решении априори непредусмотренных задач\scnsupergroupsign}
		
		\scnitem{\textbf{способность к целесообразному и целенаправленному поведению\scnsupergroupsign}}
		
	\end{scnrelfromlist}
		
	\end{SCn}
	
\end{frame}

\begin{frame}{Качество внутренней\\ информационной модели}
	\topline
	\justifying
	
	\small
	
	\begin{SCn}
		
	\scnheader{качество внутренней информационной модели окружающей среды\scnsupergroupsign}
	\begin{scnrelfromlist}{параметр-фактор}
		\scnitem{объем внутренней информационной модели окружающей среды\scnsupergroupsign}
		\scnitem{многообразие знаний, входящих в состав внутренней информационной модели окружающей среды\scnsupergroupsign}
		\scnitem{непротиворечивость и синтаксическая безошибочность внутренней информационной модели окружающей среды\scnsupergroupsign}
		\scnitem{семантическая корректность внутренней информационной модели окружающей среды\scnsupergroupsign}
		\scnitem{семантическая полнота внутренней информационной модели окружающей среды\scnsupergroupsign}
		\scnitem{информационная чистота\scnsupergroupsign}
	\end{scnrelfromlist}
		
	\end{SCn}
	
\end{frame}

\begin{frame}{Качество внутренней\\ информационной модели}
	\topline
	\justifying
	
	\small
	
	\begin{SCn}

	\scnheader{качество внутренней информационной модели окружающей среды\scnsupergroupsign}
	\begin{scnrelfromlist}{параметр-фактор}	
		\scnitem{полнота описания собственного Я\scnsupergroupsign}
		\scnitem{синтаксическая и семантическая совместимость знаний, входящих в состав внутренней информационной модели окружающей среды\scnsupergroupsign}
		\scnitem{уровень структуризации и систематизации внутренней информационной модели окружающей среды с помощью различных видов метаинформации\scnsupergroupsign}
		\scnitem{уровень развития языковых средств, используемых во внутренней информационной модели окружающей среды для описания структуры и принципов функционирования собственной физической оболочки}
		\scnitem{способность кибернетической системы к минимизации числа рассматриваемых сущностей, необходимых для выполнения её действий}
	\end{scnrelfromlist}
		
	\end{SCn}
	
\end{frame}

\begin{frame}{Способность к целесообразному\\ и целенаправленному поведению}
	\topline
	\justifying
	
	\small
	
	\begin{SCn}
		
	\scnheader{способность к целесообразному и целенаправленному поведению\scnsupergroupsign}
	\begin{scnrelfromlist}{параметр-фактор}
		\scnitem{способность к целеполаганию и планированию действий\scnsupergroupsign}
		\scnitem{способность адекватно оценивать свои возможности\scnsupergroupsign}
		\scnitem{способность кибернетической системы осознавать (выделять) задачи (действия), обязательные для выполнения\scnsupergroupsign}
		\scnitem{способность кибернетической системы грамотно сочетать свои обязательные действия и свои необязательные для текущего момента действия}
		\scnitem{способность кибернетической системы осуществлять достаточно качественное прогнозирование значимых и, прежде всего, опасных для системы ситуаций и событий в окружающей среде\scnsupergroupsign}
	\end{scnrelfromlist}
		
	\end{SCn}
	
\end{frame}

\begin{frame}{Способность к целесообразному\\ и целенаправленному поведению}
	\topline
	\justifying
	
	\small
	
	\begin{SCn}
		
	\scnheader{способность к целесообразному и целенаправленному поведению\scnsupergroupsign}
	\begin{scnrelfromlist}{параметр-фактор}
		\scnitem{способность осознавать свои главные (стратегические) цели (установки, мотивы, ограничения, принципы) и, соответственно этому, отличать свои полезные воздействия на окружающую среду от возможных вредных воздействий\scnsupergroupsign}
		\scnitem{адекватность и корректность целеполагания\scnsupergroupsign}
		\scnitem{адекватность и целенаправленность непосредственного поведения}
		\scnitem{целеустремленность\scnsupergroupsign}
	\end{scnrelfromlist}
		
	\end{SCn}
	
\end{frame}

\begin{frame}{\\Способность понимать}
	\topline
	\justifying
	
	\small
	
	\begin{SCn}
		
	\scnheader{способность понимать\scnsupergroupsign}
	\begin{scnrelfromlist}{параметр-фактор}
		\scnitem{способность понимать сообщения от других кибернетических систем\scnsupergroupsign}
		\begin{scnindent}
			\begin{scnrelfromlist}{параметр-фактор}
				\scnitem{способность понимать разного уровня сложности команды или пожелания, полученные от других кибернетических систем, и в частности оценивать возможность, своевременность и качество их выполнения\scnsupergroupsign}
			\end{scnrelfromlist}
		\end{scnindent}
		\scnitem{способность оценивать важность и актуальность приобретаемой информации\scnsupergroupsign}
		\scnitem{способность понимать сенсорную информацию (в частности обнаруживать и распознавать важные объекты, ситуации, события, процессы)\scnsupergroupsign}
	\end{scnrelfromlist}
	
	\end{SCn}
	
\end{frame}

\begin{frame}{Скорость эволюции\\ кибернетической системы}
	\topline
	\justifying
	
	\begin{SCn}
	
	\scnheader{скорость эволюции кибернетической системы\scnsupergroupsign}
	\scnidtf{эволюционируемость\scnsupergroupsign}
	\scnidtf{уровень способности (приспособленности) кибернетической системы эволюционировать как с помощью внешних субъектов (учителей, разработчиков), так и самостоятельно\scnsupergroupsign}
	\scnidtf{умение кибернетической системы эволюционировать (в том числе учиться)\scnsupergroupsign}
		
	\end{SCn}
	
\end{frame}

\begin{frame}{Ускорение эволюции\\ кибернетической системы}
	\topline
	\justifying
	
	\begin{SCn}
		
	\scnheader{ускорение эволюции кибернетической системы\scnsupergroupsign}
	\scnidtf{уровень познания законов эволюции и обусловленный этим уровень осознанности, активности и самостоятельности выполнения эволюционного процесса\scnsupergroupsign}
	\scnidtf{способность кибернетической системы к эволюции своей способности эволюционировать\scnsupergroupsign}
	\scnidtf{способность кибернетической системы к осознанной, осмысленной, целенаправленной самоэволюции\scnsupergroupsign}
	\scntext{примечание}{Кибернетическая система должна не только уметь эволюционировать (в том числе уметь учиться), но также и уметь учиться тому, как надо эволюционировать (в том числе учиться) лучше --- то есть высшей формой эволюционных способностей кибернетической системы является переход на метауровень эволюционного процесса.}
		
	\end{SCn}
	
\end{frame}

\begin{frame}{\\Следует отличать}
	\topline
	\justifying
	
	\begin{SCn}
		
	\scnheader{следует отличать*}
	\begin{scnhaselementset}
		\scnitem{текущий уровень возможностей кибернетической системы\scnsupergroupsign}
		\scnitem{эволюция кибернетической системы}
		\begin{scnindent}
			\scnsubset{процесс}
		\end{scnindent}	
		\scnitem{скорость эволюции кибернетической системы\scnsupergroupsign}
		\begin{scnindent}
			\scnidtf{способность кибернетической системы эволюционировать\scnsupergroupsign}
		\end{scnindent}
		\scnitem{ускорение эволюции кибернетической системы\scnsupergroupsign}
	\end{scnhaselementset}
		
	\end{SCn}
	
\end{frame}

\begin{frame}{\\Следует отличать}
	\topline
	\justifying
	
	\vspace{5mm}
	\begin{SCn}
		
	\scnheader{следует отличать*}
	\begin{scnhaselementset}
		\scnitem{физические возможности кибернетической системы}
		\scnitem{интеллект\scnsupergroupsign}
		\begin{scnindent}
			\scnidtf{когнитивные способности}
		\end{scnindent}
		\scnitem{базовый интеллект\scnsupergroupsign}
		\scnitem{интеллект индивидуальной кибернетической системы\scnsupergroupsign}
		\scnitem{интеллект агента сообщества\scnsupergroupsign}
		\scnitem{интеллект сообщества кибернетических систем\scnsupergroupsign}
	\end{scnhaselementset}
		
	\end{SCn}
	
\end{frame}

\begin{frame}{Физические возможности\\ кибернетической системы}
	\topline
	\justifying
	
	\begin{SCn}
	
	\scnheader{физические возможности кибернетической системы}
	\scnidtf{физические способности и характеристики кибернетической системы}
	\scnidtf{физические возможности кибернетической системы по её взаимодействию с внешней средой и собственной материальной оболочкой --- что система может обнаружить (заметить) и как физически или химически она может взаимодействовать --- например, может ли она перемещаться, может ли она "слышать"{} в ультразвуковом диапазоне и т.д.}
		
	\end{SCn}
	
\end{frame}

\begin{frame}{\\Базовый интеллект}
	\topline
	\justifying
	
	\begin{SCn}
		
	\scnheader{базовый интеллект\scnsupergroupsign}
	\scnidtf{когнитивные способности, которыми должны обладать все кибернетические системы независимо от их структурного типа (от того, являются ли они индивидуальными, агентами или сообществами)}
	\begin{scnrelfromlist}{параметр-фактор}
		\scnitem{текущие когнитивные способности (возможности) кибернетической системы}
		\scnitem{способность эволюционировать}
		\scnitem{уровень осознания законов эволюции и качества практического использования этих знаний}
	\end{scnrelfromlist}
		
	\end{SCn}
	
\end{frame}


\begin{frame}{\\Базовый интеллект}
	\topline
	\justifying
	
	\begin{SCn}
		
	\scnheader{интеллект индивидуальной кибернетической системы\scnsupergroupsign}
	\scnidtf{базовый интеллект индивидуальной кибернетической системы\scnsupergroupsign}
		
	\end{SCn}
	
\end{frame}

\begin{frame}{\\Интеллект агента сообщества}
	\topline
	\justifying
	
	\vspace{5mm}
	
	\begin{SCn}
		
	\scnheader{интеллект агента сообщества\scnsupergroupsign}
	\begin{scnrelfromlist}{параметр-фактор}
		\scnitem{базовый интеллект агента сообщества}
		\scnitem{интероперабельность агента сообщества}
		\begin{scnindent}
			\scnidtf{социальный интеллект агента сообщества}
			\scnidtf{способность эффективно существовать в социальной среде}
			\begin{scnrelfromlist}{параметр-фактор}
				\scnitem{текущий уровень интероперабельности}
				\scnitem{способность повышать уровень интероперабельности}
				\scnitem{способность ускорять повышение уровня интероперабельности путём познания законов эволюции интероперабельности и их грамотного применения}
			\end{scnrelfromlist}
		\end{scnindent}
	\end{scnrelfromlist}
		
	\end{SCn}
	
\end{frame}

\begin{frame}{\\Интеллект сообщества}
	\topline
	\justifying
	
	\vspace{5mm}
	
	\begin{SCn}
	
	\scnheader{интеллект сообщества кибернетических систем\scnsupergroupsign}
	\scnidtf{базовый интеллект сообщества кибернетических систем как самостоятельного субъекта деятельности}
	\begin{scnrelfromlist}{параметр-фактор}
		\scnitem{синергия взаимодействия агентов сообщества\scnsupergroupsign}
		\begin{scnindent}
			\scnidtf{эффективность (качество организации) взаимодействия агентов сообщества в процессе коллективного решения задач (в том числе сложных)}
			\begin{scnrelfromlist}{параметр-фактор}
				\scnitem{текущий уровень синергии}
				\scnitem{способность повышать уровень синергии}
				\scnitem{способность ускорять повышение уровня синергии}
			\end{scnrelfromlist}
		\end{scnindent}
		\scnitem{суммарный интеллектуальный потенциал, а также среднее значение и дисперсия уровня интероперабельности всех агентов сообщества}
	\end{scnrelfromlist}
	
	\end{SCn}
	
\end{frame}

\begin{frame}{\\Заключение}
	\topline
	\justifying
	
	\vspace{5mm}
	
	\begin{textitemize}
		\item Можно говорить не о наличии или отсутствии свойства интеллектуальности, а об \textbf{уровне интеллекта};
		\item Ключевую важность имеет не только текущий уровень интеллекта, но и \textbf{способность эволюционировать} и \textbf{способность к повышению способности эволюционировать};
		\item \textbf{Архитектура интеллектуальных систем} соответствует архитектуре кибернетических систем, но можно говорить о параметрах, определяющих качество отдельных компонентов интеллектуальных систем.
	\end{textitemize}
	
	
\end{frame}
