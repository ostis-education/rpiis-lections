\title{Лекция 12\\Пользовательские интерфейсы интеллектуальных систем \vspace{-2em}}   
\author[]{Шункевич Д.В.}
\institute[]{Белорусский государственный университет информатики и радиоэлектроники}

\begin{frame}
	\titlepage
\end{frame}

\begin{frame}{\\Содержание лекции}
	\topline
	\justifying
	\vspace*{\fill}\\
	\begin{SCn}
	\scnheader{Лекция 12. Пользовательские интерфейсы интеллектуальных систем}
	\begin{scnrelfromset}{структура}
		\scnitem{Пользовательские интерфейсы как интеллектуальные системы}
		\scnitem{Внешние языки общения с пользователями: универсальные и специализированные}
		\scnitem{Редакторы внешних информационных конструкций}
	\end{scnrelfromset}
	\end{SCn}
\end{frame}

\begin{frame}{Пользовательские интерфейсы как интеллектуальные системы}
	\topline
	\justifying
	\vspace*{\fill}\\
	\begin{SCn}
		\scnheader{интерфейс}
		\scntext{определение}{\textbf{\textit{интерфейс}} --- совокупность технических, программных и методических (протоколов, правил, соглашений) средств, обеспечивающих обмен информацией между пользователем и устройствами и программами, а также между устройствами и другими устройствами и программами.}		
	\end{SCn}
\end{frame}


\begin{frame}{Пользовательские интерфейсы как интеллектуальные системы}
	\topline
	\justifying
	\vspace*{\fill}\\
	\vspace{5mm}
	\footnotesize
	\begin{SCn}
		
		\scnheader{интерфейс}
		\begin{scnrelfromset}{разбиение}
			
		\scnitem{пользовательский интерфейс}
		\begin{scnindent}
			\scnidtf{один из наиболее важных компонентов компьютерной системы, представляющих собой совокупность аппаратных и программных средств, обеспечивающих обмен информацией между пользователем и компьютерной системой}		
		\end{scnindent}

		\scnitem{программный интерфейс}
		\begin{scnindent}
			\scnidtf{система унифицированных связей, предназначенных для обмена информацией между компонентами вычислительной системы} 
			\scntext{примечание}{Программный интерфейс задается набором необходимых процедур, их параметров и способов обращения.}		
		\end{scnindent}
			
		\scnitem{физический интерфейс}
		\begin{scnindent}
			\scnidtf{устройство, преобразующее сигналы и передающее их от одного компонента оборудования к другому}
			\scntext{примечание}{Физический интерфейс определяется набором электрических связей и характеристиками сигналов.}
		\end{scnindent}
		
		\end{scnrelfromset}
		
		
	\end{SCn}
\end{frame}


\begin{frame}{Пользовательские интерфейсы как интеллектуальные системы}
	\topline
	\justifying
	\vspace*{\fill}\\
	\begin{SCn}
		
		\scnheader{пользовательский интерфейс}
		\scntext{определение}{\textbf{\textit{пользовательский интерфейс}} --- интерфейс, обеспечивающий передачу информации между пользователем-человеком и программно-аппаратными компонентами компьютерной системы.}		
	\end{SCn}
\end{frame}

\begin{frame}{Внешние языки\\ общения с пользователями}
	\topline
	\justifying
	\vspace*{\fill}\\
	\begin{SCn}
		\scnheader{Пользовательский интерфейс}
			\scnsubset {\textit{командный пользовательский интерфейс}}		
			\scnsubset {WIMP-\textit{интерфейс}}
			\begin{scnindent}
				\scnidtf {Window, Image, Menu, Pointer - интерфейс}
				\scnidtf {Окно, Образ, Меню, Указатель - интерфейс}
			\end{scnindent}
			\scnsubset {SILK-\textit{интерфейс}}
			\begin{scnindent}
				\scnidtf {Speech, Image, Language, Knowledge - интерфейс}
				\scnidtf {Речь, Образ, Язык, Знание - интерфейс}
			\end{scnindent}
	\end{SCn}
\end{frame}

\begin{frame}{Внешние языки\\ общения с пользователями}
	\topline
	\justifying
	\vspace*{\fill}\\
	\begin{SCn}
		
		\scnheader{командный пользовательский интерфейс}
		\scntext{определение}{\textbf{\textit{командный пользовательский интерфейс}} --- пользовательский интерфейс, при котором обмен информацией между компьютерной системой и пользователем осуществляется путем написания текстовых инструкций или команд.}		
		
		\scnheader{WIMP-интерфейс}
		\scntext{определение}{\textbf{\textit{WIMP-интерфейс}} --- пользовательский интерфейс, при котором обмен информацией между компьютерной системой и пользователем осуществляется в форме диалога при помощью окон, меню и других элементов управления.}
		
	\end{SCn}
\end{frame}

\begin{frame}{Внешние языки\\ общения с пользователями}
	\topline
	\justifying
	\vspace*{\fill}\\
	\begin{SCn}
		
		\scnheader{SILK-интерфейс}
		\scntext{определение}{\textbf{\textit{SILK-интерфейс}} --- пользовательский интерфейс, наиболее приближенный к естественной для человека форме общения. Компьютерная система находит для себя команды, анализируя человеческую речь и находя в ней ключевые фразы. Результат выполнения команд преобразуется в понятную человеку форму, например, в естественно-языковую форму или изображение.}
		
	\end{SCn}
\end{frame}


\begin{frame}{Внешние языки\\ общения с пользователями}
	\topline
	\justifying
	\vspace*{\fill}\\
	
	\begin{SCn}
		\scnheader{SILK-интерфейс}
		\begin{scnrelfromset}{разбиение}
			\scnitem{естественно-языковой интерфейс}
			\scnitem{речевой интерфейс}
		\end{scnrelfromset}
		
	\end{SCn}
\end{frame}

\begin{frame}{Внешние языки\\ общения с пользователями}
	\topline
	\justifying
	\vspace*{\fill}\\
	\begin{SCn}
		\scnheader{естественно-языковой интерфейс}
		\scntext{определение}{\textbf{\textit{естественно-языковой интерфейс}} --- SILK-интерфейс, обмен информацией
			между компьютерной системой и пользователем в котором происходит за счёт диалога. Диалог ведётся на одном из естественных языков.}		
		\scnheader{речевой интерфейс}
		\scntext{определение}{\textbf{\textit{речевой интерфейс}} --- SILK-интерфейс, обмен информацией в котором происходит за
			счёт диалога, в процессе которого компьютерная система и пользователь общаются с помощью речи.}
		
	\end{SCn}
\end{frame}

\begin{frame}{Внешние языки\\ общения с пользователями}
	\topline
	\justifying
	\vspace*{\fill}\\
	\begin{SCn}
		
		\scnheader{адаптивный интерфейс}
		\scntext{определение}{\textbf{\textit{адаптивный интерфейс}} --- пользовательский интерфейс, который изменяется на основе потребностей пользователя или контекста.}		
		
		\scnheader{интеллектуальный интерфейс}
		\scntext{определение}{\textbf{\textit{интеллектуальный интерфейс}} --- пользовательский интерфейс, который может предположить дальнейшие действия пользователей и представить информацию на основе этого предположения.}
	\end{SCn}
\end{frame}

\begin{frame}{Внешние языки\\ общения с пользователями}
	\topline
	\justifying
	\vspace*{\fill}\\
	\begin{SCn}
		\scnheader{мультимодальный интерфейс}
		\scntext{определение}{\textbf{\textit{мультимодальный интерфейс}} --- пользовательский интерфейс, предназначенный для обработки двух или более комбинированных режимов пользовательского ввода, таких как речь, перо, касание, ручные жесты и взгляд, скоординированным образом с выводом мультимедийной системы.}
		
	\end{SCn}
\end{frame}

