\title{Лекция 13\\Анализ внешней информации в интеллектуальных системах \vspace{-2em}}   
\author[]{Шункевич Д.В.}
\institute[]{Белорусский государственный университет информатики и радиоэлектроники}

\begin{frame}
	\titlepage
\end{frame}

\begin{frame}{\\Содержание лекции}
	\topline
	\justifying
	Синтаксический анализ вводимой информации. Семантический анализ вводимой информации. Естественно-языковой интерфейс. Интеграция знаний. Модель понимания.
\end{frame}

\begin{frame}{\\Содержание лекции}
	\topline
	\justifying
    \begin{scnrelfromset}{Разбиение}
        \scnitem{Синтаксический анализ вводимой информации}
        \scnitem{Семантический анализ вводимой информации}
        \scnitem{Естественно-языковой интерфейс}
        \scnitem{Интеграция знаний}
        \scnitem{Модель понимания}
    \end{scnrelfromset}
\end{frame}

\begin{frame}{\\Синтаксический анализ вводимой информации}
	\topline
 	\begin{SCn}
    \scnheader{Синтаксический анализ вводимой информации}
    \begin{scnrelfromset}{Определение}
	   \scnitem{это анализ синтаксической структуры предложений, то есть на связях между словами и фразами в предложении. Он определяет, какие слова являются подлежащими, сказуемыми, объектами и т.д., и как они связаны друг с другом.}
    \end{scnrelfromset}
    \begin{scnrelfromset}{Возможности}
        \scnitem{направлен на определение грамматической корректности предложений и структурирование текста на уровне грамматики. Он отвечает на вопросы "какие слова связаны и как они связаны?".}
    \end{scnrelfromset}
    \end{SCn}
\end{frame}

\begin{frame}{\\Синтаксический анализ вводимой информации}
	\topline
 	\begin{SCn} 
    \scnheader{Синтаксический анализ вводимой информации}
    \begin{scnrelfromset}{Цели анализа}
            \scnitem{Проверка синтаксической корректности: Основная цель синтаксического анализа - установить, соответствует ли вводимая информация определенным правилам синтаксиса или грамматики языка.}
            \scnitem{Разбор и структурирование: Синтаксический анализ также позволяет разбить вводимую информацию на составляющие (токены или единицы), определить их взаимосвязи и построить структуру данных, которая отображает синтаксическую структуру текста.}
            \scnitem{Понимание текста: Синтаксический анализ является одной из основных составляющих обработки естественного языка (NLP) и играет важную роль в понимании текста. Путем анализа синтаксической структуры текста можно определить связи между словами, их роли в предложении и контекстуальные зависимости.}
    \end{scnrelfromset}
    \end{SCn}
\end{frame}

\begin{frame}{\\Синтаксический анализ вводимой информации}
	\topline
 	\begin{SCn} 
    \scnheader{Синтаксический анализ вводимой информации}
    \begin{scnrelfromset}{Где применяется}
            \scnitem{Поисковые системы: Синтаксический анализ может быть использован для улучшения поисковых запросов и результатов поиска.}
            \scnitem{Голосовые помощники: Синтаксический анализ используется в голосовых помощниках для распознавания и понимания речи пользователя. Анализируя синтаксическую структуру предложений, система может интерпретировать команды и предоставлять соответствующие ответы или действия.}
            \scnitem{Анализаторы кода: Синтаксический анализ используется в инструментах разработки программного обеспечения для анализа и проверки синтаксической корректности кода.}
    \end{scnrelfromset}
    \end{SCn}
\end{frame}

\begin{frame}{\\Семантический анализ вводимой информации}
	\topline
 	\begin{SCn}
    \scnheader{Семантический анализ вводимой информации}
    \begin{scnrelfromset}{Определение}
	   \scnitem{это процесс анализа значения и смысла вводимых данных с целью понимания и интерпретации их семантики, то есть смыслового содержания.}
    \end{scnrelfromset}
    \begin{scnrelfromset}{Возможности}
        \scnitem{направлен на понимание смысла и значения предложений и текстов. Он отвечает на вопросы "что означает это предложение?" или "какие отношения и значения выражаются в этом тексте?".}
    \end{scnrelfromset}
    \end{SCn}
\end{frame}

\begin{frame}{\\Семантический анализ вводимой информации}
	\topline
 	\begin{SCn} 
    \scnheader{Семантический анализ вводимой информации}
    \begin{scnrelfromset}{Цели анализа}
            \scnitem{Понимание смысла: Основная цель семантического анализа текста - это понимание смысла и значения текстовой информации.}
            \scnitem{Извлечение информации: Семантический анализ позволяет системам извлекать конкретные факты, события, именованные сущности и другую релевантную информацию из текстовых источников.}
            \scnitem{Классификация и категоризация: Семантический анализ помогает классифицировать и категоризировать тексты в соответствии с определенными классами или категориями.}
            \scnitem{Семантический поиск: Семантический анализ улучшает результаты поиска, учитывая семантическую связь и смысл между словами и фразами.}
    \end{scnrelfromset}
    \end{SCn}
\end{frame}

\begin{frame}{\\Семантический анализ вводимой информации}
	\topline
 	\begin{SCn} 
    \scnheader{Семантический анализ вводимой информации}
    \begin{scnrelfromset}{Где применяется}
            \scnitem{Поисковые системы: Синтаксический анализ может быть использован для улучшения поисковых запросов и результатов поиска.}
            \scnitem{Вопросно-ответные системы: Семантический анализ помогает системам понимать вопросы пользователей и предоставлять соответствующие ответы.}
            \scnitem{Анализ контента и автоматическая индексация: Семантический анализ используется для анализа и классификации контента, например, при индексации и каталогизации больших объемов текстовой информации.}
    \end{scnrelfromset}
    \end{SCn}
\end{frame}

\begin{frame}{\\Естественно-языковой интерфейс}
	\topline
 	\begin{SCn}
    \scnheader{Естественно-языковой интерфейс}
    \begin{scnrelfromset}{Определение}
	   \scnitem{это форма взаимодействия между человеком и компьютерной системой, основанная на использовании естественного языка для передачи команд, запросов или инструкций.}
    \end{scnrelfromset}
    \begin{scnrelfromset}{Возможности}
        \scnitem{позволяет пользователям взаимодействовать с компьютером или программным приложением, используя естественный язык, такой как английский, русский и другие языки, вместо того, чтобы использовать формализованные команды или язык программирования.}
    \end{scnrelfromset}
    \end{SCn}
\end{frame}

\begin{frame}{\\Естественно-языковой интерфейс}
	\topline
 	\begin{SCn} 
    \scnheader{Семантический анализ вводимой информации}
    \begin{scnrelfromset}{Основные преимущества}
            \scnitem{Удобство использования: возможность общения с компьютером на естественном языке без необходимости изучения специальных команд или языков программирования}
            \scnitem{Интуитивность: пользовательский опыт, близкий к общению с другими людьми, что снижает порог входа для широкого круга пользователей}
            \scnitem{Большая гибкость: способность обрабатывать сложные запросы, команды и вопросы с различными вариациями фраз и формулировок}
            \scnitem{Широкий спектр применений: естественно-языковые интерфейсы применяются в различных областях, включая мобильные приложения, умные ассистенты, системы управления данными, поисковые системы и другие}
    \end{scnrelfromset}
    \end{SCn}
\end{frame}

\begin{frame}{\\Естественно-языковой интерфейс}
	\topline
 	\begin{SCn} 
    \scnheader{Семантический анализ вводимой информации}
    \begin{scnrelfromset}{Ключевые возможности}
            \scnitem{Обработка и анализ текста: синтаксический и семантический анализ вводимого текста для понимания смысла и интента пользователя}
            \scnitem{Извлечение информации: способность системы извлекать конкретные факты, события, именованные сущности из текстовых источников}
            \scnitem{Классификация и категоризация: возможность классифицировать и категоризировать тексты по определенным классам или категориям}
            \scnitem{Анализ тональности и сентимента: определение тональности и сентимента текста, позволяющее оценить отношение автора к обсуждаемой теме}
    \end{scnrelfromset}
    \end{SCn}
\end{frame}

\begin{frame}{\\Естественно-языковой интерфейс}
	\topline
 	\begin{SCn} 
    \scnheader{Естественно-языковой интерфейс}
    \begin{scnrelfromset}{Области применения}
            \scnitem{Мобильные приложения}
            \scnitem{Умные ассистенты (например, Siri, Google Assistant)}
            \scnitem{Системы управления данными}
            \scnitem{Поисковые системы}
            \scnitem{Анализ социальных медиа}
            \scnitem{Клиентская обратная связь и обслуживание клиентов}
    \end{scnrelfromset}
    \end{SCn}
\end{frame}

\begin{frame}{\\Интеграция знаний}
	\topline
 	\begin{SCn}
    \scnheader{Интеграция знаний}
    \begin{scnrelfromset}{Определение}
	   \scnitem{это процесс объединения и совмещения информации и знаний из различных источников или предметных областей с целью создания более полного и целостного представления.}
    \end{scnrelfromset}
    \end{SCn}
\end{frame}

\begin{frame}{\\Интеграция знаний}
	\topline
 	\begin{SCn} 
    \scnheader{Интеграция знаний}
    \begin{scnrelfromset}{Основные преимущества}
            \scnitem{Улучшенная доступность информации: объединение различных источников данных позволяет легко получать целостную информацию в одном месте}
            \scnitem{Более полное представление данных: интеграция знаний позволяет объединить разрозненные данные для создания более полной картины и лучшего понимания предметной области}
            \scnitem{Усиленная аналитика: интеграция знаний предоставляет больший объем данных для проведения более глубокого анализа и выявления скрытых взаимосвязей}
            \scnitem{Улучшенное принятие решений: доступ к более широкому набору данных помогает принимать лучше обоснованные и информированные решения}
    \end{scnrelfromset}
    \end{SCn}
\end{frame}

\begin{frame}{\\Интеграция знаний}
	\topline
 	\begin{SCn} 
    \scnheader{Интеграция знаний}
    \begin{scnrelfromset}{Процесс интеграции знаний}
            \scnitem{Сбор и агрегация данных: обзор процесса сбора информации из различных источников и их агрегации в единую базу знаний}
            \scnitem{Семантическое согласование: объединение терминологии и согласование семантических моделей для обеспечения совместимости данных}
            \scnitem{Разрешение конфликтов: выявление и разрешение несоответствий и конфликтов между различными источниками данных}
            \scnitem{Создание единой модели знаний: формирование единой структуры и модели данных, объединяющей разные аспекты предметной области}
    \end{scnrelfromset}
    \end{SCn}
\end{frame}

\begin{frame}{\\Интеграция знаний}
	\topline
 	\begin{SCn} 
    \scnheader{Интеграция знаний}
    \begin{scnrelfromset}{Применение в различных областях}
            \scnitem{Бизнес и управление: использование интеграции знаний для аналитики данных, принятия решений, управления проектами и улучшения бизнес-процессов}
            \scnitem{Здравоохранение: объединение медицинских данных из разных источников для повышения эффективности диагностики, лечения и мониторинга пациентов}
            \scnitem{Информационные системы: создание целостных информационных систем, объединяющих различные компоненты и источники данных для улучшения пользовательского опыта и функциональности}
            \scnitem{Научные исследования: интеграция данных и знаний для проведения более комплексных исследований, открытия новых связей и расширения познаний в различных научных областях}
    \end{scnrelfromset}
    \end{SCn}
\end{frame}

\begin{frame}{\\Модель понимания}
	\topline
 	\begin{SCn}
    \scnheader{Модель понимания}
    \begin{scnrelfromset}{Определение}
	   \scnitem{это абстрактное представление или структура, созданная компьютерной системой для описания и понимания естественного языка или семантического контекста. Она позволяет системе анализировать, интерпретировать и извлекать смысл из вводимой информации.}
    \end{scnrelfromset}
    \end{SCn}
\end{frame}

\begin{frame}{\\Модель понимания}
	\topline
 	\begin{SCn} 
    \scnheader{Модель понимания}
    \begin{scnrelfromset}{Ключевые принципы модели понимания}
            \scnitem{Лексический анализ: разбор текста на отдельные слова и лексические единицы}
            \scnitem{Синтаксический анализ: выявление структуры предложений и установление грамматических отношений}
            \scnitem{Семантический анализ: раскрытие смысла слов, выявление смысловых отношений и контекстуального значения}
    \end{scnrelfromset}
    \end{SCn}
\end{frame}

\begin{frame}{\\Модель понимания}
	\topline
 	\begin{SCn} 
    \scnheader{Модель понимания}
    \begin{scnrelfromset}{Применение модели понимания}
            \scnitem{Естественно-языковые интерфейсы: создание интеллектуальных ассистентов, систем вопросно-ответной обработки и диалоговых систем}
            \scnitem{Обработка текстов и документов: извлечение информации, анализ содержания и классификация текстовых данных}
            \scnitem{Анализ медиа-контента: распознавание речи, обработка изображений и видео с целью понимания и классификации контента}
            \scnitem{Робототехника и автономные системы: понимание команд и инструкций, взаимодействие с окружающей средой}
    \end{scnrelfromset}
    \end{SCn}
\end{frame}