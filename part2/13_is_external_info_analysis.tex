\title{Лекция 13\\Анализ внешней информации в интеллектуальных системах \vspace{-2em}}   
\author[]{Шункевич Д.В.}
\institute[]{Белорусский государственный университет информатики и радиоэлектроники}

\begin{frame}
	\titlepage
\end{frame}

\begin{frame}{\\Содержание лекции}
	\topline
	\justifying
	Синтаксический анализ вводимой информации. Семантический анализ вводимой информации. Естественно-языковой интерфейс. Интеграция знаний. Модель понимания.
\end{frame}

\begin{frame}{\\Структура лекции}
	\topline
	\justifying

	\begin{SCn}
		\scnheader{Лекция 13. Анализ вводимой информации}
		\begin{scnrelfromset}{разбиение}
			\scnitem{Синтаксический анализ вводимой информации.}
			\scnitem{Семантический анализ вводимой информации.}
			\scnitem{Естественно-языковой интерфейс.}
			\scnitem{Интеграция знаний.}
                \scnitem{Модель понимания.}
		\end{scnrelfromset}
	\end{SCn}
\end{frame}

\begin{frame}{\\Синтаксический анализ вводимой информации}
	\topline
 	\begin{SCn}
    \scnheader{Синтаксический анализ вводимой информации}
    \begin{scnrelfromset}{определение}
	   \scnitem{это процесс анализа и понимания структуры вводимого текста с целью выявления синтаксических правил и отношений между словами.}
    \end{scnrelfromset}
    \begin{scnrelfromset}{Возможности}
        \scnitem{направлен на понимание и интерпретирование синтаксических правил, которыми руководствуется текст.}
        \scnitem{позволяет генерировать текст, имитирующий естественный язык.}
    \end{scnrelfromset}
    \end{SCn}
\end{frame}

\begin{frame}{\\Синтаксический анализ вводимой информации}
	\topline
 	\begin{SCn} 
    \scnheader{Синтаксический анализ вводимой информации}
    \begin{scnrelfromset}{цели анализа}
            \scnitem{Понимание синтаксической структуры: синтаксический анализ помогает понять грамматическую структуру предложений, введенных пользователем или полученных из других источников.}
            \scnitem{Обработка и понимание текста: синтаксический анализ помогает компьютерным системам обрабатывать и понимать введенную информацию на более высоком уровне.}
            \scnitem{Разрешение неоднозначностей: введенная информация может содержать неоднозначности, например, двусмысленности или различные возможные трактовки. }
            \scnitem{Улучшение качества обработки текста: синтаксический анализ может быть использован для улучшения качества различных задач обработки текста.}
            \scnitem{Поддержка автоматического анализа и обработки: синтаксический анализ является важным компонентом в области обработки естественного языка (NLP).}
    \end{scnrelfromset}
    \end{SCn}
\end{frame}

\begin{frame}{\\Синтаксический анализ вводимой информации}
	\topline
 	\begin{SCn} 
    \scnheader{Синтаксический анализ вводимой информации}
    \begin{scnrelfromset}{применение в различных областях}
            \scnitem{Применяетя в исследовании и разработке подсистем обработки естественно-языковых текстов.
            }
            \scnitem{Используется для анализа больших объемов текстовой информации, например, при поиске информации или анализе документов.
            }
            \scnitem{Играет важную роль в задаче машинного перевода и работы с многоязычной информацией.
            }
    \end{scnrelfromset}
    \end{SCn}
\end{frame}

\begin{frame}{\\Семантический анализ вводимой информации}
	\topline
 	\begin{SCn}
    \scnheader{Семантический анализ вводимой информации}
    \begin{scnrelfromset}{определение}
	   \scnitem{это процесс анализа и понимания смысловых аспектов текста, введенного пользователем или полученного из других источников.}
    \end{scnrelfromset}
    \begin{scnrelfromset}{возможности}
        \scnitem{направлен на понимание и интерпретирование значения слов, фраз и предложений в контексте.}
    \end{scnrelfromset}
    \end{SCn}
\end{frame}

\begin{frame}{\\Семантический анализ вводимой информации}
	\topline
 	\begin{SCn} 
    \scnheader{Семантический анализ вводимой информации}
    \begin{scnrelfromset}{цели анализа}
            \scnitem{Стремится понять содержание вводимой информации, выявить основные темы, идеи, события или факты, которые содержатся в тексте.}
            \scnitem{Извлекает структурированную информацию из текста, такую как именованные сущности (имена, места, организации), даты, события и другие важные факты.}
            \scnitem{Стремится выявить отношения и связи между различными элементами текста, такими как субъекты, объекты, действия и атрибуты.}
            \scnitem{Используется для определения тональности или сентимента текста, то есть понимания, выражено ли в тексте положительное, отрицательное или нейтральное отношение к чему-либо.}
    \end{scnrelfromset}
    \end{SCn}
\end{frame}

\begin{frame}{\\Семантический анализ вводимой информации}
	\topline
 	\begin{SCn} 
    \scnheader{Семантический анализ вводимой информации}
    \begin{scnrelfromset}{применение в различных областях}
            \scnitem{Используется для улучшения результатов поиска, позволяя системе понять смысл запроса пользователя и предоставить наиболее релевантные результаты.}
            \scnitem{Является важной составляющей многих задач NLP, таких как извлечение информации, вопросно-ответные системы, чат-боты и многое другое. Анализ семантики текста позволяет системам более точно понимать запросы пользователей и обрабатывать текстовую информацию.}
    \end{scnrelfromset}
    \end{SCn}
\end{frame}

\begin{frame}{\\Естественно-языковой интерфейс}
	\topline
 	\begin{SCn}
    \scnheader{Естественно-языковой интерфейс}
    \begin{scnrelfromset}{определение}
	   \scnitem{SILK-интерфейс (Speech – речь, Image – образ, Language – язык, Knowledge – знание), обмен информацией между компьютерной системой и пользователем в котором происходит засчёт диалога. Диалог ведётся на одном из естественных языков}
    \textbf{естественно-языковой интерфейс}
    
    \scnsuperset{речевой интерфейс}
    \begin{scnrelfromset}{определение}
        \scnitem{SILK-интерфейс, обмен информацией в котором происходит за счёт диалога, в процессе
которого компьютерная система и пользователь общаются с помощью речи. Данный вид интерфейса
наиболее приближен к естественному общению между людьми}
    \end{scnrelfromset}
    \end{scnrelfromset}
    \end{SCn}
\end{frame}

\begin{frame}{\\Естественно-языковой интерфейс}
	\topline
 	\begin{SCn}
    \begin{scnrelfromset}{возможности}
        \scnitem{Предоставляет возможность взаимодействия с компьютерной системой с помощью естественного языка, а не специализированных команд или языков программирования.}
    \end{scnrelfromset}
    \end{SCn}
\end{frame}

\begin{frame}{\\Естественно-языковой интерфейс}
	\topline
 	\begin{SCn}
    \scnheader{Решатель задач естественно-языкового интерфейса}
    \begin{scnrelfromset}{декомпозиция абстрактного sc-агента}
	   \scnitem{Абстрактный sc-агент лексического анализа}
    \begin{scnrelfromset}{декомпозиция абстрактного sc-агента}
        \scnitem{Абстрактный sc-агент декомпозиции текста на токены}
        \scnitem{Абстрактный sc-агент сопоставления токенов с лексемами}
    \end{scnrelfromset}
    \scnitem{Абстрактный sc-агент синтаксического анализа}
    \scnitem{Абстрактный sc-агент понимания сообщения}
    \end{scnrelfromset}
    \end{SCn}
\end{frame}

\begin{frame}{\\Естественно-языковой интерфейс}
	\topline
 	\begin{SCn}
    \scnheader{Агент понимания сообщения}
    \begin{scnrelfromset}{декомпозиция абстрактного sc-агента}
     \scnitem{Абстрактный sc-агент генерации вариантов значения сообщения}
     \scnitem{Абстрактный sc-агент выбора и обновления контекста}
    \begin{scnrelfromset}{декомпозиция абстрактного sc-агента}
        \scnitem{Абстрактный sc-агент разрешения контекста}
        \scnitem{Абстрактный sc-агент выбора смысла сообщения на основе контекста}
        \scnitem{Абстрактный sc-агент погружения сообщения в контекст}
    \end{scnrelfromset}
    \end{scnrelfromset}
    \end{SCn}
\end{frame}

\begin{frame}{\\Естественно-языковой интерфейс}
	\topline
 	\begin{SCn} 
    \scnheader{Естественно-языковой интерфейс}
    \begin{scnrelfromset}{основные преимущества}
            \scnitem{Позволяет пользователю взаимодействовать с системой так же, как он общается с другими людьми, без необходимости изучать специализированные команды или языки программирования.}
            \scnitem{Позволяет общаться с компьютером на естественном языке, что делает его доступным для широкого круга пользователей.}
            \scnitem{Позволяет пользователям более быстро и эффективно выполнять задачи.}
            \scnitem{Позволяет системе лучше понять контекст и смысл вводимой информации.}
    \end{scnrelfromset}
    \end{SCn}
\end{frame}

\begin{frame}{\\Естественно-языковой интерфейс}
	\topline
 	\begin{SCn} 
    \scnheader{Синтаксический анализ вводимой информации}
    \begin{scnrelfromset}{ключевые возможности}
            \scnitem{Обработка и анализ текста: синтаксический и семантический анализ вводимого текста для понимания смысла и интента пользователя}
            \scnitem{Извлечение информации: способность системы извлекать конкретные факты, события, именованные сущности из текстовых источников}
            \scnitem{Классификация и категоризация: возможность классифицировать и категоризировать тексты по определенным классам или категориям}
            \scnitem{Анализ тональности и сентимента: определение тональности и сентимента текста, позволяющее оценить отношение автора к обсуждаемой теме}
    \end{scnrelfromset}
    \end{SCn}
\end{frame}

\begin{frame}{\\Естественно-языковой интерфейс}
	\topline
 	\begin{SCn} 
    \scnheader{Синтаксический анализ вводимой информации}
    \begin{scnrelfromset}{ключевые возможности}
            \scnitem{Позволяет определить структуру вводимой информации, выявляя синтаксические отношения между словами и фразами.}
            \scnitem{Помогает обнаружить ошибки вводимой информации, такие как грамматические ошибки, неправильный порядок слов, недостающие или лишние элементы в предложении.}
            \scnitem{Помогает извлечь синтаксическую информацию из вводимого текста.}
            \scnitem{Помогает понять синтаксическую структуру текста и сделать более точные выводы на основе этой информации.}
    \end{scnrelfromset}
    \end{SCn}
\end{frame}

\begin{frame}{\\Естественно-языковой интерфейс}
	\topline
 	\begin{SCn} 
    \scnheader{Семантический анализ вводимой информации}
    \begin{scnrelfromset}{ключевые возможности}
            \scnitem{Помогает понять смысл вводимой информации, а не только ее формальную структуру.}
            \scnitem{Позволяет сопоставлять вводимую информацию с предварительно определенными знаниями или моделями.}
            \scnitem{Позволяет разрешать неоднозначность вводимой информации.}
            \scnitem{Используется для анализа настроений и сентимента вводимой информации.}
    \end{scnrelfromset}
    \end{SCn}
\end{frame}

\begin{frame}{\\Естественно-языковой интерфейс}
	\topline
 	\begin{SCn} 
    \scnheader{Синтаксический анализ вводимой информации}
    \begin{scnrelfromset}{ключевые возможности}
            \scnitem{Обработка и анализ текста: синтаксический и семантический анализ вводимого текста для понимания смысла и интента пользователя}
            \scnitem{Извлечение информации: способность системы извлекать конкретные факты, события, именованные сущности из текстовых источников}
            \scnitem{Классификация и категоризация: возможность классифицировать и категоризировать тексты по определенным классам или категориям}
            \scnitem{Анализ тональности и сентимента: определение тональности и сентимента текста, позволяющее оценить отношение автора к обсуждаемой теме}
    \end{scnrelfromset}
    \end{SCn}
\end{frame}

\begin{frame}{\\Естественно-языковой интерфейс}
	\topline
 	\begin{SCn} 
    \scnheader{Естественно-языковой интерфейс}
    \begin{scnrelfromset}{применение в различных областях}
            \scnitem{Мобильные устройства}
            \scnitem{Поисковые системы}
            \scnitem{Виртуальные ассистенты}
    \end{scnrelfromset}
    \end{SCn}
\end{frame}

\begin{frame}{\\Интеграция знаний}
	\topline
 	\begin{SCn}
    \scnheader{Интеграция знаний}
    \begin{scnrelfromset}{определение}
    \scnitem{Квазибинарное отношение, каждая пара которого связывает множество интегрируемых сущностей с результатом интеграции.*}
    \scnitem{Множество процессов интеграции множества заданных сущностей.}
    \end{scnrelfromset}
    \end{SCn}
\end{frame}

\begin{frame}{\\Интеграция знаний}
	\topline
 	\begin{SCn} 
    \scnheader{Интеграция знаний}
    \begin{scnrelfromset}{основные преимущества}
            \scnitem{Позволяет объединить информацию из разных источников, что приводит к более обширному и разнообразному набору знаний.}
            \scnitem{Помогает улучшить качество данных путем их проверки, очистки и устранения дубликатов.}
            \scnitem{Позволяет установить связи и отношения между различными элементами знаний.}
            \scnitem{упрощает поиск и извлечение информации путем объединения данных из различных источников в единый информационный ресурс.}
            \scnitem{Предоставляет более широкий набор данных для анализа и моделирования.}
    \end{scnrelfromset}
    \end{SCn}
\end{frame}

\begin{frame}{\\Интеграция знаний}
	\topline
 	\begin{SCn} 
    \scnheader{Интеграция знаний}
    \begin{scnrelfromset}{процесс интеграции знаний}
            \scnitem{Анализ источников знаний, которые требуется интегрировать.}
            \scnitem{Определяется модель данных и схема интеграции, которая будет использоваться для объединения знаний.}
            \scnitem{Создание связей и отношений между элементами знаний.}
            \scnitem{Выявление и разрешение конфликтов.}
            \scnitem{Обновление и поддержка.}
    \end{scnrelfromset}
    \end{SCn}
\end{frame}

\begin{frame}{\\Интеграция знаний}
	\topline
 	\begin{SCn} 
    \scnheader{Интеграция знаний}
    \begin{scnrelfromset}{применение в различных областях}
            \scnitem{Применяется в разработке систем искусственного интеллекта и алгоритмов машинного обучения.}
            \scnitem{Применяется для объединения данных из различных баз данных или информационных систем.}
            \scnitem{Используется для разработки экспертных систем и консультационных сервисов, которые предоставляют специализированную информацию и рекомендации в различных областях, например, юриспруденции, финансах или технике.}
    \end{scnrelfromset}
    \end{SCn}
\end{frame}

\begin{frame}{\\Модель понимания}
	\topline
 	\begin{SCn}
    \scnheader{Модель понимания}
    \begin{scnrelfromset}{определение}
	   \scnitem{это структура или формализованная репрезентация знаний, которая используется для понимания и обработки информации.}
    \end{scnrelfromset}
    \end{SCn}
\end{frame}

\begin{frame}{\\Модель понимания}
	\topline
 	\begin{SCn} 
    \scnheader{Модель понимания}
    \begin{scnrelfromset}{ключевые принципы модели понимания}
            \scnitem{Должна учитывать контекст информации, в котором она используется.}
            \scnitem{Должна быть основана на четкой и ясной семантике. Она должна определять понятия, отношения и правила, которые используются для представления знаний.}
            \scnitem{Должна иметь возможность интегрировать различные источники знаний.}
            \scnitem{Должна уметь работать с неопределенностью и нечеткостью в информации.}
            \scnitem{Способна улучшаться и совершенствоваться на основе обратной связи и новых данных.}
    \end{scnrelfromset}
    \end{SCn}
\end{frame}

\begin{frame}{\\Модель понимания}
	\topline
 	\begin{SCn} 
    \scnheader{Модель понимания}
    \begin{scnrelfromset}{применение в различных областях}
            \scnitem{Используется для анализа и интерпретации текстов на естественных языках.}
            \scnitem{Используется для создания интеллектуальных помощников и виртуальных ассистентов, которые могут взаимодействовать с пользователем на естественном языке.}
            \scnitem{Применяется для анализа больших объемов данных и извлечения ценных знаний.}
            \scnitem{Применяется для автоматизации различных процессов}
    \end{scnrelfromset}
    \end{SCn}
\end{frame}

