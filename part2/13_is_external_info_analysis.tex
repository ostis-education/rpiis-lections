\title{Лекция 13\\Анализ внешней информации в интеллектуальных системах \vspace{-2em}}   
\author[]{Шункевич Д.В.}
\institute[]{Белорусский государственный университет информатики и радиоэлектроники}

\begin{frame}
	\titlepage
\end{frame}

\begin{frame}{\\Содержание лекции}
	\topline
	\justifying
	Синтаксический анализ вводимой информации. Семантический анализ вводимой информации. Естественно-языковой интерфейс. Интеграция знаний. Модель понимания.
\end{frame}

\begin{frame}{\\Структура лекции}
	\topline
	\justifying

	\begin{SCn}
		\scnheader{Лекция 13. Анализ вводимой информации}
		\begin{scnrelfromset}{структура}
			\scnitem{Синтаксический анализ вводимой информации.}
			\scnitem{Семантический анализ вводимой информации.}
			\scnitem{Естественно-языковой интерфейс.}
			\scnitem{Интеграция знаний.}
                \scnitem{Модель понимания.}
		\end{scnrelfromset}
	\end{SCn}
\end{frame}

\begin{frame}{Синтаксический анализ\\ вводимой информации}
	\topline
 	\begin{SCn}
 		
    \scnheader{синтаксический анализ вводимой информации}
    \scntext{определение}{Это процесс анализа и понимания структуры вводимого текста с целью выявления синтаксических правил и отношений между словами.}
    
    \begin{scnrelfromset}{возможности}
        \scnfileitem{Направлен на понимание и интерпретирование синтаксических правил, которыми руководствуется текст.}
        \scnfileitem{Позволяет генерировать текст, имитирующий естественный язык.}
    \end{scnrelfromset}
    
    \end{SCn}
\end{frame}

\begin{frame}{Синтаксический анализ\\ вводимой информации}
	\topline
 	\begin{SCn} 
 	
 	\vspace{5mm}
 	\small	
    \scnheader{синтаксический анализ вводимой информации}
    \begin{scnrelfromset}{цели}
        \scnfileitem{Понимание синтаксической структуры: синтаксический анализ помогает понять грамматическую структуру предложений, введенных пользователем или полученных из других источников.}
        \scnfileitem{Обработка и понимание текста: синтаксический анализ помогает компьютерным системам обрабатывать и понимать введенную информацию на более высоком уровне.}
        \scnfileitem{Разрешение неоднозначностей: введенная информация может содержать неоднозначности, например, двусмысленности или различные возможные трактовки. }
        \scnfileitem{Улучшение качества обработки текста: синтаксический анализ может быть использован для улучшения качества различных задач обработки текста.}
        \scnfileitem{Поддержка автоматического анализа и обработки: синтаксический анализ является важным компонентом в области обработки естественного языка (NLP).}
	\end{scnrelfromset}
    
    \end{SCn}
\end{frame}

\begin{frame}{Синтаксический анализ\\ вводимой информации}
	\topline
 	\begin{SCn} 
    
    \scnheader{cинтаксический анализ вводимой информации}
    \begin{scnrelfromset}{применение}
        \scnfileitem{Применяетя в исследовании и разработке подсистем обработки естественно-языковых текстов.}
        \scnfileitem{Используется для анализа больших объемов текстовой информации, например, при поиске информации или анализе документов.}
        \scnfileitem{Играет важную роль в задаче машинного перевода и работы с многоязычной информацией.}
    \end{scnrelfromset}
    
    \end{SCn}
\end{frame}

\begin{frame}{Семантический анализ\\ вводимой информации}
	\topline
 	\begin{SCn}
 		
    \scnheader{семантический анализ вводимой информации}
    \scntext{определение}{\textit{\textbf{семантический анализ вводимой информации}} --- это процесс анализа и понимания смысловых аспектов текста, введенного пользователем или полученного из других источников.}
    \begin{scnrelfromset}{возможности}
        \scnfileitem{Направлен на понимание и интерпретирование значения слов, фраз и предложений в контексте.}
    \end{scnrelfromset}
    
    \end{SCn}
\end{frame}

\begin{frame}{Семантический анализ\\ вводимой информации}
	\topline
 	\begin{SCn} 
    \scnheader{семантический анализ вводимой информации}
    \begin{scnrelfromset}{цели}
        \scnfileitem{Стремится понять содержание вводимой информации, выявить основные темы, идеи, события или факты, которые содержатся в тексте.}
        \scnfileitem{Извлекает структурированную информацию из текста, такую как именованные сущности (имена, места, организации), даты, события и другие важные факты.}
        \scnfileitem{Стремится выявить отношения и связи между различными элементами текста, такими как субъекты, объекты, действия и атрибуты.}
        \scnfileitem{Используется для определения тональности или сентимента текста, то есть понимания, выражено ли в тексте положительное, отрицательное или нейтральное отношение к чему-либо.}
    \end{scnrelfromset}
    \end{SCn}
\end{frame}

\begin{frame}{Семантический анализ\\ вводимой информации}
	\topline
 	\begin{SCn}
 		 
    \scnheader{семантический анализ вводимой информации}
    \begin{scnrelfromset}{применение}
        \scnfileitem{Используется для улучшения результатов поиска, позволяя системе понять смысл запроса пользователя и предоставить наиболее релевантные результаты.}
        \scnfileitem{Является важной составляющей многих задач NLP, таких как извлечение информации, вопросно-ответные системы, чат-боты и многое другое. Анализ семантики текста позволяет системам более точно понимать запросы пользователей и обрабатывать текстовую информацию.}
    \end{scnrelfromset}
    \end{SCn}
\end{frame}

\begin{frame}{\\Естественно-языковой интерфейс}
	\topline
 	\begin{SCn}
 		
    \scnheader{естественно-языковой интерфейс}
    \scntext{определение}{SILK-интерфейс (Speech – речь, Image – образ, Language – язык, Knowledge – знание), обмен информацией между компьютерной системой и пользователем в котором происходит засчёт диалога. Диалог ведётся на одном из естественных языков.}
    
    \scnsuperset{речевой интерфейс}
    \begin{scnindent}
    	\scntext{определение}{SILK-интерфейс, обмен информацией в котором происходит за счёт диалога, в процессе которого компьютерная система и пользователь общаются с помощью речи. Данный вид интерфейса наиболее приближен к естественному общению между людьми.}
    \end{scnindent}
    
    \end{SCn}
\end{frame}

\begin{frame}{\\Естественно-языковой интерфейс}
	\topline
 	\begin{SCn}
 		
    \scnheader{естественно-языковой интерфейс}
    \begin{scnrelfromset}{возможности}
        \scnfileitem{Предоставляет возможность взаимодействия с компьютерной системой с помощью естественного языка, а не специализированных команд или языков программирования.}
    \end{scnrelfromset}
    
    \end{SCn}
\end{frame}

\begin{frame}{\\Естественно-языковой интерфейс}
	\topline
 	\begin{SCn} 
    \scnheader{естественно-языковой интерфейс}
    \begin{scnrelfromset}{основные преимущества}
        \scnfileitem{Позволяет пользователю взаимодействовать с системой так же, как он общается с другими людьми, без необходимости изучать специализированные команды или языки программирования.}
        \scnfileitem{Позволяет общаться с компьютером на естественном языке, что делает его доступным для широкого круга пользователей.}
        \scnfileitem{Позволяет пользователям более быстро и эффективно выполнять задачи.}
        \scnfileitem{Позволяет системе лучше понять контекст и смысл вводимой информации.}
    \end{scnrelfromset}
    \end{SCn}
\end{frame}

\begin{frame}{\\Естественно-языковой интерфейс}
	\topline
 	\begin{SCn} 
 		
    \scnheader{синтаксический анализ вводимой информации}
    \begin{scnrelfromset}{ключевые возможности}
        \scnfileitem{Обработка и анализ текста: синтаксический и семантический анализ вводимого текста для понимания смысла и интента пользователя.}
        \scnfileitem{Извлечение информации: способность системы извлекать конкретные факты, события, именованные сущности из текстовых источников.}
        \scnfileitem{Классификация и категоризация: возможность классифицировать и категоризировать тексты по определенным классам или категориям.}
        \scnfileitem{Анализ тональности и сентимента: определение тональности и сентимента текста, позволяющее оценить отношение автора к обсуждаемой теме.}
    \end{scnrelfromset}
    
    \end{SCn}
\end{frame}

\begin{frame}{\\Естественно-языковой интерфейс}
	\topline
 	\begin{SCn}
 		 
    \scnheader{синтаксический анализ вводимой информации}
    \begin{scnrelfromset}{ключевые возможности}
        \scnfileitem{Позволяет определить структуру вводимой информации, выявляя синтаксические отношения между словами и фразами.}
        \scnfileitem{Помогает обнаружить ошибки вводимой информации, такие как грамматические ошибки, неправильный порядок слов, недостающие или лишние элементы в предложении.}
        \scnfileitem{Помогает извлечь синтаксическую информацию из вводимого текста.}
        \scnfileitem{Помогает понять синтаксическую структуру текста и сделать более точные выводы на основе этой информации.}
    \end{scnrelfromset}
    
    \end{SCn}
\end{frame}

\begin{frame}{\\Естественно-языковой интерфейс}
	\topline
 	\begin{SCn} 
 		
    \scnheader{семантический анализ вводимой информации}
    \begin{scnrelfromset}{ключевые возможности}
        \scnfileitem{Помогает понять смысл вводимой информации, а не только ее формальную структуру.}
        \scnfileitem{Позволяет сопоставлять вводимую информацию с предварительно определенными знаниями или моделями.}
        \scnfileitem{Позволяет разрешать неоднозначность вводимой информации.}
        \scnfileitem{Используется для анализа настроений и сентимента вводимой информации.}
    \end{scnrelfromset}
    
    \end{SCn}
\end{frame}

\begin{frame}{\\Естественно-языковой интерфейс}
	\topline
 	\begin{SCn} 
 		
    \scnheader{синтаксический анализ вводимой информации}
    \begin{scnrelfromset}{ключевые возможности}
        \scnfileitem{Обработка и анализ текста: синтаксический и семантический анализ вводимого текста для понимания смысла и интента пользователя.}
        \scnfileitem{Извлечение информации: способность системы извлекать конкретные факты, события, именованные сущности из текстовых источников.}
        \scnfileitem{Классификация и категоризация: возможность классифицировать и категоризировать тексты по определенным классам или категориям.}
        \scnfileitem{Анализ тональности и сентимента: определение тональности и сентимента текста, позволяющее оценить отношение автора к обсуждаемой теме.}
    \end{scnrelfromset}
    
    \end{SCn}
\end{frame}

\begin{frame}{\\Естественно-языковой интерфейс}
	\topline
 	\begin{SCn} 
 		
    \scnheader{естественно-языковой интерфейс}
    \begin{scnrelfromset}{применение}
        \scnfileitem{Мобильные устройства}
        \scnfileitem{Поисковые системы}
        \scnfileitem{Виртуальные ассистенты}
	\end{scnrelfromset}
    
    \end{SCn}
\end{frame}

\begin{frame}{\\Интеграция знаний}
	\topline
 	\begin{SCn}
 		
    \scnheader{интеграция знаний}
    \begin{scnrelfromset}{определение}
	    \scnfileitem{Квазибинарное отношение, каждая пара которого связывает множество интегрируемых сущностей с результатом интеграции.}
	    \scnfileitem{Множество процессов интеграции множества заданных сущностей.}
	\end{scnrelfromset}
    
    \end{SCn}
\end{frame}

\begin{frame}{\\Интеграция знаний}
	\topline
 	\begin{SCn} 
    \scnheader{интеграция знаний}
    \begin{scnrelfromset}{основные преимущества}
            \scnfileitem{Позволяет объединить информацию из разных источников, что приводит к более обширному и разнообразному набору знаний.}
            \scnfileitem{Помогает улучшить качество данных путем их проверки, очистки и устранения дубликатов.}
            \scnfileitem{Позволяет установить связи и отношения между различными элементами знаний.}
            \scnfileitem{упрощает поиск и извлечение информации путем объединения данных из различных источников в единый информационный ресурс.}
            \scnfileitem{Предоставляет более широкий набор данных для анализа и моделирования.}
    \end{scnrelfromset}
    \end{SCn}
\end{frame}

\begin{frame}{\\Интеграция знаний}
	\topline
 	\begin{SCn} 
    \scnheader{интеграция знаний}
    \begin{scnrelfromvector}{этапы процесса}
            \scnfileitem{Анализ источников знаний, которые требуется интегрировать.}
            \scnfileitem{Определяется модель данных и схема интеграции, которая будет использоваться для объединения знаний.}
            \scnfileitem{Создание связей и отношений между элементами знаний.}
            \scnfileitem{Выявление и разрешение конфликтов.}
            \scnfileitem{Обновление и поддержка.}
    \end{scnrelfromvector}
    \end{SCn}
\end{frame}

\begin{frame}{\\Интеграция знаний}
	\topline
 	\begin{SCn} 
    \scnheader{интеграция знаний}
    \begin{scnrelfromset}{применение}
            \scnfileitem{Применяется в разработке систем искусственного интеллекта и алгоритмов машинного обучения.}
            \scnfileitem{Применяется для объединения данных из различных баз данных или информационных систем.}
            \scnfileitem{Используется для разработки экспертных систем и консультационных сервисов, которые предоставляют специализированную информацию и рекомендации в различных областях, например, юриспруденции, финансах или технике.}
    \end{scnrelfromset}
    \end{SCn}
\end{frame}

\begin{frame}{\\Модель понимания}
	\topline
 	\begin{SCn}
 		
    \scnheader{модель понимания}
    \scntext{определение}{Структура или формализованная репрезентация знаний, которая используется для понимания и обработки информации.}
    
    \end{SCn}
\end{frame}

\begin{frame}{\\Модель понимания}
	\topline
 	\begin{SCn} 
    \scnheader{модель понимания}
    \begin{scnrelfromset}{ключевые принципы}
        \scnfileitem{Должна учитывать контекст информации, в котором она используется.}
        \scnfileitem{Должна быть основана на четкой и ясной семантике. Она должна определять понятия, отношения и правила, которые используются для представления знаний.}
        \scnfileitem{Должна иметь возможность интегрировать различные источники знаний.}
        \scnfileitem{Должна уметь работать с неопределенностью и нечеткостью в информации.}
        \scnfileitem{Способна улучшаться и совершенствоваться на основе обратной связи и новых данных.}
    \end{scnrelfromset}
    \end{SCn}
\end{frame}

\begin{frame}{\\Модель понимания}
	\topline
 	\begin{SCn} 
    \scnheader{модель понимания}
    \begin{scnrelfromset}{применение}
        \scnfileitem{Используется для анализа и интерпретации текстов на естественных языках.}
        \scnfileitem{Используется для создания интеллектуальных помощников и виртуальных ассистентов, которые могут взаимодействовать с пользователем на естественном языке.}
        \scnfileitem{Применяется для анализа больших объемов данных и извлечения ценных знаний.}
        \scnfileitem{Применяется для автоматизации различных процессов.}
    \end{scnrelfromset}
    \end{SCn}
\end{frame}

