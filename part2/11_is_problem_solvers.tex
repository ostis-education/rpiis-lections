\title{Лекция 11\\Решатели задач интеллектуальных систем \vspace{-2em}}   
\author[]{Шункевич Д.В.}
\institute[]{Белорусский государственный университет информатики и радиоэлектроники}

\begin{frame}
	\titlepage
\end{frame}

\begin{frame}{\\Содержание лекции}
	\topline
	\justifying
\begin{SCn}
	\scnheader{Структура лекции}

	\begin{scnrelfromset}{разбиение}
				\scnitem{Модели решения задач при отсутствии хранимых способов их решения}
				\scnitem{Правдоподобные рассуждения}
				\scnitem{Машина обработки знаний интеллектуальной системы как многоагентная система}
                \scnitem{Понятие многоагентной системы}
	\end{scnrelfromset}
				
\end{SCn}
\end{frame}

\begin{frame}{\\Модели решения задач при отсутствии хранимых способов их решения}
\topline
\justifying
\begin{SCn}
	\scnheader{Решатель задач}

	\scnidtf{программные инструменты, которые используются для автоматического решения различных задач.}

\end{SCn}
\end{frame}

\begin{frame}{\\Модели решения задач при отсутствии хранимых способов их решения}
\topline
\justifying
\begin{SCn}
	\scnheader{Решатели задач в интеллектуальных системах}

	\begin{scnrelfromset}{особенности применения}
				\scnitem{Использование решателей задач позволяет автоматизировать процессы и уменьшить затраты на работу экспертов}
				\scnitem{Решатели задач могут обрабатывать большие объемы данных, оперативно получать необходимую информацию и делать выводы}
                \scnitem{Использование решателей задач позволяет создавать и сравнивать различные модели для решения задач, оптимизируя их и повышая эффективность}
			\end{scnrelfromset}

\end{SCn}
\end{frame}

\begin{frame}{\\Модели решения задач при отсутствии хранимых способов их решения}
\topline
\justifying
\begin{SCn}
	\scnheader{Решатели задач в интеллектуальных системах}

	\begin{scnrelfromset}{особенности применения}
                \scnitem{Решатели задач могут быть интегрированы с другими системами, что позволяет использовать их в комплексе с другими приложениями}
                \scnitem{Решатели задач позволяют повысить точность решения задач, благодаря использованию различных алгоритмов и методов, а также улучшению качества данных}
			\end{scnrelfromset}

\end{SCn}
\end{frame}

\begin{frame}{\\Модели решения задач при отсутствии хранимых способов их решения}
\topline
\justifying
\begin{SCn}
	\scnheader{Модели решения задач}

	\begin{scnrelfromset}{разбиение}
				\scnitem{Процедурный подход}
				\scnitem{Модель символьных вычислений}
				\scnitem{Модель машинного обучения}
                \scnitem{Модель решения задач на основе знаний}
                \scnitem{Модель решения на основе сведении задачи к подзадачам}
                \scnitem{Модель эволюционных алгоритмов}
                \scnitem{Модель решения на основе переформулировки задачи}
	\end{scnrelfromset}

\end{SCn}
\end{frame}

\begin{frame}{\\Модели решения задач при отсутствии хранимых способов их решения}
\topline
\justifying
\begin{SCn}
	\scnheader{Процедурный подход}

	\scntext{пояснение}{Заключается в том, что для каждой задачи разрабатывается отдельная процедура решения, которая может быть реализована на компьютере. Такой подход осуществляется с помощью алгоритмического программирования.}

    \scnheader{Модель символьных вычислений}

    \scntext{пояснение}{Математические операции, выполненные на символах вместо чисел. В этой модели используется набор математических методов и алгоритмов для решения задач.}

\end{SCn}
\end{frame}

\begin{frame}{\\Модели решения задач при отсутствии хранимых способов их решения}
\topline
\justifying
\begin{SCn}
    \scnheader{Модель машинного обучения}

    \scntext{пояснение}{Модель основана на использовании алгоритмов машинного обучения для поиска решения задачи.}

    \scnheader{Модель решения задач на основе знаний}

    \scntext{пояснение}{Модель используется для задач, которые требуют знаний эксперта. Знания о задачах хранятся в базе знаний, и система использует их для решения задач.}

    \scnheader{Модель решения на основе сведении задачи к подзадачам}

    \scntext{пояснение}{Свести исходную задачу к семейству подзадач, для которых методы их решения в текущий момент известны.}

\end{SCn}
\end{frame}

\begin{frame}{\\Модели решения задач при отсутствии хранимых способов их решения}
\topline
\justifying
\begin{SCn}
    \scnheader{Модель эволюционных алгоритмов}

    \scntext{пояснение}{Модель основана на использовании принципов эволюционной биологии для решения задач. Алгоритмы живут, размножаются и эволюционируют, изменяясь, чтобы улучшить свою производительность и точность.}

    \scnheader{Модель решения на основе переформулировки задачи}

    \scntext{пояснение}{Переформулировать задачу, то есть сгенерировать (логически вывести) логически эквивалентную формулировку исходной задачи, для которой метод её решения в текущий момент является известным.}

\end{SCn}
\end{frame}

\begin{frame}{\\Правдоподобные рассуждения}
\topline
\justifying
\begin{SCn}
	\scnheader{Правдоподобные рассуждения}

	\scnidtf{логические выкладки, основанные на анализе предоставленной информации, чтобы определить, насколько вероятными являются различные факты и события.}

\end{SCn}
\end{frame}

\begin{frame}{\\Правдоподобные рассуждения}
\topline
\justifying
\begin{SCn}
	\scnheader{Правдоподобные рассуждения}

	\begin{scnrelfromset}{сфера использования}
                \scnitem{Используются для выявления логических связей между различными фактами и предложениями, чтобы помочь системе принять правильное решение}
                \scnitem{Могут быть направлены на проверку качества информации, которую система получила, чтобы определить, насколько вероятными являются различные факты и предположения, используемые в процессе принятия решения}
                \scnitem{Используются чтобы помочь системе проверить логическую последовательность рассуждений, которые она выполняет, и убедиться, что ее выводы соответствуют логическим правилам и человеческой интуиции}
			\end{scnrelfromset}

\end{SCn}
\end{frame}

\begin{frame}{\\Правдоподобные рассуждения}
\topline
\justifying
\begin{SCn}
	\scnheader{Правдоподобные рассуждения}

	\scntext{примечание}{В решателе задач, который занимается диагностикой медицинских заболеваний, правдоподобные рассуждения может быть использованы для выявления логических связей между различными симптомами и возможными диагнозами. Система может использовать базу данных с информацией о медицинских случаях, чтобы статистически анализировать, какие симптомы чаще всего связаны с определенными диагнозами. На основании этих данных система может сделать правдоподобные выводы и дополнительно попросить пользователя предоставить дополнительную информацию, в том числе о тех симптомах, которые система не учла.}

\end{SCn}
\end{frame}

\begin{frame}{\\Машина обработки знаний интеллектуальной системы как многоагентная система}
\topline
\justifying
\begin{SCn}
	\scnheader{Машина обработки знаний}

	\scnidtf{компонент системы, который отвечает за управление и обработку знаний из разных источников.}

    \scntext{пояснение}{Под машиной обработки знаний будем понимать совокупность интерпретаторов всех навыков, составляющих некоторый решатель задач.}

\end{SCn}
\end{frame}

\begin{frame}{\\Машина обработки знаний интеллектуальной системы как многоагентная система}
\topline
\justifying
\begin{SCn}
	\scnheader{Машина обработки знаний}

	\begin{scnrelfromset}{сфера использования}
                \scnitem{Используется для создания и управления базой знаний, которая хранит и организует информацию в логическом порядке для удобства поиска и использования}
			\end{scnrelfromset}

   \scntext{примечание}{База знаний может быть общей для всей системы и использоваться различными агентами в процессе решения задач.}

\end{SCn}
\end{frame}

\begin{frame}{\\Машина обработки знаний интеллектуальной системы как многоагентная система}
\topline
\justifying
\begin{SCn}
	\scnheader{Машина обработки знаний}

   \scntext{примечание}{В многоагентной системе, каждый агент может иметь свою собственную базу знаний, которая может быть объединена с другими агентами для решения общей задачи. Каждый агент может использовать распределенную обработку для эффективного обмена информацией и координации действий.Таким образом, машина обработки знаний как многоагентная система может существенно повысить эффективность и точность решателей задач интеллектуальных систем.}

\end{SCn}
\end{frame}

\begin{frame}{\\Многоагентная система}
\topline
\justifying
\begin{SCn}
	\scnheader{Многоагентная система}

    \scnsubset{Кибернетическая система}

   \scnidtf{коллектив взаимодействующих автономных кибернетических систем, имеющих общую среду обитания (жизнедеятельности).}

   \scntext{пояснение}{Кибернетическая система, представляющая собой множество кибернетических систем, способных коммуницировать, т.е. обмениваться информацией друг с другом (причем не обязательно каждый с каждым)}

\end{SCn}
\end{frame}

\begin{frame}{\\Многоагентная система}
\topline
\justifying
\begin{SCn}
	\scnheader{Многоагентная система}

   \begin{scnrelfromset}{разбиение}
                \scnitem{многоагентная система без общей памяти}
                \scnitem{многоагентная система с общей памятью}
			\end{scnrelfromset}

   \begin{scnrelfromset}{разбиение}
                \scnitem{многоагентная система, в которой управление агентами осуществляется только путем обмена сообщениями между ними}
                \scnitem{многоагентная система, в которой управление агентами осуществляется через общую для них память}
			\end{scnrelfromset}

\end{SCn}
\end{frame}

\begin{frame}{\\Многоагентная система}
\topline
\justifying
\begin{SCn}
	\scnheader{Многоагентная система}

   \begin{scnrelfromset}{разбиение}
                \scnitem{многоагентная система с централизованным управлением агентами}
                \scnitem{многоагентная система с децентрализованным управлением агентами}
			\end{scnrelfromset}

   \begin{scnrelfromset}{разбиение}
                \scnitem{многоагентная система, в которой областью деятельности всех ее агентов является только внешняя среда этой системы}
                \scnitem{многоагентная система, в которой областью деятельности ее агентов является как внешняя среда, так и память этой системы}
			\end{scnrelfromset}

\end{SCn}
\end{frame}

\begin{frame}{\\Многоагентная система}
\topline
\justifying
\begin{SCn}
	\scnheader{Многоагентная система}

   \begin{scnrelfromset}{разбиение}
                \scnitem{коллектив из простых кибернетических систем}
                \scnitem{коллектив из индивидуальных кибернетических систем}
                \scnitem{коллектив из индивидуальных и простых кибернетических систем}
			\end{scnrelfromset}

   \begin{scnrelfromset}{разбиение}
                \scnitem{одноуровневый коллектив кибернетических систем}
                \scnitem{иерархический коллектив кибернетических систем}
			\end{scnrelfromset}

\end{SCn}
\end{frame}

\begin{frame}{\\Многоагентная система}
\topline
\justifying
\begin{SCn}
	\scnheader{Коллектив индивидуальных кибернетических систем}

    \scnidtf{многоагентная система, агентами (членами) которой являются индивидуальные кибернетические системы.}

    \scntext{примеры}{коллективы людей, коллективы компьютерных систем и людей.}
    
\end{SCn}
\end{frame}

\begin{frame}{\\Многоагентная система}
\topline
\justifying
\begin{SCn}
	\scnheader{Одноуровневый коллектив кибернетических систем}

    \scnidtf{многоагентная система, агентами которой не могут быть многоагентные системы.}

    \scnidtf{специализированное средство решения задач, реализующее либо одну модель параллельного (распределенного) решения задач соответствующего класса, либо комбинацию фиксированного числа разных и параллельно реализованных моделей решения задач.}

   \scnheader{Иерархический коллектив кибернетических систем}

    \scnidtf{многоагентная система, по крайней мере одним агентом которой является многоагентная система}

\end{SCn}
\end{frame}

\begin{frame}
	
	\begin{center}
		\begin{LARGE}
		\textbf{Спасибо за внимание!}
		\end{LARGE}

		Остались ли вопросы?
	\end{center}
\end{frame}