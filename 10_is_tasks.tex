\title{Лекция 10\\Задачи в интеллектуальных системах \vspace{-2em}}   
\author[]{Шункевич Д.В.}
\institute[]{Белорусский государственный университет информатики и радиоэлектроники}

\begin{frame}
	\titlepage
\end{frame}

\begin{frame}{\\Содержание лекции}
	\topline
	\justifying
	Вопросы и информационные задачи. Поведенческие цели и поведенческие задачи. Информационно-поисковые задачи. Интеллектуализация информационного поиска. Трудноформализуемые задачи. Задачи, для которых отсутствует четкая постановка. Классы задач и хранимые (интерпретируемые) способы их решения – программы. Типология хранимых программ и их интерпретаторов.
\end{frame}

\begin{frame}{\\Содержание лекции}
	\topline
	\justifying
    \begin{scnrelfromset}{Разбиение}
        \scnitem{Вопросы и информационные задачи}
        \scnitem{Поведенческие цели и поведенческие задачи}
        \scnitem{Информационно-поисковые задачи}
        \scnitem{Интеллектуализация информационного поиска}
        \scnitem{Трудноформализуемые задачи}
        \scnitem{Задачи, для которых отсутствует четкая постановка}
        \scnitem{Классы задач и хранимые (интерпретируемые) способы их решения – программы}
        \scnitem{Типология хранимых программ и их интерпретаторов}
    \end{scnrelfromset}
\end{frame}

\begin{frame}{\\Информационная задача}
	\topline
	\justifying
	
	\begin{SCn}
		\scnheader{информационная задача}
		\scnidtf{предполагает необходимость получения какой-либо информации.}
		\begin{scnrelfromset}{включение}
			\scnitem{задача на вычисление.}
			\scnitem{задача на установление истинности атомарного высказывания.}
			\scnitem{задача на установление истинности неатомарного высказывания.}
			\scnitem{задача на анализ.}
		\end{scnrelfromset}
	\end{SCn}
\end{frame}

\begin{frame}{\\Задачи на вычисление}
	\topline
	\justifying
	
	\begin{SCn}
		\scnheader{задачи на вычисление}
		\scnidtf{Задачи данного класса предполагают, что требуется узнать значение какой-либо величины.}
	\end{SCn}
\end{frame}

\begin{frame}{\\Задачи на установление истинности атомарного высказывания}
	\topline
	\justifying
	
	\begin{SCn}
		\scnheader{атомарное высказывание}
		\scnidtf{высказывание, не содержащее логических связок, таких как импликация, конъюнкция и т.д., и кванторов.}
        \begin{scnrelfromset}{пример}
			\scnitem{два указанных треугольника конгруэнтны}
			\scnitem{бобр занесен в красную книгу Республики Беларусь}
		\end{scnrelfromset}
	\end{SCn}
\end{frame}

\begin{frame}{\\Задачи на установление истинности неатомарного высказывания}
	\topline
	\justifying
	
	\begin{SCn}
		\scnheader{неатомарное высказывание}
		\scnidtf{высказывание, состоящее из одного и более атомарного высказывания и удовлетворяющее хотя бы одному из двух условий.}
        \begin{scnrelfromset}{условия}
			\scnitem{атомарные высказывания в рамках неатомарного связаны логическими отношениями}
			\scnitem{на все высказывание или его часть накладывается какой-либо квантор}
		\end{scnrelfromset}
	\end{SCn}
\end{frame}

\begin{frame}{\\Задачи на анализ}
	\topline
	\justifying
	
	\begin{SCn}
		\scnheader{задачи на анализ}
		\scnidtf{задачи, которые необходимо решить для ответов на вопросы «Как связаны два указанных понятия», «В чем различие между указанными объектами» и т.д}
        \end{SCn}
        Поиск ответа на такого рода вопросы часто ограничивается простым поиском в базе знаний, однако не менее часто возникают ситуации, когда для ответа необходимо сгенерировать новые знания на основе имеющихся
\end{frame}

\begin{frame}{\\Поведенческая задача}
	\topline
	\justifying
	
	\begin{SCn}
		\scnheader{поведенческая задача}
		\scnidtf{предполагает наличие некоторой среды, в которой находится агент, призванный решить данную задачу.}
	\end{SCn}
 Задача агента состоит в том, чтобы преобразовать окружающую среду из начального состояния в конечное, используя при этом допустимые операции преобразования из имеющегося множества
\end{frame}

\begin{frame}{\\Поведенческая задача}
	\topline
	\justifying
	
	\begin{SCn}
		\begin{scnrelfromset}{пример}
			\scnitem{физические задачи}
			\scnitem{геометрические задачи}
			\scnitem{химические задачи}
		\end{scnrelfromset}
	\end{SCn}
\end{frame}

\begin{frame}{\\Поведенческая цель}
	\topline
	\justifying
	
	\begin{SCn}
		\scnheader{поведенческая цель}
		\scnidtf{конечное состояние среды, в которой находится агент, призванный решить поведенческую задачу.}
	\end{SCn}
\end{frame}

\begin{frame}{\\Информационно-поисковые задачи}
	\topline
	\justifying
	
	\begin{SCn}
		\scnheader{информационно-поисковые задачи}
		\scnidtf{способность кибернетической системы находить в текущем состоянии хранимой информации релевантные ответы на запросы (вопросы) самого различного вида.}
        \scnheader{вопрос}
        \scnidtf{запрос}
        \scnsuperset{запрос изоморфных или гомоморфных фрагментов хранимой информации по заданному образцу с указанием знаков известных сущностей}
	\end{SCn}
\end{frame}

\begin{frame}{\\Информационно-поисковые задачи}
	\topline
	\justifying
	
	\begin{SCn}
		\scnsuperset{вопрос типа "как связаны между собой заданные две сущности"{}}
  \begin{scnrelfromset}{пояснение}
			\scnitem{Две сущности будем считать связанными в том и только в том случае, если существует маршрут, соединяющий указанные две сущности, в состав которого входят связки, принадлежащие в общем случаем разным отношениям}
		\end{scnrelfromset}
  \begin{scnrelfromset}{примечание}
			\scnitem{Здесь принципиально важным является учет \textit{семантической силы связей} между сущностями, которая определяется \textit{семантической силой отношений}, которым принадлежат связки, входящие в состав связей (маршрутов) между сущностями.}
		\end{scnrelfromset}
	\end{SCn}
\end{frame}

\begin{frame}{\\Информационно-поисковые задачи}
	\topline
	\justifying
	
	\begin{SCn}
		\scnsuperset{вопрос типа "что это такое"{}}
     \scnidtf{запрос спецификации (описания) заданной сущности}
    \scnrelfrom{класс частных вопросов}{запрос определения}
    \scnidtf{запрос определения заданного понятия}
	\end{SCn}
\end{frame}

\begin{frame}{\\Информационно-поисковые задачи}
	\topline
	\justifying
	
	\begin{SCn}
		\scnsuperset{почему-вопрос}
            \scnidtf{запрос объяснения корректности заданного высказывания, которое, в частности, может быть порождено (сгенерировано) в процессе решения некоторой задачи с помощью некоторого метода (алгоритма, искусственной нейронной сети логического исчисления и т.п.)}
            \scnsuperset{запрос доказательства заданной теоремы}
	\end{SCn}
\end{frame}

\begin{frame}{\\Интеллектуализация информационного поиска}
	\topline
 	\begin{SCn}
    \scnheader{Интеллектуализация информационного поиска}
    \begin{scnrelfromset}{определение}
	   \scnitem{это процесс применения различных технологий и методов искусственного интеллекта для улучшения качества и эффективности информационного поиска.}
    \end{scnrelfromset}
    \begin{scnrelfromset}{цель}
	   \scnitem{Цель состоит в том, чтобы понять запрос пользователя, интерпретировать его и предоставить наиболее релевантные результаты.}
    \end{scnrelfromset}
    \end{SCn}
\end{frame}

\begin{frame}{\\Интеллектуализация информационного поиска}
	\topline
 	\begin{SCn} 
    \scnheader{Интеллектуализация информационного поиска}
    \begin{scnrelfromset}{причины}
            \scnitem{при получении пользователем большого объема информации в результате автоматизированного поиска много времени затрачивается на ее просмотр и выбор, в то время как даже простой выбор необходимой информации зачастую представляет собой нелегкую проблему}
            \scnitem{выбор информации, осуществляемый человеком, нередко не является рациональным и строго последовательным, что сущест­венно осложняет поиск информации}
            \scnitem{пользователь при поиске информации обычно не строго определяет цель поиска, т. е. использует нечетко определенные понятия}
    \end{scnrelfromset}
    \end{SCn}
\end{frame}

\begin{frame}{\\Интеллектуализация информационного поиска}
	\topline
 	\begin{SCn} 
    \scnheader{Интеллектуализация информационного поиска}
    \begin{scnrelfromset}{преимущества}
            \scnitem{Системы могут использовать алгоритмы машинного обучения и анализ данных для предоставления наиболее релевантных результатов пользователю.}
            \scnitem{Интеллектуализация позволяет расширить функциональность поиска, включая фильтрацию, классификацию, анализ и другие функции.}
            \scnitem{Системы могут учитывать предыдущие действия и предпочтения пользователя для предоставления персонализированных результатов.}
    \end{scnrelfromset}
    \end{SCn}
\end{frame}

\begin{frame}{\\Трудноформализуемые задачи}
	\topline
 	\begin{SCn}
    \scnheader{трудноформализуемые задачи}
    \scnidtf{интеллектуальные задачи}
    \begin{scnrelfromset}{Определение}
	   \scnitem{это задачи, для которых сложно разработать точный и эффективный алгоритм решения, или для которых нет известного полиномиального алгоритма решения.}
    \end{scnrelfromset}
    \end{SCn}
\end{frame}

\begin{frame}{\\Трудноформализуемые задачи}
	\topline
 	\begin{SCn}
    \scnheader{трудноформализуемые задачи}
    \scnidtf{признаки трудноформализуемых задач}
\begin{scnrelfromset}{признаки}
    \scnitem{неточность и недостоверность исходных данных;}
    \scnitem{отсутствие критерия качества результата;}
    \scnitem{невозможность или высокая трудоемкость разработки алгоритма;}
    \scnitem{необходимость учета контекста задачи;}
\end{scnrelfromset}
    \end{SCn}
\end{frame}

\begin{frame}{\\Трудноформализуемые задачи}
	\topline
 	\begin{SCn}
        \scnheader{задача, предполагающая использование информации, обладающей различного рода не-факторами}
        \scnidtf{трудноформализуемая задача}
        \scnsuperset{задача проектирования}
        \scnsuperset{задача распознавания}
        \scnsuperset{задача прогнозирования}
        \scnsuperset{задача целеполагания}
        \scnsuperset{задача планирования}
    \end{SCn}
\end{frame}

\begin{frame}{\\Трудноформализуемые задачи}
	\topline
 	\begin{SCn}
    \scnheader{задача проектирования}
    \scnidtf{интеллектуальная задача}
    \begin{scnrelfromset}{Определение}
	   \scnitem{разработка решения для определенной проблемы или потребности. Она включает анализ требований, исследование возможных решений, создание моделей или прототипов, а также оценку и тестирование разработанных решений.}
    \end{scnrelfromset}
    \begin{scnrelfromset}{пример}
	   \scnitem{проектирование архитектуры ПО}
	   \scnitem{проектирование зданий}
    \end{scnrelfromset}
    \end{SCn}
\end{frame}

\begin{frame}{\\Трудноформализуемые задачи}
	\topline
 	\begin{SCn}
    \scnheader{задача распознавания}
    \scnidtf{интеллектуальная задача}
    \begin{scnrelfromset}{Определение}
	   \scnitem{определение или классификация объектов, сигналов или данных на основе предоставленных признаков или характеристик.}
    \end{scnrelfromset}
    \begin{scnrelfromset}{пример}
	   \scnitem{распознавание речи}
	   \scnitem{распознавание объектов на изображении}
	   \scnitem{распознавание текста}
    \end{scnrelfromset}
    \end{SCn}
\end{frame}

\begin{frame}{\\Трудноформализуемые задачи}
	\topline
 	\begin{SCn}
    \scnheader{задача прогнозирования}
    \scnidtf{интеллектуальная задача}
    \begin{scnrelfromset}{Определение}
	   \scnitem{предсказание будущих событий, состояний или значений на основе доступных данных и моделей.}
    \end{scnrelfromset}
    \begin{scnrelfromset}{пример}
	   \scnitem{прогнозирование погоды}
	   \scnitem{прогнозирование цен}
    \end{scnrelfromset}
    \end{SCn}
\end{frame}

\begin{frame}{\\Трудноформализуемые задачи}
	\topline
 	\begin{SCn}
    \scnheader{задача целеполагания}
    \scnidtf{интеллектуальная задача}
    \begin{scnrelfromset}{Определение}
	   \scnitem{определение целей или задач, которые должны быть достигнуты в определенной области или ситуации.}
    \end{scnrelfromset}
    \begin{scnrelfromset}{пример}
	   \scnitem{определение приоритетов в проекте}
	   \scnitem{установление целей}
    \end{scnrelfromset}
    \end{SCn}
\end{frame}

\begin{frame}{\\Трудноформализуемые задачи}
	\topline
 	\begin{SCn}
    \scnheader{задача планирования}
    \scnidtf{интеллектуальная задача}
    \begin{scnrelfromset}{Определение}
	   \scnitem{определение последовательности действий, ресурсов и временных рамок для достижения поставленных целей или выполнения определенной задачи.}
    \end{scnrelfromset}
    \begin{scnrelfromset}{пример}
	   \scnitem{планирование в проекте}
	   \scnitem{планирование производственных операций}
    \end{scnrelfromset}
    \end{SCn}
\end{frame}

\begin{frame}{\\Класс задач}
	\topline
 	\begin{SCn}
    \scnheader{класс задач}
    \scnidtf{множество аналогичных задач}
    \scnidtf{множество задач, для которого можно построить обобщенную формулировку задач, соответствующую
    всему этому множеству задач}
    \begin{scnrelfromset}{примечание}
	   \scnitem{Каждая обобщенная формулировка задач       
        соответствующего класса по сути есть не что иное, как строгое
        логическое определение указанного класса задач.}
    \end{scnrelfromset}
    \begin{scnrelfromset}{семейство подмножеств}
	   \scnitem{задача}
    \end{scnrelfromset}
    \end{SCn}
\end{frame}

\begin{frame}{\\Хранимые в памяти методы решения задач}
	\topline
 	\begin{SCn}
    \scnheader{хранимые методы решения задач}
    \scnidtf{интепретируемые методы решения задач}
    \begin{scnrelfromset}{разбиение}
	   \scnitem{методы верхнего уровня – интерпретируемые методы;}
	   \scnitem{методы базового уровня, представленные на базовом языке программирования, который интерпретируется непосредственно процессором кибернетической системы;}
	   \scnitem{метаметоды, описывающие интерпретацию методов верхнего уровня.}
    \end{scnrelfromset}
    \end{SCn}
\end{frame}

\begin{frame}{\\Интепретатор хранимых программ}
	\topline
 	\begin{SCn}
    \scnheader{интепретатор хранимых программ}
    \scnidtf{процессор кибернетической системы}
    \begin{scnrelfromset}{определение}
	   \scnitem{физически (аппаратно реализованный) интерпретатор хранимых в памяти кибернетической системы методов
(программ), соответствующих базовой (для данной кибернетической системы) модели решения задач, т.е.
такой модели решения задач, которая для данной кибернетической системы является моделью решения
задач самого нижнего уровня и, следовательно, не может быть интерпретирована с помощью другой модели
решения задач, используемой этой же кибернетической системой, а может быть проинтерпретирована либо
путем аппаратной реализации такого интерпретатора, либо путём его программной реализации.}
    \end{scnrelfromset}
    \end{SCn}
\end{frame}