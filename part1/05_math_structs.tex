\title{Лекция 5\\Представление и обработка математических структур \vspace{-2em}}   
%\author[]{Шункевич Д.В.}
%\institute[]{Белорусский государственный университет информатики и радиоэлектроники}

\begin{frame}
	\titlepage
\end{frame}

\begin{frame}{\\Содержание лекции}
	\topline
	\justifying
	
	\begin{SCn}
	\scnheader{Лекция 5. Представление и обработка математических структур}
    \begin{scnrelfromset}{структура}
        \scnitem{Классификация средств обработки числовой информации}
        \scnitem{Представление и обработка чисел в компьютере}
        \scnitem{Представление основных математических структур в памяти компьютерных систем}
        \scnitem{Оценка временной сложности алгоритмов}
    \end{scnrelfromset}
    \end{SCn}
    
\end{frame}


\begin{frame}{\\Средства обработки числовой информации}
	\topline
	\justifying
 	\begin{SCn}
    \scnheader{Программные средства обработки числовой информации}
    \begin{scnrelfromset}{разбиение}
        \scnitem{Электронные калькуляторы}
        \scnitem{Табличные процессоры}
        \scnitem{Базы данных}
        \scnitem{Пакеты прикладных программ}
    \end{scnrelfromset}
    \end{SCn}
\end{frame}

\begin{frame}{\\Средства обработки числовой информации}
	\topline
	\justifying
	\newline
 	\begin{SCn}
 		
    \scnheader{табличный процессор}
	\scntext{пояснение}{\textbf{Табличный процессор} -- программа обработки электронных таблиц}
    \scntext{примечание}{Первую программа этого класса VisiCalc создали в 1979 г. студенты Гарварда Д. Бриклин и Б.Франкстон}    
    \begin{scnrelfromset}{современные представители}
        \scnitem{MS Excel}
        \scnitem{OpenOffice.org Calc}
        \scnitem{Google Spreadsheets (http://spreadsheets.google.com)}
    \end{scnrelfromset}
    \end{SCn}
\end{frame}


\begin{frame}{\\Средства обработки числовой информации}
	\topline
 	\begin{SCn}
 	\footnotesize
    \scnheader{база данных}
	\scntext{пояснение}{\textbf{База данных} -- большой массив однородной информации, в том числе и числовой, организованный специальным образом.}

    \scnheader{система управления базами данных}
    \scnidtf{CУБД}
    \scntext{пояснение}{\textbf{СУБД} -- программное средство, обеспечивающее целостность хранимых данных, доступ к ним.}
    \begin{scnrelfromset}{примеры}
    	\scnitem{Oracle}
    	\scnitem{MySQL}
    	\scnitem{MS SQL Server}
		\scnitem{PostgreSQL}
		\scnitem{Neo4j}
    \end{scnrelfromset}
    
    \end{SCn}
\end{frame}

\begin{frame}{\\Средства обработки числовой информации}
	\topline
 	\begin{SCn}
 		
    \scnheader{пакет прикладных программ}
    \scntext{пояснение}{\textbf{Пакеты прикладных команд} -- наборы программных приложений, предназначенных для выполнения конкретных задач в рамках определенной области деятельности.}
    \scntext{пояснение}{Используются специалистами в различных предметных областях для решения соответствующих прикладных задач.}
    \scntext{принцип}{Принцип функциональной полноты}
    \begin{scnindent}
    	\scntext{пояснение}{Пакет обеспечивает решение всех задач некого класса.}
	\end{scnindent}    	
    \end{SCn}
\end{frame}

\begin{frame}{\\Средства обработки числовой информации}
	\topline
 	\begin{SCn}
 		
    \scnheader{пакеты прикладных программ}
    \begin{scnrelfromset}{примеры}
        \scnitem{Statistica -- пакет статистического анализа}
        \scnitem{Maple -- система компьютерной алгебры}
        \scnitem{MATLAB -- пакет для задач технических вычислений}
       	\begin{scnindent}
        \begin{scnrelfromset}{достоинства}
	        \scnfileitem{большое количество математических функций}
	        \scnfileitem{свой язык программирования}
	        \scnfileitem{ряд расширений (toolbox) для решения задач}
	        \begin{scnindent}
		        \begin{scnrelfromset}{примеры}
		            \scnfileitem{обработка сигналов}
		            \scnfileitem{анализ нейронных сетей}
		        \end{scnrelfromset}
			\end{scnindent}
		\end{scnrelfromset}
		\end{scnindent}	
    \end{scnrelfromset}
    \end{SCn}
\end{frame}


\begin{frame}{\\Представление и обработка чисел}
	\topline
 	\begin{SCn}
 		
    \scnheader{представление чисел в ЭВМ}
    \begin{scnrelfromset}{разбиение}
        \scnitem{представление целых чисел}
        \scnitem{представление вещественных чисел}
	    \begin{scnindent}
        \begin{scnrelfromset}{разбиение}
            \scnitem{представление вещественных чисел с фиксированной запятой}
            \scnitem{представление вещественных чисел с плавающей запятой}
        \end{scnrelfromset}
		\end{scnindent}
    \end{scnrelfromset}
    \end{SCn}
\end{frame}

\begin{frame}{\\Представление и обработка чисел}
    \topline
    \begin{SCn}
        \scnheader{представление целых чисел}
        \scntext{пояснение}{Целые числа в ЭВМ представляются в двоичном коде, используя фиксированное количество битов (обычно от 8 до 64, в зависимости от архитектуры процессора). Каждый бит может иметь два состояния -- 0 или 1 -- что позволяет представить $2^n$ различных значений, где $n$ -- количество битов.}
    \end{SCn}
\end{frame}

\begin{frame}{\\Представление и обработка чисел}
    \topline
    \begin{SCn}
	
	\scnheader{Представление вещественных чисел с фиксированной запятой}
	\scntext{пример}{Если мы используем 8 битов для представления числа, то мы можем выделить, скажем, 4 бита для целой части и 4 бита для дробной.}
	\scntext{пояснение}{Для представления вещественных чисел с фиксированной запятой используется формат Q=m.n, где m обозначает количество битов, выделенных для целой части числа, а n -- количество битов, выделенных для дробной части.}
	
    \end{SCn}
\end{frame}

\begin{frame}{\\Представление и обработка чисел}
    \topline
    \begin{SCn}
        \scnheader{Представление вещественных чисел с плавающей запятой}
        \scntext{пояснение}{Формат представления вещественных чисел в компьютерах, где число представляется в виде мантиссы, экспоненты и знака. Мантисса представляет собой дробное число с фиксированным количеством битов, экспонента определяет порядок числа, а знак указывает на его положительность или отрицательность.}
        \scntext{пояснение}{формат задается стандартом IEEE 754, который определяет различные размеры мантиссы и экспоненты, а также способы обработки переполнения и неопределенных значений.}
    \end{SCn}
\end{frame}

\begin{frame}{\\Представление и обработка чисел}
    \topline
    \begin{SCn}
        \scnheader{Представление вещественных чисел с плавающей запятой}
           \begin{figure}[h]
          \centering
          \includegraphics[width=1\textwidth]{part1/images/plav_tochka.png}
    \end{figure}
    \end{SCn}
\end{frame}


\begin{frame}{Представление основных математических структур в памяти компьютерных систем}
    \topline
    \begin{SCn}
        \scnheader{Основные математические структуры, которые могут храниться в памяти компьютерных систем}
        \begin{scneqtoset}
            \scnitem{числа}
            \scnitem{массивы}
            \scnitem{строки}
            \scnitem{структуры данных}
            \scnitem{множества}
            \scnitem{графы}
        \end{scneqtoset}
    \end{SCn}
\end{frame}

\begin{frame}{Представление основных математических структур в памяти компьютерных систем}
    \topline
    \begin{SCn}
        \scnheader{Хранение массива как математической структуры в памяти компьютерной системы}
        \begin{scnrelfromlist}{пояснение}
            \scnfileitem{Массив представляет собой упорядоченный набор элементов одного типа, расположенных в памяти компьютерной системы последовательно друг за другом.}
            \scnfileitem{Для хранения массива в памяти компьютера, выделяется блок памяти фиксированного размера, достаточного для хранения всех элементов массива.}
            \scnfileitem{Каждый элемент массива имеет свой индекс, который указывает на его позицию в массиве. Индексы начинаются с нуля для первого элемента и продолжаются до последнего элемента массива.}
        \end{scnrelfromlist}
    \end{SCn}
\end{frame}

\begin{frame}{Представление основных математических структур в памяти компьютерных систем}
    \topline
    \begin{SCn}
        \scnheader{Хранение массива как математической структуры в памяти компьютерной системы}
            \begin{scnrelfromset}{разбиение по способу хранения в памяти}
            \scnfileitem{\textbf{Статический}: размер массива определяется в момент его создания, и он не может быть изменен в процессе работы программы}
            \scnfileitem{\textbf{Динамический}: размер может изменяться в процессе выполнения программы}
            \end{scnrelfromset}
    \end{SCn}
\end{frame}

\begin{frame}{Представление основных математических структур в памяти компьютерных систем}
    \topline
    \begin{SCn}
        \scnheader{Хранение строки как математической структуры в памяти компьютерной системы}
        \begin{scnrelfromlist}{пояснение}
	        \scnfileitem{Строка в компьютерной системе представляет собой упорядоченный набор символов, где каждый символ хранится в памяти компьютера в виде кода символа, например, в кодировке ASCII, Unicode или другой.}
	        \scnfileitem{Для хранения строки в памяти компьютера используется последовательность ячеек памяти, где каждая ячейка хранит код одного символа.}
        \end{scnrelfromlist}
    \end{SCn}
\end{frame}

\begin{frame}{Представление основных математических структур в памяти компьютерных систем}
    \topline
    \begin{SCn}
        \scnheader{Хранение множеств как математической структуры в памяти компьютерной системы}
        \scntext{пояснение}{Множество в математике представляет собой коллекцию уникальных элементов без порядка. В компьютерной системе множество может быть представлено в виде структуры данных, которая содержит уникальные элементы.}
        
        \scnheader{Способы хранения множеств в памяти компьютера}
        \begin{scneqtoset}
            \scnitem{массив с флагами}
            \scnitem{двоичное дерево поиска}
            \scnitem{хеш-таблица}
        \end{scneqtoset}
    
    \end{SCn}
\end{frame}

\begin{frame}{Представление основных математических структур в памяти компьютерных систем}
    \topline
    \newline
    \begin{SCn}
        \scnheader{Хранение графа как математической структуры в памяти компьютерной системе}
        \scntext{пояснение}{Граф в математике -- совокупность вершин и ребер, связывающих эти вершины. В компьютерной системе граф может быть представлен в виде структуры данных, которая содержит вершины и ребра, связывающие эти вершины.}
        
        \scnheader{Способы хранения графов в памяти компьютера}
        \begin{scneqtoset}
            \scnitem{список смежности}
            \scnitem{матрица смежности}
            \scnitem{список ребер}
            \scnitem{список инцидентности}
            \scnitem{матрица инцидентности}
        \end{scneqtoset}

    \end{SCn}
\end{frame}

\begin{frame}{Представление основных математических структур в памяти компьютерных систем}
    \topline
    \begin{SCn}
        \scnheader{список смежности}
        \scntext{пояснение}{В этом подходе каждая вершина графа представлена списком вершин, с которыми она связана. Для каждого ребра хранится информация о связанных вершинах и весе ребра. Операции добавления и удаления вершин и ребер выполняются за время $O(1)$, но поиск определенного ребра может потребовать времени $O(E)$, где $E$ -- количество ребер в графе.}
    \end{SCn}
\end{frame}

\begin{frame}{Представление основных математических структур в памяти компьютерных систем}
    \topline
    \begin{SCn}
        \scnheader{матрица смежности}
        \scntext{пояснение}{В этом подходе граф представлен в виде матрицы размера $n x n$, где $n$ -- количество вершин в графе. Если вершины $i$ и $j$ связаны ребром, то элемент матрицы $A(i,j)$ равен весу ребра, в противном случае он равен $0$ или бесконечности. Операции поиска, добавления и удаления ребер выполняются за время $O(1)$, но операции поиска вершины могут потребовать времени $O(n)$.}
    \end{SCn}
\end{frame}


\begin{frame}{Представление основных математических структур в памяти компьютерных систем}
    \topline
    \begin{SCn}
        \scnheader{список ребер}
        \scntext{пояснение}{В этом подходе граф представлен списком ребер, где каждое ребро содержит информацию о связанных вершинах и весе ребра. Операции поиска, добавления и удаления ребер выполняются за время $O(1)$, но поиск вершин может потребовать времени $O(E)$.}
    \end{SCn}
\end{frame}

\begin{frame}{Представление основных математических структур в памяти компьютерных систем}
    \topline
    \begin{SCn}
        \scnheader{список инцидентности}
        \scntext{пояснение}{В этом подходе каждая вершина графа представлена списком ребер, которые инцидентны данной вершине. Для каждого ребра хранится информация о связанных вершинах и весе ребра. Операции добавления и удаления вершин и ребер выполняются за время $O(1)$, но поиск определенного ребра может потребовать времени $O(E)$, где $E$ -- количество ребер в графе.}
    \end{SCn}
\end{frame}

\begin{frame}{Представление основных математических структур в памяти компьютерных систем}
    \topline
    \begin{SCn}
        \scnheader{матрица инцидентности}
        \scntext{пояснение}{В этом подходе граф представлен в виде матрицы размера $n x m$, где $n$ -- количество вершин в графе, $m$ -- количество ребер в графе. Если ребро $i$ инцидентно вершинам $j$ и $k$, то элемент матрицы $A(j,i)$ и $A(k,i)$ равны соответственно $1$ и $-1$ (или наоборот), а остальные элементы равны $0$.}
    \end{SCn}
\end{frame}


\begin{frame}{\\Оценка временной сложности алгоритмов}
    \topline
    \begin{SCn}
    	
        \scnheader{оценка временной сложности алгоритмов}
       	\scnidtf{процесс определения того, как изменяется время выполнения алгоритма при увеличении размера входных данных.}
        \scntext{пояснение}{Оценка временной сложности обычно производится путем анализа количества операций, которые должен выполнить алгоритм для обработки заданных входных данных.}
        \scntext{пояснение}{Сложность алгоритма обычно описывается с помощью O-нотации (<<О большое>>), которая указывает на оценку сверху алгоритмической сложности.}
        \begin{scnindent}
        	\scntext{пример}{Если алгоритм имеет временную сложность $O(n^2)$, то время выполнения алгоритма увеличивается пропорционально квадрату размера входных данных.}
        \end{scnindent}
            
    \end{SCn}
\end{frame}