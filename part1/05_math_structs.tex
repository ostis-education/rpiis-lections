\title{Лекция 5\\Представление и обработка математических структур \vspace{-2em}}   
\author[]{Шункевич Д.В.}
\institute[]{Белорусский государственный университет информатики и радиоэлектроники}

\begin{frame}
	\titlepage
\end{frame}

\begin{frame}{\\Содержание лекции}
	\topline
	\justifying
	История развития и классификация средств обработки числовой информации. Представление и обработка чисел в компьютере. Представление основных математических структур в памяти компьютерных систем. Оценка временной сложности алгоритмов.
\end{frame}



\begin{frame}{\\Содержание лекции}
	\topline
	\justifying
    \begin{scnrelfromset}{Разбиение}
        \scnitem{Классификация средств обработки числовой информации}
        \scnitem{Представление и обработка чисел в компьютере}
        \scnitem{Представление основных математических структур в памяти компьютерных систем}
        \scnitem{Оценка временной сложности алгоритмов}
    \end{scnrelfromset}
\end{frame}


\begin{frame}{\\Средства обработки числовой информации}
	\topline
 	\begin{SCn}
    \scnheader{Программные средства обработки числовой информации}
    \begin{scnrelfromset}{Разбиение}
        \scnitem{Электронные калькуляторы;}
        \scnitem{Табличные процессоры;}
        \scnitem{Базы данных;}
        \scnitem{Пакеты прикладных программ}
    \end{scnrelfromset}
    \end{SCn}
\end{frame}

\begin{frame}{\\Средства обработки числовой информации}
	\topline
 	\begin{SCn}
    \scnheader{Табличный процессор}
    
    \begin{scnrelfromset}{Определение}
        \scnitem{программа обработки электронных таблиц}
    \end{scnrelfromset}

    \begin{scnrelfromset}{Пояснение}
        \scnitem{Первую программа этого класса VisiCalc создали в 1979 г. студенты Гарварда Д. Бриклин и Б.Франкстон}
    \end{scnrelfromset}
    \scnheader{Современные аналоги}
    \begin{scnrelfromset}{Разбиение}
        \scnitem{MS Excel;}
        \scnitem{OpenOffice.org Calc;}
        \scnitem{Google Spreadsheets (http://spreadsheets.google.com)}
    \end{scnrelfromset}
    \end{SCn}
\end{frame}


\begin{frame}{\\Средства обработки числовой информации}
	\topline
 	\begin{SCn}
    \scnheader{База данных}
    \begin{scnrelfromset}{Определение}
        \scnitem{большой массив однородной информации, в том числе и числовой, организованный специальным образом.}
    \end{scnrelfromset}

    \scnheader{Система управления базой данных (CУБД)}
    \begin{scnrelfromset}{Определение}
        \scnitem{программное средство, обеспечивающее целостность хранимых данных, доступ к ним.}
    \end{scnrelfromset}
    \end{SCn}
\end{frame}

\begin{frame}{\\Средства обработки числовой информации}
	\topline
 	\begin{SCn}
    \scnheader{Примеры СУБД}
    \begin{scnrelfromset}{Разбиение}
        \scnitem{Oracle;}
        \scnitem{mySQL;}
        \scnitem{IBM DB2.}
    \end{scnrelfromset}
    \end{SCn}
\end{frame}


\begin{frame}{\\Средства обработки числовой информации}
	\topline
 	\begin{SCn}
    \scnheader{Пакеты прикладных программ}
    \begin{scnrelfromset}{Определение}
        \scnitem{наборы программных приложений, предназначенных для выполнения конкретных задач в рамках определенной области деятельности.}
    \end{scnrelfromset}

    \begin{scnrelfromset}{Пояснение}
        \scnitem{Используются специалистами в различных предметных областях для решения соответствующих прикладных задач.}
    \end{scnrelfromset}

    \scnheader{Принцип функциональной полноты}
    \begin{scnrelfromset}{Пояснение}
        \scnitem{Пакет обеспечивает решение всех задач некого класса.}
    \end{scnrelfromset}
    \end{SCn}
\end{frame}

\begin{frame}{\\Средства обработки числовой информации}
	\topline
 	\begin{SCn}
    \scnheader{Пакеты прикладных программ}
    \begin{scnrelfromset}{Примеры}
        \scnitem{Statistica – пакет статистического анализа;}
        \scnitem{Maple – система компьютерной алгебры;}
        \scnitem{MATLAB – пакет для задач технических вычислений:}
        \begin{scnrelfromset}{Разбиение}
        \scnitem{большое количество математических функций;}
        \scnitem{свой язык программирования;}
        \scnitem{ряд расширений (toolbox) для решения задач:}
        \begin{scnrelfromset}{Разбиение}
            \scnitem{обработка сигналов;}
            \scnitem{анализ нейронных сетей.}
        \end{scnrelfromset}
            
        \end{scnrelfromset}
    \end{scnrelfromset}
    \end{SCn}
\end{frame}


\begin{frame}{\\Представление и обработка чисел}
	\topline
 	\begin{SCn}
    \scnheader{Представление чисел в ЭВМ}
    \begin{scnrelfromset}{Разбиение}
        \scnitem{Представление целых значений;}
        \scnitem{Представление вещественных чисел:}
        \begin{scnrelfromset}{Представление вещественных чисел:}
            \scnitem{Представление вещественных чисел с фиксированной запятой;}
            \scnitem{Представление вещественных чисел с плавающей запятой.}
        \end{scnrelfromset}
    \end{scnrelfromset}
    \end{SCn}
\end{frame}

\begin{frame}{\\Представление и обработка чисел}
    \topline
    \begin{SCn}
        \scnheader{Представление целых значений}
        \begin{scnrelfromset}{Пояснение}
            \scnitem{Целые числа в ЭВМ представляются в двоичном коде, используя фиксированное количество битов (обычно от 8 до 64, в зависимости от архитектуры процессора). Каждый бит может иметь два состояния - 0 или 1 - что позволяет представить $2^n$ различных значений, где $n$ - количество битов.}
        \end{scnrelfromset}
    \end{SCn}
\end{frame}

\begin{frame}{\\Представление и обработка чисел}
    \topline
    \begin{SCn}
        \scnheader{Представление вещественных чисел с фиксированной запятой}
        \begin{scnrelfromset}{Пояснение}
            \scnitem{Представление вещественных чисел с фиксированной запятой}
            \begin{scnrelfromset}{Пример}
                \scnitem{если мы используем 8 битов для представления числа, то мы можем выделить, скажем, 4 бита для целой части и 4 бита для дробной}
            \end{scnrelfromset}
            \scnitem{Для представления вещественных чисел с фиксированной запятой используется формат Qm.n, где m обозначает количество битов, выделенных для целой части числа, а n - количество битов, выделенных для дробной части}
        \end{scnrelfromset}
    \end{SCn}
\end{frame}

\begin{frame}{\\Представление и обработка чисел}
    \topline
    \begin{SCn}
        \scnheader{Представление вещественных чисел с плававющей запятой}
        \begin{scnrelfromset}{Определение}
            \scnitem{формат представления вещественных чисел в компьютерах, где число представляется в виде мантиссы, экспоненты и знака. Мантисса представляет собой дробное число с фиксированным количеством битов, экспонента определяет порядок числа, а знак указывает на его положительность или отрицательность.}
        \end{scnrelfromset}
        \begin{scnrelfromset}{Пояснение}
            \scnitem{формат задается стандартом IEEE 754, который определяет различные размеры мантиссы и экспоненты, а также способы обработки переполнения и неопределенных значений }
        \end{scnrelfromset}
    \end{SCn}
\end{frame}

\begin{frame}{\\Представление и обработка чисел}
    \topline
    \begin{SCn}
        \scnheader{Представление вещественных чисел с плававющей запятой}
           \begin{figure}[h]
          \centering
          \includegraphics[width=1\textwidth]{part1/images/plav_tochka.png}
    \end{figure}
    \end{SCn}
\end{frame}


\begin{frame}{\\Представление основных математических структур в памяти компьютерных систем}
    \topline
    \begin{SCn}
        \scnheader{Основные математические структуры, которые могут храниться в памяти компьютерных систем}
        \begin{scnrelfromset}{Разбиение}
            \scnitem{Числа}
            \scnitem{Массивы}
            \scnitem{Строки}
            \scnitem{Структуры данных}
            \scnitem{Множества}
            \scnitem{Графы}
        \end{scnrelfromset}
    \end{SCn}
\end{frame}

\begin{frame}{\\Представление основных математических структур в памяти компьютерных систем}
    \topline
    \begin{SCn}
        \scnheader{Хранение массива как математической структуры в памяти компьютерной системы}
        \begin{scnrelfromset}{Пояснение}
            \scnitem{Массив представляет собой упорядоченный набор элементов одного типа, расположенных в памяти компьютерной системы последовательно друг за другом. }
            \scnitem{Для хранения массива в памяти компьютера, выделяется блок памяти фиксированного размера, достаточного для хранения всех элементов массива.}
            \scnitem{Каждый элемент массива имеет свой индекс, который указывает на его позицию в массиве. Индексы начинаются с нуля для первого элемента и продолжаются до последнего элемента массива.}
        \end{scnrelfromset}
    \end{SCn}
\end{frame}

\begin{frame}{\\Представление основных математических структур в памяти компьютерных систем}
    \topline
    \begin{SCn}
        \scnheader{Хранение массива как математической структуры в памяти компьютерной системы}
        \begin{scnrelfromset}{Пояснение}
            \begin{scnrelfromset}{Разбиение по способу хранения в памяти}
            \scnitem{Статически: размер массива определяется в момент его создания, и он не может быть изменен в процессе работы программы}
            \scnitem{Динамический:  размер может изменяться в процессе выполнения программы}
                
            \end{scnrelfromset}
        \end{scnrelfromset}
    \end{SCn}
\end{frame}

\begin{frame}{\\Представление основных математических структур в памяти компьютерных систем}
    \topline
    \begin{SCn}
        \scnheader{Хранение строки как математической структуры в памяти компьютерной системы}
        \begin{scnrelfromset}{Пояснение}
        \scnitem{Строка в компьютерной системе представляет собой упорядоченный набор символов, где каждый символ хранится в памяти компьютера в виде кода символа, например, в кодировке ASCII, Unicode или другой.}
        \scnitem{Для хранения строки в памяти компьютера используется последовательность ячеек памяти, где каждая ячейка хранит код одного символа.}
        \end{scnrelfromset}
    \end{SCn}
\end{frame}

\begin{frame}{\\Представление основных математических структур в памяти компьютерных систем}
    \topline
    \begin{SCn}
        \scnheader{Хранение множеств как математической структуры в памяти компьютерной системы}
        \begin{scnrelfromset}{Пояснение}
        \scnitem{Множество в математике представляет собой коллекцию уникальных элементов без порядка. В компьютерной системе множество может быть представлено в виде структуры данных, которая содержит уникальные элементы.}
        \scnheader{способы хранения множеств в памяти компьютера}
        \begin{scnrelfromset}{Разбиение}
            \scnitem{Массив с флагами}
            \scnitem{Двоичное дерево поиска}
            \scnitem{Хеш-таблица}
        \end{scnrelfromset}
        \end{scnrelfromset}
    \end{SCn}
\end{frame}

\begin{frame}{\\Представление основных математических структур в памяти компьютерных систем}
    \topline
    \begin{SCn}
        \scnheader{Хранение графа как математической структуры в памяти компьютерной системе }
        \begin{scnrelfromset}{Пояснение}
        \scnitem{Граф в математике представляет собой совокупность вершин и ребер, связывающих эти вершины. В компьютерной системе граф может быть представлен в виде структуры данных, которая содержит вершины и ребра, связывающие эти вершины.}
        \scnheader{способы хранения графов в памяти компьютера}
        \begin{scnrelfromset}{Разбиение}
            \scnitem{Список смежности}
            \scnitem{Матрица смежности}
            \scnitem{Список ребер}
            \scnitem{Список инцидентности}
            \scnitem{Матрица инцидентности}
        \end{scnrelfromset}
        \end{scnrelfromset}
    \end{SCn}
\end{frame}

\begin{frame}{\\Представление основных математических структур в памяти компьютерных систем}
    \topline
    \begin{SCn}
        \scnheader{Список смежности}
        \begin{scnrelfromset}{Пояснение}
        \scnitem{ В этом подходе каждая вершина графа представлена списком вершин, с которыми она связана. Для каждого ребра хранится информация о связанных вершинах и весе ребра. Операции добавления и удаления вершин и ребер выполняются за время O(1), но поиск определенного ребра может потребовать времени O(E), где E - количество ребер в графе.}
        \end{scnrelfromset}
    \end{SCn}
\end{frame}

\begin{frame}{\\Представление основных математических структур в памяти компьютерных систем}
    \topline
    \begin{SCn}
        \scnheader{Матрица смежности}
        \begin{scnrelfromset}{Пояснение}
        \scnitem{В этом подходе граф представлен в виде матрицы размера n x n, где n - количество вершин в графе. Если вершины i и j связаны ребром, то элемент матрицы A(i,j) равен весу ребра, в противном случае он равен 0 или бесконечности. Операции поиска, добавления и удаления ребер выполняются за время O(1), но операции поиска вершины могут потребовать времени O(n).}
        \end{scnrelfromset}
    \end{SCn}
\end{frame}


\begin{frame}{\\Представление основных математических структур в памяти компьютерных систем}
    \topline
    \begin{SCn}
        \scnheader{Список ребер}
        \begin{scnrelfromset}{Пояснение}
        \scnitem{В этом подходе граф представлен списком ребер, где каждое ребро содержит информацию о связанных вершинах и весе ребра. Операции поиска, добавления и удаления ребер выполняются за время O(1), но поиск вершин может потребовать времени O(E).}
        \end{scnrelfromset}
    \end{SCn}
\end{frame}

\begin{frame}{\\Представление основных математических структур в памяти компьютерных систем}
    \topline
    \begin{SCn}
        \scnheader{Список инцидентности}
        \begin{scnrelfromset}{Пояснение}
        \scnitem{В этом подходе каждая вершина графа представлена списком ребер, которые инцидентны данной вершине. Для каждого ребра хранится информация о связанных вершинах и весе ребра. Операции добавления и удаления вершин и ребер выполняются за время O(1), но поиск определенного ребра может потребовать времени O(E), где E - количество ребер в графе.}
        \end{scnrelfromset}
    \end{SCn}
\end{frame}

\begin{frame}{\\Представление основных математических структур в памяти компьютерных систем}
    \topline
    \begin{SCn}
        \scnheader{Матрица инцидентности}
        \begin{scnrelfromset}{Пояснение}
        \scnitem{В этом подходе граф представлен в виде матрицы размера n x m, где n - количество вершин в графе, m - количество ребер в графе. Если ребро i инцидентно вершинам j и k, то элемент матрицы A(j,i) и A(k,i) равны соответственно 1 и -1 (или наоборот), а остальные элементы равны 0.}
        \end{scnrelfromset}
    \end{SCn}
\end{frame}


\begin{frame}{\\Оценка временной сложности алгоритмов}
    \topline
    \begin{SCn}
        \scnheader{Оценка временной сложности алгоритмов}
        \begin{scnrelfromset}{Определение}
        \scnitem{процесс определения того, как изменяется время выполнения алгоритма при увеличении размера входных данных.}
        \end{scnrelfromset}
        \begin{scnrelfromset}{Пояснение}
            \scnitem{Оценка временной сложности обычно производится путем анализа количества операций, которые должен выполнить алгоритм для обработки заданных входных данных.}
            
        \end{scnrelfromset}
    \end{SCn}
\end{frame}


\begin{frame}{\\Оценка временной сложности алгоритмов}
    \topline
    \begin{SCn}
        \scnheader{Пояснение}
        \begin{scnrelfromset}{Определение}
        \scnitem{Сложность алгоритма обычно описывается с помощью O-нотации («О большое»), которая указывает на оценку сверху алгоритмической сложности.}
        \end{scnrelfromset}
        \begin{scnrelfromset}{Пример}
            \scnitem{Если алгоритм имеет временную сложность $O(n^2)$, то время выполнения алгоритма увеличивается пропорционально квадрату размера входных данных.}
        \end{scnrelfromset}
    \end{SCn}
\end{frame}









