\title{Лекция 2\\Информационные технологии: основные понятия \vspace{-2em}}   
\author[]{Шункевич Д.В.}
\institute[]{Белорусский государственный университет информатики и радиоэлектроники}

\begin{frame}
	\titlepage
\end{frame}

\begin{frame}{\\Содержание лекции}
	\topline
	\justifying
    \scnheader{Лекция 2. Информационные технологии: основные понятия}
    \begin{scnrelfromset}{разбиение}
        \scnitem{Информационные технологии}
        \scnitem{Понятие информации и ее свойства}
        \scnitem{Модель информационного обмена}
        \scnitem{Представление и обработка информации.}
    \end{scnrelfromset}
\end{frame}


\begin{frame}{\\Информационные технологии}
    \topline
    \justifying
    \begin{SCn}
    \scnheader{Информационная технология}
    \scnidtf{совокупность средств и методов сбора, накопления, обработки и передачи данных (первичной информации) для получения информации нового качества о состоянии объекта, процесса или явления (информационного продукта).}
    \end{SCn}
\end{frame}


\begin{frame}{\\Информационные технологии}
    \topline
    \justifying
	\begin{SCn}
    \scnheader{Информационная технология}
    \begin{scnrelfromset}{цель}
	   \scnitem{в результате целенаправленных действий по переработке первичной информации получить необходимую для пользователя информацию.}
    \end{scnrelfromset}
    \begin{scnrelfromset}{примеры}
	   \scnitem{наскальная живопись}
        \scnitem{руны}
        \scnitem{сказки}
        \scnitem{книги}
        \scnitem{каталоги}
    \end{scnrelfromset}
    \end{SCn}
\end{frame}

\begin{frame}{\\Информационные технологии}
    \topline
    \justifying
    \begin{SCn}
    \scnheader{Компьютерная информационная технология(КИТ)}
    \scnidtf{совокупность способов, методов и соответствующих программных и аппаратных средств, обеспечивающих \textbf{сбор, хранение, обработку, управление и передачу} информации согласно требованиям пользователей.}
    \end{SCn}
\end{frame}

\begin{frame}{\\Информационные технологии}
    \topline
    \justifying
	\begin{SCn}
    \scnheader{Компьютерная информационная технология(КИТ)}
    \begin{scnrelfromset}{составные части}
	  \scnitem{\begin{scnindent}{
                \textbf{Исходные данные\\}}
                \scnidtf{информация, представленная в каком-то виде.}
                \end{scnindent}
            }
        \scnitem{
            \begin{scnindent}{
            \textbf{Средства, используемые для работы с данными} \\}
            \scnidtf{ аппаратное и программное обеспечение.}
            \end{scnindent}
            }
        \scnitem{\textbf{Правила, приемы или алгоритмы обработки данных.}}
        \scnitem{
        \begin{scnindent}{
        \textbf{Результат} \\}
        \scnidtf{ информация, представленная в требуемом виде.}
        \end{scnindent}}
    \end{scnrelfromset}
    \end{SCn}
\end{frame}

\begin{frame}{\\Информационные технологии}
    \topline
    \justifying
	\begin{SCn}
    \scnheader{Компьютерная информационная технология(КИТ)}
    \begin{scnrelfromset}{структура}
	  \scnitem{\begin{scnindent}{\textbf{Аппаратное обеспечение}\\} \scnidtf{ компьютер и подключаемые устройства.}\end{scnindent}}
        \scnitem{\begin{scnindent}{\textbf{Программное обеспечение} \\} \scnidtf{ совокупность компьютерные программ, используемые для решения задачи.}\end{scnindent}}
        \scnitem{\begin{scnindent}{\textbf{Математическое обеспечение} \\} \scnidtf{ разделы математики, на основе которых разработаны алгоритмы, реализованные в виде программного обеспечения.}\end{scnindent}}
        \scnitem{\textbf{Информационное обеспечение.}}
    \end{scnrelfromset}
    \end{SCn}
\end{frame}

\begin{frame}{\\Аппаратное обеспечение}
    \topline
    \justifying
	\begin{SCn}
    \scnheader{Аппаратное обеспечение}
        \scnhaselement{Компьютеры}
        \scnhaselement{Подключаемые устройства (мышь, сканер, принтер)}
        \scnhaselement{Сетевое оборудование (концентраторы, кабели и т.д.)}
    \end{SCn}
\end{frame}

\begin{frame}{\\Аппаратное обеспечение}
    \topline
    \justifying
	\begin{SCn}
        \scnheader{Аппаратное обеспечение}
        \begin{scnrelfromset}{разбиение по условиям эксплуатации}
            \scnitem{Универсальные (офисные)}
            \scnitem{Специализированные}
        \end{scnrelfromset}
    \end{SCn}
\end{frame}

\begin{frame}{\\Аппаратное обеспечение}
    \topline
    \justifying
	\begin{SCn}
        \scnheader{Аппаратное обеспечение}
        \begin{scnrelfromset}{разбиение по производительности}
            \scnitem{Суперкомпьютеры (\,  более 100 терафлопс )\,}
            \scnitem{\begin{scnindent}{
                    Персональные компьютеры (\, примерно 70 гигафлопс )\,}
                    \begin{scnrelfromset}{разбиение}
                        \scnitem{настольные}
                        \scnitem{персоные}
                    \end{scnrelfromset}
                    \end{scnindent}}
            \scnitem{карманные компьтеры (\, 1-2 мегафлопс)\,}
        \end{scnrelfromset}
        \scnrelfrom{пояснение}
            {[1 флопс - Floating point Operations Per Second – число операций с плавающей точкой в секунду]}
    \end{SCn}
\end{frame}

\begin{frame}{\\Программное обеспечение}
    \topline
    \justifying
    \begin{SCn}
    \scnheader{Программное обеспечение}
    \scnidtf{совокупность программ обработки данных и необходимых для их эксплуатации документов. }
    \end{SCn}
\end{frame}

\begin{frame}{\\Программное обеспечение}
    \topline
    \justifying
	\begin{SCn}
        \scnheader{Программное обеспечение}
        \begin{scnrelfromset}{разбиение по назначению}
            \scnitem{Системное программное обеспечение}
            \scnitem{Прикладное программное обеспечение}
            \scnitem{Инструментальные средства разработки программ}
        \end{scnrelfromset}
    \end{SCn}
\end{frame}

\begin{frame}{\\Программное обеспечение}
    \topline
    \justifying
	\begin{SCn}
        \scnheader{Системное ПО}
        \scnidtf{совокупность взаимосвязанных программ, которые обеспечивают функционирование компьютеров как таковых.}
        \begin{scnrelfromset}{разбиение}
            \scnitem{Базовое ПО – операционные системы, драйвера и т.д.}
            \scnitem{Сервисное ПО – антивирусное ПО, архиваторы и т.д.}
        \end{scnrelfromset}
    \end{SCn}
\end{frame}


\begin{frame}{\\Программное обеспечение}
    \topline
    \justifying
	\begin{SCn}
        \scnheader{Прикладное ПО}
        \scnidtf{совокупность программ, для выполнения конкретных пользовательских задач по работе с информацией.}
        \begin{scnrelfromset}{разбиение}
            \scnitem{Программы общего назначения – текстовые и табличные процессоры, графические редакторы.}
            \scnitem{Специализированные пакеты – решение задач в определенной предметной области – бухгалтерские пакеты, обучающие программы и т.д.}
        \end{scnrelfromset}
    \end{SCn}
\end{frame}


\begin{frame}{\\Программное обеспечение}
    \topline
    \justifying
	\begin{SCn}
        \scnheader{Инструментальное ПО}
        \scnidtf{совокупность программ для разработки, отладки и внедрения новых программных продуктов.}
    \end{SCn}
\end{frame}


\begin{frame}{\\Программное обеспечение}
    \topline
    \justifying
	\begin{SCn}
        \scnheader{Программное обеспечение}
        \begin{scnrelfromset}{разбиение по способу распространения}
            \scnitem{Закрытое ПО – автор сохраняет за собой определенные права (запрет изменения, распространения, исходный код скрыт);}
            \scnitem{Открытое ПО – дается гарантия свободного распространения копии программы вместе с исходным кодом, изменять и использовать программу для новых разработок}
            \scnitem{Свободное ПО – использование программ бесплатно, исходный код программ доступен}
        \end{scnrelfromset}
    \end{SCn}
\end{frame}

\begin{frame}{\\Математическое обеспечение}
    \topline
    \justifying
	\begin{SCn}
        \scnheader{Математическое обеспечение}
        \scnidtf{разделы математики, использующиеся при создании компьютерных технологий, на основе которых разработаны алгоритмы, реализованные в виде программного обеспечения.}
        
    \end{SCn}
\end{frame}

\begin{frame}{\\Информационное  обеспечение}
    \topline
    \justifying
	\begin{SCn}
        \scnheader{Информационное  обеспечение}
            \scnidtf{совокупность данных и документов, хранящихся в информационной системе и используемых в процессе работы. Включает также средства классификации документов, инструкции по работе с ними, правила именования и организации информации в системе и т.д.}
    \end{SCn}
\end{frame}

\begin{frame}{\\Информационные технологии}
    \topline
    \justifying
	\begin{SCn}
        \scnheader{Этапы развития компьютерных технологий}
        \begin{scnrelfromset}{разбиение}
            \scnitem{1947-1962 – железный век, преобладают аппаратные проблемы}
            \scnitem{1954-1970 – бронзовый век, возможности ПО не соответствуют аппаратным}
            \scnitem{1970-1990 – серебряный век, проблемы представления данных}
            \scnitem{1990 и по наши дни – гуманитарный век, удобство работы конечного пользователя}
        \end{scnrelfromset}
    \end{SCn}
\end{frame}


\begin{frame}{Этапы развития процессов обработки и передачи информации}
    \topline
    \justifying
	\begin{SCn}
        \scnheader{Этапы развития процессов
        обработки и передачи информации}
        \begin{scnrelfromset}{разбиение}
            \scnitem{Хранение и передача информации в
            устной форме;}
            \scnitem{Изобретение письменности;}
            \scnitem{Изобретение книгопечатания;}
            \scnitem{Использование электричества
            (телеграф, телевизор, телефон);}
            \scnitem{Использование компьютеров;}
            \scnitem{Появление Интернета.}
        \end{scnrelfromset}
    \end{SCn}
\end{frame}

\begin{frame}{\\Понятие информационного общества}
    \topline
    \justifying
	\begin{SCn}
        \scnheader{Информационное общество}
        \scnidtf{\small{концепция постиндустриального
        общества; новая историческая фаза развития цивилизации, в
        которой главными продуктами производства являются информация
        и знания.}}
        \scnheader{Отличительные черты Информационного общества}
        \begin{scnrelfromset}{разбиение}
            \scnitem{увеличение роли информации и знаний в жизни общества;}
            \scnitem{возрастание доли информационных коммуникаций, продуктов и услуг в
            валовом внутреннем продукте;}
            \scnitem{создание глобального информационного пространства,
            обеспечивающего эффективное информационное взаимодействие
            людей, их доступ к мировым информационным ресурсам.}
            
        \end{scnrelfromset}
    \end{SCn}
\end{frame}

\begin{frame}{\\Понятие информации}
    \topline
    \justifying
	\begin{SCn}
        \scnheader{Информация}
        \scnidtf{сведения, передаваемые одними людьми
            другим людям устным, письменным или
            каким-нибудь другим способом.}
        \scnidtf{сведения о чем-то
        независимо от формы их представления.}
    \end{SCn}
\end{frame}

\begin{frame}{\\Информационное
взаимодействие}
    \topline
    \justifying
	\begin{SCn}
        \scnheader{Информационное взаимодействие}
        \scnidtf{любое взаимодействие между объектами, в
        процессе которого один приобретает некую
        субстанцию, а другой ее не теряет. При этом
        эта субстанция называется информацией.}
        \begin{scnrelfromset}{пример}
            \scnitem{Механическое взаимодействие;}
            \scnitem{Каталитическое взаимодействие;}
            \scnitem{Передача генетической информации;}
            \scnitem{Восприятие окружающей среды;}
            \scnitem{Подача звуковых сигналов об
опасности;}
            \scnitem{Танцы пчел.}
        \end{scnrelfromset}
    \end{SCn}
\end{frame}

\begin{frame}{\\Информация}
    \topline
    \justifying
	\begin{SCn}
        \scnheader{Свойства информация}
        \begin{scnrelfromset}{разбиение}
            \scnitem{\begin{scnindent}{Атрибутивные \\} \scnidtf{свойства, без которых
            информации не существует}\end{scnindent}}
            \scnitem{\begin{scnindent}{Динамические \\} \scnidtf{свойства,
            характеризующие динамику изменения
            информации во времени}\end{scnindent}}
            \scnitem{\begin{scnindent}{Прагматические \\} \scnidtf{свойства
            информации, проявляющиеся при ее
            использовании}\end{scnindent}}
        \end{scnrelfromset}
    \end{SCn}
\end{frame}


\begin{frame}{\\Атрибутивные
свойства информации}
    \topline
    \justifying
	\begin{SCn}
        \scnheader{Атрибутивные свойства информации}
        \begin{scnrelfromset}{разбиение}
            \scnitem{Неотрывность информации от
            физического носителя и языковая
            природа информации}
            \scnitem{\begin{scnindent}{Дискретность \\} \scnidtf{можно разбить на
            отдельные факты, части и работать с ними
            отдельно;}\end{scnindent}}
            \scnitem{\begin{scnindent}{Непрерывность \\} \scnidtf{новая информация сливается с уже накопленной.}\end{scnindent}}
        \end{scnrelfromset}
    \end{SCn}
\end{frame}

\begin{frame}{\\Динамические
свойства информации}
    \topline
    \justifying
	\begin{SCn}
        \scnheader{Динамические свойства информации}
        \begin{scnrelfromset}{разбиение}
            \scnitem{Рост и кумулирование информации}
            \scnitem{\begin{scnindent}{Старение \\} \scnidtf{износ физ. носителя, потеря
            актуальности самой информации;}\end{scnindent}}
            \scnitem{Многократное распространение}
        \end{scnrelfromset}
    \end{SCn}
\end{frame}

\begin{frame}{\\Прагматические
свойства информации}
    \topline
    \justifying
	\begin{SCn}
        \scnheader{Атрибутивные свойства информации}
        \begin{scnrelfromset}{разбиение}
            \scnitem{\begin{scnindent}{Ценность \\} \scnidtf{для каждого потребителя своя;}\end{scnindent}}
            \scnitem{\begin{scnindent}{Смысл и новизна \\} \scnidtf{получатель понимает
            смысл и извлекает из нее что-то новое;}\end{scnindent}}
            \scnitem{\begin{scnindent}{Понятность \\} \scnidtf{физическое представление
            информации может предоставить информацию
            только для потребителя, который способен
            воспринять и распознать это представление.}\end{scnindent}}
        \end{scnrelfromset}
    \end{SCn}
\end{frame}

\begin{frame}{\\Информация}
    \topline
    \justifying
	\begin{SCn}
        \scnheader{Способы восприятия информации}
        \begin{scnrelfromset}{разбиение}
            \scnitem{Визуальная информация (90\%);}
            \scnitem{Аудиальная информация (9\%);}
            \scnitem{Тактильная информация;}
            \scnitem{Органолептическая информация (вкус и запах).}
        \end{scnrelfromset}
    \end{SCn}
\end{frame}



\begin{frame}{\\Информация}
    \topline
    \justifying
	\begin{SCn}
        \scnheader{информация, представленная в компьютере}
        \begin{scnrelfromset}{разбиение}
            \scnitem{Текстовая информация;}
            \scnitem{Числовая информация;}
            \scnitem{Графическая информация;}
            \scnitem{Звуковая информация;}
            \scnitem{Мультимедийная (комбинированная).}
        \end{scnrelfromset}
    \end{SCn}
\end{frame}

\begin{frame}{\\Этапы передачи
информации}
    \topline
    \justifying
	\begin{SCn}
        \scnheader{Этапы передачи
        информации}
        \begin{scnrelfromset}{разбиение}
            \scnitem{Отправитель (источник информации)
            кодирует информацию (идею, мысль
            и т.д.) в сообщение;}
            \scnitem{Отправитель передаёт сообщение
            через среду передачи получателю;}
            \scnitem{Получатель (приемник информации)
            получает сообщение;}
            \scnitem{Получатель декодирует смысл
            сообщения (и что-то делает).}
        \end{scnrelfromset}
    \end{SCn}
\end{frame}

\begin{frame}{\\Кодирование информации}
    \topline
    \justifying
	\begin{SCn}
        \scnheader{Кодирование информации}
        \scnidtf{представление информации в определенной форме.}
        \scnrelfrom{требование}{
            [Получатель должен быть
            способен воспринять форму
            представления информации и
            проинтерпретировать сообщение.]}
        \end{SCn}
        Информация кодируется \textbf{знаками} и \textbf{сигналами}.
\end{frame}

\begin{frame}{\\Знак}
    \topline
    \justifying
	\begin{SCn}
        \scnheader{Знак}
        \scnidtf{материальный объект, который при определенных условиях представляет
            другой предмет, явление, свойство или
            отношение; единство формы
            (означающего) и содержания
            (означаемого).}
        \end{SCn}
        Знак может составным.
\end{frame}

\begin{frame}{\\Сигнал}
    \topline
    \justifying
	\begin{SCn}
        \scnheader{Сигнал}
        \scnidtf{изменение физической
        величины, передающее информацию,
        кодированную определённым способом.}
        \begin{scnrelfromset}{пример}
            \scnitem{электронные импульсы}
            \scnitem{свет}
            \scnitem{радиоволны}
        \end{scnrelfromset}
        \end{SCn}
\end{frame}

\begin{frame}{\\Данные}
    \topline
    \justifying
	\begin{SCn}
        \scnheader{Данные}
        \scnidtf{сведения, представленные в
        определенной знаковой системе и на
        определенном материальном носителе
        для обеспечения возможностей хранения,
        передачи, приема и обработки.}
            \scnheader{Формат данных}
            \scnidtf{соглашение о том как различные фрагменты
                информации располагаются внутри файла, а
                также алгоритм преобразования данных из
                двоичного вида в более удобный для
                пользователя вид.}
        \begin{scnrelfromset}{разбиение}
            \scnitem{Бинарные;}
            \scnitem{Текстовые}
            \scnitem{Открытые}
            \scnitem{Закрытые}
        \end{scnrelfromset}
        \end{SCn}
\end{frame}

\begin{frame}{\\Информация и знания}
    \topline
    \justifying
	\begin{SCn}
        \scnheader{Информация}
        \scnidtf{то, что мы получаем
в результате интерпретации полученных
данных;}
        \scnheader{Знания}
        \scnidtf{уже полученная ранее и сохраненная
    информация об окружающем мире;}
    \scnheader{Информация и знания}
    \begin{scnrelfromset}{различия}
        \scnitem{новая информация полученная из данных
        может добавиться к знаниям или быть
        малозначимой, не актуальной}
        \scnitem{Знания – уже обработанная информация,
        отличаются высокой степенью структуризации
        и упорядоченности;}
    \end{scnrelfromset}
\end{SCn}  
\end{frame}

\begin{frame}{Этапы представления и обработки информации}
\topline
\justifying
    \begin{SCn}
        \scnheader{Этапы представления и обработки информации}
        \begin{scnrelfromset}{разбиение}
            \scnitem{Захват информации: сбор информации из различных источников;}
            \scnitem{Кодирование информации: преобразование информации в форму, которую можно обработать компьютером;}
            \scnitem{Хранение информации: сохранение информации на устройствах хранения данных;}
        \end{scnrelfromset}
    \end{SCn}
\end{frame}

\begin{frame}{Этапы представления и обработки информации}
\topline
\justifying
    \begin{SCn}
    \scnheader{Этапы представления и обработки информации}
    \begin{scnrelfromset}{разбиение}
        \scnitem{Обработка информации: вычисления, анализ данных и другие задачи, выполняемые компьютером;}
        \scnitem{Передача информации: передача информации через сеть связи;}
        \scnitem{Восприятие информации: интерпретация информации человеком или другой системой для принятия решений или выполнения задач.}
    \end{scnrelfromset}
    \end{SCn}
\end{frame}

\begin{frame}{Методы и технологии представления и обработки информации}
\topline
\justifying
    \begin{SCn}
    \scnheader{Методы и технологии на этапах представления и обработки информации}
    \begin{scnrelfromset}{захват}
        \scnitem{Датчики, клавиатура, микрофон, сканеры и другие устройства для сбора информации.}
    \end{scnrelfromset}
    \begin{scnrelfromset}{кодирование}
        \scnitem{Программное обеспечение для преобразования информации в форму, которую можно обработать компьютером}
    \end{scnrelfromset}
    \begin{scnrelfromset}{хранение}
        \scnitem{Устройства хранения данных, такие как жесткие диски, флеш-накопители, облачные сервисы и т.д.}
    \end{scnrelfromset}
    \end{SCn}
\end{frame}

\begin{frame}{Методы и технологии на этапах представления и обработки информации (продолжение)}
\topline
\justifying
    \begin{SCn}
        \scnheader{Методы и технологии на этапах представления и обработки информации (продолжение)}
        \begin{scnrelfromset}{разбиение}
            \scnitem{Программное обеспечение для выполнения вычислений, анализа данных и других задач, такое как языки программирования, математические пакеты, системы управления базами данных и т.д.}
            \scnitem{Протоколы и сетевое оборудование для передачи информации через сеть, такие как протоколы TCP/IP, маршрутизаторы, коммутаторы и т.д.}
            \scnitem{Интерфейсы пользователя, такие как графические пользовательские интерфейсы (GUI), голосовые интерфейсы, интерфейсы командной строки и т.д.}
        \end{scnrelfromset}
    \end{SCn}
\end{frame}

\begin{frame}{Примеры алгоритмов обработки информации}
\topline
\justifying
    \begin{SCn}
    \scnheader{Примеры алгоритмов обработки информации}
    \begin{scnrelfromset}{обработка}
        \scnitem{Алгоритмы машинного обучения для классификации данных, кластеризации данных и прогнозирования;}
        \scnitem{Алгоритмы компьютерного зрения для обработки изображений, распознавания образов и т.д.;}
        \scnitem{Алгоритмы обработки естественного языка для анализа текстов, распознавания речи и т.д.}
    \end{scnrelfromset}
    \end{SCn}
\end{frame}

\begin{frame}{Роль искусственного интеллекта в обработке информации}
\topline
\justifying
    \begin{SCn}
    \scnheader{Роль искусственного интеллекта в обработке информации}
    \begin{scnrelfromset}{разбиение}
        \scnitem{Автоматическое принятие решений на основе больших объемов данных;}
        \scnitem{Автоматическое извлечение полезной информации из больших объемов данных;}
        \scnitem{Автоматическое создание новых знаний на основе анализа данных.}
    \end{scnrelfromset}
    \end{SCn}
\end{frame}
