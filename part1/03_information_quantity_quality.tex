\title{Лекция 3\\Качество и количество информации \vspace{-2em}}   
\author[]{Шункевич Д.В.}
\institute[]{Белорусский государственный университет информатики и радиоэлектроники}

\begin{frame}
	\titlepage
\end{frame}

\begin{frame}{\\Содержание лекции}
	\topline
	\justifying
	Качественные характеристики информации. Проблемы количественной оценки информации. Подходы к оценке информации.
\end{frame}

\begin{frame}{\\Содержание лекции}
	\topline
	\justifying
	\scnheader{Лекция №3}
    \begin{scnrelfromset}{структура}
        \scnitem{Качество информации}
        \scnitem{Качественные характеристики информации}
        \scnitem{Проблемы количественной оценки информации}
        \scnitem{Подходы к оценке информации}
    \end{scnrelfromset}
\end{frame}


\begin{frame}{\\Качество информации}
	\topline
 	\begin{SCn}
    \scnheader{качество информации}
	\scnidtf{совокупность свойств, отражающих степень пригодности конкретной информации об объектах и их взаимосвязях для достижения целей, стоящих перед пользователем}
	\scnidtf{степень соответствия информации требованиям и потребностям пользователя}
    \end{SCn}
\end{frame}

\begin{frame}{\\Качество информации}
	\topline
 	\begin{SCn} 
    \scnheader{информация}
    \begin{scnrelfromset}{цели получения}
        \scnfileitem{Принятие решений: информация помогает принимать решения на основе фактов и анализа данных}
        \scnfileitem{Разработка стратегий: информация позволяет оценить текущую ситуацию, выделить тренды и прогнозировать будущие события}
        \scnfileitem{Исследование: информация используется для исследования различных явлений и процессов}
    \end{scnrelfromset}
    \end{SCn}
\end{frame}

\begin{frame}{\\Качество информации}
	\topline
 	\begin{SCn} 
    \scnheader{информация}
    \begin{scnrelfromset}{цели получения}
        \scnfileitem{Управление ресурсами: информация помогает управлять ресурсами, такими как время, деньги и материалы}
        \scnfileitem{Контроль: информация используется для контроля за выполнением задач и процессов}
        \scnfileitem{Коммуникация: информация используется для коммуникации между людьми}
    \end{scnrelfromset}
    \end{SCn}
\end{frame}

\begin{frame}{\\Качественные характеристики информации}
	\topline
 	\begin{SCn} 
    \scnheader{Качественные характеристики информации}
    \begin{scneqtoset}
        \scnitem{Объективность}
        \scnitem{Достоверность}
        \scnitem{Точность}
        \scnitem{Полнота}
        \scnitem{Доступность}
        \scnitem{Понятность}
        \scnitem{Актуальность}
        \scnitem{Полезность (ценность)}
    \end{scneqtoset}
    \end{SCn}
\end{frame}

\begin{frame}{\\Качественные характеристики информации}
	\topline
 	\begin{SCn} 
    \scnheader{объективность}
	\scntext{пояснение}{Информация объективна, если она не зависит от чьего-либо мнения, суждения}
    \begin{scnrelfromset}{примеры}
        \scnfileitem{Сообщение <<На улице тепло>> несет субъективную информацию}
        \scnfileitem{Сообщение <<На улице 22\degree С>> несет объективную информацию}
    \end{scnrelfromset}
    \end{SCn}
\end{frame}

\begin{frame}{\\Качественные характеристики информации}
	\topline
 	\begin{SCn} 
    \scnheader{достоверность}
	\scntext{пояснение}{Информация достоверна, если она отражает истинное положение дел.}
    
    \scnheader{недостоверность информации}
    \begin{scnrelfromset}{причины}
        \scnfileitem{Преднамеренное искажение (дезинформация) или непреднамеренное искажение субъективного толка (слухи, сплетни, рыбацкие истории)}
        \scnfileitem{Искажение в результате воздействия помех (<<испорченный телефон>>) или недостаточно точных средств ее фиксации}
    \end{scnrelfromset}
    \end{SCn}
\end{frame}

\begin{frame}{\\Качественные характеристики информации}
	\topline
 	\begin{SCn} 
    \scnheader{достоверность}
    \begin{scnrelfromset}{подтверждение}
	    \scnitem{Ссылка на источник}
	    \scnitem{Аргументация}
	    \scnitem{Экспертное мнение}
	    \scnitem{Консистентность}        
    \end{scnrelfromset}
    \end{SCn}
\end{frame}

\begin{frame}{\\Качественные характеристики информации}
	\topline
 	\begin{SCn} 
    \scnheader{точность}
    \scntext{пояснение}{Отражает степень подробности описания реального состояния объекта, процесса, явления и т. п.}
    \end{SCn}
\end{frame}



\begin{frame}{\\Качественные характеристики информации}
	\topline
 	\begin{SCn} 
    \scnheader{полнота}
    \scntext{пояснение}{Информацию можно назвать полной, если ее достаточно для понимания и принятия решения.}
    \scntext{пояснение}{Неполная информация может быть воспринята неверно, привести к неверному решению.}
    \end{SCn}
\end{frame}



\begin{frame}{\\Качественные характеристики информации}
	\topline
 	\begin{SCn} 
    \scnheader{доступность}
    \scntext{пояснение}{Данное свойство отражает удобство формы представления информации для восприятия потребителем, удобство получения доступа к информации.}
    \end{SCn}
\end{frame}



\begin{frame}{\\Качественные характеристики информации}
	\topline
 	\begin{SCn} 
    \scnheader{понятность}
    \scntext{пояснение}{Отражает соответствие содержания информации уровню знаний потребителя.}
    \scntext{пояснение}{Язык изложения информации должен быть доступен получателю.}
    \end{SCn}
\end{frame}


\begin{frame}{\\Качественные характеристики информации}
	\topline
 	\begin{SCn} 
    \scnheader{актуальность}
    \scnidtf{своевременность}
    \scnidtf{степень ценности информации на момент использования в зависимости от срока возникновения и динамики изменения информации}

    \scnheader{неактуальность}
    \begin{scnrelfromset}{причины}
	    \scnfileitem{устаревшая информация}
	    \scnfileitem{незначимая или ненужная информация}
    \end{scnrelfromset}
    \end{SCn}
\end{frame}

\begin{frame}{\\Качественные характеристики информации}
	\topline
 	\begin{SCn} 
    \scnheader{полезность}
    \scntext{пояснение}{Рассматривается только применительно к нуждам конкретных людей.}
    \scntext{пояснение}{Степень полезность информации оценивается по тем задачам, которые конкретный человек может решить с ее помощью.}
    \end{SCn}
\end{frame}

\begin{frame}{\\Проблемы количественной оценки информации}
	\topline
 	\begin{SCn} 
    \scnheader{Точность количественной оценки информации}
    \begin{scnrelfromset}{проблемы}
    \scnitem{Качество источника информации}
    \scnitem{Виды информации}
    \scnitem{Разнообразие информации}
    \scnitem{Языковые и культурные различия}
    \scnitem{Субъективность оценки}
    \scnitem{Необходимость предварительной обработки}
    \scnitem{Динамичность информации}
    \end{scnrelfromset}
    \end{SCn}
\end{frame}

\begin{frame}{\\Проблемы количественной оценки информации}
	\topline
 	\begin{SCn} 
    \scnheader{качество источника информации}
    \scnidtf{степень достоверности, полезности и авторитетности информации, полученной из данного источника}
    
    \scnheader{Факторы, влияющие на качество источника информации}
    \begin{scneqtoset}
    \scnitem{Авторитетность}
    \scnitem{Объективность}
    \scnitem{Актуальность}
    \scnitem{Репутация}
    \scnitem{Достоверность}
    \end{scneqtoset}
    \end{SCn}
\end{frame}

\begin{frame}{\\Проблемы количественной оценки информации}
	\topline
 	\begin{SCn} 
    \scnheader{информация}
    \begin{scnrelfromset}{разбиение}
    \scnitem{научная информация}
    \scnitem{художественная информация (картины, фильмы, произведения искусства)}
    \scnitem{эмоциональная информация}
    \end{scnrelfromset}
    \end{SCn}
\end{frame}

\begin{frame}{\\Проблемы количественной оценки информации}
	\topline
 	\begin{SCn} 
 		
    \scnheader{разнообразие информации}
    \scntext{пояснение}{Разнообразие информации означает наличие большого количества различных типов информации, которые могут быть представлены в различных формах. Это может включать в себя текстовые материалы, звуковые и видеофайлы, графические изображения, базы данных и т.д.}
    
    \end{SCn}
\end{frame}

\begin{frame}{\\Проблемы количественной оценки информации}
	\topline
 	\begin{SCn} 
 		
    \scnheader{языковые различия}
    \scntext{пояснение}{Языковые и культурные различия играют важную роль в передаче и восприятии информации.}
    \scntext{пояснение}{Языковые различия могут привести к неправильному пониманию текста или его перевода, особенно при переводе между языками с разными лингвистическими структурами и грамматикой.}
   
    \end{SCn}
\end{frame}

\begin{frame}{\\Проблемы количественной оценки информации}
	\topline
 	\begin{SCn} 
 		
    \scnheader{культурные различия}
    \scntext{пояснение}{Культурные различия также могут привести к неправильному пониманию информации, так как разные культуры имеют свои собственные ценности, убеждения, традиции и обычаи.}
    \scntext{пример}{Некоторые детали или символы могут иметь разное значение в разных культурах, что может привести к недопониманию информации.}
    
    \end{SCn}
\end{frame}

\begin{frame}{\\Проблемы количественной оценки информации}
	\topline
 	\begin{SCn} 
 		
    \scnheader{культурные различия}
    \scntext{пояснение}{Культурные различия могут повлиять на то, как люди воспринимают и интерпретируют информацию.}
    \scntext{пример}{В некоторых культурах принято выражать свои мысли косвенным образом, используя метафоры или аллегории, тогда как в других культурах прямолинейность и конкретность ценятся больше.}
	\scntext{пример}{В Болгарии кивок головой обозначает отрицание, в отличие от многих других культур.}

    \end{SCn}
\end{frame}


\begin{frame}{\\Проблемы количественной оценки информации}
	\topline
 	\begin{SCn}
 		
    \scnheader{субъективность оценки информации}
    \scntext{пояснение}{Субъективность оценки информации заключается в том, что разные люди могут по-разному оценивать информацию и приходить к различным выводам, в зависимости от своих предпочтений, убеждений, опыта и других факторов. Это может привести к искажению или неправильному пониманию информации.}
    
    \end{SCn}
\end{frame}

\begin{frame}{\\Проблемы количественной оценки информации}
	\topline
 	\begin{SCn}     

    \scnheader{Факторы, влияющие на субъективность оценки информации}
    \begin{scneqtoset}
    \scnfileitem{Личные убеждения и предпочтения, которые могут влиять на интерпретацию информации}
    \scnfileitem{Опыт и знания в определенной области, которые могут сделать одну информацию более понятной и ценной, чем другую}
    \scnfileitem{Эмоциональное состояние, которое может повлиять на оценку информации}
    \begin{scnindent}
        \scntext{пример}{информация, которая вызывает положительные эмоции, может быть оценена выше, чем информация, которая вызывает негативные эмоции}
    \end{scnindent}
	\scnfileitem{Культурные и языковыме различиями, которые могут влиять на восприятие и оценку информации}
    \end{scneqtoset}

    \end{SCn}
\end{frame}

\begin{frame}{\\Проблемы количественной оценки информации}
	\topline
 	\begin{SCn}     
    \scnheader{предварительная обработка информации}
    \scntext{пояснение}{процесс обработки данных с целью подготовки их к анализу, использованию и хранению. Это может включать в себя различные виды обработки данных, такие как сортировка, фильтрация, преобразование и агрегирование.}
    \end{SCn}
\end{frame}

\begin{frame}{\\Проблемы количественной оценки информации}
	\topline
 	\begin{SCn}     
 		
    \scnheader{Факторы необходимости предварительной обработки информации}
    \begin{scneqtoset}
    \scnitem{Качество данных}
    \scnitem{Объем данных}
    \scnitem{Необходимость анализа}
    \scnitem{Совместимость}
    \scnitem{Экономия времени}
    \end{scneqtoset}

    \end{SCn}
\end{frame}


\begin{frame}{\\Проблемы количественной оценки информации}
	\topline
 	\begin{SCn} 
 		    
    \scnheader{динамичность информации}
    \scntext{пояснение}{Динамичность информации означает, что она постоянно изменяется и обновляется в соответствии с изменением условий и событий, которые она описывает. Это связано с тем, что мир постоянно меняется, а информация является отражением этого мира.}
    \scntext{пояснение}{Для работы с динамической информацией необходимо использовать соответствующие инструменты и методы.}
    \begin{scnrelfromset}{примеры}
    \scnitem{мониторинг новостей}
    \scnitem{анализ трендов в социальных сетях} 
    \end{scnrelfromset}
    \end{SCn}
\end{frame}


\begin{frame}{\\Проблемы количественной оценки информации}
	\topline
 	\begin{SCn}
 		
    \scnheader{Объективные методы информации}
    \scntext{пояснение}{Должны отражать существенные характеристики информации в сообщении.}
    \scntext{пример}{Измерение количества информации в сообщении путем вычисления его объема его машинного представления.}
    \begin{scnindent}
	    \scntext{недостаток}{Информация может быть перекодирована без потери смысла, но занимать другой объем.}
    \end{scnindent}
    \scntext{примечание}{Должен быть метод однозначного вычисления количества информации, т.е. предложена некая матем. формула.}
 
    \end{SCn}
\end{frame}


\begin{frame}{\\Подходы к оценке количества информации}
	\topline
 	\begin{SCn}     
 		
    \scnheader{Подходы к оценке количества информации}
    \begin{scneqtoset}
    \scnitem{Статистический подход}
    \scnitem{Семантический подход}
    \scnitem{Прагматический подход}
    \end{scneqtoset}

    \end{SCn}
\end{frame}




\begin{frame}{\\Подходы к оценке количества информации}
	\topline
 	\begin{SCn}     
 		
    \scnheader{Статистический подход}
    \begin{scnrelfromset}{пояснения}
    	\scnfileitem{Подход, который применяется для оценки при передаче информации в каналах связи}
	    \scnfileitem{Сообщение уменьшает степень неопределенности, которая существовала до его получения.}
	    \scnfileitem{Количество уменьшения неопределенности после опыта можно отожествить с количеством полученной информации.}
	    \scnfileitem{Необходимо научиться измерять неопределенность до получения сообщения и после получения сообщения.}
    \end{scnrelfromset}

    \end{SCn}
\end{frame}

\begin{frame}{\\Подходы к оценке количества информации}
	\topline
 	\begin{SCn}     
 		
    \scnheader{Требования к формуле количества информации}
    \begin{scneqtoset}
        \scnfileitem{Количество получаемой информации больше в том опыте, у которого больше число возможных исходов}
        \scnfileitem{Опыт с единственным возможным исходом несет количество информации равное 0}
        \scnfileitem{Количество информации от двух независимых опытов равно сумме количества информации от каждого из них}
    \end{scneqtoset}

    \end{SCn}
\end{frame}

\begin{frame}{\\Подходы к оценке количества информации}
	\topline
 	\begin{SCn}     
    \scnheader{Формулы для подсчета количества информации}
    \begin{scneqtoset}
        \scnitem{Формула Хартли(1928г.):
        \begin{figure}[h]
          \centering
          \includegraphics[width=0.2\textwidth]{part1/images/image.png}
        \end{figure}
        }
        \scnitem{Формула Шеннона(1948 г.):
        \begin{figure}[h]
          \centering
          \includegraphics[width=0.7\textwidth]{part1/images/image2.png}
        \end{figure}
        }
    \end{scneqtoset}
    \end{SCn}
\end{frame}


\begin{frame}{\\Подходы к оценке количества информации}
	\topline
 	\begin{SCn}     
 		
    \scnheader{Формулы для подсчета количества информации}
    \begin{scnrelfromset}{пояснения}
    \scnfileitem{Единица измерения -- \textit{бит} (от англ. binary digit -- двоичная цифра) введена К.Шенном}
    \scnfileitem{1 бит -- количество информации, которую мы получаем, устраняя неопределенность из двух выборов}
    \begin{scnindent}
		\scntext{пример}{При броске кубика мы получаем количество информации $I = \log_{2} 6 = 2.6$ бит}
	\end{scnindent}        
    \end{scnrelfromset}

    \end{SCn}
\end{frame}

\begin{frame}{\\Подходы к оценке количества информации}
    \topline
    \begin{SCn}     
    \scnheader{Семантический подход}
    \begin{scnrelfromset}{пояснения}
    \scnfileitem{Для понимания и использования информации получатель должен обладать определенным запасом знаний – тезаурусом.}
    \scnfileitem{Ю.А. Шрейдер: количество семантической информации, содержащиеся в сообщении определяется степенью изменения тезауруса получателя.}
    \end{scnrelfromset}
    \end{SCn}
\end{frame}

\begin{frame}{\\Подходы к оценке количества информации}
	\topline
 \begin{SCn}     
    \scnheader{Прагматический подход}
    \begin{scnrelfromset}{пояснения}
    \scnfileitem{Ценность информации, используемой в системах управления оценивается по эффекту, которая она оказывает на результат.}
    \scnfileitem{Формула А.А. Харкевича:
    \begin{figure}[h]
          \centering
          \includegraphics[width=0.2\textwidth]{part1/images/screen3.png}
    \end{figure}}
	\begin{scnindent}
	    \begin{scnrelfromset}{пояснения}
	        \scnfileitem{$pi$ – вероятность достижения цели после получения информации}
	        \scnfileitem{$p$ – вероятность достижения цели до получения информации}
	    \end{scnrelfromset}
	\end{scnindent}
    
    \end{scnrelfromset}
    \end{SCn}
\end{frame}


