\title{Лекция 3\\Качество и количество информации \vspace{-2em}}   
\author[]{Шункевич Д.В.}
\institute[]{Белорусский государственный университет информатики и радиоэлектроники}

\begin{frame}
	\titlepage
\end{frame}

\begin{frame}{\\Содержание лекции}
	\topline
	\justifying
	Качественные характеристики информации. Проблемы количественной оценки информации. Подходы к оценке информации.
\end{frame}

\begin{frame}{\\Содержание лекции}
	\topline
	\justifying
    \begin{scnrelfromset}{Разбиение}
        \scnitem{Качество информации}
        \scnitem{Качественные характеристики информации}
        \scnitem{Проблемы количественной оценки информации}
        \scnitem{Подходы к оценке информации}
    \end{scnrelfromset}
\end{frame}


\begin{frame}{\\Качество информации}
	\topline
 	\begin{SCn}
    \scnheader{Качество информации}
    \begin{scnrelfromset}{Определение}
	   \scnitem{совокупность свойств, отражающих степень пригодности конкретной информации об объектах и их взаимосвязях для достижения целей, стоящих перед пользователем}
        \scnitem{степень соответствия информации требованиям и потребностям пользователя}
    \end{scnrelfromset}
    \end{SCn}
\end{frame}

\begin{frame}{\\Качество информации}
	\topline
 	\begin{SCn} 
    \scnheader{Информация}
    \begin{scnrelfromset}{Цели получения}
            \scnitem{Принятие решений: информация помогает принимать решения на основе фактов и анализа данных}
            \scnitem{Разработка стратегий: информация позволяет оценить текущую ситуацию, выделить тренды и прогнозировать будущие события}
            \scnitem{Исследование: информация используется для исследования различных явлений и процессов}
            
    \end{scnrelfromset}
    \end{SCn}
\end{frame}

\begin{frame}{\\Качество информации}
	\topline
 	\begin{SCn} 
    \scnheader{Информация}
    \begin{scnrelfromset}{Цели получения}
            \scnitem{Управление ресурсами: информация помогает управлять ресурсами, такими как время, деньги и материалы}
            \scnitem{Контроль: информация используется для контроля за выполнением задач и процессов}
            \scnitem{Коммуникация: информация используется для коммуникации между людьми}
    \end{scnrelfromset}
    \end{SCn}
\end{frame}

\begin{frame}{\\Качественные характеристики информации}
	\topline
 	\begin{SCn} 
    \scnheader{Качественные характеристики информации}
    \begin{scnrelfromset}{Разбиение}
            \scnitem{Объективность}
            \scnitem{Достоверность}
            \scnitem{Точность}
            \scnitem{Полнота}
            \scnitem{Доступность}
            \scnitem{Понятность}
            \scnitem{Актуальность}
            \scnitem{Полезность (ценность)}
    \end{scnrelfromset}
    \end{SCn}
\end{frame}

\begin{frame}{\\Качественные характеристики информации}
	\topline
 	\begin{SCn} 
    \scnheader{Объективность}
    \begin{scnrelfromset}{Пояснение}
            \scnitem{Информация объективна, если она не зависит от чьего-либо мнения, суждения}
    \end{scnrelfromset}
    \begin{scnrelfromset}{Примеры}
        \scnitem{Сообщение «На улице тепло» несет субъективную информацию;}
        \scnitem{Сообщение «На улице 22' С» несет объективную информацию}
    \end{scnrelfromset}
    \end{SCn}
\end{frame}

\begin{frame}{\\Качественные характеристики информации}
	\topline
 	\begin{SCn} 
    \scnheader{Достоверность}
    \begin{scnrelfromset}{Пояснение}
            \scnitem{Информация достоверна, если она отражает истинное положение дел.}
    \end{scnrelfromset}
    \scnheader{Недостоверность информации}
    \begin{scnrelfromset}{Причины}
        \scnitem{преднамеренное искажение (дезинформация) или непреднамеренное искажение субъективного толка (слухи, сплетни, рыбацкие истории);}
        \scnitem{искажение в результате воздействия помех («испорченный телефон») или недостаточно точных средств ее фиксации.}
    \end{scnrelfromset}
    \end{SCn}
\end{frame}

\begin{frame}{\\Качественные характеристики информации}
	\topline
 	\begin{SCn} 
    \scnheader{Достоверность}
    \begin{scnrelfromset}{Подтверждение}
    \scnitem{Ссылка на источник}
    \scnitem{Аргументация}
    \scnitem{Экспертное мнение}
    \scnitem{Консистентность}
        
    \end{scnrelfromset}
    \end{SCn}
\end{frame}

\begin{frame}{\\Качественные характеристики информации}
	\topline
 	\begin{SCn} 
    \scnheader{Точность}
    \begin{scnrelfromset}{Пояснение}
    \scnitem{Отражает степень подробности описания реального состояния объекта, процесса, явления и т. п.}
    \end{scnrelfromset}
    \end{SCn}
\end{frame}



\begin{frame}{\\Качественные характеристики информации}
	\topline
 	\begin{SCn} 
    \scnheader{Полнота}
    \begin{scnrelfromset}{Пояснение}
    \scnitem{Информацию можно назвать полной, если ее достаточно для понимания и принятия решения.}
    \scnitem{Неполная информация может быть воспринята неверно, привести к неверному решению.}
    \end{scnrelfromset}
    \end{SCn}
\end{frame}



\begin{frame}{\\Качественные характеристики информации}
	\topline
 	\begin{SCn} 
    \scnheader{Доступность}
    \begin{scnrelfromset}{Пояснение}
    \scnitem{Данное свойство отражает удобство формы представления информации для восприятия потребителем, удобство получения доступа к информации.}
    \end{scnrelfromset}
    \end{SCn}
\end{frame}



\begin{frame}{\\Качественные характеристики информации}
	\topline
 	\begin{SCn} 
    \scnheader{Понятность}
    \begin{scnrelfromset}{Пояснение}
    \scnitem{Отражает соответствие содержания информации уровню знаний потребителя.}
    \scnitem{Язык изложения информации должен быть доступен получателю.}
    \end{scnrelfromset}
    \end{SCn}
\end{frame}


\begin{frame}{\\Качественные характеристики информации}
	\topline
 	\begin{SCn} 
    \scnheader{Актуальность(своевременность)}
    \begin{scnrelfromset}{Определение}
    \scnitem{степень ценности информации на момент использования в зависимости от срока возникновения и динамики изменения информации}
    \end{scnrelfromset}
    \scnheader{Неактуальность}
    \begin{scnrelfromset}{Причины}
    \scnitem{устаревшая информация;}
    \scnitem{незначимая или ненужная информация.}
    \end{scnrelfromset}
    \end{SCn}
\end{frame}

\begin{frame}{\\Качественные характеристики информации}
	\topline
 	\begin{SCn} 
    \scnheader{Полезность}
    \begin{scnrelfromset}{Пояснение}
    \scnitem{Рассматривается только применительно к нуждам конкретных людей;}
    \scnitem{Степень полезность информации оценивается по тем задачам, которые конкретный человек может решить с ее помощью. }
    \end{scnrelfromset}
    \end{SCn}
\end{frame}

\begin{frame}{\\Проблемы количественной оценки информации}
	\topline
 	\begin{SCn} 
    \scnheader{Точночть количественной оценки информации}
    \begin{scnrelfromset}{Разбиение по проблемам}
    \scnitem{Качество источника информации;}
    \scnitem{Виды информации;}
    \scnitem{Разнообразие информации;}
    \scnitem{Языковые и культурные различия;}
    \scnitem{Субъективность оценки;}
    \scnitem{Необходимость предварительной обработки;}
    \scnitem{Динамичность информации;}
    \end{scnrelfromset}
    \end{SCn}Мдфв
\end{frame}

\begin{frame}{\\Проблемы количественной оценки информации}
	\topline
 	\begin{SCn} 
    \scnheader{Качество источника информации}
    \begin{scnrelfromset}{Определение}
    \scnitem{степень достоверности, полезности и авторитетности информации, полученной из данного источника.}
    \end{scnrelfromset}
    
    \scnheader{Влияющие факторы на качество источника информации}
    \begin{scnrelfromset}{Разбиение}
    \scnitem{Авторитетность;}
    \scnitem{Объективность;}
    \scnitem{Актуальность;}
    \scnitem{Репутация;}
    \scnitem{Достоверность}
    \end{scnrelfromset}
    \end{SCn}
\end{frame}

\begin{frame}{\\Проблемы количественной оценки информации}
	\topline
 	\begin{SCn} 
    \scnheader{Информация}
    \begin{scnrelfromset}{Разбиение по видам}
    \scnitem{научная информация;}
    \scnitem{художественная информация (картины, фильмы, произведения искусства);}
    \scnitem{эмоциональная информация}
    \end{scnrelfromset}
    \end{SCn}
\end{frame}

\begin{frame}{\\Проблемы количественной оценки информации}
	\topline
 	\begin{SCn} 
    \scnheader{Разнообразие информации}
    \begin{scnrelfromset}{Пояснение}
    \scnitem{Разнообразие информации означает наличие большого количества различных типов информации, которые могут быть представлены в различных формах. Это может включать в себя текстовые материалы, звуковые и видеофайлы, графические изображения, базы данных и т.д.
}
    \end{scnrelfromset}
    \end{SCn}
\end{frame}

\begin{frame}{\\Проблемы количественной оценки информации}
	\topline
 	\begin{SCn} 
    \scnheader{Языковые различия}
    \begin{scnrelfromset}{Пояснение}
    \scnitem{Языковые и культурные различия играют важную роль в передаче и восприятии информации.}
    \scnitem{Языковые различия могут привести к неправильному пониманию текста или его перевода, особенно при переводе между языками с разными лингвистическими структурами и грамматикой.}
    
    \end{scnrelfromset}
    \end{SCn}
\end{frame}

\begin{frame}{\\Проблемы количественной оценки информации}
	\topline
 	\begin{SCn} 
    \scnheader{Культурные различия}
    \begin{scnrelfromset}{Пояснение}
    \scnitem{Культурные различия также могут привести к неправильному пониманию информации, так как разные культуры имеют свои собственные ценности, убеждения, традиции и обычаи.}
        \begin{scnrelfromset}{Пример}
    \scnitem{некоторые детали или символы могут иметь разное значение в разных культурах, что может привести к недопониманию информации.}
    \end{scnrelfromset}
    \end{scnrelfromset}

    \end{SCn}
\end{frame}

\begin{frame}{\\Проблемы количественной оценки информации}
	\topline
 	\begin{SCn} 
    \scnheader{Культурные различия}
    \begin{scnrelfromset}{Пояснение}

    \scnitem{культурные различия могут повлиять на то, как люди воспринимают и интерпретируют информацию. }
        \begin{scnrelfromset}{Пример}
    \scnitem{ В некоторых культурах принято выражать свои мысли косвенным образом, используя метафоры или аллегории, тогда как в других культурах прямолинейность и конкретность ценятся больше.}
    \end{scnrelfromset}
    \end{scnrelfromset}

    \end{SCn}
\end{frame}


\begin{frame}{\\Проблемы количественной оценки информации}
	\topline
 	\begin{SCn} 
    \scnheader{Субъективность оценки информации}
    \begin{scnrelfromset}{Пояснение}

    \scnitem{Субъективность оценки информации заключается в том, что разные люди могут по-разному оценивать информацию и приходить к различным выводам, в зависимости от своих предпочтений, убеждений, опыта и других факторов. Это может привести к искажению или неправильному пониманию информации.}
    \end{scnrelfromset}
    \end{SCn}
\end{frame}

\begin{frame}{\\Проблемы количественной оценки информации}
	\topline
 	\begin{SCn}     
    \scnheader{Факторы, влияющие на субъективность оценки информации}
    \begin{scnrelfromset}{Разбиение}

    \scnitem{Личные убеждения и предпочтения, которые могут влиять на интерпретацию информации}
    \scnitem{Опыт и знания в определенной области, которые могут сделать одну информацию более понятной и ценной, чем другую}
    \scnitem{Эмоциональное состояние, которое может повлиять на оценку информации}
        \begin{scnrelfromset}{Пример}
            \scnitem{информация, которая вызывает положительные эмоции, может быть оценена выше, чем информация, которая вызывает негативные эмоции}
        \end{scnrelfromset}
    \end{scnrelfromset}

    \end{SCn}
\end{frame}

\begin{frame}{\\Проблемы количественной оценки информации}
	\topline
 	\begin{SCn}     
    \scnheader{Факторы, влияющие на субъективность оценки информации}
    \begin{scnrelfromset}{Разбиение}

    \scnitem{Личные убеждения и предпочтения, которые могут влиять на интерпретацию информации}
    \scnitem{Опыт и знания в определенной области, которые могут сделать одну информацию более понятной и ценной, чем другую}
    \scnitem{Эмоциональное состояние, которое может повлиять на оценку информации}
        \begin{scnrelfromset}{Пример}
            \scnitem{информация, которая вызывает положительные эмоции, может быть оценена выше, чем информация, которая вызывает негативные эмоции}
        \end{scnrelfromset}
    \end{scnrelfromset}

    \end{SCn}
\end{frame}

\begin{frame}{\\Проблемы количественной оценки информации}
	\topline
 	\begin{SCn}     
    \scnheader{Факторы, влияющие на субъективность оценки информации}
    \begin{scnrelfromset}{Разбиение}
    \scnitem{Культурные и языковыме различиями, которые могут влиять на восприятие и оценку информации}
    \end{scnrelfromset}
    \end{SCn}
\end{frame}

\begin{frame}{\\Проблемы количественной оценки информации}
	\topline
 	\begin{SCn}     
    \scnheader{Предварительная обработка информации}
    \begin{scnrelfromset}{Определение}
    \scnitem{процесс обработки данных с целью подготовки их к анализу, использованию и хранению. Это может включать в себя различные виды обработки данных, такие как сортировка, фильтрация, преобразование и агрегирование.}
    \end{scnrelfromset}
    \end{SCn}
\end{frame}

\begin{frame}{\\Проблемы количественной оценки информации}
	\topline
 	\begin{SCn}     
    \scnheader{Факторы необходимости предварительной обработки информации}
    \begin{scnrelfromset}{Разбиение}
    \scnitem{Качество данных}
    \scnitem{Объем данных}
    \scnitem{Необходимость анализа}
    \scnitem{Совместимость}
    \scnitem{Экономия времени}
        
    \end{scnrelfromset}
    \end{SCn}
\end{frame}


\begin{frame}{\\Проблемы количественной оценки информации}
	\topline
 	\begin{SCn}     
    \scnheader{Динамичность информации}
    \begin{scnrelfromset}{Пояснение}
    \scnitem{Динамичность информации означает, что она постоянно изменяется и обновляется в соответствии с изменением условий и событий, которые она описывает. Это связано с тем, что мир постоянно меняется, а информация является отражением этого мира.}
    \scnitem{Для работы с динамической информацией необходимо использовать соответствующие инструменты и методы. }
    \begin{scnrelfromset}{Пример}
    \scnitem{мониторинг новостей}
    \scnitem{анализ трендов в социальных сетях} 
    \end{scnrelfromset}
    \end{scnrelfromset}
    \end{SCn}
\end{frame}


\begin{frame}{\\Проблемы количественной оценки информации}
	\topline
 	\begin{SCn}     
    \scnheader{Объективные методы информации}
    \begin{scnrelfromset}{Пояснение}
    \scnitem{Должны отражать существенные характеристики информации в сообщении}
    \begin{scnrelfromset}{Пример}
    \scnitem{измерение количества информации в сообщении путем вычисления его объема его машинного представления}
    \begin{scnrelfromset}{Недостатки}
        \scnitem{информация может быть перекодирована без потери смысла, но занимать другой объем}
    \end{scnrelfromset}
    \end{scnrelfromset}
    \scnitem{Должен быть метод однозначного вычисления количества информации, т.е. предложена некая матем. формула.}

    \end{scnrelfromset}
    \end{SCn}
\end{frame}


\begin{frame}{\\Подходы к оценке количества информации}
	\topline
 	\begin{SCn}     
    \scnheader{Подходы к оценке количества информации}
    \begin{scnrelfromset}{Разбиение}
    \scnitem{Статистический подход}
    \scnitem{Семантический подход}
    \scnitem{Прагматический подход}
    \end{scnrelfromset}
    \end{SCn}
\end{frame}




\begin{frame}{\\Подходы к оценке количества информации}
	\topline
 	\begin{SCn}     
    \scnheader{Статистический подход}
    \begin{scnrelfromset}{Определение}
    \scnitem{подход, который применяется для оценки при передаче информации в каналах связи}
    \end{scnrelfromset}

    \begin{scnrelfromset}{Пояснение}
    \scnitem{Сообщение уменьшает степень неопределенности, которая существовала до его получения.}
    \scnitem{Количество уменьшения неопределенности после опыта можно отожествить с количеством полученной информации.}
    \scnitem{Необходимо научиться измерять неопределенность до получения сообщения и после получения сообщения.}
    \end{scnrelfromset}
    \end{SCn}
\end{frame}

\begin{frame}{\\Подходы к оценке количества информации}
	\topline
 	\begin{SCn}     
    \scnheader{Требования к формуле количества информации}
    \begin{scnrelfromset}{Разделение}
        \scnitem{Количество получаемой информации больше в том опыте, у которого больше число возможных исходов;}
        \scnitem{Опыт с единственным возможным исходом несет количество информации равное 0;}
        \scnitem{Количество информации от двух независимых опытов равно сумме количества информации от каждого из них.}
    \end{scnrelfromset}
    \end{SCn}
\end{frame}

\begin{frame}{\\Подходы к оценке количества информации}
	\topline
 	\begin{SCn}     
    \scnheader{Формулы для подсчета количества информации}
    \begin{scnrelfromset}{Разделение}
        \scnitem{Формула Хартли(1928г.):
        \begin{figure}[h]
          \centering
          \includegraphics[width=0.2\textwidth]{part1/images/image.png}
        \end{figure}
        }
        \scnitem{Формула Шеннона(1948 г.):
        \begin{figure}[h]
          \centering
          \includegraphics[width=0.7\textwidth]{part1/images/image2.png}
        \end{figure}
        }
    \end{scnrelfromset}
    \end{SCn}
\end{frame}


\begin{frame}{\\Подходы к оценке количества информации}
	\topline
 	\begin{SCn}     
    \scnheader{Формулы для подсчета количества информации}
    \begin{scnrelfromset}{Пояснение}
    \scnheader{Единица измерения}
    \begin{scnrelfromset}{Пояснение}
    \scnitem{Единица измерения - бит (от англ. binary digit – двоичная цифра) введена К.Шенном;}
    \scnitem{1 бит - количество информации, которую мы получаем, устраняя неопределенность из двух выборов;}
    \begin{scnrelfromset}{Пример}
        \scnitem{при броске кубика мы получаем количество информации I=log26 = 2.6 бит }
    \end{scnrelfromset}
        
    \end{scnrelfromset}
    \end{scnrelfromset}
    \end{SCn}
\end{frame}

\begin{frame}{\\Подходы к оценке количества информации}
    \topline
    \begin{SCn}     
    \scnheader{Семантический подход}
    \begin{scnrelfromset}{Пояснение}
    \scnitem{Для понимания и использования информации получатель должен обладать определенным запасом знаний – тезаурусом.}
    \scnitem{Ю.А. Шрейдер: "количество семантической информации, содержащиеся в сообщении определяется степенью изменения тезауруса получателя."}
    \end{scnrelfromset}
    \end{SCn}
\end{frame}

\begin{frame}{\\Подходы к оценке количества информации}
	\topline
 \begin{SCn}     
    \scnheader{Прагматический подход}
    \begin{scnrelfromset}{Пояснение}
    \scnitem{Ценность информации, используемой в системах управления оценивается по эффекту, которая она оказывает на результат.}
    \scnitem{Формула А.А. Харкевича:}
    \begin{figure}[h]
          \centering
          \includegraphics[width=0.2\textwidth]{part1/images/screen3.png}
    \end{figure}

    \begin{scnrelfromset}{Пояснение}
        \scnitem{Pi – вероятность достижения цели после получения информации}
        \scnitem{p – вероятность достижения цели до получения информации}
    \end{scnrelfromset}
    
    \end{scnrelfromset}
    \end{SCn}
\end{frame}

