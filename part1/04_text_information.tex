\title{Лекция 4\\Представление и обработка текстовой информации \vspace{-2em}}   
\author[]{Шункевич Д.В.}
\institute[]{Белорусский государственный университет информатики и радиоэлектроники}

\begin{frame}
	\titlepage
\end{frame}

\begin{frame}{\\Содержание лекции}
	\topline
	\justifying

	\begin{SCn}
		\scnheader{Структура лекции}
		\begin{scnrelfromset}{разбиение}
			\scnitem{Текст и гипертекст. Структура текста.}
			\scnitem{Представление текстовой информации в компьютере.}
			\scnitem{Средства редактирования текста.}
		\end{scnrelfromset}
	\end{SCn}

\end{frame}

\begin{frame}{\\Структура лекции}
	\topline
	\justifying
	
	\begin{SCn}
		\scnheader{Лекция 4. Представление и обработка текстовой информации}
		\begin{scnrelfromset}{разбиение}
			\scnitem{\textbf{Текст и гипертекст. Структура текста.}}
			\scnitem{Представление текстовой информации в компьютере.}
			\scnitem{Средства редактирования текста.}
		\end{scnrelfromset}
	\end{SCn}
	
\end{frame}

\begin{frame}{\\Понятие текста}
	\topline
	\justifying

	\begin{SCn}
		\scnheader{Текст}
		\scntext{определение}{\textbf{Текст} - связная последовательность знаков, образующая законченное сообщение.}
		\scnhaselement{начало}
		\scnhaselement{конец}
		\scnhaselement{внутренняя структура}

		\begin{scnrelfromlist}{свойство}
			\scnitem{линейность}
			\scnitem{дискретность}
		\end{scnrelfromlist}
	
	\end{SCn}

\end{frame}

\begin{frame}{\\Понятие гипертекста}
	\topline
	\justifying

	\begin{SCn}
		\scnheader{Гипертекст}
		\begin{scnrelfromlist}{утверждение}
			\scnfileitem{В гипертексте отсутствует единый путь прочтения, в точках ветвления порядок определяет читатель.}
			\scnfileitem{Гипертекст разбивается на цельные фрагменты, между которыми устанавливаются связи.}
			\scnfileitem{Гипертекст имеет множество путей прочтения текста.}
		\end{scnrelfromlist}
	\end{SCn}
	
\end{frame}

\begin{frame}{\\Структура текста}
	\topline
	\justifying
	
	\begin{SCn}
		\scnheader{Текст}
		\scntext{цель}{Цель текста – передача некой информации читателю.}
		\begin{scnrelfromset}{разбиение}
			\scnitem{форма}
			\begin{scnindent}
				\scntext{пояснение}{Оформление текста должно помогать читателю воспринять логику изложения.}
			\end{scnindent}
			\scnitem{содержание}
			\begin{scnindent}
				\scntext{пояснение}{Содержание – основные идеи, логику и порядок изложения задает автор текста.}
			\end{scnindent}
		\end{scnrelfromset}
	\end{SCn}

\end{frame}

\begin{frame}{\\Структура текста}
	\topline
	\justifying

	\begin{SCn}
		\scnheader{Структура текста}
		\scntext{определение}{\textbf{Структура} – это совокупность логических частей и элементов, из которых он состоит, и связей между ними.}		
		\begin{scnrelfromlist}{пояснение}
			\scnfileitem{Структура текста зачастую определяется неявно, через \scnkeyword{форму}.}
			\scnfileitem{Оформление задач по принципу: Дано-найти-решение.}
		\end{scnrelfromlist}
	\end{SCn}

\end{frame}

\begin{frame}{\\Иерархическая структура}
	\topline
	\vspace{3em}
	\begin{center}
		\includegraphics[scale=0.5]{part1/img/text_hierarchy.png}
	\end{center}
\end{frame}

\begin{frame}{\\Правила построения структуры текста}
	\topline
	\justifying

	\begin{SCn}
		\scnheader{Структура текста}
		\begin{scnrelfromset}{правила построения}
			\scnfileitem{Человек в состоянии удержать в памяти 7±2 объекта – число частей документа.}
			\scnfileitem{Разделы документа должны соответствовать основным мыслям текста.}
			\scnfileitem{В каждом разделе выделяются основные положения, которые он содержит.}
			\scnfileitem{Структура должна быть визуально подчеркнута.}
		\end{scnrelfromset}		
	\end{SCn}
\end{frame}

\begin{frame}{\\Задачи визуального оформления документа}
	\topline
	\justifying

	\begin{SCn}
		\scnheader{Визуальное оформление документа}
		\begin{scnrelfromlist}{цель}
			\scnfileitem{Выделить структуру документа.}
			\scnfileitem{Выделить важные элементы в тексте.}
			\scnfileitem{Подчеркнуть похожие элементы.}
			\scnfileitem{Ориентировать читателя в документе, сообщить информацию о документе.}
		\end{scnrelfromlist}
	\end{SCn}

\end{frame}

\begin{frame}{\\Способы визуализации структуры документа}
	\topline
	\justifying

	\begin{SCn}
		\scnheader{Структура текста}
		\begin{scnrelfromset}{способы визуализации}
			\scnitem{размер шрифта}
			\scnitem{отступ}
			\scnitem{нумерация заголовков}
		\end{scnrelfromset}
	\end{SCn}

\end{frame}

\begin{frame}{\\Структура лекции}
	\topline
	\justifying

	\begin{SCn}
		\scnheader{Лекция 4. Представление и обработка текстовой информации}
		\begin{scnrelfromset}{разбиение}
			\scnitem{Текст и гипертекст. Структура текста.}
			\scnitem{\textbf{Представление текстовой информации в компьютере.}}
			\scnitem{Средства редактирования текста.}
		\end{scnrelfromset}
	\end{SCn}

\end{frame}

\begin{frame}{\\Удобство представления текста на компьютере}
	\topline
	\justifying

	\begin{SCn}
		\scnheader{Компьютерное представление текста}
		\begin{scnrelfromset}{требования}
			\scnitem{простота кодирования}
			\scnitem{точность представления}
			\scnitem{удобство хранения}
			\scnitem{удобство передачи по каналам связи}
		\end{scnrelfromset}
	\end{SCn}
\end{frame}

\begin{frame}{\\Однобайтное кодирование}
	\topline
	\justifying

	\begin{SCn}
		\scnheader{Компьютерное представление текста}
		\scnsuperset{однобайтное кодирование}
		\begin{scnindent}
			\scnsuperset{ASCII}
			\begin{scnindent}
				\scnidtf{American Standard Code for Information Interchange}
				\scnsuperset{US-ASCII}
			\end{scnindent}
			\scnsuperset{EBCDIC}
			\begin{scnindent}
				\scnidtf{Extended Binary Coded Decimal Interchange Code}
			\end{scnindent}
		\end{scnindent}
	\end{SCn}

\end{frame}

\begin{frame}{\\Терминология}
	\topline
	\justifying

	\begin{SCn}
		\scnheader{Кодовая страница}
		\scnidtf{code page}
		\scntext{пояснение}{\textbf{Кодовая страница} -- таблица, сопоставляющая каждому значению байта некоторый символ (или его отсутствие).}

		\scnheader{Набор символов}
		\scnidtf{character set}
		\scntext{пояснение}{\textbf{Набор символов} -- определённая таблица кодировки конечного множества знаков. Такая таблица сопоставляет каждому символу последовательность длиной в один или несколько байтов.}
		\begin{scnindent}
			\scnrelfrom{источник}{RFC 2278}
		\end{scnindent}

	\end{SCn}

\end{frame}

\begin{frame}{\\Кодовые страницы с кириллицей}
	\topline
	\justifying

	\begin{SCn}
		\scnheader{Кодовая страница}
		\scnsubset{кодовая страница с кириллицей}
		\begin{scnindent}
			\scnhaselement{Windows-1251}
			\begin{scnindent}
				\scnidtf{ANSI Cyrillic}
				\scniselement{Windows}
			\end{scnindent}
			\scnhaselement{KOI8}
			\begin{scnindent}
				\scniselement{UNIX}
			\end{scnindent}
			\scnhaselement{ГОСТ 19768-87}
			\scnhaselement{IBM code page 866}
			\begin{scnindent}
				\scniselement{DOS}
			\end{scnindent}
			\scnhaselement{MacCyrillic}
			\begin{scnindent}
				\scniselement{Macintosh}
			\end{scnindent}
		\end{scnindent}
	\end{SCn}
\end{frame}

\begin{frame}{\\Проблемы однобайтного кодирования}
	\topline
	\justifying

	\begin{SCn}
		\scnheader{Однобайтное кодирование}
		\begin{scnrelfromset}{недостатки}
			\scnitem{проблема отображения документов в неправильной кодировке}
			\begin{scnindent}
				\scnidtf{крокозябры}
			\end{scnindent}
			\scnitem{проблема ограниченности набора символов}
			\scnitem{проблема преобразования одной кодировки в другую}
			\scnitem{проблема многоязычных документов}
			\scnitem{проблема дублирования шрифтов}
		\end{scnrelfromset}
	\end{SCn}
\end{frame}

\begin{frame}{\\Стандарт Unicode}
	\topline
	\justifying

	\begin{SCn}
		\scnheader{Unicode}
		\scntext{цель}{универсальное представление знаков всех письменных языков}
		\scnsubset{многобайтное кодирование}
		\scnrelfrom{год начала разработки}{1991 г.}
		\scnrelfrom{текущая версия}{Unicode 15.0}
		\begin{scnindent}
			\scnrelfrom{год выпуска}{2022 г.}
			\scnrelfrom{количество поддерживаемых символов}{149 тыс.}
		\end{scnindent}
		\scntext{сайт проекта}{\url{https://www.unicode.org/}}
	\end{SCn}
\end{frame}

\begin{frame}{\\Структура стандарта Unicode}
	\topline
	\justifying

	\begin{SCn}
		\scnheader{Unicode}
		\scnsuperset{универсальный набор символов}
		\begin{scnindent}
			\scnidtf{UCS}
			\scnidtf{Universal Character Set}
		\end{scnindent}
		\scnsuperset{формат преобразования кода символа Unicode}
		\begin{scnindent}
			\scnidtf{UTF}
			\scnidtf{Unicode Transformation Format}
		\end{scnindent}
	\end{SCn}

\end{frame}

\begin{frame}{\\Универсальный набор символов UCS}
	\topline
	\justifying

	\begin{SCn}
		\scnheader{Символ Unicode}
		\scntext{утверждение}{Каждый символ Unicode имеет свой уникальный код – целое неотрицательное число}
		\scntext{утверждение}{Обозначение символов: U+XXXX, XXXX – номер символа в наборе в 16-ричном виде.}
		\scnhaselement{Й}
		\begin{scnindent}
			\scnrelfrom{код}{U+0419}
		\end{scnindent}
		\scnhaselement{Эмодзи клоуна}
		\begin{scnindent}
			\scnrelfrom{код}{U+1F921}
		\end{scnindent}
	\end{SCn}

\end{frame}

\begin{frame}{Способы представления кодов символов Unicode}
	\topline
	\justifying

	\begin{SCn}
		\scnheader{Unicode Encoding Form}
		\scntext{пояснение}{Unicode Encoding Form определяет способ представления кодов UCS в виде последовательности байт на компьютере}

		\scnheader{Unicode}
		\begin{scnhaselementrolelist}{способ представления кодов}
			\scnitem{UTF-8}
			\scnitem{UTF-16}
			\scnitem{UTF-32}
		\end{scnhaselementrolelist}
	\end{SCn}
\end{frame}

\begin{frame}{\\UTF-8}
	\topline
	\justifying

	\begin{SCn}
		\scnheader{UTF-8}
		\scnrelfrom{размер символа}{1-4 байт}
		\begin{scnrelfromset}{преимущества}
			\scnfileitem{совместимость с ASCII}
			\scnfileitem{меньший размер файла по сравнению с другими способами}
		\end{scnrelfromset}
		\begin{scnrelfromset}{недостатки}
			\scnfileitem{переменное число байт кода затрудняет разбор документа в UTF-8}
		\end{scnrelfromset}
	\end{SCn}

\end{frame}

\begin{frame}{\\Кодирование в UTF-8}
	\topline
	\justifying

	\vspace{3em}
	\includegraphics[scale=0.49]{part1/img/utf8_encoding.png}
\end{frame}

\begin{frame}{\\UTF-16 и UTF-32}
	\topline
	\justifying

	\begin{SCn}
		\scnheader{UTF-16}
		\scnrelfrom{размер символа}{2 байта}

		\scnheader{UTF-32}
		\scnrelfrom{размер символа}{4 байта}

		\scnheader{BOM*}
		\scnidtf{Byte Order Mark*}
		\scntext{пояснение}{BOM -- метка порядка байтов -- неразрывный пробел U+FEFF.}
	\end{SCn}

\end{frame}

\begin{frame}{\\UTF-16 и UTF-32}
	\topline
	\justifying

	\begin{SCn}
		\scnheader{Little-endian}
		\scnhaselement{UTF-16LE}
		\begin{scnindent}
			\scntext{BOM}{FF FE}
		\end{scnindent}
		\scnhaselement{UTF-32LE}
		\begin{scnindent}
			\scntext{BOM}{FF FE 00 00}
		\end{scnindent}

		\scnheader{Big-endian}
		\scnhaselement{UTF-16BE}
		\begin{scnindent}
			\scntext{BOM}{FE FF}
		\end{scnindent}
		\scnhaselement{UTF-32BE}
		\begin{scnindent}
			\scntext{BOM}{00 00 FE FF}
		\end{scnindent}
	\end{SCn}

\end{frame}

\begin{frame}{Форматы представления текстовой информации}
	\topline
	\justifying

	\begin{SCn}
		\scnheader{Формат представления текстовой информации}
		\scnhaselement{формат TXT}
		\scnhaselement{формат \TeX}
		\begin{scnindent}
			\scnidtf{формат \LaTeX}
			\scnidtf{формат DVI}
		\end{scnindent}
		\scnhaselement{формат HTML}
		\scnhaselement{формат RTF}
		\scnhaselement{формат DOC}
		\begin{scnindent}
			\scnidtf{формат DOCX}
		\end{scnindent}
		\scnhaselement{формат ODT}
		\begin{scnindent}
			\scnidtf{формат Open Document}
		\end{scnindent}
	\end{SCn}

\end{frame}

\begin{frame}{Классификация форматов представления текста}
	\topline
	\justifying

	\begin{SCn}
		\scnheader{Формат представления текстовой информации}
		\begin{scnrelfromset}{разбиение}
			\scnitem{бинарный}
			\scnitem{текстовый}
		\end{scnrelfromset}
		\begin{scnrelfromset}{разбиение}
			\scnitem{закрытый}
			\scnitem{открытый}
		\end{scnrelfromset}
		\begin{scnrelfromset}{разбиение}
			\scnitem{стандартный}
			\scnitem{нестандартный}
		\end{scnrelfromset}
	\end{SCn}

\end{frame}

\begin{frame}{\\Структура лекции}
	\topline
	\justifying

	\begin{SCn}
		\scnheader{Лекция 4. Представление и обработка текстовой информации}
		\begin{scnrelfromset}{разбиение}
			\scnitem{Текст и гипертекст. Структура текста.}
			\scnitem{Представление текстовой информации в компьютере.}
			\scnitem{\textbf{Средства редактирования текста.}}
		\end{scnrelfromset}
	\end{SCn}

\end{frame}

\begin{frame}{Программное обеспечение для работы с текстом}
	\topline
	\justifying

	\begin{SCn}
		\scnheader{Программное обеспечение для работы с текстом}
		\begin{scnrelfromset}{разбиение}
			\scnitem{текстовый редактор}
			\scnitem{текстовый процессор}
			\begin{scnindent}
				\scnsuperset{WYSIWYG-редактор}
			\end{scnindent}
			\scnitem{браузер}
		\end{scnrelfromset}
	\end{SCn}

\end{frame}

\begin{frame}{\\Текстовые процессоры}
	\topline
	\justifying

	\begin{SCn}
		\scnheader{Текстовый процессор}
		\scnhaselement{Microsoft Word}
		\scnhaselement{OpenOffice.org Writer}
		\scnhaselement{IBM Lotus Symphony}
		\scnhaselement{Google Docs}
		\begin{scnindent}
			\scniselement{онлайн-сервис}
			\scntext{url}{\url{https://docs.google.com/}}
		\end{scnindent}
	\end{SCn}

\end{frame}

\begin{frame}{\\Регулярные выражения}
	\topline
	\justifying

	\begin{SCn}
		\scnheader{Регулярные выражения}
		\scnidtf{regular expressions}
		\scntext{пояснение}{\textbf{Регулярные выражения} -- способ (язык) описания множества строк}
		\begin{scnrelfromlist}{используется в}
			\scnitem{текстовый редактор}
			\scnitem{язык программирования}
			\begin{scnindent}
				\scnhaselement{PHP}
				\scnhaselement{Perl}
			\end{scnindent}
		\end{scnrelfromlist}
	\end{SCn}

\end{frame}

\begin{frame}{\\Шаблон поиска}
	\topline
	\justifying
	\newline
	\begin{SCn}
		\scnheader{Шаблон поиска}
		\scntext{пояснение}{\textbf{Шаблон поиска} -- строка-описание желаемого результата поиска.}
		\scntext{пример}{a*.txt -- поиск всех текстовых файлов с именем на a}

		\scnheader{Литерал}
		\scntext{пояснение}{\textbf{Литерал} -- символ в шаблоне поиска, который соответствует самому себе.}

		\scnheader{Метасимвол}
		\scntext{пояснение}{\textbf{Метасимвол} -- символ в шаблоне поиска, имеющий особое значение, обозначает какой-то другой символ или последовательность символов.}
	\end{SCn}

\end{frame}

\begin{frame}{\\Символьные классы}
	\topline
	\justifying

	\begin{SCn}
		\scnheader{Символьный класс}
		\scntext{пояснение}{\textbf{Символьный класс} – любой символ из указанного набора символов}
	\end{SCn}

	\begin{center}
		\includegraphics[scale=0.5]{part1/img/symbol_classes.png}
	\end{center}

\end{frame}

\begin{frame}{\\Сокращенная запись символьных классов}
	\topline
	\justifying

	\begin{SCn}
		\scnheader{Сокращенная запись символьных классов}
		\scntext{пояснение}{Для наиболее распространенных символьных классов введены специальные обозначения}
	\end{SCn}

	\begin{center}
		\includegraphics[scale=0.5]{part1/img/short_symbol_classes.png}
	\end{center}

\end{frame}

\begin{frame}{\\Квантификаторы}
	\topline
	\justifying

	\begin{SCn}
		\scnheader{Квантификатор}
		\scntext{пояснение}{\textbf{Квантификаторы} -- метасимволы, указывающие сколько раз должен встретиться элемент перед ними.}

		\scnheader{Жадное поведение}
		\scntext{пояснение}{\textbf{Жадное поведение} -- ищется строка максимальной длины, удовлетворяющая шаблону.}

		\scnheader{Ленивое поведение}
		\scntext{пояснение}{\textbf{Ленивое поведение} -- ищется строка минимальной длины, удовлетворяющая шаблону.}
	\end{SCn}

\end{frame}

\begin{frame}{\\Квантификаторы}
	\topline

	\begin{center}
		\vspace{3em}
		\includegraphics[scale=0.5]{part1/img/quantificators.png}
	\end{center}
\end{frame}

\begin{frame}{\\Примеры квантификаторов}
	\topline
	\justifying

	\begin{SCn}
		\scnheader{пример жадного поведения}
		\scntext{исходная строка}{Иванов -- 28 лет; Петров -- 25 лет;}
		\scntext{шаблон поиска}{.+ -- \textbackslash d\textbackslash d~лет;}
		\scntext{результат}{Иванов -- 28 лет; Петров -- 25 лет;}

		\scnheader{пример ленивого поведения}
		\scntext{исходная строка}{Иванов -- 28 лет; Петров -- 25 лет;}
		\scntext{шаблон поиска}{.+?	-- \textbackslash d\textbackslash d~лет;}
		\scntext{результат}{Иванов -- 28 лет;}
	\end{SCn}

\end{frame}

\begin{frame}{\\Позиционирование в строке}
	\topline
	
	\begin{center}
		\includegraphics[scale=0.5]{part1/img/regex_positioning.png}
	\end{center}

\end{frame}

\begin{frame}{\\Группы и подстановки}
	\topline
	\justifying

	\begin{SCn}
		\scnheader{Группа}
		\scntext{пояснение}{\textbf{Группа} -- часть шаблона, которая обрабатывается как единое целое.}
		\scntext{пример}{(\textbackslash w+) (\textbackslash w+)\textbackslash. -- поиск имени и фамилии}

		\scnheader{Подстановка}
		\scntext{пояснение}{\textbf{Подстановка} -- использование при замене найденных подстрок}
		\scntext{пример}{\textbackslash 2 \textbackslash 1\textbackslash. -- перестановка местами имени и фамилии из примера выше.}
	\end{SCn}

\end{frame}

\begin{frame}{\\Альтернативы}
	\topline
	\justifying

	\begin{SCn}
		\scnheader{Альтернатива}
		\scntext{пояснение}{\textbf{Альтернатива} -- выбор из нескольких вариантов}
		\scntext{пример}{Я люблю (яблоки|бананы)}
	\end{SCn}

\end{frame}

\begin{frame}
	
	\begin{center}
		\begin{LARGE}
		\textbf{Спасибо за внимание!}
		\end{LARGE}

		Остались ли вопросы?
	\end{center}
\end{frame}

\begin{frame}{\\Вопросы к зачёту}
	\topline
	\justifying

	\begin{itemize}
		\item 17. Понятие текста и гипертекста. 
		\item 18. Понятие структуры текста. Способы визуализации структуры текста для читателя. 
		\item 19. Представление текстовой информации на компьютере. Основные подходы. 
		\item 20. Однобайтное кодирование. Достоинства подхода и присущие ему проблемы.
		\item 21. Многобайтное кодирование. Структура стандарта Unicode. 
		\item 22. Многобайтное кодирование. Представление кодов символов в UTF-8. 
		\item 23. Многобайтное кодирование. Представление кодов символов в UTF-16 и UTF 32. Метка порядка байтов. 
	\end{itemize}
	
\end{frame}
