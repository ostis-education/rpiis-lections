\title{Лекция 1\\Основные понятия. Цели и задачи курса \vspace{-2em}}   
\author[]{Шункевич Д.В.}
\institute[]{Белорусский государственный университет информатики и радиоэлектроники}

\begin{frame}
	\titlepage
\end{frame}

\begin{frame}{\\Содержание лекции}
	\topline
	\justifyin
	\begin {SCn}
	\scnheader{Структура лекции}
	\begin{scnrelfromset}{разбиение}
		\scnitem{Сущность понятия технологии}
		\scnitem{Типы технологий}
		\scnitem{Этапы развития процессов обработки и передачи информации}
	\end{scnrelfromset}
\end{SCn}
\end{frame}


\begin{frame}{\\Сущность понятия технологии}
	\topline
	\justifying
	\begin{SCn}
		\scnheader{Технология}
		\scntext{определение}{\textbf{\textit{Технология}} — от греч. \textit{t\'echne} — искусство, мастерство, умение и греч. \textit{logos} — изучение.}
	\end{SCn}
\end{frame} 

\begin{frame}{\\Сущность технологий}
	\topline
	\justifying
	\begin{SCn}
		\scnheader{История появления понятия}
		\begin{scnrelfromset}{разбиение}
			\scnitem{термин \flqq{}технология\frqq}
			\begin{scnindent}
				\scnrelfrom{предложен}{И. Беккманом}
				\scnrelfrom{период}{1777 год}
			\end{scnindent}
		\end{scnrelfromset}
		
	\end{SCn}
\end{frame}

\begin{frame}{\\Сущность понятия технологии}
	\topline
	\justifying
	\begin{SCn}
		\scnheader{Технология}
		\scntext{определение}{\textbf{\textit{Технология}} — совокупность метеодов и инструментов для решения поставленных задач, достижения желаемых результатов. Точное определение ускользает из-за постоянного прогресса человечества и параллельного развития смысла понятия.}
		\scnidtf {\textbf{\textit{Технология}} — это совокупность наук, сведений о способах переработки того или иного сырья в фабрикат, в готовое изделие; совокупность процессов такой переработки} 
		\scnidtf{\textbf{\textit{Технология}} — совокупность производственных методов и процессов в определенной отрасли производства, а также научное описание способов производства}
	\end{SCn}
\end{frame}


\begin{frame}{\\Сущность понятия технологии}
	\topline
	\justifying
	\begin{SCn}
		\scnidtf {\textbf{\textit{Технология}} — практическое применение знания и использование методов в производственной деятельности.}
		\scntext{пояснение}{Точное определение понятия ускользает в связи с постоянным развитием смысла понятия параллельно с прогрессом человечества.}
	\end{SCn}
\end{frame}

\begin{frame}{\\Сущность понятия технологии}
	\topline
	\justifying
	\begin{SCn}
		\scnheader{изображение}
		\scnrelfrom{пример}{\includegraphics[scale=0.17]{./part1/01_pics/fire.jpg}}
		\scnidtf{пояснение}{Технология разжигания огня с помощью лука.}
		
	\end{SCn}
\end{frame}


\begin{frame}{\\Сущность понятия технология}
	\topline
	\justifying
	\begin{SCn}
		\scnheader{Цель технологии}
		\scntext{определение}{\textbf{\textit{Цель технологии}}– выпуск продукции, удовлетворяющей потребности человека или системы}
	\end{SCn}
\end{frame}

\begin{frame}{\\Сущность понятия технология}
	\topline
	\justifying
	\begin{SCn}
		\scnheader{Составные части технологии}
		\begin{scnrelfromset}{разбиение}
			\scnitem{Исходные материалы, сырье}
			\scnitem{Инструменты, орудия производства}
			\scnitem{Правила действий}
			\scnitem{Результат, конечный продукт}
		\end{scnrelfromset}
	\end{SCn}
\end{frame}
\begin{frame}{\\Типы технологии}
\topline
\justifying
\begin{SCn}
	\scnheader{Типы технологий}
	\begin{scnrelfromset}{разбиение}
		\scnitem{Производственные технологии}
		\scnitem{Информационные технологии}
		\scnitem{Космические технологии}
		\scnitem{Транспортные технологии}
		\scnitem{Военные технологии}
		\scnitem{Высокие технологии}
		\scnitem{Социальные технологии}
		\scnitem{Телекоммуникационные технологии}
	\end{scnrelfromset}
	\scntext{пояснение}{Классификация по отрасли применения}
\end{SCn}
\end{frame}


\begin{frame}{\\Типы технологии}
	\topline
	\justifying
	\begin{SCn}
		\scnheader{Типы технологий}
		\begin{scnrelfromset}{разбиение}
			\scnitem{Пионерные технологии}
			\scnitem{Ноу-хау технологии}
			\scnitem{Оригинальные технологии}
			\scnitem{Улучшающая технология}
			\scnitem{Закрывающая технология}
		\end{scnrelfromset}
		\scntext{пояснение}{Классификация по степени радикальности нововведений}
	\end{SCn}
\end{frame}


\begin{frame}{\\\small{Этапы развития процессов обработки и передачи информации}}
	\topline
	\justifying
	\begin{SCn}
		\scnheader{Информация}
		\scntext{определение}{\textbf{\textit{Информация}} — от лат. \textit{informātiō} — \flqq{}разъяснение, представление, понятие о чём-либо\frqq{}}
		\scnidtf{\textbf{\textit{Информация}} — све­де­ния (со­об­ще­ния, дан­ные) не­за­ви­си­мо от фор­мы их пред­став­ле­ния.}
		
	\end{SCn}
\end{frame}

\begin{frame}{\\Этапы развития процессов обработки и передачи информации}
	\topline
	\justifying
	\begin{SCn}
		\scnheader{Информация}
		\scntext{определение}{\textbf{\textit{Информация}} — от лат. \textit{informātiō} — \flqq{}разъяснение, представление, понятие о чём-либо\frqq{}}
		\scnidtf{\textbf{\textit{Информация}} — све­де­ния (со­об­ще­ния, дан­ные) не­за­ви­си­мо от фор­мы их пред­став­ле­ния.}
		
	\end{SCn}
\end{frame} 

\begin{frame}{\\Этапы развития процессов обработки и передачи информации}
	\topline
	\justifying
	\begin{SCn}
		\scnheader{Информационные технологии}
		\scntext{пояснение}{За процессы передачи и распространения информации отвечают \textbf{\textit{информационные технологии}}}
		
	\end{SCn}
\end{frame} 

\begin{frame}{\\Этапы развития процессов обработки и передачи информации}
	\topline
	\justifying
	\begin{SCn}
		\scnheader{Этапы развития}
		\begin{scnrelfromset}{разбитие}
			\scnitem{\flqq{}ручная\frqq{} информационная технология}
			\scnitem{\flqq{механическая}\frqq{} информационная технология}
			\scnitem{\flqq{}электрическая\frqq{} информационная технология}
			\scnitem{\flqq{}электронная\frqq{} информационная технология}
			\scnitem{\flqq{}компьютерная\frqq{} информационная технология}
		\end{scnrelfromset}
	\end{SCn}
\end{frame} 


