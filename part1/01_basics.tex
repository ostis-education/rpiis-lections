\title{Лекция 1\\Основные понятия. Цели и задачи курса \vspace{-2em}}   
\author[]{Шункевич Д.В.}
\institute[]{Белорусский государственный университет информатики и радиоэлектроники}

\begin{frame}
	\titlepage
\end{frame}


\begin{frame}{\\Содержание лекции}
	\topline
	\justifying
	\begin {SCn}
	\scnheader{Лекция 1}
	\begin{scnrelfromset}{структура лекции}
		\scnitem{Сущность понятия технологии}
		\scnitem{Типы технологий}
		\scnitem{Этапы развития процессов обработки и передачи информации}
	\end{scnrelfromset}
	\end{SCn}
\end{frame}


\begin{frame}{\\Сущность понятия технологии}
	\topline
	\justifying
	\begin{SCn}
		\scnheader{технология}
		\scntext{определение}{\textbf{\textit{технология}} — от греч. \textit{t\'echne} — искусство, мастерство, умение и греч. \textit{logos} — изучение.}
	\end{SCn}
\end{frame} 

\begin{frame}{\\Сущность понятия технологии}
	\topline
	\justifying
	\begin{SCn}
		\scnheader{История появления понятия технологии}
		\scntext{пояснение}{Термин "технология"{} предложен И. Беккманом в 1777 году}		
	\end{SCn}
\end{frame}

\begin{frame}{\\Сущность понятия технологии}
	\topline
	\justifying
	\begin{SCn}
		\scnheader{технология}
		\scnidtf{совокупность методов и инструментов для решения поставленных задач, достижения желаемых результатов}
		\begin{scnindent}
			\scntext{примечание}{Точное определение ускользает из-за постоянного прогресса человечества и параллельного развития смысла понятия.}
		\end{scnindent}			
		\scnidtf{совокупность наук, сведений о способах переработки того или иного сырья в фабрикат, в готовое изделие; совокупность процессов такой переработки} 
		\scnidtf{совокупность производственных методов и процессов в определенной отрасли производства, а также научное описание способов производства}
	\end{SCn}
\end{frame}


\begin{frame}{\\Сущность понятия технологии}
	\topline
	\justifying
	\begin{SCn}
		\scnheader{технология}
		\scnidtf{практическое применение знания и использование методов в производственной деятельности}
		\begin{scnindent}
			\scntext{примечание}{Точное определение ускользает из-за постоянного прогресса человечества и параллельного развития смысла понятия.}
		\end{scnindent}	
	\end{SCn}
\end{frame}

\begin{frame}{\\Сущность понятия технологии}
	\topline
	\justifying
	\begin{SCn}
		\scnheader{технология}
		\scntext{пример}{\includegraphics[scale=0.17]{./part1/img/fire.jpg}}
		\begin{scnindent}
			\scntext{пояснение}{Технология разжигания огня с помощью лука.}
		\end{scnindent}
	\end{SCn}
\end{frame}


\begin{frame}{\\Сущность понятия технология}
	\topline
	\justifying
	\begin{SCn}
		\scnheader{технология}
		\scntext{цель}{\textbf{\textit{Цель технологии}} -- выпуск продукции, удовлетворяющей потребности человека или системы}
	\end{SCn}
\end{frame}

\begin{frame}{\\Сущность понятия технология}
	\topline
	\justifying
	\begin{SCn}
		\scnheader{технология}
		\begin{scnrelfromset}{составные части}
			\scnitem{исходные материалы, сырье}
			\scnitem{инструменты, орудия производства}
			\scnitem{правила действий}
			\scnitem{результат, конечный продукт}
		\end{scnrelfromset}
	\end{SCn}
\end{frame}
\begin{frame}{\\Типы технологии}
\topline
\justifying
\begin{SCn}
	\scnheader{технологии}
	\begin{scnrelfromset}{разбиение}
		\scnitem{производственные технологии}
		\scnitem{информационные технологии}
		\scnitem{космические технологии}
		\scnitem{транспортные технологии}
		\scnitem{военные технологии}
		\scnitem{высокие технологии}
		\scnitem{социальные технологии}
		\scnitem{телекоммуникационные технологии}
		\scnitem{...}
	\end{scnrelfromset}
	\begin{scnindent}
		\scntext{пояснение}{Классификация по отрасли применения}
	\end{scnindent}
\end{SCn}
\end{frame}


\begin{frame}{\\Типы технологии}
	\topline
	\justifying
	\begin{SCn}
		\scnheader{технология}
		\begin{scnrelfromset}{разбиение}
			\scnitem{пионерная технологии}
			\scnitem{ноу-хау технология}
			\scnitem{оригинальная технология}
			\scnitem{улучшающая технология}
			\scnitem{закрывающая технология}
		\end{scnrelfromset}
		\begin{scnindent}
			\scntext{пояснение}{Классификация по степени радикальности нововведений}
		\end{scnindent}
	\end{SCn}
\end{frame}

\begin{frame}{Этапы развития процессов обработки и передачи информации
	\vspace{2em}}
	\topline
	\justifying
	\begin{SCn}
		\scnheader{информация}
		\scntext{определение}{\textbf{\textit{Информация}} — от лат. \textit{informātiō} — \flqq{}разъяснение, представление, понятие о чём-либо\frqq{}}
		\scntext{определение}{\textbf{\textit{Информация}} — све­де­ния (со­об­ще­ния, дан­ные) не­за­ви­си­мо от фор­мы их пред­став­ле­ния.}
		
	\end{SCn}
\end{frame}

\begin{frame}{Этапы развития процессов обработки и передачи информации
	\vspace{2em}}
	\topline
	\justifying
	\begin{SCn}
		\scnheader{информационные технологии}
		\scntext{пояснение}{За процессы передачи и распространения информации отвечают \textbf{\textit{информационные технологии}}}
		
	\end{SCn}
\end{frame} 

\begin{frame}{Этапы развития процессов обработки и передачи информации
	\vspace{2em}}
	\topline
	\justifying
	\begin{SCn}
		\scnheader{информационная технология}
		\begin{scnrelfromset}{разбиение}
			\scnitem{\flqq{}ручная\frqq{} информационная технология}
			\scnitem{\flqq{механическая}\frqq{} информационная технология}
			\scnitem{\flqq{}электрическая\frqq{} информационная технология}
			\scnitem{\flqq{}электронная\frqq{} информационная технология}
			\scnitem{\flqq{}компьютерная\frqq{} информационная технология}
		\end{scnrelfromset}
		\begin{scnindent}
			\scntext{пояснение}{Этапы развития информационных технологий}
		\end{scnindent}
	\end{SCn}
\end{frame} 
