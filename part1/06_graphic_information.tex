\title{Лекция 6\\Представление и обработка графической информации}   
\author[]{Шункевич Д.В.}
\institute[]{Белорусский государственный университет информатики и радиоэлектроники}

\begin{frame}
	\titlepage
\end{frame}

\begin{frame}{\\Содержание лекции}
	\topline
	\justifying
\begin{SCn}
	\scnheader{Лекция 6. Представление и обработка графической информации}
	\begin{scnrelfromset}{структура}
		\scnitem{История развития технологий обработки графической информации}
		\scnitem{Представление графической информации в  компьютере}
		\scnitem{Основные технологии обработки графической информации}
	\end{scnrelfromset}
				
\end{SCn}
\end{frame}


\begin{frame}{\\История развития}
\topline
\justifying
\begin{SCn}
	\scnheader{графическая информация}	
	\scnsuperset{наскальная живопись}
	\scnsuperset{роспись посуды, фрески и мозаики}
	\scnsuperset{атласы и карты}

\end{SCn}
\end{frame}


\begin{frame}{\\История развития}
\topline
\justifying
\begin{SCn}
	\scnheader{Использование графической информации}

	\begin{scnrelfromlist}{пример}
			\scnitem{Первый вывод информации на дисплей}
			\begin{scnindent}
					\scnrelfrom{машина}{Whirlwind-I}
					\scnrelfrom{место проведения}{Массачусетский университет}
					\scnrelfrom{год}{1950}
			\end{scnindent}
	\end{scnrelfromlist}

\end{SCn}
\end{frame}


\begin{frame}{\\История развития}
\topline
\justifying
\begin{SCn}
	\scnheader{Использование графической информации}

	\begin{scnrelfromlist}{пояснение}
			\scnitem{Термин <<компьютерная графика>>}
			\begin{scnindent}
					\scnrelfrom{автор}{Уильям Феттер}
					\scnrelfrom{год}{1960}
			\end{scnindent}
	\end{scnrelfromlist}

\end{SCn}
\end{frame}


\begin{frame}{\\История развития}
\topline
\justifying
\begin{SCn}
	\scnheader{Использование графической информации}

	\begin{scnrelfromlist}{пример}
			\scnitem{программа для рисования Sketchpad}
			\begin{scnindent}
					\scnrelfrom{автор}{Айвен Сазерленд}
					\scnrelfrom{год}{1961}
			\end{scnindent}
	\end{scnrelfromlist}

\end{SCn}
\end{frame}


\begin{frame}{\\История развития}
\topline
\justifying
\begin{SCn}
	\scnheader{Использование графической информации}

	\begin{scnrelfromlist}{пример}
			\scnitem{первая игра <<Spacewar>>}
			\begin{scnindent}
					\scnrelfrom{автор}{Стив Рассел}
					\scnrelfrom{год}{1961}
			\end{scnindent}
	\end{scnrelfromlist}

\end{SCn}
\end{frame}



\begin{frame}{\\История развития}
\topline
\justifying
\begin{SCn}
	
	\scnheader{компьютерная графика}
	\begin{scnrelfromset}{периоды развития}
		\scnitem{...}
		\begin{scnindent}
			\scnrelfrom{разрабатываются}{методы}
			\scnrelfrom{разрабатываются}{алгоритмы}
			\scnrelfrom{период}{1960-1970-е}
		\end{scnindent}
	\end{scnrelfromset}



\end{SCn}
\end{frame}


\begin{frame}{\\История развития}
\topline
\justifying
\begin{SCn}

	\scnheader{компьютерная графика}
	\begin{scnrelfromset}{периоды развития}
		\scnitem{...}
		\begin{scnindent}
			\scnrelfrom{развитие}{прикладная дисциплина}
			\scnrelfrom{период}{1980-е}
		\end{scnindent}
	\end{scnrelfromset}

\end{SCn}
\end{frame}


\begin{frame}{\\История развития}
\topline
\justifying
\begin{SCn}

	\scnheader{компьютерная графика}
	\begin{scnrelfromset}{периоды развития}
		\scnitem{...}
		\begin{scnindent}
			\scnrelfrom{развитие}{средство организации диалога <<человек-компьютер>>}
			\scnrelfrom{период}{1990-е}
		\end{scnindent}
	\end{scnrelfromset}

\end{SCn}
\end{frame}


\begin{frame}{\\ Представление}
\topline
\justifying
\begin{SCn}

	\scnheader{изображение}
	\scnrelfrom{природа}{непрерывная}
	\scntext{пояснение}{В отличие от текста, графическая информация представляется в компьютере  с потерями.}
	\begin{scnrelfromset}{способы представления}
		\scnitem{векторный}
		\scnitem{растровый}
	\end{scnrelfromset}	

\end{SCn}
\end{frame}


\begin{frame}{\\ Представление}
\topline
\justifying
\begin{SCn}

	\scnheader{Изображение}
	\scntext{пример}{\includegraphics[scale=0.2]{./part1/06_pics/Blackie.png}}	

\end{SCn}
\end{frame}


\begin{frame}{\\ Представление}
\topline
\justifying
\begin{SCn}

	\scnheader{растровая графика}
	\scntext{определение}{\textbf{\textit{Растровая графика}} -- это графическое изображение, состоящее из массива сетки пикселей, или точек различных  цветов, которые имеют одинаковый размер и форму.}
	
	\scnheader{пиксель}
	\scntext{определение}{\textbf{\textit{Пиксель}} -- наименьший  элемент двумерного цифрового изображения в растровой графике, или элемент матрицы дисплеев, формирующих изображение.}

\end{SCn}
\end{frame}

\begin{frame}{\\ Представление}
\topline
\justifying
\begin{SCn}

	\scnheader{разрешение}
	\scntext{определение}{\textbf{\textit{Разрешение}} -- количество пикселей, которыми представлено изображение.}

\end{SCn}
\end{frame}


\begin{frame}{\\ Растровая графика}
\topline
\justifying
\begin{SCn}
	
	\scnheader{растровая графика}
		\begin{scnrelfromset}{достоинства}
			\scnfileitem{Универсальность применения, возможность воспроизвести любое самое сложное изображение.}
			\scnfileitem{Простота создания и редактирования изображения по частям.}
			\scnfileitem{Легкость преобразования файлов для вывода.}
		\end{scnrelfromset}
		\begin{scnrelfromset}{недостатки}
			\scnfileitem{Большой объем выходного файла.}
			\scnfileitem{Изображение плохо поддается масштабированию и другим преобразованиям.}
		\end{scnrelfromset}

\end{SCn}
\end{frame}


\begin{frame}{\\ Векторная графика}
\topline
\justifying
\begin{SCn}
	
	\scnheader{векторная графика}
	\scnrelfrom{логический элемент}{примитив векторной графики}
	
	\scnheader{примитив векторной графики}
	\scnidtf{простая геометрическая фигура -- отрезок, окружность, кривая и т.д.}
	\scntext{задание}{Для каждого примитива необходимо задать только его базовые координаты.
		
	Итоговое изображение описывается как последовательность команд создания таких примитивов.}
\end{SCn}
\end{frame}


\begin{frame}{\\ Векторная графика}
\topline
\justifying
\begin{SCn}
	
	\scnheader{векторная графика}
	\begin{scnrelfromset}{достоинства}
		\scnfileitem{Векторные изображения имеют малый объем.}
		\scnfileitem{Легкость преобразования изображения.}
	\end{scnrelfromset}
	\begin{scnrelfromset}{недостатки}
		\scnfileitem{Проблематичность его использования для передачи сложных изображений.}
		\scnfileitem{При выводе изображение может выглядеть иначе из-за отличий в реализации команд.}
		\scnfileitem{Визуализация векторного изображения может занять больше времени чем аналога в растре.}
	\end{scnrelfromset}
	
\end{SCn}
\end{frame}


\begin{frame}{\\ Глубина цвета}
\topline
\justifying
\begin{SCn}
	
	\scnheader{глубина цвета}
	\begin{scnrelfromset}{семейство подклассов}
	\scnitem{1 бит}
		\begin{scnindent}
			\scnrelfrom{количество различных цветов}{2 (монохромное изображение)}
		\end{scnindent}
	\scnitem{4 бит}
		\begin{scnindent}
			\scnrelfrom{количество различных цветов}{16 различных цветов}
		\end{scnindent}
	\scnitem{8 бит}
		\begin{scnindent}
			\scnrelfrom{количество различных цветов}{256 возможных цветов}
		\end{scnindent}
	\end{scnrelfromset}
	
\end{SCn}
\end{frame}

\begin{frame}{\\ Глубина цвета}
\topline
\justifying
\begin{SCn}
	
	\scnheader{глубина цвета}
	\begin{scnrelfromset}{семейство подклассов}
	\scnitem{16 бит}
		\begin{scnindent}
			\scnrelfrom{количество различных цветов}{65 536 цветов (High Color)}
		\end{scnindent}
	\scnitem{24 бит}
		\begin{scnindent}
			\scnrelfrom{количество различных цветов}{16 777 216 цветов (True Color)}
		\end{scnindent}
	\end{scnrelfromset}

\end{SCn}
\end{frame}


\begin{frame}{\\  Цвет}
\topline
\justifying
\begin{SCn}
	
	\scnheader{цвет}
	\scntext{определение}{\textbf{\textit{Цвет}} -- наше восприятие прямых или отраженных лучей.}
	
	\scnheader{цветовая модель}
	\scntext{определение}{\textbf{\textit{Цветовая модель}} -- способ разделения цвета на составляющие компоненты.}
	\begin{scnrelfromset}{разбиение}
		\scnitem{аддитивная модель RGB}
		\scnitem{субтрактивная модель CMYK}
		\scnitem{модель HSB}
	\end{scnrelfromset}
	
\end{SCn}
\end{frame}


\begin{frame}{\\  RGB}
\topline
\justifying
\begin{SCn}
	
	\scnheader{Цветовая модель RGB}
	\scntext{пояснение}{\textbf{\textit{RGB}} = Red -- Green -- Blue.}
	\scntext{пояснение}{Каждый цвет кодируется тремя байтами, которые задают интенсивность базовых цветов.}
	\begin{scnrelfromlist}{пример}
		\scnfileitem{000000 -- черный.}
		\scnfileitem{FFFFFF -- белый.}
		\scnfileitem{FF00FF -- лиловый.}
	\end{scnrelfromlist}
\end{SCn}
\end{frame}

\begin{frame}{\\  RGB}
\topline
\justifying
\begin{SCn}

	\scnheader{Цветовая модель RGB}
	\scntext{пояснение}{\includegraphics[scale=1.2]{./part1/06_pics/RGB.jpg}}

\end{SCn}
\end{frame}


\begin{frame}{\\  CMYK}
\topline
\justifying
\begin{SCn}
	
	\scnheader{Цветовая модель CMYK}
	\scntext{пояснение}{Соответствует печати красками на бумаге, ориентирована на работу с отраженным цветом.}
	\scntext{пояснение}{Основные цвета: голубой (Cyan). Лиловый (Magenta), желтый (Yellow) и черный (Key Color --  Black). Обозначение показывает какой процент каждой краски должен быть использован.}
	\begin{scnrelfromlist}{пример}
		\scnfileitem{(0, 0, 0, 0) -- белый цвет.}
		\scnfileitem{(100,100,100,100) -- черный цвет.}
	\end{scnrelfromlist}
\end{SCn}
\end{frame}

\begin{frame}{\\  CMYK}
\topline
\justifying
\begin{SCn}

	\scnheader{Цветовая модель CMYK}
	\scntext{пояснение}{\includegraphics[scale=0.3]{./part1/06_pics/CMYK.png}}

\end{SCn}
\end{frame}



\begin{frame}{\\  HSB}
\topline
\justifying
\begin{SCn}
	
	\scnheader{Цветовая модель HSB}
	\scntext{пояснение}{Цветовая модель HSB наиболее удобна для человека, т. к. она хорошо согласуется с моделью восприятия цвета человеком. Модель HSB наиболее удобна для восприятия человеком, но носит теоретический характер.}
	
\end{SCn}
\end{frame}

\begin{frame}{\\  HSB}
\topline
\justifying
\begin{SCn}
	
	\scnheader{Цветовая модель HSB}
	\begin{scnrelfromset}{компоненты}
		\scnitem{тон}
		\begin{scnindent}
			\scntext{пояснение}{\textbf{\textit{тон (Hue)}} -- это конкретный оттенок цвета}
		\end{scnindent}
		\scnitem{насыщенность}
		\begin{scnindent}
			\scntext{пояснение}{\textbf{\textit{насыщенность (Saturation)}} -- характеризует его интенсивность или чистоту}
		\end{scnindent}
		\scnitem{яркость цвета}
		\begin{scnindent}
			\scntext{пояснение}{\textbf{\textit{яркость цвета (Brightness)}} -- зависит от примеси черной краски, добавленной к данному цвету.}
		\end{scnindent}
	\end{scnrelfromset}
	
\end{SCn}
\end{frame}

\begin{frame}{\\  HSB}
\topline
\justifying
\begin{SCn}

	\scnheader{Цветовая модель HSB}
	\scntext{пояснение}{\includegraphics[scale=0.125]{./part1/06_pics/HSB.png}}

\end{SCn}
\end{frame}


\begin{frame}{\\  Форматы графических файлов}
\topline
\justifying
\begin{SCn}
	
	\scnheader{Форматы растровой графики}
	\scntext{пояснение}{\includegraphics[scale=0.4]{./part1/06_pics/Formats.png}}
\end{SCn}
\end{frame}


\begin{frame}{\\  Представление}
\topline
\justifying
\begin{SCn}
	
	\scnheader{Форматы векторной графики}
	\begin{scneqtoset}
		\scnitem{WMF (Windows MetaFile).}
		\scnitem{EPS (Encapsulated Postscript).}
		\scnitem{CDR (CorelDraw Document).}
		\scnitem{CGM (Computer Graphic Metafile).}
		\scnitem{SWF (Shockwave Flash).}
		\scnitem{PDF (Portable Document Format).}
	\end{scneqtoset}
\end{SCn}
\end{frame}

\begin{frame}{\\  ИТ обработки графической информации}
\topline
\justifying
\begin{SCn}
	
	\scnheader{Основные направления ИТ работы с графикой}
	\begin{scneqtoset}
		\scnitem{компьютерная графика}
		\scnitem{обработка изображений}
		\scnitem{распознавание образов}
		\scnitem{когнитивная графика}
	\end{scneqtoset}
\end{SCn}
\end{frame}

\begin{frame}{\\Компьютерная графика}
\topline
\justifying
\begin{SCn}
	
	\scnheader{Компьютерная графика}
	\begin{scnrelfromset}{разбиение}
		\scnitem{двухмерная графика}
		\scnitem{трехмерная графика}
		\scnitem{растровая графика}
		\scnitem{векторная графика}
		\scnitem{фрактальная графика}
	\end{scnrelfromset}
		
	\scnheader{фрактал}
	\scntext{пояснение}{\textbf{\textit{Фрактал}} -- объект, отдельные элементы которого наследуют свойства родительских структур.}
\end{SCn}
\end{frame}

\begin{frame}{\\  ИТ обработки графической информации}
\topline
\justifying
\begin{SCn}
	
	\scnheader{Множество Мандельброта}
	\scniselement{фрактал}
	\scntext{пояснение}{\includegraphics[scale=0.2]{./part1/06_pics/mandelbrot.jpg}}
\end{SCn}
\end{frame}


\begin{frame}{\\  ИТ обработки графической информации}
\topline
\justifying
\begin{SCn}
	
	\scnheader{программное средство обработки графической информации}
	\begin{scnrelfromset}{разбиение}
		\scnitem{растровый графический редактор}
		\begin{scnindent}
			\scntext{пример}{Adobe Photoshop, Gimp, Corel Paint, Paint.}
		\end{scnindent}
		
		\scnitem{векторный графический редактор}
		\begin{scnindent}
			\scntext{пример}{CorelDRAW, AutoCAD, 3D Studio Max.}
		\end{scnindent}
		
		\scnitem{система деловой и научной графики}	
	\end{scnrelfromset}
\end{SCn}
\end{frame}
