\title{Лекция 6\\Представление и обработка графической информации}   
\author[]{Шункевич Д.В.}
\institute[]{Белорусский государственный университет информатики и радиоэлектроники}

\begin{frame}
	\titlepage
\end{frame}

\begin{frame}{\\Содержание лекции}
	\topline
	\justifying
\begin{SCn}
	\scnheader{Структура лекции}

	\begin{scnrelfromset}{разбиение}
				\scnitem{История развития технологий обработки графической информации}
				\scnitem{Представление графической информации в  компьютере}
				\scnitem{Основные технологии обработки графической информации}
	\end{scnrelfromset}
				
\end{SCn}
\end{frame}


\begin{frame}{\\История развития}
\topline
\justifying
\begin{SCn}
	\scnheader{Использование графической информации}

	\begin{scnrelfromset}{разбиение}
			\scnitem{Наскальная живопись}
			\scnitem{Роспись посуды, фрески и мозаики}
			\scnitem{Атласы и карты}
	\end{scnrelfromset}

\end{SCn}
\end{frame}


\begin{frame}{\\История развития}
\topline
\justifying
\begin{SCn}
	\scnheader{Использование графической информации}

	\begin{scnrelfromset}{разбиение}
			\scnitem{Первый вывод информации на дисплей}
			\begin{scnindent}
					\scnrelfrom{машина}{Whirlwind-I}
					\scnrelfrom{место проведения}{Массачусетский университет}
					\scnrelfrom{год}{1950}
			\end{scnindent}
	\end{scnrelfromset}

\end{SCn}
\end{frame}


\begin{frame}{\\История развития}
\topline
\justifying
\begin{SCn}
	\scnheader{Использование графической информации}

	\begin{scnrelfromset}{разбиение}
			\scnitem{Термин «компьютерная графика»}
			\begin{scnindent}
					\scnrelto{автор}{Уильям Феттер}
					\scnrelfrom{год}{1960}
			\end{scnindent}
	\end{scnrelfromset}

\end{SCn}
\end{frame}


\begin{frame}{\\История развития}
\topline
\justifying
\begin{SCn}
	\scnheader{Использование графической информации}

	\begin{scnrelfromset}{разбиение}
			\scnitem{программа для рисования Sketchpad}
			\begin{scnindent}
					\scnrelto{автор}{Айвен Сазерленд}
					\scnrelfrom{год}{1961}
			\end{scnindent}
	\end{scnrelfromset}

\end{SCn}
\end{frame}


\begin{frame}{\\История развития}
\topline
\justifying
\begin{SCn}
	\scnheader{Использование графической информации}

	\begin{scnrelfromset}{разбиение}
			\scnitem{первая игра «Spacewar>>}
			\begin{scnindent}
					\scnrelto{автор}{Стив Рассел}
					\scnrelfrom{год}{1961}
			\end{scnindent}
	\end{scnrelfromset}

\end{SCn}
\end{frame}



\begin{frame}{\\История развития}
\topline
\justifying
\begin{SCn}
	\scnheader{Использование графической информации}

	\begin{scnrelfromset}{разбиение}
			\scnitem{компьютерная графика}
			\begin{scnindent}
					\scnrelfrom{разрабатываются}{методы}
					\scnrelfrom{разрабатываются}{алгоритмы}
					\scnrelfrom{период}{1960-1970-е}
			\end{scnindent}
	\end{scnrelfromset}

\end{SCn}
\end{frame}


\begin{frame}{\\История развития}
\topline
\justifying
\begin{SCn}
	\scnheader{Использование графической информации}

	\begin{scnrelfromset}{разбиение}
			\scnitem{компьютерная графика}
			\begin{scnindent}
					\scnrelfrom{развитие}{прикладная дисциплина}
					\scnrelfrom{период}{1980-е}
			\end{scnindent}
	\end{scnrelfromset}

\end{SCn}
\end{frame}


\begin{frame}{\\История развития}
\topline
\justifying
\begin{SCn}
	\scnheader{Использование графической информации}

	\begin{scnrelfromset}{разбиение}
			\scnitem{компьютерная графика}
			\begin{scnindent}
					\scnrelfrom{средство организации диалога <<человек-компьютер>>}{методы}
					\scnrelfrom{период}{1990-е}
			\end{scnindent}
	\end{scnrelfromset}

\end{SCn}
\end{frame}


\begin{frame}{\\ Представление}
\topline
\justifying
\begin{SCn}

	\scnheader{Изображение}
			\scnrelfrom{природа}{непрерывная}
			\scntext{пояснение}{В отличие от текста, графическая информация представляется в компьютере  с потерями.}
			\begin{scnrelfromset}{способы представления}
				\scnitem{Векторный}
				\scnitem{Растровый}
			\end{scnrelfromset}
	

\end{SCn}
\end{frame}


\begin{frame}{\\ Представление}
\topline
\justifying
\begin{SCn}

	\scnheader{Изображение}

			\scnrelfrom{пример}{\includegraphics[scale=0.2]{./part1/06_pics/Blackie.png}}
			
	

\end{SCn}
\end{frame}


\begin{frame}{\\ Представление}
\topline
\justifying
\begin{SCn}
	\scnheader{Растровая графика}
		\scntext{определение}{\textbf{\textit{Растровая графика}} — это графическое изображение, состоящее из массива сетки пикселей, или точек 							различных  цветов, которые имеют одинаковый размер и форму. }
	
	\scnheader{Пиксель}
		\scntext{определение}{\textbf{\textit{Пиксель}} - наименьший  элемент двумерного цифрового изображения в растровой графике, или  								элемент матрицы дисплеев, формирующих изображение.}

\end{SCn}
\end{frame}

\begin{frame}{\\ Представление}
\topline
\justifying
\begin{SCn}

	\scnheader{Разрешение}
		\scntext{определение}{\textbf{\textit{Разрешение}} – количество пикселей, которыми представлено изображение.}

\end{SCn}
\end{frame}


\begin{frame}{\\ Представление}
\topline
\justifying
\begin{SCn}
	
	\scnheader{Растровая графика}
		\begin{scnrelfromset}{достоинство}
				\scnitem{Универсальность применения, возможность воспроизвести любое самое сложное изображение.}
				\scnitem{Простота создания и редактирования изображения по частям.}
				\scnitem{Легкость преобразования файлов для вывода.}
			\end{scnrelfromset}
		\begin{scnrelfromset}{недостаток}
				\scnitem{Большой объем выходного файла.}
				\scnitem{Изображение плохо поддается масштабированию и другим преобразованиям.}
			\end{scnrelfromset}

\end{SCn}
\end{frame}


\begin{frame}{\\ Представление}
\topline
\justifying
\begin{SCn}
	
	\scnheader{Векторная графика}
		\scnrelfrom{логический элемент}{Простая геометрическая фигура – отрезок, окружность, 	кривая и т.д.}
	\scnheader{Примитив векторной графики}
			\scnrelfrom{задание}{Для каждого примитива необходимо задать только его базовые координаты.
						Итоговое изображение описывается как последовательность команд создания таких примитивов.}
	

\end{SCn}
\end{frame}


\begin{frame}{\\  Представление}
\topline
\justifying
\begin{SCn}
	
	\scnheader{Векторная графика}
		\begin{scnrelfromset}{достоинство}
				\scnitem{Векторные изображения имеют малый объем.}
				\scnitem{Легкость преобразования изображения.}
			\end{scnrelfromset}
		\begin{scnrelfromset}{недостаток}
				\scnitem{Проблематичность его использования для передачи сложных изображений.}
				\scnitem{При выводе изображение может выглядеть иначе, из-за отличий в реализации команд.}
				\scnitem{Визуализация векторного изображения может занять больше времени чем аналога в растре.}
			\end{scnrelfromset}
	
\end{SCn}
\end{frame}


\begin{frame}{\\  Представление}
\topline
\justifying
\begin{SCn}
	
	\scnheader{Глубина цвета}
			\begin{scnrelfromset}{бит }
				\scnitem{1 бит}
					\begin{scnindent}
						\scnrelfrom{колическтво различных цветов}{монохромное изображение.}
					\end{scnindent}
				\scnitem{4 бит}
					\begin{scnindent}
						\scnrelfrom{колическтво различных цветов}{16 различных цветов.}
					\end{scnindent}
				\scnitem{8 бит}
					\begin{scnindent}
						\scnrelfrom{колическтво различных цветов}{256 возможных цветов.}
					\end{scnindent}
			\end{scnrelfromset}
	
\end{SCn}
\end{frame}

\begin{frame}{\\  Представление}
\topline
\justifying
\begin{SCn}
	
	\scnheader{Глубина цвета}
			\begin{scnrelfromset}{бит }
				\scnitem{16 бит}
					\begin{scnindent}
						\scnrelfrom{колическтво различных цветов}{65 536 цветов (High Color).}
					\end{scnindent}
				\scnitem{24 бит}
					\begin{scnindent}
						\scnrelfrom{колическтво различных цветов}{16 777 216 цветов (True Color).}
					\end{scnindent}
			\end{scnrelfromset}
	
\end{SCn}
\end{frame}


\begin{frame}{\\  Представление}
\topline
\justifying
\begin{SCn}
	
	\scnheader{Цвет}
		\scntext{определение}{\textbf{\textit{Цвет}} - наше восприятие прямых или отраженных лучей.}
	
	\scnheader{Цветовая модель}
		\scntext{определение}{\textbf{\textit{Цветовая модель}} - способ разделения цвета на составляющие компоненты.}
	\begin{scnrelfromset}{разбиение}
				\scnitem{аддитивная модель RGB.}
				\scnitem{субтрактивная модель CMYK.}
				\scnitem{модель HSB.}
			\end{scnrelfromset}
	
\end{SCn}
\end{frame}


\begin{frame}{\\  Представление}
\topline
\justifying
\begin{SCn}
	
	\scnheader{Цветовая модель RGB}
		\scntext{пояснение}{\textbf{\textit{RGB}} - RGB = Red – Green – Blue.}
		\scntext{пояснение}{Каждый цвет кодируется тремя байтами, которые задают интенсивность базовых цветов.}
	\begin{scnrelfromset}{пример}
				\scnitem{000000 – черный.}
				\scnitem{FFFFFF – белый.}
				\scnitem{FF00FF – лиловый.}
			\end{scnrelfromset}
\end{SCn}
\end{frame}

\begin{frame}{\\  Представление}
\topline
\justifying
\begin{SCn}

	\scnheader{Цветовая модель RGB}
			\scnrelfrom{пояснение}{\includegraphics[scale=1.2]{./part1/06_pics/RGB.jpg}}

\end{SCn}
\end{frame}


\begin{frame}{\\  Представление}
\topline
\justifying
\begin{SCn}
	
	\scnheader{Цветовая модель CMYK}
		\scntext{пояснение}{Соответствует печати красками на бумаге, ориентирована на работу с отраженным цветом.}
		\scntext{пояснение}{Основные цвета: голубой (Cyan). Лиловый (Magenta), желтый (Yellow) и черный (Key Color - Black).Обозначение 							показывает какой процент каждой краски должен быть использован.}
	\begin{scnrelfromset}{пример}
				\scnitem{(0, 0, 0, 0) – белый цвет.}
				\scnitem{(100,100,100,100) – черный цвет.}
			\end{scnrelfromset}
\end{SCn}
\end{frame}

\begin{frame}{\\  Представление}
\topline
\justifying
\begin{SCn}

	\scnheader{Цветовая модель CMYK}
			\scnrelfrom{пояснение}{\includegraphics[scale=0.3]{./part1/06_pics/CMYK.png}}

\end{SCn}
\end{frame}



\begin{frame}{\\  Представление}
\topline
\justifying
\begin{SCn}
	
	\scnheader{Цветовая модель HSB}
		\scntext{пояснение}{Цветовая модель HSB наиболее удобна для человека, т. к. она хорошо согласуется с моделью восприятия цвета человеком. Модель HSB наиболее удобна для восприятия человеком, но носит теоретический характер.}
	\begin{scnrelfromset}{компонента}
				\scnitem{тон}
					\begin{scnindent}
					\scntext{пояснение}{\textbf{\textit{тон (Hue)}} – это конкретный оттенок цвета}
					\end{scnindent}
				\scnitem{насыщенность}
					\begin{scnindent}
					\scntext{пояснение}{\textbf{\textit{насыщенность (Saturation)}} –  характеризует его интенсивность или чистоту}
					\end{scnindent}
			\end{scnrelfromset}
\end{SCn}
\end{frame}

\begin{frame}{\\  Представление}
\topline
\justifying
\begin{SCn}
	
	\scnheader{Цветовая модель HSB}
	\begin{scnrelfromset}{компонента}
				
				\scnitem{яркость цвета}
					\begin{scnindent}
					\scntext{пояснение}{\textbf{\textit{яркость цвета (Brightness)}} –зависит от примеси черной краски, добавленной к данному цвету.}
					\end{scnindent}
			\end{scnrelfromset}
\end{SCn}
\end{frame}

\begin{frame}{\\  Представление}
\topline
\justifying
\begin{SCn}

	\scnheader{Цветовая модель HSB}
			\scnrelfrom{пояснение}{\includegraphics[scale=0.125]{./part1/06_pics/HSB.png}}

\end{SCn}
\end{frame}


\begin{frame}{\\  Представление}
\topline
\justifying
\begin{SCn}
	
	\scnheader{Форматы графических файлов}
			\scnrelfrom{пояснение}{\includegraphics[scale=0.4]{./part1/06_pics/Formats.png}}
\end{SCn}
\end{frame}


\begin{frame}{\\  Представление}
\topline
\justifying
\begin{SCn}
	
	\scnheader{Форматы векторной графики}
		\begin{scnrelfromset}{разбиение}
				\scnitem{WMF (Windows MetaFile).}
				\scnitem{EPS (Encapsulated Postscript).}
				\scnitem{CDR (CorelDraw Document).}
				\scnitem{CGM (Computer Graphic Metafile).}
				\scnitem{SWF (Shockwave Flash).}
				\scnitem{PDF (Portable Document Format).}
			\end{scnrelfromset}
\end{SCn}
\end{frame}

\begin{frame}{\\  ИТ обработки графической информации}
\topline
\justifying
\begin{SCn}
	
	\scnheader{Основные направления ИТ работы с графикой}
		\begin{scnrelfromset}{разбиение}
				\scnitem{Компьютерная графика.}
				\scnitem{Обработка изображений.}
				\scnitem{Распознавание образов.}
				\scnitem{Когнитивная графика.}
			\end{scnrelfromset}
\end{SCn}
\end{frame}

\begin{frame}{\\ИТ обработки графической информации}
\topline
\justifying
\begin{SCn}
	
	\scnheader{Компьютерная графика}
		\begin{scnrelfromset}{разбиение}
				\scnitem{Двухмерная графика.}
				\scnitem{Трехмерная графика.}
				\scnitem{Растровая графика.}
				\scnitem{Векторная графика.}
				\scnitem{Фрактальная графика}

			\end{scnrelfromset}
		\scnheader{Фрактал}
					\scntext{пояснение}{\textbf{\textit{Фрактал}} — объект, отдельные элементы которого наследуют свойства родительских 											структур.}
\end{SCn}
\end{frame}


\begin{frame}{\\  ИТ обработки графической информации}
\topline
\justifying
\begin{SCn}
	
	\scnheader{Множество Мандельброта}
			\scnrelfrom{пояснение}{\includegraphics[scale=0.2]{./part1/06_pics/mandelbrot.jpg}}
\end{SCn}
\end{frame}


\begin{frame}{\\  ИТ обработки графической информации}
\topline
\justifying
\begin{SCn}
	
	\scnheader{Классификация программных средств}
		\begin{scnrelfromset}{разбиение}

			\scnitem{Растровые графические редакторы}
					\begin{scnindent}
						\scntext{пример}{Adobe Photoshop, Gimp, Corel Paint, Paint.}
					\end{scnindent}

			\scnitem{Векторные графические редакторы}
					\begin{scnindent}
						\scntext{пример}{CorelDRAW, AutoCAD, 3D Studio Max.}
					\end{scnindent}

			\scnitem{Системы деловой и научной графики}
			
		\end{scnrelfromset}
\end{SCn}
\end{frame}
