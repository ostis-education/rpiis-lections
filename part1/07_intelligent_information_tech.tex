\title{Лекция 7\\Интеллектуальные информационные технологии \vspace{-2em}}   
\author[]{Шункевич Д.В.}
\institute[]{Белорусский государственный университет информатики и радиоэлектроники}

\begin{frame}
	\titlepage
\end{frame}

\begin{frame}{Интеллектуальные информационные технологии}
	\topline
	\justifying
	\begin{SCn}
	    \scnheader{Лекция 7 Интеллектуальные информационные технологии}
            \begin{scnrelfromset}{разбиение}
                \scnitem{Понятие искусственного интеллекта}
                \scnitem{История развития интеллектуальных информационных технологий}
                \scnitem{Понятие знаний, модели представления знаний}
                \scnitem{Онтологии}
                \scnitem{Основные направления развития ИИ}
            \end{scnrelfromset}
	\end{SCn}
\end{frame}

\begin{frame}{\\Понятие искусственного интеллекта}
    \topline
    \justifying
    \begin{SCn}
        \scnheader{Искусственный интеллект}
        \scnidtf{свойство автоматических систем брать на себя отдельные (творческие) функции интеллекта человека, например, выбирать и принимать оптимальные решения}
        \scnidtf{научное направление, в рамках которого ставятся и решаются задачи аппаратного или программного моделирования тех видов человеческой деятельности, которые традиционно считаются интеллектуальными}
        \scnidtf{одно из направлений информатики, целью которого является разработка аппаратно-программных средств, позволяющих пользователю-непрограммисту ставить и решать свои, традиционно считающиеся интеллектуальными задачи, общаясь с ЭВМ на ограниченном подмножестве естественного языка}
    \end{SCn}
\end{frame}


\begin{frame}{\\Интеллектуальная система}
    \topline
    \justifying
    \begin{SCn}
        \scnheader{Интеллектуальная система}
        \begin{scnrelfromset}{отличительные черты}
            \scnitem{имеет функцию представления и обработки знаний}
            \scnitem{имеет функцию рассуждения}
            \scnitem{имеет функцию общения (в удобном для человека виде)}
            \scnitem{обладает способностью обучения и самообучения}
        \end{scnrelfromset}
    \end{SCn}
\end{frame}

\begin{frame}{\\Интеллектуальные задачи}
    \topline
    \justifying
    \begin{SCn}
        \scnheader{Интеллектуальные задачи}
        \scnidtf{Задачи, связанные с отысканием алгоритма решения класса задач определенного типа}
        \scnheader{Поведенческие определения ИИ}
        \begin{scnrelfromset}{разбиение}
            \scnitem{Критерий А.Н. Колмогорова}
            \scnitem{Критерий А. Тьюринга}
        \end{scnrelfromset}
    \end{SCn}
\end{frame}

\begin{frame}{\\Критерий Тьюринга}
    \topline
    \justifying
    \begin{SCn}
        \scnheader{Критерий Тьюринга}
        \scnidtf{Испытатель через посредника общается с невидимым для него собеседником – человеком или системой}
        \scnheader{Интреллектуальная система}
        \scnidtf{система, которую испытатель в процессе такого общения не может отличить от человека}
    \end{SCn}
\end{frame}

\begin{frame}{\\Критерий Тьюринга}
    \topline
    \justifying
    \begin{SCn}
        \scnheader{Критерий Тьюринга}
        \begin{scnrelfromset}{достоинства}
            \scnitem{Широта тем для обсуждения}
        \end{scnrelfromset}
        \begin{scnrelfromset}{недостатки}
            \scnitem{Проверяется только способность машины походить на человека, а не разумность машины вообще}
            \scnitem{Непрактичность (несоответствие реальным задачам, решаемым в области ИИ)}
            \scnitem{Тест отслеживает только на поведение}
        \end{scnrelfromset}
    \end{SCn}
\end{frame}

\begin{frame}{Цели интеллектуальных информационных технологий}
    \topline
    \justifying
    \begin{SCn}
        \scnheader{Цели интеллектуальных информационных технологий}
        \begin{scnrelfromset}{разбиение}
            \scnitem{расширение круга задач, решаемых с помощью компьютеров, особенно в слабоструктурированных предметных областях}
            \scnitem{повышение уровня интеллектуальной информационной поддержки современного специалиста}
        \end{scnrelfromset}
    \end{SCn}
\end{frame}

\begin{frame}{\\Предметная область}
    \topline
    \justifying
    \begin{SCn}
        \scnheader{Предметная область}
        \scnidtf{область человеческой деятельности, для которой разрабатывается система}
        \begin{scnrelfromset}{разбиение}
            \scnitem{\begin{scnindent}{Слабо структурированная ПрО \\} \scnidtf{области, алгоритм действий в которых заранее не известен}\end{scnindent}}
            \scnitem{\begin{scnindent}{Хорошо структурированная ПрО \\} \scnidtf{области, в которых уже существуют апробированные алгоритмы и методы решения задач}\end{scnindent}}
        \end{scnrelfromset}
    \end{SCn}
\end{frame}

\begin{frame}{История развития искусственного\\ интеллекта}
    \topline
    \justifying
    \begin{SCn}
        \scnheader{История развития искусственного интеллекта}
        \begin{scnrelfromset}{разбиение}
            \scnitem{Начало развития направления 1940 гг}
            \scnitem{1943 г. – первая статья в области ИИ (Мак-Коллок и Питтс модель нейронов)}
            \scnitem{Конец 40-х – задача автоматического перевода с одного языка на другой}
            \scnitem{1950 г. – критерий Тьюринга (в статье Computing Machinery and Intelligence)}
            \scnitem{1956 г. появление термина «искусственный интеллект», автор Джон Маккарти, семинар по логическим рассуждениям}
            \scnitem{1957 г. – Розенблат предложил устройство для распознавания образов - персептрон}
        \end{scnrelfromset}
    \end{SCn}
\end{frame}

\begin{frame}{История развития искусственного\\ интеллекта}
    \topline
    \justifying
    \begin{SCn}
        \scnheader{История развития искусственного интеллекта}
        \begin{scnrelfromset}{разбиение}
            \scnitem{1959 г. – Саймон и Ньюэлл разработали программу GPS (General Problem Solving program) – универсальный решатель задач}
            \scnitem{50-е годы – моделирование человеческого разума, создание аналогичного ему искусственного}
            \scnitem{60-е годы – эвристическое программирование – разработка стратегии действий на основе заранее известных эвристик (Эвристика – теоретически необоснованное правило, позволяющее сократить число переборов)}
            \scnitem{65-75 гг. – развитие логических методов, доказательство теорем}
        \end{scnrelfromset}
    \end{SCn}
\end{frame}

\begin{frame}{История развития искусственного\\ интеллекта}
    \topline
    \justifying
    \begin{SCn}
        \scnheader{История развития искусственного интеллекта}
        \begin{scnrelfromset}{разбиение}
            \scnitem{1973 г. – создан язык Пролог}
            \scnitem{75-е – переход от поиска универсального алгоритма мышления к моделированию знаний экспертов. Основное направление- представление знаний}
        \end{scnrelfromset}
    \end{SCn}
\end{frame}

\begin{frame}{\\Центры исследований в области ИИ}
    \topline
    \justifying
    \begin{SCn}
        \scnheader{Центры исследований в области ИИ}
        \begin{scnrelfromset}{разбиение}
            \scnitem{США: Беркли, Стенфорд, Массачусетс}
            \scnitem{Япония: Университет Токио, Sony labs}
            \scnitem{Европа: Марсель, Эдинбург}
            \scnitem{Россия: Новосибирск (академгородок), Переславль-Залесский, МЭИ, МГТУ им. Баумана}
            \scnitem{Беларусь: Минск, Брест}
        \end{scnrelfromset}
    \end{SCn}
\end{frame}

\begin{frame}{Основные направления исследований в области ИИ}
    \topline
    \justifying
    \begin{SCn}
        \scnheader{Основные направления исследований в области ИИ}
        \begin{scnrelfromset}{разбиение}
            \scnitem{Нейрокибернетика или бионический}
            \scnitem{Символьный или логический}
            \scnitem{Кибернетика «черного ящика» или программно-прагматический}
            \scnitem{Подход, ориентированный на создание смешанных систем человек-компьютер}
        \end{scnrelfromset}
    \end{SCn}
\end{frame}


\begin{frame}{\\Данные и знания}
    \topline
    \justifying
    \begin{SCn}
        \scnheader{Данные}
        \scnidtf{сведения, представленные в определенной знаковой системе и на определенном материальном носителе для обеспечения возможностей хранения, передачи, приема и обработки}
        \scnheader{Знания}
        \scnidtf{хорошо структурированные данные или данные о данных (метаданные)}
    \end{SCn}
    
\end{frame}

\begin{frame}{\\Определение знаний}
    \topline
    \justifying
    \begin{SCn}
        \scnheader{Знания}
        \scnidtf{это основные закономерности предметной области, позволяющие человеку решать конкретные производственные, научные и другие задачи, т. е. факты, понятия, взаимосвязи, оценки, правила, эвристики(иначе фактические знания), а также стратегии принятия решения в этой области (иначе стратегические знания)}
        \scnheader{Знания о предметной области}
        \scnidtf{совокупность реальных или абстрактных объектов (сущностей), связей и отношении между этими объектами, а также процедур преобразования этих объектов для решения задач в предметной области}
    \end{SCn}
    
\end{frame}


\begin{frame}{\\Свойства знаний}
    \topline
    \justifying
    \begin{SCn}
        \scnheader{Свойства знаний}
        \begin{scnrelfromset}{разбиение}
            \scnitem{Более сложная структура}
            \scnitem{\begin{scnindent}{Внутренняя активность \\} \scnidtf{изменение знаний влияет на поведение системы}\end{scnindent}}
        \end{scnrelfromset}
    \end{SCn}
\end{frame}


\begin{frame}{\\Классификация знаний}
    \topline
    \justifying
    \begin{SCn}
        \scnheader{Знания}
        \begin{scnrelfromset}{разбиение}
            \scnitem{\begin{scnindent}{Процедурные знания \\} \scnidtf{описывают последовательности действий, которые могут использоваться при решении задач}\end{scnindent}}
            \scnitem{\begin{scnindent}{Декларативные знания \\} \scnidtf{все знания, не являющиеся процедурными, например, статьи в толковых словарях и энциклопедиях, формулировки законов в физике, химии и других науках, собрание исторических фактов и т. п.}\end{scnindent}}
        \end{scnrelfromset}
    \end{SCn}
\end{frame}


\begin{frame}{\\Основные задачи инженерии знаний}
    \topline
    \justifying
    \begin{SCn}
        \scnheader{Основные задачи инженерии знаний}
        \begin{scnrelfromset}{разбиение}
            \scnitem{Представление знаний}
            \scnitem{Получение знаний}
            \scnitem{Верификация знаний}
            \scnitem{Использование знаний}
        \end{scnrelfromset}
    \end{SCn}
\end{frame}


\begin{frame}{\\Требования к представлению знаний}
    \topline
    \justifying
    \begin{SCn}
        \scnheader{Требования к представлению знаний}
        \begin{scnrelfromset}{разбиение}
            \scnitem{Наглядность}
            \scnitem{Удобство представления знаний}
            \scnitem{Универсальность}
            \scnitem{Расширяемость}
        \end{scnrelfromset}
    \end{SCn}
\end{frame}

\begin{frame}{\\Модель представления знаний}
    \topline
    \justifying
    \begin{SCn}
        \scnheader{Модель представления знаний}
        \scnidtf{формализм, предназначенный для описания статических и динамических свойств предметной области (соглашение - как описывать знания)}
        \begin{scnrelfromset}{разбиение}
            \scnitem{Универсальные модели представления знаний}
            \scnitem{Специализированные модели представления знаний}
        \end{scnrelfromset}
    \end{SCn}
\end{frame}


\begin{frame}{\\Основные модели представления знаний}
    \topline
    \justifying
    \begin{SCn}
        \scnheader{Основные модели представления знаний}
        \begin{scnrelfromset}{разбиение}
            \scnitem{Формальные логические модели}
            \scnitem{Продукционные модели}
            \scnitem{Семантические сети}
            \scnitem{Фреймы}
        \end{scnrelfromset}
    \end{SCn}
\end{frame}


\begin{frame}{\\Формальные логические модели}
    \topline
    \justifying
    \begin{SCn}
        \scnheader{Формальные логические модели}
        \begin{scnrelfromset}{виды}
            \scnitem{Исчисление высказываний}
            \scnitem{Исчисление предикатов}
            \scnitem{Нечеткая логика}
        \end{scnrelfromset}
        \scntext{примечание}{не используются в промышленных разработках}
    \end{SCn}
\end{frame}

\begin{frame}{\\Формальные логические модели}
    \topline
    \justifying
    \begin{SCn}
        \scnheader{Формальные логические модели}
        \begin{scnrelfromset}{достоинства}
            \scnitem{Высокий уровень формализации, что обеспечивает точность получения результата}
            \scnitem{Согласованность}
            \scnitem{Единый способ описания знаний о предметной области и способов решения задач в предметной области}
        \end{scnrelfromset}
        
        \begin{scnrelfromset}{недостатки}
            \scnitem{Ненаглядность представления знаний}
            \scnitem{Очень строгие ограничения, накладываемые структурой представления знаний}
        \end{scnrelfromset}
    \end{SCn}
\end{frame}


\begin{frame}{\\Исчисление высказываний}
    \topline
    \justifying
    \begin{SCn}
        \scnheader{Высказывание}
        \scnidtf{неделимое грамматически правильное предложение, являющееся истинным или ложным}
        \scnheader{Сложное высказывание}
        \scnidtf{комбинация простых при помощи логических связок}
    \end{SCn}
\end{frame}


\begin{frame}{\\Исчисление предикатов}
    \topline
    \justifying
    \begin{SCn}
        \scnheader{Исчисление предикатов}
        \scnidtf{система моделирования некой среды и проверки гипотез относительно этой среды при помощи разработанной модели}
        \scnheader{Предикат}
        \scnidtf{функция на множестве M=M1*M2*…*Mn, принимающая значение истина или ложь}
    \end{SCn}
\end{frame}


\begin{frame}{\\Продукционные модели}
    \topline
    \justifying
    \begin{SCn}
        \scnheader{Продукционные модели}
        \begin{scnrelfromset}{достоинства}
            \scnitem{Простота и наглядность правил}
            \scnitem{Простота пополнения базы знаний}
            \scnitem{Простота вывода в базе знаний}
        \end{scnrelfromset}
        
        \begin{scnrelfromset}{недостатки}
            \scnitem{Несоответствие представлению знаний человеком}
            \scnitem{Сложно управлять выводом при больших БЗ}
            \scnitem{Сложность оценки непротиворечивости БЗ}
        \end{scnrelfromset}
    \end{SCn}
\end{frame}

\begin{frame}{\\Семантические сети}
    \topline
    \justifying
    \begin{SCn}
        \scnheader{Семантические сети}
        \scnidtf{ориентированный граф, вершины которого понятия, а дуги – отношения между ними}
        \scnidtf{Наиболее общий способ представления знаний}
        \scntext{пояснение}{поиск решения сводится к задаче поиска фрагмента семантической сети, соответствующего запросу}
    \end{SCn}
\end{frame}

\begin{frame}{\\Семантические сети}
    \topline
    \justifying
    \begin{SCn}
        \scnheader{Семантические сети}
        \begin{scnrelfromset}{классификация по количеству типов отношений}
            \scnitem{однородные}
            \scnitem{неоднородные}
        \end{scnrelfromset}
        \begin{scnrelfromset}{классификация по типу отношений}
            \scnitem{бинарные}
            \scnitem{N-арные}
        \end{scnrelfromset}
    \end{SCn}
\end{frame}


\begin{frame}{\\Семантические сети}
    \topline
    \justifying
    \begin{SCn}
        \scnheader{Семантические сети}
        \begin{scnrelfromset}{достоинства}
            \scnitem{Наглядность, универсальность, простота понимания}
            \scnitem{Соответствуют представлению знаний у человека}
        \end{scnrelfromset}
        
        \begin{scnrelfromset}{недостатки}
            \scnitem{Сложность организации процессов вывода на семантической сети}
            \scnitem{Смешение различных групп знаний}
        \end{scnrelfromset}
    \end{SCn}
\end{frame}

\begin{frame}{\\Фреймы}
    \topline
    \justifying
    \begin{SCn}
        \scnheader{Фрейм}
        \scnidtf{абстрактный образ для представления некоего стереотипа восприятия}
        \begin{scnrelfromset}{классификация}
            \scnitem{Фреймы-образцы(прототипы)}
            \scnitem{Фреймы-экземпляры}
        \end{scnrelfromset}

        \scntext{примечание}{В ходе вывода во фреймовой модели сначала подбирается прототип, а потом идет его уточнение применительно к образу}
    \end{SCn}
\end{frame}


\begin{frame}{\\Фреймы}
    \topline
    \justifying
    \begin{SCn}
        \scnheader{Фрейм}
        \begin{scnrelfromset}{достоинства}
            \scnitem{Интеграция знаний (декларативных и процедурных)}
            \scnitem{Соответствие принципам хранения знаний человеком}
            \scnitem{Наглядность, гибкость, однородность}
        \end{scnrelfromset}
        
        \begin{scnrelfromset}{недостатки}
            \scnitem{Сложность управления выводом}
            \scnitem{Низкая эффективность*}
        \end{scnrelfromset}
    \end{SCn}
\end{frame}


\begin{frame}{\\Методы извлечения знаний}
    \topline
    \justifying
    \begin{SCn}
        \scnheader{Методы извлечения знаний}
        \begin{scnrelfromset}{разбиение}
            \scnitem{\begin{scnindent}{Коммуникативные методы} 
                \begin{scnrelfromset}{разбиение}
                    \scnitem{Активные}
                    \scnitem{Пассивные}
                \end{scnrelfromset}
                \end{scnindent}}
            \scnitem{Текстологические методы}
            \scnitem{Наглядность, гибкость, однородность}
        \end{scnrelfromset}
    \end{SCn}
\end{frame}


\begin{frame}{\\Онтологический подход}
    \topline
    \justifying
    \begin{SCn}
        \scnheader{Онтология}
        \scnidtf{раздел философии, в котором изучаются наиболее общие характеристики бытия и сущностей}
        \scnidtf{это точная спецификация концептуализации, формализованное представление основных понятий и связей между ними}
        \scnidtf{эксплицитная спецификация определенной темы}
        \scnheader{Концептуализация}
        \scnidtf{процесс перехода от представления предметной области на естественном языке к точной спецификации этого описания на некотором формальном языке, ориентированном на компьютерное представление}
    \end{SCn}
\end{frame}

\begin{frame}{\\Виды онтологий}
    \topline
    \justifying
    \begin{SCn}
        \scnheader{Виды онтологий}
        \begin{scnrelfromset}{разбиение}
            \scnitem{\begin{scnindent}{Управляемый словарь терминов\\}
                    \scnidtf{ограниченный список слов и терминов, используемых для индексации и категоризации информации на сайте}\end{scnindent}}
            \scnitem{\begin{scnindent}{Кольцо синонимов\\}
                    \scnidtf{управляемый словарь с перечнем терминов, их синонимов, без указания предпочтительных синонимов}\end{scnindent}}
            \scnitem{\begin{scnindent}{Тезаурус\\}
                    \scnidtf{управляемый словарь с иерархической структурой, со связями и зависимостями между терминами}\end{scnindent}}
            \scnitem{\begin{scnindent}{Онтология \\}
                    \scnidtf{сложный тезаурус с настраиваемыми семантическими связями}\end{scnindent}}
        \end{scnrelfromset}
    \end{SCn}
\end{frame}



\begin{frame}{\\Цели создания онтологий}
    \topline
    \justifying
    \begin{SCn}
        \scnheader{Цели создания онтологий}
        \begin{scnrelfromset}{разбиение}
            \scnitem{Для совместного использования людьми или программными агентами общего понимания структуры информации}
            \scnitem{Для возможности повторного использования знаний в предметной области}
            \scnitem{Для того чтобы сделать допущения в предметной области явными}
            \scnitem{Для отделения  знаний в предметной области от оперативных знаний}
            \scnitem{Для анализа знаний в предметной области}
        \end{scnrelfromset}
    \end{SCn}
\end{frame}



\begin{frame}{\\Задачи решаемые с помощью онтологий}
    \topline
    \justifying
    \begin{SCn}
        \scnheader{Задачи решаемые с помощью онтологий}
        \begin{scnrelfromset}{разбиение}
            \scnitem{Создание и использование БЗ}
            \scnitem{Организация эффективного поиска в БД, информационных каталогах, БЗ}
            \scnitem{Создание систем, реализующих механизмы рассуждений}
            \scnitem{Организация поиска по смыслу в текстовой информации}
            \scnitem{Семантический поиск в Internet}
            \scnitem{Представление смысла в метаданных об информационных ресурсах}
            \scnitem{Построение и использование баз общих знаний (common knowledge) для различных интеллектуальных систем}
        \end{scnrelfromset}
    \end{SCn}
\end{frame}


\begin{frame}{\\Принципы создания онтологий}
    \topline
    \justifying
    \begin{SCn}
        \scnheader{Принципы создания онтологий}
        \begin{scnrelfromset}{разбиение}
            \scnitem{Ясность}
            \scnitem{Согласованность}
            \scnitem{Расширяемость}
            \scnitem{Минимум влияния кодирования}
            \scnitem{Минимум онтологических обязательств}
        \end{scnrelfromset}
    \end{SCn}
\end{frame}


\begin{frame}{\\Языки описания онтологий}
    \topline
    \justifying
    \begin{SCn}
        \scnheader{Языки описания онтологий}
        \scnidtf{формальный язык, используемый для кодирования онтологии}
        \begin{scnrelfromset}{разбиение}
            \scnitem{OWL(ontology web language), стандарт W3C}
            \scnitem{KIF (Knowledge Interchange Format или формат обмена знаниями)}
            \scnitem{CycL — онтологический язык проекта Cyc}
            \scnitem{DAML+OIL (FIPA)}
            \scnitem{SL/EL}
        \end{scnrelfromset}
    \end{SCn}
\end{frame}


\begin{frame}{\\Основные направления развития ИИ}
    \topline
    \justifying
    \begin{SCn}
        \scnheader{Имитация творчества}
        \begin{scnrelfromset}{разбиение}
            \scnitem{Решение игровых задач (шахматы, шашки, домино, го)}
            \scnitem{Автоматическое доказательство теорем}
            \scnitem{Программы анализа и синтеза музыкальных произведений}
            \scnitem{Генерация стихов, сказок, афоризмов}
            \scnitem{Программы моделирующие поведение}
        \end{scnrelfromset}
    \end{SCn}
\end{frame}

\begin{frame}{\\Основные направления развития ИИ}
    \topline
    \justifying
    \begin{SCn}
        \scnheader{Создание средств разработки для ИИ}
        \begin{scnrelfromset}{разбиение}
            \scnitem{\begin{scnindent}
            {Разработка инструментальных средств для создания интеллектуальных систем} 
                    \begin{scnrelfromset}{разбиение}
                        \scnitem{Разработка новых языков программирования, ориентированных на задачи ИИ}
                        \scnitem{Создание программ-оболочек для наполнения базой знаний}
                    \end{scnrelfromset}
                \end{scnindent}
                }
            \scnitem{Автоматический синтез программ}
        \end{scnrelfromset}
    \end{SCn}
\end{frame}



\begin{frame}{\\Основные направления развития ИИ}
    \topline
    \justifying
    \begin{SCn}
        \scnheader{Анализ и обработка естественного языка}
        \begin{scnrelfromset}{разбиение}
            \scnitem{Создание ЕЯ – интерфейсов}
            \scnitem{Автоматическое реферирование}
            \scnitem{Автоматическая классификация документов}
            \scnitem{Машинный перевод(морфология, синтаксис, лингвистика)}
            \scnitem{Извлечение фактов из текстов}
            \scnitem{Анализ текстов на предмет авторского права}
        \end{scnrelfromset}
    \end{SCn}
\end{frame}

\begin{frame}{\\Новые архитектуры компьютеров}
    \topline
    \justifying
    \begin{SCn}
    \small{
        \scntext{примечание}{Существующие ЭВМ используют архитектура фон Неймана и неэффективны в плане символьной обработки.
                            Основная цель – разработать ЭВМ лучше подходящие для решения отдельных типов задач.}
        \scnheader{Архитектура фон Неймана}
        \begin{scnrelfromset}{принципы}
            \scnitem{двоичное представление}
            \scnitem{программное управление}
            \scnitem{однородность памяти}
            \scnitem{адресуемость памяти}
            \scnitem{последовательное программное управление}
            \scnitem{условные переходы}
        \end{scnrelfromset}
    }\end{SCn}
\end{frame}

\begin{frame}{\\Новые архитектуры компьютеров}
    \topline
    \justifying
    \begin{SCn}
        \scnheader{Новые архитектуры компьютеров}
        \begin{scnrelfromset}{разбиение}
            \scnitem{Ассоциативные процессоры}
            \scnitem{Машины баз данных}
            \scnitem{Параллельные компьютеры}
            \scnitem{Векторные компьютеры}
            \scnitem{Графодинамический подход}
        \end{scnrelfromset}
    \end{SCn}
\end{frame}

\begin{frame}{\\Интеллектуальные роботы}
    \topline
    \justifying
    \begin{SCn}
        \scnheader{Робот}
        \scnidtf{электромеханическое устройство для автоматизации человеческого труда}
        \begin{scnrelfromset}{проблемы}
            \scnitem{машинное зрение}
            \scnitem{хранение и обработка трехмерной визуальной информации}
        \end{scnrelfromset}
    \end{SCn}
\end{frame}