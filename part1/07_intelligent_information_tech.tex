\title{Лекция 7\\Интеллектуальные информационные технологии \vspace{-2em}}   
\author[]{Шункевич Д.В.}
\institute[]{Белорусский государственный университет информатики и радиоэлектроники}

\begin{frame}
	\titlepage
\end{frame}

\begin{frame}{Интеллектуальные информационные технологии}
	\topline
	\justifying
	\begin{SCn}
	    \scnheader{Лекция 7}
            \begin{scnrelfromset}{структура лекции}
                \scnitem{Понятие искусственного интеллекта}
                \scnitem{История развития интеллектуальных информационных технологий}
                \scnitem{Понятие знаний, модели представления знаний}
                \scnitem{Онтологии}
                \scnitem{Основные направления развития ИИ}
            \end{scnrelfromset}
	\end{SCn}
\end{frame}

\begin{frame}
	\centering
	\Huge
	\textbf{Понятие искусственного интеллекта}
\end{frame}

\begin{frame}{\\Понятие искусственного интеллекта}
    \topline
    \justifying
    \begin{SCn}
        \scnheader{искусственный интеллект}
        \scnidtf{свойство автоматических систем брать на себя отдельные (творческие) функции интеллекта человека, например, выбирать и принимать оптимальные решения}
        \scnidtf{научное направление, в рамках которого ставятся и решаются задачи аппаратного или программного моделирования тех видов человеческой деятельности, которые традиционно считаются интеллектуальными}
        \scnidtf{одно из направлений информатики, целью которого является разработка аппаратно-программных средств, позволяющих пользователю-непрограммисту ставить и решать свои, традиционно считающиеся интеллектуальными задачи, общаясь с ЭВМ на ограниченном подмножестве естественного языка}
    \end{SCn}
\end{frame}


\begin{frame}{\\Интеллектуальная система}
    \topline
    \justifying
    \begin{SCn}
        \scnheader{интеллектуальная система}
        \begin{scnrelfromset}{отличительные черты}
            \scnitem{имеет функцию представления и обработки знаний}
            \scnitem{имеет функцию рассуждения}
            \scnitem{имеет функцию общения (в удобном для человека виде)}
            \scnitem{обладает способностью обучения и самообучения}
        \end{scnrelfromset}
    \end{SCn}
\end{frame}

\begin{frame}{\\Интеллектуальные задачи}
    \topline
    \justifying
    \begin{SCn}
        \scnheader{интеллектуальные задачи}
        \scnidtf{задачи, связанные с отысканием алгоритма решения класса задач определенного типа}
        \scnheader{поведенческое определение ИИ}
        \begin{scnrelfromset}{разбиение}
            \scnitem{критерий А.Н. Колмогорова}
            \scnitem{критерий А. Тьюринга}
        \end{scnrelfromset}
    \end{SCn}
\end{frame}

\begin{frame}{\\Критерий Тьюринга}
    \topline
    \justifying
    \begin{SCn}
        \scnheader{критерий Тьюринга}
        \scnidtf{испытатель через посредника общается с невидимым для него собеседником – человеком или системой}
        \scnheader{интреллектуальная система}
        \scnidtf{система, которую испытатель в процессе такого общения не может отличить от человека}
    \end{SCn}
\end{frame}

\begin{frame}{\\Критерий Тьюринга}
    \topline
    \justifying
    \begin{SCn}
        \scnheader{критерий Тьюринга}
        \begin{scnrelfromset}{достоинства}
            \scnitem{широта тем для обсуждения}
        \end{scnrelfromset}
        \begin{scnrelfromset}{недостатки}
            \scnitem{проверяется только способность машины походить на человека, а не разумность машины вообще}
            \scnitem{непрактичность (несоответствие реальным задачам, решаемым в области ИИ)}
            \scnitem{тест отслеживает только поведение}
        \end{scnrelfromset}
    \end{SCn}
\end{frame}

\begin{frame}{Цели интеллектуальных информационных технологий}
    \topline
    \justifying
    \begin{SCn}
        \scnheader{интеллектуальные информационные технологии}
        \begin{scnrelfromset}{цели}
            \scnitem{расширение круга задач, решаемых с помощью компьютеров, особенно в слабоструктурированных предметных областях}
            \scnitem{повышение уровня интеллектуальной информационной поддержки современного специалиста}
        \end{scnrelfromset}
    \end{SCn}
\end{frame}

\begin{frame}{\\Предметная область}
    \topline
    \justifying
    \begin{SCn}
        \scnheader{предметная область}
        \scnidtf{область человеческой деятельности, для которой разрабатывается система}
        \begin{scnrelfromset}{разбиение}
            \scnitem{\begin{scnindent}{слабо структурированная ПрО \\} \scnidtf{область, алгоритм действий в которой заранее не известен}\end{scnindent}}
            \scnitem{\begin{scnindent}{хорошо структурированная ПрО \\} \scnidtf{область, в которой уже существуют апробированные алгоритмы и методы решения задач}\end{scnindent}}
        \end{scnrelfromset}
    \end{SCn}
\end{frame}

\begin{frame}
	\centering
	\Huge
	\textbf{История развития искусственного интеллекта}
\end{frame}

\begin{frame}{История развития искусственного\\ интеллекта}
    \topline
    \justifying
    \begin{SCn}
        \scnheader{история развития искусственного интеллекта}
        \begin{scnrelfromset}{разбиение}
            \scnitem{1940 гг.– Начало развития направления}
            \scnitem{1943 г. – первая статья в области ИИ (Мак-Коллок и Питтс модель нейронов)}
            \scnitem{Конец 1940-х – задача автоматического перевода с одного языка на другой}
            \scnitem{1950 г. – критерий Тьюринга (в статье Computing Machinery and Intelligence)}
            \scnitem{1956 г. – появление термина «искусственный интеллект», автор Джон Маккарти, семинар по логическим рассуждениям}
            \scnitem{1957 г. – Розенблат предложил устройство для распознавания образов - персептрон}
        \end{scnrelfromset}
    \end{SCn}
\end{frame}

\begin{frame}{История развития искусственного\\ интеллекта}
    \topline
    \justifying
    \begin{SCn}
        \scnheader{история развития искусственного интеллекта}
        \begin{scnrelfromset}{разбиение}
            \scnitem{1959 г. – Саймон и Ньюэлл разработали программу GPS (General Problem Solving program) – универсальный решатель задач}
            \scnitem{1950-е годы – моделирование человеческого разума, создание аналогичного ему искусственного}
            \scnitem{1960-е годы – эвристическое программирование – разработка стратегии действий на основе заранее известных эвристик (Эвристика – теоретически необоснованное правило, позволяющее сократить число переборов)}
            \scnitem{1965-1975 гг. – развитие логических методов, доказательство теорем}
        \end{scnrelfromset}
    \end{SCn}
\end{frame}

\begin{frame}{История развития искусственного\\ интеллекта}
    \topline
    \justifying
    \begin{SCn}
        \scnheader{история развития искусственного интеллекта}
        \begin{scnrelfromset}{разбиение}
            \scnitem{1973 г. – создан язык Пролог}
            \scnitem{1975-е – переход от поиска универсального алгоритма мышления к моделированию знаний экспертов. Основное направление- представление знаний}
        \end{scnrelfromset}
    \end{SCn}
\end{frame}

\begin{frame}{\\Центры исследований в области ИИ}
    \topline
    \justifying
    \begin{SCn}
        \scnheader{центры исследований в области ИИ}
        \begin{scnrelfromset}{разбиение}
            \scnitem{США: Беркли, Стенфорд, Массачусетс}
            \scnitem{Япония: Университет Токио, Sony labs}
            \scnitem{Европа: Марсель, Эдинбург}
            \scnitem{Россия: Новосибирск (академгородок), Переславль-Залесский, МЭИ, МГТУ им. Баумана}
            \scnitem{Беларусь: Минск, Брест}
        \end{scnrelfromset}
    \end{SCn}
\end{frame}

\begin{frame}{Основные направления исследований в области ИИ}
    \topline
    \justifying
    \begin{SCn}
        \scnheader{исследования в области ИИ}
        \begin{scnrelfromset}{основные направления}
            \scnitem{\begin{scnindent}{нейрокибернетика \\} \scnidtf{бионика}\end{scnindent}}
            \scnitem{\begin{scnindent}{символьное \\} \scnidtf{логическое}\end{scnindent}}
            \scnitem{\begin{scnindent}{кибернетика «черного ящика»\\} \scnidtf{программно-прагматическое}\end{scnindent}}
            \scnitem{направление, ориентированное на создание смешанных систем человек-компьютер}
        \end{scnrelfromset}
    \end{SCn}
\end{frame}

\begin{frame}
\centering
\Huge
\textbf{Понятие знания, модели представления знаний}
\end{frame}

\begin{frame}{\\Данные и знание}
    \topline
    \justifying
    \begin{SCn}
        \scnheader{данные}
        \scnidtf{сведения, представленные в определенной знаковой системе и на определенном материальном носителе для обеспечения возможностей хранения, передачи, приема и обработки}
        \scnheader{знание}
        \scnidtf{хорошо структурированные данные}
        \scnidtf{данные о данных (метаданные)}
    \end{SCn}
    
\end{frame}

\begin{frame}{\\Определение знания}
    \topline
    \justifying
    \begin{SCn}
        \scnheader{знание}
        \scnidtf{основные закономерности предметной области, позволяющие человеку решать конкретные производственные, научные и другие задачи, т. е. факты, понятия, взаимосвязи, оценки, правила, эвристики(иначе фактические знания), а также стратегии принятия решения в этой области (иначе стратегические знания)}
        \scnheader{знания о предметной области}
        \scnidtf{совокупность реальных или абстрактных объектов (сущностей), связей и отношении между этими объектами, а также процедур преобразования этих объектов для решения задач в предметной области}
    \end{SCn}
    
\end{frame}


\begin{frame}{\\Свойства знания}
    \topline
    \justifying
    \begin{SCn}
        \scnheader{знание}
        \begin{scnrelfromset}{свойства}
            \scnitem{сложная структура}
            \scnitem{\begin{scnindent}{внутренняя активность \\} \scnidtf{изменение знаний влияет на поведение системы}\end{scnindent}}
        \end{scnrelfromset}
    \end{SCn}
\end{frame}


\begin{frame}{\\Классификация знаний}
    \topline
    \justifying
    \begin{SCn}
        \scnheader{знание}
        \begin{scnrelfromset}{разбиение}
            \scnitem{\begin{scnindent}{процедурные знания \\} \scnidtf{описывают последовательности действий, которые могут использоваться при решении задач}\end{scnindent}}
            \scnitem{\begin{scnindent}{декларативные знания \\} \scnidtf{все знания, не являющиеся процедурными}\end{scnindent}}
            \begin{scnindent}
	            \begin{scnrelfromset}{разбиение}
	            	\scnitem{статьи в толковых словарях и энциклопедиях}
	            	\scnitem{формулировки законов в физике, химии и других науках}
	            	\scnitem{собрание исторических фактов}
	            	\scnitem{...}
	            \end{scnrelfromset}
        	\end{scnindent}
        \end{scnrelfromset}
    \end{SCn}
\end{frame}


\begin{frame}{\\Основные задачи инженерии знаний}
    \topline
    \justifying
    \begin{SCn}
        \scnheader{инженерия знаний}
        \begin{scnrelfromset}{задачи}
            \scnitem{представление знаний}
            \scnitem{получение знаний}
            \scnitem{верификация знаний}
            \scnitem{использование знаний}
        \end{scnrelfromset}
    \end{SCn}
\end{frame}


\begin{frame}{\\Требования к представлению знаний}
    \topline
    \justifying
    \begin{SCn}
        \scnheader{представление знаний}
        \begin{scnrelfromset}{требования}
            \scnitem{наглядность}
            \scnitem{удобство представления знаний}
            \scnitem{универсальность}
            \scnitem{расширяемость}
        \end{scnrelfromset}
    \end{SCn}
\end{frame}

\begin{frame}{\\Модель представления знаний}
    \topline
    \justifying
    \begin{SCn}
        \scnheader{модель представления знаний}
        \scnidtf{формализм, предназначенный для описания статических и динамических свойств предметной области (соглашение - как описывать знания)}
        \begin{scnrelfromset}{разбиение}
            \scnitem{универсальные модели представления знаний}
            \scnitem{специализированные модели представления знаний}
        \end{scnrelfromset}
    \end{SCn}
\end{frame}


\begin{frame}{\\Основные модели представления знаний}
    \topline
    \justifying
    \begin{SCn}
        \scnheader{модели представления знаний}
        \begin{scnrelfromset}{разбиение}
            \scnitem{формальные логические модели}
            \scnitem{продукционные модели}
            \scnitem{семантические сети}
            \scnitem{фреймы}
        \end{scnrelfromset}
    \end{SCn}
\end{frame}


\begin{frame}{\\Формальные логические модели}
    \topline
    \justifying
    \begin{SCn}
        \scnheader{формальные логические модели}
        \begin{scnrelfromset}{виды}
            \scnitem{исчисление высказываний}
            \scnitem{исчисление предикатов}
            \scnitem{нечеткая логика}
        \end{scnrelfromset}
        \scntext{примечание}{не используются в промышленных разработках}
    \end{SCn}
\end{frame}

\begin{frame}{\\Формальные логические модели}
    \topline
    \justifying
    \begin{SCn}
        \scnheader{формальные логические модели}
        \begin{scnrelfromset}{достоинства}
            \scnitem{высокий уровень формализации, что обеспечивает точность получения результата}
            \scnitem{согласованность}
            \scnitem{единый способ описания знаний о предметной области и способов решения задач в предметной области}
        \end{scnrelfromset}
        
        \begin{scnrelfromset}{недостатки}
            \scnitem{ненаглядность представления знаний}
            \scnitem{очень строгие ограничения, накладываемые структурой представления знаний}
        \end{scnrelfromset}
    \end{SCn}
\end{frame}


\begin{frame}{\\Исчисление высказываний}
    \topline
    \justifying
    \begin{SCn}
        \scnheader{высказывание}
        \scnidtf{неделимое грамматически правильное предложение, являющееся истинным или ложным}
        \scnheader{сложное высказывание}
        \scnidtf{комбинация простых высказываний при помощи логических связок}
    \end{SCn}
\end{frame}


\begin{frame}{\\Исчисление предикатов}
    \topline
    \justifying
    \begin{SCn}
        \scnheader{исчисление предикатов}
        \scnidtf{система моделирования некой среды и проверки гипотез относительно этой среды при помощи разработанной модели}
        \scnheader{предикат}
        \scnidtf{функция на множестве $M=M_1 \times M_2 \times \dots \times M_n$, принимающая значение истина или ложь}
    \end{SCn}
\end{frame}


\begin{frame}{\\Продукционные модели}
    \topline
    \justifying
    \begin{SCn}
        \scnheader{продукционные модели}
        \begin{scnrelfromset}{достоинства}
            \scnitem{простота и наглядность правил}
            \scnitem{простота пополнения базы знаний}
            \scnitem{простота вывода в базе знаний}
        \end{scnrelfromset}
        
        \begin{scnrelfromset}{недостатки}
            \scnitem{несоответствие представлению знаний человеком}
            \scnitem{сложно управлять выводом при больших БЗ}
            \scnitem{сложность оценки непротиворечивости БЗ}
        \end{scnrelfromset}
    \end{SCn}
\end{frame}

\begin{frame}{\\Семантические сети}
    \topline
    \justifying
    \begin{SCn}
        \scnheader{семантические сети}
        \scnidtf{ориентированный граф, вершины которого понятия, а дуги – отношения между ними}
        \scnidtf{наиболее общий способ представления знаний}
        \scntext{пояснение}{поиск решения сводится к задаче поиска фрагмента семантической сети, соответствующего запросу}
    \end{SCn}
\end{frame}

\begin{frame}{\\Семантические сети}
    \topline
    \justifying
    \begin{SCn}
        \scnheader{семантические сети}
        \begin{scnrelfromset}{разбиение}
            \scnitem{однородные}
            \scnitem{неоднородные}
        \end{scnrelfromset}
        \begin{scnindent}
        \scntext{пояснение}{Классификация по количеству типов отношений}
        \end{scnindent}
        \begin{scnrelfromset}{разбиение}
            \scnitem{бинарные}
            \scnitem{N-арные}
        \end{scnrelfromset}
        \begin{scnindent}
        \scntext{пояснение}{Классификация по типу отношений}
        \end{scnindent}
    \end{SCn}
\end{frame}


\begin{frame}{\\Семантические сети}
    \topline
    \justifying
    \begin{SCn}
        \scnheader{семантические сети}
        \begin{scnrelfromset}{достоинства}
            \scnitem{наглядность, универсальность, простота понимания}
            \scnitem{соответствие представлению знаний у человека}
        \end{scnrelfromset}
        
        \begin{scnrelfromset}{недостатки}
            \scnitem{сложность организации процессов вывода на семантической сети}
            \scnitem{смешение различных групп знаний}
        \end{scnrelfromset}
    \end{SCn}
\end{frame}

\begin{frame}{\\Фреймы}
    \topline
    \justifying
    \begin{SCn}
        \scnheader{фрейм}
        \scnidtf{абстрактный образ для представления некоего стереотипа восприятия}
        \begin{scnrelfromset}{классификация}
            \scnitem{фреймы-образцы(прототипы)}
            \scnitem{фреймы-экземпляры}
        \end{scnrelfromset}

        \scntext{примечание}{В ходе вывода во фреймовой модели сначала подбирается прототип, а потом идет его уточнение применительно к образу}
    \end{SCn}
\end{frame}


\begin{frame}{\\Фреймы}
    \topline
    \justifying
    \begin{SCn}
        \scnheader{фрейм}
        \begin{scnrelfromset}{достоинства}
            \scnitem{интеграция знаний (декларативных и процедурных)}
            \scnitem{соответствие принципам хранения знаний человеком}
            \scnitem{наглядность, гибкость, однородность}
        \end{scnrelfromset}
        
        \begin{scnrelfromset}{недостатки}
            \scnitem{сложность управления выводом}
            \scnitem{низкая эффективность}
        \end{scnrelfromset}
    \end{SCn}
\end{frame}


\begin{frame}{\\Методы извлечения знаний}
    \topline
    \justifying
    \begin{SCn}
        \scnheader{извлечение знаний}
        \begin{scnrelfromset}{методы}
            \scnitem{\begin{scnindent}{коммуникативные методы} 
                \begin{scnrelfromset}{разбиение}
                    \scnitem{активные}
                    \scnitem{пассивные}
                \end{scnrelfromset}
                \end{scnindent}}
            \scnitem{текстологические методы}
        \end{scnrelfromset}
    \end{SCn}
\end{frame}

\begin{frame}
\centering
\Huge
\textbf{Онтологии}
\end{frame}


\begin{frame}{\\Онтологический подход}
    \topline
    \justifying
    \begin{SCn}
        \scnheader{онтология}
        \scnidtf{раздел философии, в котором изучаются наиболее общие характеристики бытия и сущностей}
        \scnidtf{это точная спецификация концептуализации, формализованное представление основных понятий и связей между ними}
        \scnidtf{эксплицитная спецификация определенной темы}
        \scnheader{концептуализация}
        \scnidtf{процесс перехода от представления предметной области на естественном языке к точной спецификации этого описания на некотором формальном языке, ориентированном на компьютерное представление}
    \end{SCn}
\end{frame}

\begin{frame}{\\Виды онтологий}
    \topline
    \justifying
    \begin{SCn}
        \scnheader{онтология}
        \begin{scnrelfromset}{виды}
            \scnitem{\begin{scnindent}{управляемый словарь терминов\\}
                    \scnidtf{ограниченный список слов и терминов, используемых для индексации и категоризации информации на сайте}\end{scnindent}}
            \scnitem{\begin{scnindent}{кольцо синонимов\\}
                    \scnidtf{управляемый словарь с перечнем терминов, их синонимов, без указания предпочтительных синонимов}\end{scnindent}}
            \scnitem{\begin{scnindent}{тезаурус\\}
                    \scnidtf{управляемый словарь с иерархической структурой, со связями и зависимостями между терминами}\end{scnindent}}
            \scnitem{\begin{scnindent}{онтология \\}
                    \scnidtf{сложный тезаурус с настраиваемыми семантическими связями}\end{scnindent}}
        \end{scnrelfromset}
    \end{SCn}
\end{frame}



\begin{frame}{\\Цели создания онтологий}
    \topline
    \justifying
    \begin{SCn}
        \scnheader{онтология}
        \begin{scnrelfromset}{цели создания}
            \scnitem{совместное использование людьми или программными агентами общего понимания структуры информации}
            \scnitem{возможность повторного использования знаний в предметной области}
            \scnitem{явное описание допущений в предметной области}
            \scnitem{отделение знаний в предметной области от оперативных знаний}
            \scnitem{анализ знаний в предметной области}
        \end{scnrelfromset}
    \end{SCn}
\end{frame}



\begin{frame}{\\Задачи решаемые с помощью онтологий}
    \topline
    \justifying
    \begin{SCn}
        \scnheader{онтология}
        \begin{scnrelfromset}{применения}
            \scnitem{создание и использование БЗ}
            \scnitem{организация эффективного поиска в БД, информационных каталогах, БЗ}
            \scnitem{создание систем, реализующих механизмы рассуждений}
            \scnitem{организация поиска по смыслу в текстовой информации}
            \scnitem{семантический поиск в Internet}
            \scnitem{представление смысла в метаданных об информационных ресурсах}
            \scnitem{построение и использование баз общих знаний (common knowledge) для различных интеллектуальных систем}
        \end{scnrelfromset}
    \end{SCn}
\end{frame}


\begin{frame}{\\Принципы создания онтологий}
    \topline
    \justifying
    \begin{SCn}
        \scnheader{онтология}
        \begin{scnrelfromset}{принципы создания}
            \scnitem{ясность}
            \scnitem{согласованность}
            \scnitem{расширяемость}
            \scnitem{минимум влияния кодирования}
            \scnitem{минимум онтологических обязательств}
        \end{scnrelfromset}
    \end{SCn}
\end{frame}


\begin{frame}{\\Языки описания онтологий}
    \topline
    \justifying
    \begin{SCn}
        \scnheader{языки описания онтологий}
        \scnidtf{формальный язык, используемый для кодирования онтологии}
        \begin{scnrelfromset}{разбиение}
            \scnitem{OWL(ontology web language), стандарт W3C}
            \scnitem{KIF (Knowledge Interchange Format или формат обмена знаниями)}
            \scnitem{CycL — онтологический язык проекта Cyc}
            \scnitem{DAML+OIL (FIPA)}
            \scnitem{SL/EL}
        \end{scnrelfromset}
    \end{SCn}
\end{frame}

\begin{frame}
\centering
\Huge
\textbf{Основные направления развития ИИ}
\end{frame}



\begin{frame}{\\Основные направления развития ИИ}
\topline
\justifying
\begin{SCn}
	\scnheader{искусственный интеллект}
	\begin{scnrelfromset}{направления развития}
		\scnitem{имитация творчества}
		\scnitem{создание средств разработки}
		\scnitem{анализ и обработка естественного языка}
	\end{scnrelfromset}
\end{SCn}
\end{frame}

\begin{frame}{\\Основные направления развития ИИ}
    \topline
    \justifying
    \begin{SCn}
        \scnheader{имитация творчества}
        \begin{scnrelfromset}{разбиение}
            \scnitem{решение игровых задач (шахматы, шашки, домино, го)}
            \scnitem{автоматическое доказательство теорем}
            \scnitem{программы анализа и синтеза музыкальных произведений}
            \scnitem{генерация стихов, сказок, афоризмов}
            \scnitem{программы моделирующие поведение}
        \end{scnrelfromset}
    \end{SCn}
\end{frame}

\begin{frame}{\\Основные направления развития ИИ}
    \topline
    \justifying
    \begin{SCn}
        \scnheader{создание средств разработки для ИИ}
        \begin{scnrelfromset}{разбиение}
            \scnitem{\begin{scnindent}
            {разработка инструментальных средств для создания интеллектуальных систем} 
                    \begin{scnrelfromset}{разбиение}
                        \scnitem{разработка новых языков программирования, ориентированных на задачи ИИ}
                        \scnitem{создание программ-оболочек для наполнения базой знаний}
                    \end{scnrelfromset}
                \end{scnindent}
                }
            \scnitem{автоматический синтез программ}
        \end{scnrelfromset}
    \end{SCn}
\end{frame}



\begin{frame}{\\Основные направления развития ИИ}
    \topline
    \justifying
    \begin{SCn}
        \scnheader{анализ и обработка естественного языка}
        \begin{scnrelfromset}{разбиение}
            \scnitem{создание ЕЯ – интерфейсов}
            \scnitem{автоматическое реферирование}
            \scnitem{автоматическая классификация документов}
            \scnitem{машинный перевод(морфология, синтаксис, лингвистика)}
            \scnitem{извлечение фактов из текстов}
            \scnitem{анализ текстов на предмет авторского права}
        \end{scnrelfromset}
    \end{SCn}
\end{frame}

\begin{frame}{\\Новые архитектуры компьютеров}
    \topline
    \justifying
    \begin{SCn}
    \small{
        \scnheader{архитектура фон Неймана}
        \begin{scnrelfromset}{принципы}
            \scnitem{двоичное представление}
            \scnitem{программное управление}
            \scnitem{однородность памяти}
            \scnitem{адресуемость памяти}
            \scnitem{последовательное программное управление}
            \scnitem{условные переходы}
        \end{scnrelfromset}
        \scntext{примечание}{Существующие ЭВМ используют архитектуру фон Неймана и неэффективны в плане символьной обработки.
        Основная цель – разработать ЭВМ лучше подходящие для решения отдельных типов задач.}
    }\end{SCn}
\end{frame}

\begin{frame}{\\Новые архитектуры компьютеров}
    \topline
    \justifying
    \begin{SCn}
        \scnheader{новые архитектуры компьютеров}
        \begin{scnrelfromset}{разбиение}
            \scnitem{ассоциативные процессоры}
            \scnitem{машины баз данных}
            \scnitem{параллельные компьютеры}
            \scnitem{векторные компьютеры}
            \scnitem{графодинамический подход}
        \end{scnrelfromset}
    \end{SCn}
\end{frame}

\begin{frame}{\\Интеллектуальные роботы}
    \topline
    \justifying
    \begin{SCn}
        \scnheader{робот}
        \scnidtf{электромеханическое устройство для автоматизации человеческого труда}
        \begin{scnrelfromset}{проблемы}
            \scnitem{машинное зрение}
            \scnitem{хранение и обработка трехмерной визуальной информации}
        \end{scnrelfromset}
    \end{SCn}
\end{frame}